\documentclass[12pt]{article}
%\documentclass[AEJ]{AEA}
\usepackage[round, sort , authoryear]{natbib}


\usepackage{pgf}
%%% PAGE DIMENSIONS
\usepackage[margin=2cm]{geometry}
%\usepackage[top=1.5in, bottom=1.5in, left=1.5in, right=1.5in]{geometry}
\usepackage{blindtext} % to change the page dimensions
%\geometry{a4paper} % or letterpaper (US) or a5paper or....
%\geometry{margin=1.5in} % for example, change the margins to 2 inches all round
% \geometry{landscape} % set up the page for landscape
%   read geometry.pdf for detailed page layout information


%required by R
\usepackage{booktabs}
\usepackage{longtable}
\usepackage{array}
\usepackage{multirow}
\usepackage{wrapfig}
\usepackage{float}
\usepackage{colortbl}
\usepackage{pdflscape}
\usepackage{tabu}
\usepackage[normalem]{ulem}
\usepackage[normalem]{ulem}
\usepackage[utf8]{inputenc}
\usepackage{makecell}
\usepackage{xcolor}
\usepackage{dcolumn}
\usepackage{mathtools}% http://ctan.org/pkg/mathtools


%Fort title position
\usepackage{titling}

%For new line in table
\usepackage{array}
\usepackage{makecell}



%Write a vecor
\newcommand{\myvec}[1]{\ensuremath{\begin{pmatrix}#1\end{pmatrix}}}

%New environments
\newtheorem{lemma}{Lemma}
\newtheorem{assumption}{Assumption}
\newtheorem{proposition}{Proposition}
\newtheorem{definition}{Definition}
\newenvironment{proof}[1][Proof]{\noindent \textbf{#1.} }{\  \rule{0.5em}{0.5em}}

%For tables position
\usepackage{float}
\restylefloat{table}


%%% PAGE DIMENSIONS
%\usepackage[margin=2.5 cm]{geometry}
%\usepackage{blindtext} % to change the page dimensions
%\geometry{a4paper} % or letterpaper (US) or a5paper or....
%\geometry{margin=1.5cm} % for example, change the margins to 2 inches all round
% \geometry{landscape} % set up the page for landscape
%   read geometry.pdf for detailed page layout information

\usepackage[utf8]{inputenc} 
\usepackage{graphicx} % support the \includegraphics command and options
\usepackage{epstopdf}
\usepackage[hang]{footmisc}
\usepackage{lipsum}
\usepackage{setspace}
% \usepackage[parfill]{parskip} % Activate to begin paragraphs with an empty line rather than an indent
\usepackage{pgfplots}
\pgfplotsset{compat=1.13}
\usepackage{caption}
\usepackage{threeparttablex}
\usepackage{color, colortbl}
\definecolor{Gray}{gray}{0.9}
%%% PACKAGES
\usepackage{placeins}
\usepackage{booktabs} % for much better looking tables
\usepackage{array} % for better arrays (eg matrices) in maths
\usepackage{paralist} % very flexible & customisable lists (eg. enumerate/itemize, etc.)
\usepackage{verbatim} % adds environment for commenting out blocks of text & for better verbatim


% These packages are all incorporated in the memoir class to one degree or another...
%\usepackage{amsmath}
\numberwithin{table}{section}
\usepackage{cases}
\usepackage{graphicx}
%
\usepackage{float}
\usepackage{authblk}
\usepackage{pgfplots}
\usepackage{pdfpages}
\linespread{1.5}
\setlength{\footnotemargin}{4mm}
\usepackage{amssymb} 
\usepackage{tabularx}
\usepackage[linesnumbered,ruled,vlined]{algorithm2e}
\addtolength{\footnotesep}{2mm} % change to 1mm

%For Counting Figures in the Appendix
\usepackage{chngcntr}

%No Counting within section
\DeclareCaptionLabelSeparator{none}{}
\captionsetup{labelsep=none}
\counterwithout{figure}{section}
\counterwithout{table}{section}

%footnote space
\setlength{\footnotesep}{0.35cm}
\interfootnotelinepenalty=10000


%Nice Figure and table headers
\captionsetup[figure]{labelfont={sc},name={Figure},labelsep=period}
\captionsetup[table]{labelfont={sc},name={Table},labelsep=period,justification=centering}

%For subtables
\usepackage{subcaption}
\usepackage[justification=centering]{caption}
%For having the catption above tables
\usepackage{float}
\floatstyle{plaintop}
\restylefloat{table}

%For bar charts in table
\newlength\MAX  \setlength\MAX{20mm}
\newcommand*\Chart[1]{#1~\rlap{\textcolor{blue!20}{\rule{#1\MAX}{2ex}}}}
%\newcommand*\Chartguys[1]{\rlap{\textcolor{blue!30}{\rule{6ex}{\MAX}}}{\textcolor{white!30}{\rule{6ex}{#1\MAX}}}}
%\newcommand*\Chartgirls[1]{\rlap{\textcolor{red!30}{\rule{6ex}{\MAX}}}{\textcolor{white!30}{\rule{6ex}{#1\MAX}}}}
\newcommand*\Chartguys[1]{{\textcolor{blue!30}{\rule{6ex}{#1\MAX}}}}
\newcommand*\Chartgirls[1]{{\textcolor{red!30}{\rule{6ex}{#1\MAX}}}}


\renewcommand{\thesubfigure}{ (\alph{subfigure})}
\captionsetup[sub]{labelformat=simple}
%Math stuff
\DeclareMathOperator*{\argmax}{arg\,max}

\setcounter{MaxMatrixCols}{10}
%TCIDATA{OutputFilter=LATEX.DLL}
%TCIDATA{Version=5.50.0.2960}
%TCIDATA{<META NAME="SaveForMode" CONTENT="1">}
%TCIDATA{BibliographyScheme=BibTeX}
%TCIDATA{LastRevised=Monday, August 27, 2018 20:06:51}
%TCIDATA{<META NAME="GraphicsSave" CONTENT="32">}
%TCIDATA{Language=American English}



%\restylefloat{table}
%\oddsidemargin  +0.18in
%\evensidemargin +0.18in
%\topmargin 40pt \textheight 8.1in \textwidth 6.5in
%\linespread{1.5}\parskip .05in






%Nice Figure and table headers
\captionsetup[figure]{labelfont={sc},name={Figure},labelsep=none}
\captionsetup[table]{labelfont={sc},name={Table},labelsep=none,justification=centering}

\usepackage{booktabs}   % for nice tables
\usepackage[colorlinks=false, linktocpage=true]{hyperref}
%\usepackage[flushmargin]{footmisc}
%\addtolength{\footnotesep}{3mm} % change to 1mm
\hypersetup{
	colorlinks,
	linkcolor={blue!50!black},
	citecolor={blue!50!black},
	urlcolor={blue!80!black}
}
% use for hypertext
\usepackage[colorinlistoftodos]{todonotes}
\setlength{\marginparwidth}{2cm}
\newenvironment{customlegend}[1][]{%
	\begingroup
	% inits/clears the lists (which might be populated from previous
	% axes):
	\pgfplots@init@cleared@structures
	\pgfplotsset{#1}%
}{%
	% draws the legend:
	\pgfplots@createlegend
	\endgroup
}%
%For Figures, below

\usepackage{tikz}
\usetikzlibrary{shapes}
\usepgflibrary{arrows} % LATEX and plain TEX and pure pgf
\usepgflibrary[arrows] % ConTEXt and pure pgf
\usetikzlibrary{arrows} % LATEX and plain TEX when using Tik Z
\usetikzlibrary[arrows] % ConTEXt when using Tik Z
\usepackage{hyperref}



%%% The "real" document content comes below...


%%%%%%%%%%%%%%%%%%%%%%%%%%%%%%%%%%%%%%%%%%%%%%%%%%%%%%
%%%%%%%%%%%%%%%%%%%%%%%%%%%%%%%%%%%%%%%%%%%%%%%%%%%%%%
%\parskip3mm\parindent0cm


\title{(Changing) Marriage and Cohabitation Patterns in the US: do Divorce Laws Matter?\thanks{The authors wish to thank David de la Croix, Matthias Doepke, Fabio Mariani, Jesús Fernández-Villaverde, Alessandra Voena and the Northwestern Macro group for their useful comments. Computational resources were provided by the supercomputing facilities of the Université catholique de Louvain (CISM/UCL) and Northwestern University. Blasutto acknowledges financial support from the French speaking community of Belgium (ARC project 15/19-063 on \textit{family transformations} and \textit{mandat d' aspirant} FC 23613).}}
\author{\large Fabio Blasutto$^1$ \quad Egor Kozlov$^2$}
\date{}

\begin{document}
	 	\tikzstyle{block} = [draw, fill=white, rectangle, 
	minimum height=3em, minimum width=6em]
	\tikzstyle{sum} = [draw, fill=white, circle, node distance=1cm]
	\tikzstyle{input} = [coordinate]
	\tikzstyle{output} = [coordinate]
	\tikzstyle{pinstyle} = [pin edge={to-,thin,black}]
	
	
	%Gauss distribution
	\pgfmathdeclarefunction{gauss}{2}{%
		\pgfmathparse{1/(#2*sqrt(2*pi))*exp(-((x-#1)^2)/(2*#2^2))}%
	}
	% Bibliography style, important for biblatex functioning	
	\bibliographystyle{apa}
	
	\maketitle
	
\vspace{-2cm}
\begin{center}\large	
	Job Market Paper\\
	\href{https://drive.google.com/file/d/1utFkMPdUR5yb07oyJI7Vx1BJDLfYgMHi/view?usp=sharing}{Latest version available here}
\end{center}
\begin{center}\large
	\today
\end{center}



	\footnotetext[1]{IRES/LIDAM, UCLouvain \& FNRS (Belgium). Email:  \tt{fabio.blasutto@uclouvain.be}}
	\footnotetext[2]{Northwestern University. Email:  \tt{egorkozlov2020@u.northwestern.edu}.\\}
	
%\begin{abstract}
%	Marriage differs from informal cohabitation for its higher cost of breaking up. Moving from a mutual to a unilateral divorce regime drastically increases the risk of dissolution. We explore how this change affects partnership choice and the relative power within the couple. Exploiting the staggered introduction of unilateral divorce across the US states, we show that after the reform newly formed relationships are significantly more likely to be cohabitations instead of marriages, and that cohabitation spells last longer. To understand the drivers of these changes, we build and estimate a structural life cycle model with partnership choice, female labor supply and savings decisions, where the gains of marriage with respect to cohabitation come from a better risk sharing and specialization within the household, enforced through a costly divorce. In the model, the reform increases the risk of divorce, making cohabitation preferred by couples that would have had the highest risk of divorce under the mutual consent regime. Since the match quality of cohabitations is on average lower than that of marriages, the reform increases their length: we estimate that time spent cohabiting would have been four times lower if the law had never changed. In states where assets upon divorce are split equally, men would lose most of their wealth upon divorce: at equilibrium, many women accept to cohabit in exchange for a higher bargaining power. 
%\end{abstract}

%\begin{abstract}
%	This paper analyzes the role of unilateral divorce for the rise in cohabitation. Exploiting the staggered introduction of unilateral divorce across the US states, we show that after the reform singles become more likely to cohabit than to marry, and that newly formed cohabitation spells last longer. We then provide a theoretical rationale for these facts, building a life-cycle model with partnership choice, female labor supply and saving decisions. A structural estimation of the model suggests that the decline in marriage is due to an increased risk of divorce, while new cohabitation spells last longer for a selection effect. A counterfactual experiment reveals that time spent cohabiting would have been four times lower if divorce law had never changed.
%\end{abstract}


%\begin{abstract}
%	This paper analyzes the role of unilateral divorce for the rise of cohabitation. Exploiting the staggered introduction of unilateral divorce across the US states, we show that after the reform singles become more likely to cohabit than to marry, and that newly formed cohabitation spells last longer. We then provide a theoretical rationale for these facts, building a life-cycle model with partnership choice, female labor supply and saving decisions. A structural estimation of the model suggests that unilateral divorce reduces the gains from marriage, making cohabitation preferred to couples that would have had the highest risk of divorce. Since married couples are on average better matched, the reform increases the length of newly formed cohabitations because of a selection effect. A counterfactual experiment reveals that time spent cohabiting would have been four times lower if divorce law had never changed.
%\end{abstract}


\begin{abstract}
This paper analyzes the role of unilateral divorce for the rise of unmarried cohabitation. Exploiting the staggered introduction of unilateral divorce across the US states, we show that after the reform singles become more likely to cohabit than to marry, and newly formed cohabitations last longer. We then provide a theoretical rationale for these facts, building a life-cycle model with partnership choice, female labor force participation, and saving decisions. A structural estimation of the model suggests that unilateral divorce decreases couples’ stability, making cohabitation preferred to couples that would have been at high risk of divorce if they marry. As cohabiting couples formed after the reform are better matched, the average length of cohabitations increases. A counterfactual experiment reveals that the time spent cohabiting would have been almost halved if the divorce laws had never changed.
\end{abstract}


%\clearpage
%\noindent \medskip{}
\clearpage
\section{Introduction}

%Informal cohabitation is on the rise: the share of women that ever cohabited in the United States moved from 33\% in 1987 to 60\% in 2010 \citep{manning2013}. Since cohabiting couples display different outcomes than married ones, this rise definitely contributes to the overall changes in the structure and behavior of the American family. For example, \cite{bumpass2000} claim that the higher instability of cohabitation is responsible for the rise in the number of single mothers, which is linked to worse children's outcomes \citep{chetty2018}. The first step to find out why cohabitation is surging and why its outcomes differs from marriage is to better understand partnership choice, selection and cohabiting couples behavior.

%Moreover, \cite{poortman2012} point out that couple specific investments of those who cohabit are lower than those who are married.
%Since disadvantaged social classes are those who are more likely to cohabit instead of marrying, partnership choice can be an engine of economic inequality

Unmarried cohabitation is on the rise: the share of women that ever cohabited in the United States moved from 33\% in 1987 to 60\% in 2010 \citep{manning2013}. This increase contributed to the overall changes in the structure and behavior of the American family. The higher instability of cohabitation plausibly contributes to the rise in the number of single mothers (\cite{bumpass2000}), which is associated with poor outcomes for children \citep{chetty2018,mclanahan2013}. Cohabiting people spend less money on relationship-specific investments % maybe any examples here?
\citep{poortman2012}. Finally, their children's well-being is worse even after controlling for parental resources \citep{brown2004}. However, it is not clear to what extent cohabitants' outcomes are due to selection % elaborate what kind of?
versus the direct effect of the form of partnership on the couple's behavior. Quantifying these two mechanisms' relative importance can be done only by understanding the rationale for the partnerships choices. Why do people cohabit instead of marrying? Why did cohabitation become more common over time?

Our paper addresses these questions by focusing on a major US policy change that took place mostly during the 1970s. During this period, most of the states have made divorce considerably easier by switching from the mutual consent divorce regime, requiring both spouses to agree to divorce, towards unilateral divorce, in which one spouse's decision was enough to initiate the procedure. The paper explores what role does unilateral divorce have for the rise of cohabitation. Since marriage and cohabitation can be viewed as contracts whose attractiveness depends on their termination rights and costs, the switch from mutual consent to unilateral divorce represents a unique opportunity to learn about partnership choices.

We answer the questions with four contributions. First, we offer a few pieces of empirical evidence. We show that under the unilateral divorce people prefer cohabitation to marriage more often. We also argue that the newly formed cohabitations last longer. Second, we propose a theory of partnership choice and endogenous breakup/divorce. In particular, in our model the policy makes cohabitation the preferred option to couples that would have experienced the highest risk of a costly divorce. Therefore, the average match quality of cohabiting couples increases, which causes cohabitation spells to last longer. % A theory professor asked me once if something I claim to be a result is generic or is it specific to my choice of parameters. This may be a useless thing, but maybe the wording can slightly be adjusted top.
Third, we structurally estimate our model to match the empirical findings about the divorce regimes' transition. Fourth, we perform several counterfactual experiments to understand the overall role of unilateral divorce on the rise of cohabitation. Thus, this paper consists of four parts, one for each contribution, which we now describe in more detail.

%Since the seminal work of \cite{becker1981}, economists studied the incentives behind the decision to marry, among which the sharing of public goods, the division of labor to exploit comparative advantages (\citealp{chiappori1997}), and risk sharing (\citealp{voena2015} and \citealp{rigas2015}). These reasons tell us why couples live together, while they are actually silent about the choice between marriage and informal cohabitation. Those two partnerships can be viewed as contracts subject to different rules, which in particular regulate costs and rights to end these agreements. While the cost of divorce is higher than the one of just breaking up,\footnote{Henceforth we will refer to the separation from cohabitation as breakup to avoid confusion with the legal separation.}, the introduction of unilateral divorce during the 1970s shifted the property rights from the spouse who wished to stay married to the one who wished to divorce. Hence, the switch from mutual consent to unilateral divorce represents a unique opportunity to learn how people choose between marriage and cohabitation.
%In particular, unilateral divorce might have eroded the gains from marriage, both increasing the likelihood of a costly divorce and hampering couples' ability to commit, thus making some couples willing to cohabit instead of marrying. In addition, shifting to a less committed relationship hampers couples' ability to specialize and it might affect the relative bargaining  power of partners. 

%In this paper, we study the role of unilateral divorce on the rise of cohabitation. Our main contribution is to show that the rate of cohabitation surged after the changes in the US law because the increased risk of divorce made marriage less attractive. In particular, cohabitation becomes the preferred choice to couples that would have experienced the highest risk of divorce. As a consequence, the average match quality of cohabiting couples increases, which cause cohabitation spells to last longer. Interestingly, men are those who wish to cohabit the most after the reform because they would lose a relatively higher share of their wealth upon divorce. This increases the bargaining power of women entering cohabitation.

In the first part of the paper we document the effect of unilateral divorce on the choice between marriage and cohabitation and the duration of newly formed cohabitations. We use data from the National Survey of Family and the Household (NSFH) and from the National Survey of Family Growth (NSFG) to study the choice between marrying and cohabiting. Then, exploiting the exogenous variation coming from the staggered introduction of unilateral divorce over time across the US states, we estimate that couples formed after the policy change are 5-7\% more likely to choose cohabitation over marriage than in the pre-reform period. % Give some baseline numbers here?
Interestingly, the effect's size depends on how property is divided upon divorce, being strongest in states where each spouse gets half of the wealth and where the judge decides the assets' shares. This suggests that divorce allocations affect partnership choices.
%This suggests that the result cannot be entirely driven by divorce being a less binding commitment technology,\footnote{This mechanism is highlighted by \cite{lafortune2020}, where the equal division of assets within marriage makes it possible to efficiently invest in children even after the rise of unilateral divorce.} since the egalitarian property division rule can act as a substitute for the mutual consent regime.
Moreover, we analyze how unilateral divorce affected cohabitation spells duration: our estimates from a multinomial probit show that cohabitations formed after the reform last longer because people both marry less and break up less.\footnote{Hereafter we refer to the separation from cohabitation as a breakup to avoid confusion with legal separation.}

In the second part of the paper, we propose a theory to understand the mechanisms underlying the facts documented in the empirical part. We build a dynamic model of intra-household decision making and search in the mating market, where agents make decisions according to the realization of idiosyncratic permanent income shocks, their amount of wealth and couple-specific match quality. With some probability, single agents meet a potential partner drawn from an exogenous distribution of match quality, productivity, and wealth. After the draw, they decide whether to marry, cohabit, or stay single. Couples make decisions about consumption, savings, and female labor force participation. Women receive a productivity penalty for not working, and women's time can be used to produce a public good that captures utility gains from children, durable goods, and services.

In the model, cohabitation and marriage differ in their splitting costs and the way property is divided when the couple dissolves. Moreover, there is a stigma affecting cohabitations, which is modeled as an exogenous disutility flow. It can capture the shame towards out-out-wedlock births, premarital sex, and premarital cohabitation at the time of the divorce revolution. In the case of a breakup, assets are split according to individual property rights. In the case of divorce, we assume they are divided in half. We estimate the model using community property states data to be consistent with this assumption. Breakup (unilateral divorce) can be initiated unilaterally, as opposed to mutual consent divorce, which requires both partners' agreement. Following \cite{voena2015}, under mutual consent the couple always cooperates while married, which implies that the allocation of resources corresponds to the Pareto-efficient inter-temporal allocation. % Wait is this even true???
Instead, the fact that just one spouse can decide to terminate the relationship results in a lack of commitment, making the intra-household allocation of decision power responsive to shocks. Hence, abandoning the mutual consent regime affects the risk of divorce and, in turn, the surplus of marriage. 

The divorce costs and rights affect the gains of marriage relative to cohabitation through three main channels. First, by acting as commitment technologies, they enforce a better risk sharing and a more efficient specialization in public goods production within marriage. We assume that public goods are produced with money and women's time. When a woman stops working to produce such a good, her potential productivity in the labor market decreases. Second, divorce costs increase the risk of being ``trapped" in a bad marriage. Third, they affect the expected value of marriage by modifying the risk and value of divorce. The effects of tightening or relaxing the barriers to divorce depend on which channel prevails. For example, the introduction of unilateral divorce has an uncertain impact on the share of couples that cohabit and marry. The outcomes of cohabitation and marriage depend not only on the rules underlying these contracts but also on sorting on unobserved relationship quality. A cheap breakup is most attractive to couples whose match quality is low. The couples' ability to cooperate depends on the match quality itself through its effect on the couples' stability: making partnership-specific investments is more comfortable when the risk of splitting is low, which is when the match quality is high.

%In our framework marriage gains depend on the quality of the match: for couples whose value is low cohabitation is the best choice because breaking up is cheaper and their risk of splitting is high. Interestingly, the ability of the couple to cooperate is directly affected by the quality of the match. For this reason, the outcomes of cohabitation and marriage depend not only of the rules underlying these contracts, but the ability of the couple to cooperate is directly affected by the quality of the match.



%The gains of marriage with respect to cohabitation come from a better risk sharing and a more efficient specialization in the production of public goods,\footnote{We assume that public goods are produced with money and women's time. When women stop working for producing such a good their potential productivity in the labor market decreases.} which arise from a stronger commitment, enforced through the threat of a costly divorce and the mutual consent regime. On the other hand, couples facing the highest risk of splitting (i.e. because of a low match quality draw) are more likely to choose cohabitation, since breaking up is cheaper. Interestingly, the ability of the couple to cooperate is directly affected by the quality of the match. For this reason, the outcomes of cohabitation and marriage depend not only of the rules underlying these contracts, but also on sorting on match quality.

% After unilateral divorce is introduced, the dissolution of marriage can be initiated by one spouse without the other spouse's consent. We model the decision making in the couple building on the literature on limited commitment (see for example \cite{kk1996},  \cite{ligon2002}, \cite{marcet2019} and \cite{pavoni2018}), which has been applied to dynamic collective models in the household by \cite{voena2015}, \cite{mazzocco2007}, \cite{foerster2019} and \cite{lise2018} among others. In this framework, when one member of the couple wishes to split, the couple rebargains such that the binding member is made indifferent between breakup and staying in a couple.\footnote{For particularly bad draws in the match quality or productivity shock this might not be possible and hence the couple would split.}

In the third part of the paper we structurally estimate the model to understand the quantitative relevance of the mechanisms that drive partnership choice. The model is estimated by indirect inference using to match the regression results from our empirical analysis, mating market moments (NSFH), and female labor supply moments (PSID). The introduction of unilateral divorce is modeled as an unexpected policy change. The estimated model closely replicates the targeted moments. Our over-identification checks support the estimation results: the model can match several non-targeted moments, for instance, the impact of unilateral divorce on cohabitation duration.

According to the estimates, a switch from mutual consent to unilateral divorce causes couples to start cohabiting more by reducing the married couple's ability to cooperate and by increasing the likelihood of a costly divorce.\footnote{An increased likelihood of divorce can reduce the ability of the couple to cooperate by itself. Yet it also directly affects the marriage surplus by reducing the possibility of losing assets upon divorce. For example, if there was no wage uncertainty and women always participated in the labor market, unilateral divorce would affect marriage gains via the direct effect only.} Since cohabiting couples that would have married under the older regime are better matched than the average cohabiting couple, the reform increases the stability and the duration of newly formed cohabitations. To further deepen our understanding of the mechanisms, we analyze the intra-household bargaining changes that followed the policy change. We find that the average Pareto weight of cohabiting women increases because men, fearing to lose most of their assets because of divorce, convince women to cohabit instead of marrying in exchange for more power in the couple. This mechanism is specific to the divorce regime where assets are split evenly. Instead, if spouses always keep owning their assets separately, men would not need to choose cohabitation to insure their property. Consistently with empirical evidence, the impact of unilateral divorce on cohabitation likelihood is lower in the model under separate ownership. % This paragraph is still long and can be split somehow

The fourth and last part of the paper conducts a series of counterfactual experiments to understand the quantitative importance of the forces that contributed to the rise of cohabitation. To assess the role of unilateral divorce, we perform a counterfactual experiment where the unilateral divorce was never introduced. We find that people on average would have spent 1.24 years cohabiting instead of 2.19, while only 29.1\% of people would have ever cohabited instead of 43.3\%. In the second series of counterfactuals, we find that a 20\% decrease in gender productivity gap and a 10\% drop in market prices of goods increase respectively by 9\% and 3\% the share of people that ever cohabited at age 39. Both effects are driven by a reduced scope for household specialization, which is better exploited within marriage. We also study various channels of how unilateral divorce affects welfare. The possibility to cohabit limits the welfare losses for men who can secure their assets, while women suffer more because of couple-specific investments like children deteriorate their value of divorcing more than for men.

\textbf{Literature.} This paper adds to three strands of the literature. First, by documenting how divorce laws influence the choice between marriage and cohabitation, we add to the existing literature that studies the effects of unilateral divorce. This policy change has been shown to affect the rate of divorce (\citealp{friedberg1998}, \citealp{wolfers2006}), female labor supply (\citealp{stevenson2008}, \citealp{voena2015}), savings (\citealp{voena2015}), marriage rates (\citealp{rasul2003,rasul2006}), children's well-being (\citealp{gruber2004}), family violence (\citealp{stevenson2006}), marriage-specific capital (\citealp{stevenson2007}), assortative mating (\citealp{reynoso2019}), the rise in serial monogamy (\citealp{de2015}),  and prostitution (\citealp{ciacci2017}), among the others. We complement the findings of \cite{rasul2003,rasul2006} by showing that the decrease in marriage rates after the introduction of unilateral divorce is not only driven by more people staying single, but also by more people that choose to cohabit. This suggest that marriage and cohabitation are substitutes.\footnote{Cohabitation can also be a substitute for being single or dating, as \cite{rindfuss1990} point out. Moreover, \cite{blasutto2020} and \cite{brien2006} claim that cohabitation can also be a complement for marriage, which allows the couple to learn about its match quality before making the binding decision of getting married.} Our paper builds on \cite{voena2015},  which studies how the interaction of unilateral divorce with property rights upon divorce affected married couples' household behavior. We extend her work both by considering cohabitation as an alternative relationship and by analyzing selection into partnership. This paper also extends the work of \cite{fernandez2017} by showing that not considering cohabitation as an alternative to marriage biases upwards the negative impact of unilateral divorce on men's welfare. The intuition is that men can limit the losses stemming from the increased risk of divorce by cohabiting.

Second, our paper adds to the literature that studies the choice between marriage and cohabitation. A first subset thereof has focused on identifying the gains of marriage and cohabitation, highlighting the role of commitment \citep{matouschek2008}, specialization within the couple \citep{gemici2014}, learning about match quality \citep{brien2006}, income dynamics \citep{blasutto2020}, assets \citep{lafortune2017, lafortune2020} and investment in children \citep{lundberg2015}. We extend these works by showing how an increase in the easiness of divorce decreases the couple's ability to cooperate and makes divorce allocations more relevant for partnership choices, since the likelihood of divorce increases. Consequently, the relative power of potential partners and the rules about the division of assets upon divorce become crucial. These results highlight a new role of partners' relative power and assets for partnership choices, which is analyzed within a framework that extends the theory of \cite{blasutto2020} and \cite{gemici2014} by including saving decisions.\footnote{\cite{lafortune2020} also highlight the role of assets: our model features their intuition that assets can act as a commitment technology, but our framework also allows assets to influence partnership choices via a direct effect of divorce's risk. Thanks to this mechanism, we can explain why unilateral divorce caused cohabitation to increase more in community property states than in title-based ones.}

Another subset of these papers studies the effect of changes in cohabitants' rights on partnership choices and cohabiting couples' behavior, highlighting the role of alimony rights  (\citealp{chiappori2017,gousse2018}), taxation \cite{leturcq2012} and division of assets at breakup (\citealp{fisher2012,gousse2018,chigavazira2019}). We extend this literature by showing that the introduction of unilateral divorce impacts both the choice to cohabit and cohabitation's stability, even though cohabitant's rights are not directly affected. Further, the effects on the intention to cohabit depends on property division rights, which indicates that partnership choices depend on divorce allocations. This evidence suggests that changes in family law should be designed considering that marriage and cohabitation are substitutes. 
%\cite{chiappori2017},\cite{persson2020}, \cite{leturcq2012}, \cite{gousse2018}
%Second, our paper expands the literature on cohabitation, showing how the risk of divorce is essential for understanding partnership choices. Different papers in the literature highlighted various gains of marriage with respect cohabitation: commitment \citep{matouschek2008}, labor specialization within the couple \citep{gemici2014}, learning about match quality \citep{brien2006}, income dynamics \citep{blasutto2020} and investment in children \citep{lafortune2020}. A paper closely related to ours is \cite{lafortune2020}, who highlight the role of assets as a collateral that enforces commitment and hence allows for optimal investment in children. While we think that their mechanism is important, it cannot be the whole story since we observe that the switch towards cohabitation is the strongest in states where assets are divided more equally, where their model would predict that the effect is the weakest. In this paper, we quantify how much of the shift is due to the erosion of marriage's collateral versus the role of the increased risk of divorce. We extend the work of \cite{gemici2014} and \cite{blasutto2020}, who develop collective dynamic models with limited commitment, introducing asset accumulation decisions, which are shown to play an important role for commitment and hence for partnership choice, especially when interacted with property rights upon divorce. Other papers highlighted how

Finally, this paper is tied to the extensive literature that studies the changes in the American household character over the last decades. Various studies explored the role of health improvements, wage distribution and dynamics, norms and technology on the rise in female labor force participation \citep{fernandez2004,greenwood2005,albanesi2016,greenwood2016}, the changes in household formation and dissolution \citep{greenwood2016,ciscato2019}, the rise in positive assortative mating \citep{fernandez2005,greenwood2016,ciscato2019} and the increase in the age at marriage \citep{santos2016}. We extend this literature by showing that the introduction of unilateral divorce implied a rise in cohabitation. Advances in the home production technology and the reduction in the gender wage gap also contributed to the rise.

Our modelling of the decision making in the couple builds on existing literature on limited commitment (\cite{kk1996},  \cite{ligon2002}, \cite{marcet2019} and \cite{pavoni2018}), which has been applied to dynamic collective models in the household by \cite{mazzocco2007}, \cite{mazzocco2013}, \cite{bayot2015}, \cite{rigas2015}, \cite{voena2015}, \cite{abraham2018}, \cite{lise2018}, \cite{low2018}, \cite{foerster2019} and \cite{reynoso2019} among others. The fact that couples cannot commit ex-ante to any possible division of marital surplus contributes to the creation of an imperfectly transferable utility environment, under which the Becker-Coase theorem does not hold (see \cite{galichon2019} and \cite{weber2018}, who applies the theorem to related environments). The imperfect transferability feature is important: according to \cite{becker1977}, divorce laws should not affect separation decisions \textit{``if all compensations between spouses were feasible and costless"}. \cite{chiappori2015} and \cite{fella2004} discuss the assumptions of the Becker-Coase theorem more.


The paper is organized as follows. Section 2 offers an overview of US divorce laws. Section 3 documents the effect of introducing unilateral divorce on partnership choices. Section 4 presents and develops the theoretical model. Section 5 describes the model's estimation, while section 6 discusses the main mechanisms of the model. Section 7 reports the results of the welfare analysis. Section 8 performs a series of counterfactual experiments, while Section 9 contains the conclusion.

%\begin{enumerate}
%\item \cite{chigavazira2019}: What: extending  equitable  property  division  divorce  laws  to  unmarried cohabiting couples in Australia. Findings: men increase their employment and women increase time spent on housework. Couples have more children and are more likely  to  become  home  owners. Strategy: triple Diff-in-Diff: state-time-relationship(coh/mar). They select a sample of marriages and cohabitation that stays intact at the end of the time span[this seems wrong to me]. Then they claim that the effect is not due to selection, even thought they check that in a sloppy way without a duration model, which seems the thing to do.
%\item \cite{chiappori2017}: what:  granting alimony rights to cohabiting couples in Canada. Finsings: Among couples united before the reform, obtaining the right to petition for alimony led women to lower their labor force participation but not among newly formed cohabiting couples. Cool difference lead by selection+commitment
%\item \cite{persson2020}: VERY IMPORTANT PAPER. Exploiting Sweden’ s elimin ation of s ur v ivors insur ance, I demon strate that severin g this link (1) affected entry into marriage up to 50 years before expected payout, (2) raised the divorce rate by 10pp, and (3) raised the assortativeness of matching (4) increased rate of cohabitation and of separation, consistent with selection along match quality
%
%\end{enumerate}
\section{US Divorce and Cohabitation Laws: an Overview} % This should be a subsection, otherwise it is a bit short...
\textbf{Divorce Laws.} Between the late 1960s and early 1980s, most US states experienced fundamental changes in the divorce law. These changes affected both the right to initiate a divorce without the other spouse's consent and about the division of assets upon divorce.

Before the 1960s the vast majority of US states had a mutual consent divorce regime.\footnote{All the states apart from New Mexico, Oklahoma and Alaska.} Both spouses' agreement was needed to obtain a divorce for mundane reasons (i.e., without misconduct by any spouse). Both spouses' agreement was needed to obtain a divorce for mundane reasons (i.e., without misconduct by any spouse). However, divorce was still permitted for grounds showing guilt of misconduct by any of the two spouses: for those cases, the innocent party's agreement alone was enough for having a divorce granted. Examples of guilt or misconduct are adultery or abandonment.

From the late 1960s and early 1980s, most US states switched to a unilateral divorce regime. Under this regime divorce can be filed by one spouse without the consent of the other. More detailed chronology about the unilateral divorce introduction was in different states can be found in Table 1 of \cite{ciacci2017}% I think this needs an appendix

Another dimension along which divorce laws differ across states and over time is property division regimes. In the United States, there are three types of these regimes:
\begin{enumerate}
\item \textit{Community Property}. Under this regime the couple is jointly owning family wealth, both that obtained during the marriage and before. This implies that when divorce occurs, each spouse gets precisely half of the total family wealth.
\item \textit{Equitable distribution}. Under this regime, the court decides how to split family wealth between the two spouses. This decision is driven by the principle of equity, which is ambiguous. In some cases, the wealth is divided exactly in half; in others, a larger share reserved for the party that contributed the most to its accumulation.
\item \textit{Title Based Regime}. Under this regime, wealth is split according to the title of ownership, as the spouses own their assets separately.
\end{enumerate}
The possibility of signing prenuptial agreements gives to the couple the possibility of splitting assets differently than it is dictated by the law, but legal scholars believe that their effect is quite limited. In fact, these contracts could not be enforced by courts until the 1970s. After the introduction of the Uniform Premarital Agreements Act of 1983, it has been easier to enforce these contracts even though today prenuptial agreements are signed in a minority of marriages (5-10\%) according to \cite{rainer2007}, which might be due to social stigma or lack of information on their benefits \citep{mahar2003}.\\
%Rights of cohabitors,literature
%\begin{enumerate}
%	\item \cite{garrison2008} cohabitation was never regulated because it lasts for a too short time. In a nutshell, Cohabitation created no rights or obligations+ Cohabitants could not agree to create rights or obligations based on their intimate relationship. Instead, cohabitors can call for other equitable	doctrine for relief purchase money resulting trust, quantum meruit. These various forms of equitable relief ensured that, in many cases, one cohabitant who had been	cheated by the other could obtain recovery for some or all of his losses. Anyway, cohabitant claims for financial relief have not flooded the courts, partially because they are poorer and more disadvantaged
%	\item \cite{katz2008} , in a state that makes a	meaningful distinction between the legal status of legitimate and illegitimate children, children born to a non-marital cohabiting couple are illegitimate, requiring some affirmative action by the birth father if he wishes to have any legal control over his child. 
%	\item \cite{starnes2016} Cohabitation might cause the alimony rights to terminate in some States.
%	\item \cite{waggoner2016} Textual words:\textit{As far as the law is concerned, the partners are complete strangers to each other}. He means short term relationship because longer-term cohabitations have already found their way into the legal system, even though Contracts are unenforceable in some jurisdictions and, even if they are, the plaintiff cannot always prove the	existence of a contract. Other English-speaking jurisdictions have already enacted or introduced	legislation granting marital rights to cohabiting couples if their relationship meets specific criteria (Australia, Canada, Ireland,1 New Zealand,101 and Scotland1 and introduced in the United Kingdom for England and Wales. Other that the anglsphere,many european contries also does have that). 
%	
%	\item \cite{strasser2001} Definition of Domestic Partners two persons of the same or  opposite sex, not married to one another, who for a significant period of time share a  primary residence and a life together as a couple.” Whether a couple has shared a life together will be determined in light of a number of factors including the parties’“oral or written statements,” the extent to which their finances were intermingled,“ the extent to  which their relationship fostered...[either] inter-dependence” or one party’s dependence on the other, the extent to which the members of the couple acted or assumed roles in furtherance of their life together, the extent to which the relationship wrought change in the life” of either party, the emotional or physical intimacy of the relationship, and the reputation of the couple in the community
%		
%	\item \cite{mayeri2015} Enforcement of financial responsibilities for non-marital fathers was increased rapidly during the welfare reform in the 1990s, and made the income sharing guaranteed through marriage and non-marital fertility much more similar. 
%	
%	\item \cite{rossin2017} in-hospital voluntary paternity establishment enable all unmarried parents to voluntarily establish paternity by filing out a simple one-page form at the hospital at the time of childbirth, and provide information to all new fathers about their rights and obligations.  Before IHVPE, paternity establishment was a costly and complicated process involving the court system and	DNA testing; less than a third of children born out of wedlock had fathers who	established paternity in the late 1980s. This was a way to enforce child support
%\item \cite{goran2008} Common-Law Marriage requires cohabitation (relevant factors=sexual relationship, exclusive relationship, mutual children, mutual house, economic support, emotional support, shared recreation,length of cohabitation might play a role, only because cohabitation has to be permanent.) and reputation as spouses (outwardly represent themselves as spouses with respect to third parties) as continuing factors to be considered as-if-married. Inferred agreements might also play a role: couple could cohabit for years without being considered as being in a marriage agreement, with marital right and obligations. The burden of the proof is on the party alleging the existence common-law marriage, in which case it is presumed and the burden shifts: the other party has to prove that no informal marital contract has been formed.
%\item \cite{bowman2004} similar to garrison up to a certain point. then she says that in some states there exist some status-based rights. In washington the meretricious relationship implies community property but it applies to couples with common law marraige characteristics, which are rare for cohabitors and also hard to prove. it has been introduced in 1984 (does not matter much for us.) In Vermont and Massachussets only same-sex couples recive status-based cohabitation rights, apporved by most of other us states. Domestic partnerships are available in Hawaii (1997-some benefits as cara at hoispital of insurance for partners of employees), new Jeersey (2004-stringent rights for entering and exiting with few benefits) adn California (2001-2004 as marriage for same sex and hetero above age of 62.) Many of the most significant government benefits are universallyunavailable to cohabitants, however, because they derive from federal law.Social Security survivors’ benefits are available only to those who qualify as spouses as defined by state law. Some exception (california) applies for state related benefits. Concerning children, what matters is mostly biology and not the relationship status.
%
%\end{enumerate}
\textbf{Breakup/Divorce laws compared.} The regulation of cohabitation in the US is limited, and small changes have been done since the 1960s, when “Cohabitation created no rights or obligations”, see \cite{garrison2008}. She argues that \textit{“Cohabitants could not agree to create rights or obligations based on their intimate relationship”}. She continues analyzing the effect of the \textit{Marvin vs. Marvin} case (1976), where palimony --- a compensation from one member of an unmarried couple to another after breakup --- was awarded to the female partner: \textit{“[the case] have not produced results markedly different from those permissible under pre-Marvin case law.”} Finally, she argues that claims for financial relief has rarely reached the courts because 1) cohabitation is usually very short and no committed 2) cohabitants are younger and poorer than marrieds and 3) cohabitants do not usually adopt sharing behavior, unlike in the Marvin cases. Similarly, \cite{bowman2004} claims that remedies based on the contract had a limited application.

Hence, breakup resembles unilateral divorce because one partner can end cohabitation without the other partner's consent. Concerning property division rights, cohabitation de facto falls under the title-based property regime. One crucial difference between divorce and breakup is that the former requires the couple to undergo a legal process, which implies monetary and time costs, while the latter does not require this procedure. While \cite{garrison2008} argues that claims for financial reliefs after a breakup are rare, the breakup is treated like a divorce under the doctrine of common-law marriage, a legal framework under which a couple is considered as married without having formally registered their relationship. \cite{goran2008} explains that the existence of the implied contract is presumed once continuous cohabitation and reputation (holding out as husband and wife) are proven. However, it is still possible that the couple --- even if cohabiting for many years---is not considered to be in a marriage agreement, with marital rights and obligations. These rules create uncertainty regarding recognizing common-law marriage for some couples, especially those close to a breakup, where the two partners might disagree about the existence of an implied marital agreement.

The lower costs of a breakup are consistent with the findings of \cite{avellar2005}, who show that for women the drop in income following the couple's breakdown is larger for divorce than for a breakup. To further support the claim that divorce is more costly than a breakup, in appendix \ref{section:eventsa} we select from the PSID a sample of couples that divorced/broke up to study how their net-worth evolves after splitting. The point estimates of two event-studies indicate that richer couples' divorce results in a loss of assets, while we could not observe the same pattern for the divorce of poorer couples and breakups.

%When a child is born in the unmarried couple, where the situation started changing in the 1990s, when establishing paternity and the enforcement of child support became easier.\footnote{\cite{rossin2017} explains that in-hospital voluntary paternity establishments enabled all unmarried parents to voluntarily establish paternity.}

\section{Data and Empirical Evidence}
\subsection{Dataset}\label{dataset}
We begin by describing our data. We use the wave I (1987-1988) of the National Survey of Family and the Household (NSFH), and the National survey of Family Growth (NSFG), 1988 wave. Both surveys were designed to study the causes and consequences of changes happening in families and households within the United States. This is reflected in detailed questions regarding the retrospective family history of respondents, including information both about marriage and cohabitation. Moreover, primary respondents are asked a large set of questions regarding their socio economic background and the demographics of the household.\footnote{One adult per household was randomly selected as the primary respondent, while in the NSFG respondents are all women of 15-44 years of age.} While the NSFH I is the first of three longitudinal waves, NSFG is made of several repeated cross sectional samples.\footnote{We decided not to use the other two other waves of the NSFH because in the second wave all currently cohabiting households were dropped from the survey. Moreover, the 1988 wave of NSFG is the only one with publicly available information about the residence of the respondents, which is crucial to identify the divorce regime that applies to the respondent.} A drawback of using this data is that we know the state of residence of the respondents only at age 16 for the NSFH and at birth for the NSFG.\footnote{We do not have the choice of using other surveys for our analysis, since they either lack the State of residence variable, or they miss information about cohabitation history, or they do not cover people that were in a relationship age at the time the law changed.} Since we also know whether people lived all their life in the same state, we can overcome it and perform our empirical analysis both on the universe and of the subsample of never movers. We will show that point estimates turn out to be statistically not distinguishable between the two samples. Further details regarding those two surveys can be found in \cite{bumpass2017} and \cite{mosher1996}. We use this dataset to build two samples, the one of \textit{first and second relationships} and the one of \textit{first cohabitations}, that are described below.
  
\textbf{First and Second Relationships Sample.} We build a sample to analyze the type of relationship that respondents decided to have, which can be either marriage or cohabitation. The sample is made of first and second relationships.\footnote{Dating is not considered, since we cannot observe this state. Hence, people dating will fall under the category of singles.} One first relationship is defined observing the first time (if ever) a certain person started cohabiting or married. This observation is associated to the date at which the relationship starts, to the characteristics of the respondent member of the formed couple, and with a \textit{type}, which can either marriage or cohabitation. Note that the type of relationship of couples that cohabited before marriage is ``cohabitation": the transition from cohabitation to marriage is analyzed using the sample of \textit{first cohabitations}. Second relationships are defined in a similar fashion, but they include only respondents that ended the relationship with their first partner and started a new one with a different person. The way this sample is built implies that for some respondents we will have zero corresponding observations in this sample, while for others we will have one, and for others we will have two. We did not consider third or higher order relationships for our analysis since these individuals would be further away from the age at which we knew their state of residence. Finally, we consider only relationships that started when the respondent was 20 years old or older. In table \ref{table:sum_rel} we report the descriptive statistics of this sample.

\begin{table}[!htbp]\centering
	\caption{\\Descriptive statistics, relationship sample}
	\label{table:sum_rel}

% Table created by stargazer v.5.2.2 by Marek Hlavac, Harvard University. E-mail: hlavac at fas.harvard.edu
% Date and time: mar, ott 27, 2020 - 10:26:28
\begin{tabular}{@{\extracolsep{5pt}}lcccc} 
\\[-1.8ex]\hline 
\hline \\[-1.8ex] 
Statistic & \multicolumn{1}{c}{N} & \multicolumn{1}{c}{Mean} & \multicolumn{1}{c}{Median} & \multicolumn{1}{c}{St. Dev.} \\ 
\hline \\[-1.8ex] 
Unilateral Divorce Dummy & 10,533 & 0.349 & 0 & 0.477 \\ 
Age Relationship Starts & 10,533 & 25.471 & 23 & 7.214 \\ 
Married & 10,533 & 0.650 & 1 & 0.477 \\ 
College & 10,533 & 0.252 & 0 & 0.434 \\ 
Female & 10,533 & 0.655 & 1 & 0.475 \\ 
Birth year & 10,533 & 1,950.016 & 1,952 & 10.630 \\ 
NSFH Dummy & 10,533 & 0.733 & 1 & 0.442 \\ 
\hline \\[-1.8ex] 
\end{tabular} 

\end{table}
\FloatBarrier
\textbf{First Cohabitation Sample.} This sample is built to analyze the decisions of cohabiting couples to breakup or to marry. It is composed of the first non-marital cohabitation experienced by respondents. This sample includes couples that cohabited before marriage, but it also includes cohabitations experienced by people with the following marital history: marriage without premarital cohabitation, divorce, cohabitation with a different person. Each observation of this sample is associated with a starting date, a possible ending date, and an outcome, which can be still cohabiting, married or breakup. In table \ref{table:sum_coh} we report the descriptive statistics of this sample.

\begin{table}[!htbp]\centering
	\caption{\\Descriptive statistics, cohabitation sample}
	\label{table:sum_coh}
	
% Table created by stargazer v.5.2.2 by Marek Hlavac, Harvard University. E-mail: hlavac at fas.harvard.edu
% Date and time: mer, feb 12, 2020 - 15:58:49
\begin{tabular}{@{\extracolsep{5pt}}lcccc} 
\\[-1.8ex]\hline 
\hline \\[-1.8ex] 
Statistic & \multicolumn{1}{c}{N} & \multicolumn{1}{c}{Mean} & \multicolumn{1}{c}{Median} & \multicolumn{1}{c}{St. Dev.} \\ 
\hline \\[-1.8ex] 
Unilateral Divorce & 5,675 & 0.454 & 0 & 0.498 \\ 
Age Cohabitation Starts & 5,675 & 23.701 & 22 & 6.976 \\ 
Year Cohabitation Starts & 5,675 & 1,978.724 & 1,980 & 7.160 \\ 
College & 5,675 & 0.162 & 0 & 0.368 \\ 
Female & 5,675 & 0.758 & 1 & 0.428 \\ 
Cohabitation Duration (months) & 5,675 & 24.170 & 13 & 29.513 \\ 
Year of birth & 5,675 & 1,954.630 & 1,956 & 13.790 \\ 
NSFH & 5,675 & 0.562 & 1 & 0.496 \\ 
Censored & 5,675 & 0.102 & 0 & 0.303 \\ 
Married & 5,675 & 0.490 & 0 & 0.500 \\ 
Separated & 5,675 & 0.408 & 0 & 0.491 \\ 
\hline \\[-1.8ex] 
\end{tabular} 

\end{table}
\FloatBarrier

\subsection{Empirical Evidence}\label{empirics}
Does unilateral divorce affect the partnership choice of couples? We exploit the timing in the adoption of unilateral divorce as a source of exogenous variation in the right to divorce.\footnote{See table 1 in \cite{ciacci2017} for the timing of adoption of unilateral divorce.} This strategy has already been used several times by the literature\footnote{Among the others, see \cite{wolfers2006}, \cite{stevenson2008}, \cite{voena2015}, \cite{reynoso2019} and \cite{ciacci2017}.} to study the non-neutrality of the rights to divorce on various economic and demographic outcomes. According to \cite{gruber2004}, who reviews the legal literature about the topic, the introduction of unilateral divorce was not view as a tool of social policy, but rather a way to reduce the legal burden of divorce trials. This reasoning is consistent with the fact that this change was not initiated by the most liberal states: New York was the last state to introduce unilateral divorce in October 2010, almost 40 years later than Kentucky. Moreover, \cite{reynoso2019} shows that there is no geographic correlation in adoption. 
\subsubsection*{Relationship Choice}
What is the effect of unilateral divorce on the partnerships that couples choose? To answer this question, we estimate equation (\ref{eq:ols_baseline}), where $i$ are the newly formed couples, $t$ is the calendar time, and $s$ is the state: 
\begin{equation}\label{eq:ols_baseline}
\text{married}_{i,t,s}=\beta_0+\beta_1*\text{Unilateral}_{t,s}+\mathbf{\gamma'}\mathbf{X}_i+\delta_s+\nu_t+\epsilon_{i,t,s}.
\end{equation}
The dependent variable is a dummy take takes value $1$ if the couple $i$, started at time $t$ if state $s$ is a marriage, and 0 if it is cohabitation. The vector $\mathbf{X}_i$ instead includes a set of socio demographic controls, while $\delta_s$ are the state fixed effects and  $\nu_t$ are the time fixed effects. The variable $\text{Unilateral}_{t,s}$ instead is a dummy that takes value $1$ if unilateral divorce wave in place in state $s$ at time $t$: $\beta_1$ instead is the coefficient that is informative about the effect of unilateral divorce on partnership choice. The results of the estimation are reported in table \ref{table:ols_rel_standard} for different samples. Column (1) reports the results for the full sample described in section \ref{dataset}, while column (2) is restricted to observations for which we know that the person lived all its life in the reported states, ensuring that they did not migrate. Finally, columns (3) and (4) restrict the sample to respectively the NSFH and NSFG surveys only.
%MARRUAGE TO DIVORCE

	\begin{table}[H]\centering
		\caption{\\OLS Regression. Observation: first and second relationships}
		\label{table:ols_rel_standard}
		\begin{threeparttable}[t]\centering
			
% Table created by stargazer v.5.2.2 by Marek Hlavac, Harvard University. E-mail: hlavac at fas.harvard.edu
% Date and time: mar, ott 27, 2020 - 10:25:57
\begingroup 
\footnotesize 
\begin{tabular}{@{\extracolsep{5pt}}lcccc} 
\\[-1.8ex]\hline 
\hline \\[-1.8ex] 
 & \multicolumn{4}{c}{\textit{Dependent variable: Married (0/1)}} \\ 
\cline{2-5} 
\\[-1.8ex] &  &  &  & \\[-4.8ex] \\ 
 & Full Sample & Resident & NSFH & NSFG \\ 
\\[-1.8ex] & (1) & (2) & (3) & (4)\\ 
\hline \\[-1.8ex] 
 Unilateral Divorce & $-$0.069$^{***}$ & $-$0.088$^{***}$ & $-$0.077$^{***}$ & $-$0.067$^{*}$ \\ 
  & (0.020) & (0.021) & (0.025) & (0.037) \\ 
 \hline \\[-1.8ex] 
State Fixed effects & Yes & Yes & Yes & Yes \\ 
Birth Year dummies & Yes & Yes & Yes & Yes \\ 
Year started Fixed Effect & Yes & Yes & Yes & Yes \\ 
Demographic Controls & Yes & Yes & Yes & Yes \\ 
Observations & 10,533 & 6,846 & 7,722 & 2,811 \\ 
R$^{2}$ & 0.146 & 0.166 & 0.163 & 0.139 \\ 
\hline 
\hline \\[-1.8ex] 
\end{tabular} 
\endgroup 

	\begin{tablenotes}[flushleft]
		\footnotesize{\item \textsc{Notes}: standard errors are clustered at the state level.
			Coefficients that are significantly different from zero are denoted by the following system: *10\%, **5\%  and ***1\%.}
	\end{tablenotes}
\end{threeparttable}
\end{table}
\FloatBarrier
The results reported in table \ref{table:ols_rel_standard} suggest that unilateral divorce decreased the share of couples that are married by $-5\%(6\%)$ depending on the specification. These results are robust to an alternative specification that includes state specific linear trends, whose results are reported in table \ref{table:wmarlin}, and to the use of a logistic regression, reported in table \ref{table:wmar_logit}.

We then move on to better understand the heterogeneity hidden behind the effect of unilateral divorce. While in some states assets are split in the same way in both breakup and divorce, which is the case of \textit{title-based regime} states, in others this rule is different, which is the case of \textit{community property} and \textit{equitable distribution} states. Analyzing this heterogeneity is then interesting to understand how much the asset sharing rule is important for understanding relationship choices.  We hence estimate equation (\ref{eq:ols_by_prop})


\begin{equation}\label{eq:ols_by_prop}
\begin{split}
\text{married}_{i,t,s}=\beta_0+&\beta_1*\text{Unilateral*No Title Based}_{t,s}\\&+\beta_2*\text{Unilateral*Title Based}_{t,s}+\\&\quad\quad\beta_3*\text{Title Based}_{t,s}+\mathbf{\gamma'}\mathbf{X_i}+\delta_s+\nu_t+\epsilon_{i,t,s},
\end{split}
\end{equation}
whose  indexes and controls are the same of equation (\ref{eq:ols_baseline}), with the difference that now we capture the interaction of unilateral divorce with asset division regimes interacting $\text{Unilateral}_{t,s}$ with  $\text{Title Based}_{t,s}$ and $\text{No Title Based}_{t,s}$, which indicates whether state $s$ at time $t$ had or not a title-based regime. In table \ref{table:wmarc} we report the results of the estimation of equation \ref{eq:ols_by_prop}. Similarly to table \ref{table:ols_rel_standard}, column (1) reports the results for the full sample described in section \ref{dataset}, while column (2) is restricted to the observations for which we know that the person lived all her life in the reported states, which ensures that they did not migrate. Finally, columns (3) and (4) restrict the sample to respectively the NSFH and NSFG surveys only.

	\begin{table}[H]\centering
		\caption{\\OLS Regression. Observation: first and second relationships}
		\label{table:wmarc}
		\begin{threeparttable}[t]\centering
			
% Table created by stargazer v.5.2.2 by Marek Hlavac, Harvard University. E-mail: hlavac at fas.harvard.edu
% Date and time: mar, ott 27, 2020 - 10:26:54
\begingroup 
\footnotesize 
\begin{tabular}{@{\extracolsep{5pt}}lcccc} 
\\[-1.8ex]\hline 
\hline \\[-1.8ex] 
 & \multicolumn{4}{c}{\textit{Dependent variable: Married (0/1)}} \\ 
\cline{2-5} 
\\[-1.8ex] &  &  &  & \\[-4.8ex] \\ 
 & Full Sample & Resident & NSFH & NSFG \\ 
\\[-1.8ex] & (1) & (2) & (3) & (4)\\ 
\hline \\[-1.8ex] 
 UnDiv*NoTit & $-$0.074$^{***}$ & $-$0.090$^{***}$ & $-$0.084$^{***}$ & $-$0.068$^{*}$ \\ 
  & (0.020) & (0.022) & (0.025) & (0.039) \\ 
  UnDiv*Tit & $-$0.015 & $-$0.053 & $-$0.014 & $-$0.046 \\ 
  & (0.031) & (0.037) & (0.040) & (0.048) \\ 
  Tit & $-$0.014 & $-$0.011 & $-$0.011 & $-$0.017 \\ 
  & (0.021) & (0.026) & (0.027) & (0.037) \\ 
 \hline \\[-1.8ex] 
State Fixed effects & Yes & Yes & Yes & Yes \\ 
Year started Fixed Effect & Yes & Yes & Yes & Yes \\ 
Birth Year dummies & Yes & Yes & Yes & Yes \\ 
Demographic Controls & Yes & Yes & Yes & Yes \\ 
Observations & 10,533 & 6,846 & 7,722 & 2,811 \\ 
R$^{2}$ & 0.147 & 0.166 & 0.164 & 0.139 \\ 
\hline 
\hline \\[-1.8ex] 
\end{tabular} 
\endgroup 

			\begin{tablenotes}[flushleft]
				\footnotesize{\item \textsc{Notes}: standard errors are clustered at the state level.
					Coefficients that are significantly different from zero are denoted by the following system: *10\%, **5\%  and ***1\%.}
			\end{tablenotes}
		\end{threeparttable}
	\end{table}
\FloatBarrier
The results show that the effect of unilateral divorce on the likelihood that a couples chooses marriage over cohabitation in non-title-based states is significant with a magnitude of $-6\%(-7\%)$ depending on specification, while it is not significant and much smaller in title-based states. These results suggest that having a sharing rule decided by the law is not enough to replace the mutual consent regime as an alternative commitment technology. Instead, these results are consistent with the view that the richest partner starts disliking marriage when divorce becomes unilateral, since she would risk losing most of her wealth upon divorce. This was not happening in mutual consent regime, since she could have exercised her right to veto divorce. In title-based state this threat for the richest member of the couple does not exist, hence marriage surplus with respect to cohabitation does not vary significantly. Table \ref{table:wmarlinc} shows that results are robust to the inclusion of state specific linear trends.
\subsubsection*{Cohabitation Duration}
What is the effect of unilateral divorce on cohabitation duration? How much of the change is due to a variation in the risk of breakup versus the risk of marriage? In order to answer this question, we construct a model of cohabitation duration with multiple risks, namely breakup and marriage. Our model builds on \cite{jenkins1995}, who shows how that a logistic regression can be used for studying duration of events by reshaping the dataset to obtain unit of time per spells observations, where the dependent variable takes value $1$ whenever the event of interest occurs. The natural extension of this model to a multiple risk environment would be to use a multinomial logit. However, the problem with this model is that it assumes independence of irrelevant alternatives, which is particularly unappealing for our problem, since it would imply that the relative probability of choosing marriage over breakup stays the same after cohabitation is no longer an option. Hence, we chose to model cohabitation duration with a multinomial probit, where the independence of irrelevant alternatives does not need to be satisfied. We then study the choice of cohabiting couple $i$, at calendar time $t$ in state $s$ and at duration $d$ estimating the following model: 
\begin{equation}\label{eq:probit_1}
\begin{split}
&Y_{i,s,t,d}^\text{Marry}=\beta^\text{Marry}*\text{Unilateral}_{s,t}+\mathbf{\gamma^{\text{Marry'}}}\mathbf{X_i}+\alpha_d+\delta_s+\nu_t+\epsilon^{\text{Marry}}_{i,s,t,d},\\&
Y_{i,s,t,d}^\text{Cohabit}=\beta^\text{Cohabit}*\text{Unilateral}_{s,t}+\mathbf{\gamma^{\text{Cohabit'}}}\mathbf{X_i}+\alpha_d+\delta_s+\nu_t+\epsilon^{\text{Cohabit}}_{i,s,t,d},\\&
Y_{i,s,t,d}^\text{Breakup}=\beta^\text{Breakup}*\text{Unilateral}_{s,t}+\mathbf{\gamma^{\text{Breakup'}}}\mathbf{X_i}+\alpha_d+\delta_s+\nu_t+\epsilon^{\text{Breakup}}_{i,s,t,d},
\end{split}
\end{equation}
where
\begin{equation}\label{eq:probit_error}
\myvec{\epsilon_{i,s,t,d}^{\text{Marry}}\\\epsilon_{i,s,t,d}^{\text{Cohabit}}\\\epsilon_{i,s,t,d}^{\text{Breakup}}}\sim\mathcal{N}(\mathbf{0},\mathbf{\Sigma}),
\end{equation}
and
\begin{equation}\label{eq:probit_outcome}
Y_{i,s,t,d}=
\begin{cases}  	
\text{Marry}\hspace{30pt}       \text{if}\hspace{10pt}Y_{i,s,t,d}^\text{Marry}>Y_{i,s,t,d}^\text{Cohabit}\hspace{5pt}\text{and}\hspace{5pt}\hspace{5pt}Y_{i,s,t,d}^\text{Marry}>Y_{i,s,t,d}^\text{Breakup}\\
\text{Cohabit}\hspace{23pt}   \text{if}\hspace{10pt}Y_{i,s,t,d}^\text{Cohabit}>Y_{i,s,t,d}^\text{Marry}\hspace{5pt}\text{and}\hspace{5pt}\hspace{5pt}Y_{i,s,t,d}^\text{Cohabit}>Y_{i,s,t,d}^\text{Breakup}\\
\text{Breakup}\hspace{22pt}     \text{otherwise.}
\end{cases}
\end{equation}
The model described above is estimated with bayesian techniques via Markov chain Monte Carlo following the procedure of \cite{imai2005a}, which is implemented using the standard options provided by the $R$ package \textit{MNP} developed by \cite{imai2005b}. In table \ref{table:mpc} we report results from the full sample in column (1), from the resident only sample in column (2) and from the observations coming from the NSFH and NSFG surveys alone respectively in column (3) and (4). Note that to gain intuition about the size of the results, in  table \ref{table:mpc} we computed the average risk of the event of interest relatively of continue cohabiting. The results show that unilateral divorce caused an increase in the duration of cohabitation, which comes from both a reduced hazard of marriage and of breakup. While the result about the risk of marriage is not unexpected in light of the estimation results described above, the reduced risk of breakup brings new insights about the possible mechanisms underlying partnership choices. In fact, the decrease in the risk of breakup is consistent with a selection effect: some cohabiting couples would have married if mutual consent divorce was still in place. If the match quality of cohabitations is lower than the one of marriages,\footnote{This seems plausible because the risk of divorce is much lower than the risk of breakup.} unilateral divorce drives down the risk of breakup because of a selection effect. 
%Cohabitation Length4

	\begin{table}[htbp]\centering
		\caption{\\Multinomial Probit. Observation: person-month of cohabitation}
		\label{table:mpc}
		\begin{threeparttable}[t]\centering
			\footnotesize 
\begin{tabular}{@{\extracolsep{5pt}}lcccc} 
\\[-1.8ex]\hline 
\hline 
\\[-1.8ex] & \multicolumn{1}{c}{Full Sample} & \multicolumn{1}{c}{Resident}& \multicolumn{1}{c}{NSFH}& \multicolumn{1}{c}{NSFG} \\ 
\\[-1.8ex] & \multicolumn{1}{c}{(1)} & \multicolumn{1}{c}{(2)} & \multicolumn{1}{c}{(3)} & \multicolumn{1}{c}{(4)}\\ 
\hline \\[-1.8ex] 
\\[-2.2ex] & \multicolumn{4}{c}{Risk of Marriage relative to Cohabitation} \\  
 \hline \\[-1.8ex]
 Unilateral Divorce &  -0.09  &  -0.07  &  -0.12  &  -0.13  \\ 
  & ( 0.04 ) & ( 0.04 ) & ( 0.04 ) & ( 0.11 ) \\  
 \hline \\[-1.8ex]
 Average Relative Risk &  -0.12  &  -0.08  &  -0.2  &  -0.14  \\ 
 \hline \\[-1.8ex]
 \\[-2.2ex] & \multicolumn{4}{c}{Risk of Separation relative to Cohabitation} \\  
 \hline \\[-1.8ex]
 Unilateral Divorce &  -0.04  &  -0.09  &  -0.05  &  -0.06  \\ 
  & ( 0.02 ) & ( 0.02 ) & ( 0.02 ) & ( 0.1 ) \\  
 \hline \\[-1.8ex]
 Average Relative Risk &  -0.06  &  -0.18  &  -0.1  &  -0.06  \\ 
 \hline \\[-1.8ex]
State Fixed effects & Yes & Yes & Yes & Yes \\ 
Year Fixed effects & Yes & Yes & Yes & Yes \\ 
Age Polynomial & Yes & Yes & Yes & Yes \\
Picewise Duration & Yes & Yes & Yes & Yes \\ 
\hline
Observations & \multicolumn{1}{c}{ 138012 } & \multicolumn{1}{c}{ 81920 } & \multicolumn{1}{c}{ 77826 } & \multicolumn{1}{c}{ 60186 } \\ 
\hline
Censored spells(\%) & \multicolumn{1}{c}{ 10.18 } & \multicolumn{1}{c}{ 10.98 } & \multicolumn{1}{c}{ 11.6 } & \multicolumn{1}{c}{ 8.38 } \\ 
\hline 
\hline \\[-1.8ex] 
\end{tabular}

			\begin{tablenotes}[flushleft]
				\footnotesize{\item \textsc{Notes}: the values reported in the table are the mean and the standard deviation (in parenthesis) of the posterior distribution of parameters obtained using the Markov chain Monte Carlo estimation described by \cite{imai2005a}.
				Coefficients' distributions whose interpercentile range do not contain 0 are denoted by the following system: *90\%, **95\%  and ***99\%.}
			\end{tablenotes}
		\end{threeparttable}
	\end{table}

\FloatBarrier
\FloatBarrier
\section{Theory}
To identify the channels through which unilateral divorce impact partnership choice, we develop a dynamic life-cycle model of partnership formation and dissolution, savings, female labor force participation and home production. Couples act cooperatively, and according to the divorce regime they can be subject to limited commitment, which means that there might be renegotiations in response to changes in the outside options, which are assumed to be divorce or breakup. Time is discrete and in each period men and women draw their productivities. If single, with some probability they meet a potential partner: after drawing a match quality shock they decide whether to marry, cohabit or to stay single. Couples observe the match quality shock, their productivity and assets, and according to those they decide whether to stay together or to split. Cohabiting couples can also decide whether to marry. Both singles and couples make consumption and saving decisions, using their money for private or public good expenditure. Couples also make female labor participation decisions and women's time can be used to produce public goods, but this comes at the cost of a loss in productivity.
%The gains of being in a couple come from love, risk sharing and labor market specialization. Instead, the gains of marriage with respect to cohabitation comes from a more functioning risk sharing and a better specialization in time use, which derives from the high cost of divorce serving as a commitment device. These gains of marriage deteriorate when the love shock is low enough to cause a high risk of divorce and frequent renegotiations: in this case cohabitation might be the better option, since we assume that the cost of breakup will be lower than the cost of marriage.
\begin{figure}[h!]
	\label{fig:scheme}
	\caption{}
	\hspace{0.3cm}
	\resizebox{0.99\textwidth}{!}{

\begin{tikzpicture}[domain=0:1,scale=1]
\node [ellipse,fill=blue!20,minimum height=2.5cm,minimum width=7.8cm,draw, name=input,align=center] (sum) {\huge\bf Single};
\node [ellipse,fill=none,minimum height=0.5cm,minimum width=7.8cm, above right=1.5cm and 7.449cm of sum ] (controller) {\\ \huge \bf Breakup};
\node [ellipse,fill=blue!20,minimum height=2.5cm,minimum width=7.8cm,draw, below right=1.5cm and 7.449cm of controller,align=center] (system) {\huge \bf Cohabitation};
\draw [decoration={markings,mark=at position -0.95 with
    {\arrow[scale=5,>=stealth]{<}}},postaction={decorate}] (controller) --  node[above left,yshift=0.3cm]{\huge \bf}(system);
\node [output, right of=system] (output) {fe};
\node [ellipse,fill=blue!20,minimum height=2.5cm,minimum width=7.8cm,draw, below left=1.5cm and 7.449cm of system,align=center] (measurements) {\huge \bf {\color{blue!20}a}Marriage };
\node [ellipse,fill=none,minimum height=1.5cm,minimum width=4.2cm, left of=measurements, node distance=12.98cm,align=left] (divorce) {\huge \bf Divorce};
\draw [decoration={markings,mark=at position -0.92 with
    {\arrow[scale=5,>=stealth]{<}}},postaction={decorate}] (sum) -- node[above right,yshift=0.3cm]{\huge \bf } (controller);
\draw [decoration={markings,mark=at position 1 with
    {\arrow[scale=5,>=stealth]{>}}},postaction={decorate}] (sum) -- node[above]{\huge\bf} (system);
\draw [decoration={markings,mark=at position 0.99 with
    {\arrow[scale=5,>=stealth]{>}}},postaction={decorate}] (system) --node[above left]{\huge\bf } (measurements);
\draw [decoration={markings,mark=at position 0.99 with
    {\arrow[scale=5,>=stealth]{>}}},postaction={decorate}] (measurements) --node[above]{\huge\bf } (divorce);
\draw [decoration={markings,mark=at position 1 with
    {\arrow[scale=5,>=stealth]{>}}},postaction={decorate}] (sum) --node[above right]{\huge } (measurements);
\draw [decoration={markings,mark=at position 1 with
    {\arrow[scale=5,>=stealth]{>}}},postaction={decorate}] (divorce) --node[right]{\huge } (sum);
node [near end] {$$} (sum);
\end{tikzpicture}

}
	\begin{minipage}{0.99\textwidth} % choose width suitably
		\hspace{3em}
		{\footnotesize .\par}
	\end{minipage}
\end{figure}
\subsection{Preferences}
Women $f$ and men $m$ derive utility from consuming a private good  $c$ and a household public good $Q$. The public good can be interpreted in terms of both the quantity and quality of children, as well as the goods and services produced within the household, such as washing clothes or preparing meals. Preferences are separable in the two goods and across time.
Agents derives utility from a couple specific love shock $\psi$, which evolves over time and it can be interpreted as the value of love and companionship in a couple. The intra-period utility of a single agent $s\in(f,m)$ is:
\[u(c^s_t,Q^s_t)=\frac{{c^s_t}^{1-\sigma}}{1-\sigma}+\alpha\frac{{Q^s_t}^{1-\xi}}{1-\xi},\]
where the superscript $s$ on $Q$ accounts for the fact that there is no partner to share the public good. The utility for an agent $s\in(f,m)$ in a couple is:
\[u^{C}(c^s_t,Q_t)=\frac{{c_t^s}^{1-\sigma}}{1-\sigma}+\alpha\frac{Q_t^{1-\xi}}{1-\xi}+\psi_t,\]
where the match quality $\psi$ evolves according to the following law of motion:
\[\psi_t=\psi_{t-1}+\epsilon_t,\text{ where }\epsilon_t \overset{\text{i.i.d.}}{\sim}\mathcal{N}(0,\sigma^2_{\psi}). \]
The love shock at first meeting can have a different variance, denoted by $\sigma^2_{\psi,I}$. Note that if the couple is cohabiting, the utility of the two partners is decreased by $\gamma$, which captures the stigma associated with premarital sex, premarital cohabitation and out-of-wedlock births. This assumption fits the fact that for people born in 1940-1955 (whose behavior will be used to build the target moments for the structural estimation) conservative attitudes towards premarital sex were common.\footnote{The shame associated with an out-of-wedlock birth, whose interaction with technology is studied by \cite{fernandez2014}, can be a factor leading young women to prefer marriage over cohabitation even if the rules governing these two partnerships were identical. \cite{blasutto2020} can match closely marriage and cohabitation choices using a theoretical framework close to ours, without the need of introducing a stigma component towards cohabitation. This is possible because he analyzes the behavior of people born in 1980-1984, for whom the stigma towards premarital sex and premarital cohabitation was low.}

\subsection{Wages}
The labor income for agents $s\in\{f,m\}$ depends on their age $t$ and on a permanent income component $z^s_t$:
\[\ln(w^s_t)=f^s_t+z^s_t,\]
where $f^s_t$ is a gender specific function that captures the evolution of productivity over age. The permanent income component  $z^s_t$ evolves over time as:
\begin{equation}\label{eq:pcomp}
z^s_t=z^s_{t-1}-(1-P_t^s) \mu+\zeta^s_{t}\text{, where }\zeta^s_{t}\overset{\text{i.i.d.}}{\sim}\mathcal{N}(0,\sigma_\zeta^{2_s})\text{, and }\zeta^s_{1}=z^s_{1}.
\end{equation}
where $P^s_t$ is a dummy of labor force participation. Men and single women are always assumed to participate in the labor market, hence $P^m_t=1$.\footnote{The assumption that men, as opposed to women, always participate in the labor market is rather common in the literature (\citealp{ciscato2019,low2018,voena2015}) and it is in line with the gender roles typically observed in the period under analysis. In our PSID sample only 5\% of men between 20 and 60 do not supply working hours in the market.} Parameter $\mu$ is the loss in productivity that affects women that are not participating in the labor market. It can be interpreted as a reduced form way of capturing both the missed opportunity to accumulate human capital while working and the skill atrophy from interruptions \citep{adda2017}. Modeling the loss in productivity for not working is an important feature of our model as it creates an incentive to join the labor force for women that expect to divorce or breakup soon. 
\subsection{Home Production}
In our model each agent has one unit of time. Singles and men in a couple supply inelastically a fraction $1-\phi$ of their time to the labor market, while women in a couple can be out of the labor force to devote their time producing the home good $Q$. The public good can also be produced buying $d$ goods in the market. Following \cite{greenwood2016} we define the production function of home goods for couples as:
\begin{equation}\label{eq:pfunction}
Q_t=[d_t^\nu+\kappa ({2\phi+(1-P^f_t)(1-\phi))}^\nu]^{\frac{1}{\nu}}, \text{ where }0<\nu<1,
\end{equation}
while for singles of gender $s\in\{f,m\}$
\begin{equation}\label{eq:pfunctions}
Q^s_t=[(d^s_t)^\nu+\kappa \phi^\nu]^{\frac{1}{\nu}}.
\end{equation}
The parameter $\nu$ captures the degree of substitutability between women's time and the use of durables in the production of home goods. This structure implies that when the relative price of $d_t$ decreases and when wages goes up,\footnote{The relative price of $d_t$ is normalized to 1 in equations (\ref{eq:pfunction}) and (\ref{eq:pfunctions})} women spend less time producing household goods and their employment outside the home increases.
\subsection{Budget Constraints}
The budget constraint of a single agent of gender  $s\in\{f,m\}$ is:
\begin{equation}\label{eq:bcs}
a^s_{t+1}=R a^s_t+w^s_t(1-\phi)-c^s_t-d^s_t, \text{ with }a^s_{t+1}\geq0,
\end{equation}
where $a^s$ are agent's savings and $w^s$ is the wage. $c^s$ and $d^s$ are the private good consumption and the expenditure used to produce the public good.
The budget constraint for a couple is:
\begin{equation}\label{eq:bcm}
a^f_{t+1}+a^m_{t+1}=R a_t+w^m_t(1-\phi)+P^f_t w^f_t(1-\phi)-c^f_t-c^m_t-d_t, \text{ with }a_{t+1}\geq0,
\end{equation}
When a couple divorces in $t$, we assume 
\[a^m_t+a^f_t=\delta a_t,\]
where $\delta$ is the fraction of total assets $a_t$ left after divorce. We assume $\delta=1$ for breakup.\footnote{The assumption that divorce erodes a fraction of wealth is common to \cite{cubeddu2003}. In appendix \ref{section:eventsa} we provide evidence that divorce results in a loss of net worth for rich but not for poor households. Moreover, we do not find evidence for a loss of net worth following breakup for rich and poor households. In practice the cost of breakup is positive because of psychological distress associated with a separation and because looking for a new accommodation requires time. However, these costs are common with divorce and hence they do not help explaining why couples should choose one partnership over the others. This reasoning is confirmed by the fact that when we tried estimating the model allowing for a positive cost of breakup, these parameters was not identified.} 
An important feature of our model is the role of property rights, which defines how assets are divided upon divorce/breakup. Since we use data from community property states to estimate the model, this regime applies to divorce. Accordingly, upon divorce each spouse keeps half of the assets, while the division of asset upon breakup is a couple's decision. We describe the details of this choice in this section, where the problem of the cohabiting couple is presented.

\subsection{Problem of the Singles}
We start by describing the problem for a single agent $i\in\{f,m\}$ in $t$. The agent makes consumption, saving and expenditure decisions. In $t+1$ she meets a potential partner $j$ of the opposite sex with probability $\lambda_{t+1}$ and she can decide to enter a partnership, which also depend on whether the potential partner will agree. If the two decide to marry, the variable $M_{t+1}$ will take value 1, while $C_{t+1}=1$ if the couple decides to cohabit. Otherwise, $M_{t+1}$ and $C_{t+1}$ will be equal to 0. The state variable of a single is $\omega^i_t=\{a^i_t,z^i_t\}$, while her choices are represented by the vector $\mathbf{q}^i_t=\{a^i_{t+1},c^i_t,d^i_t\}$. We denote by $V_t^{i,S}(\omega^i_t)$ the value function of agent $i$, which we define as
\begin{equation}\label{eq:v_single}
\begin{split}
V_t^{iS}(\omega^i_t)=&\max_{\mathbf{q}^i_t} u(c^i_t,Q^i_t)+\beta E_t \bigg\{(1-\lambda_{t+1})V^{iS}_{t+1}(\omega^i_{t+1})+\\ & \quad\quad \lambda_{t+1}\big\{(1-M_{t+1})(1-C_{t+1})	V^{i,S}_{t+1}(\omega_{t+1})+\\ &\quad\quad\quad\quad M_{t+1} V^{i,M}_{t+1}(\Omega_{t+1})+ C_{t+1} V^{i,C}_{t+1}(\Omega_{t+1}) \big\}\bigg\},
\\ &\text{s.t. \eqref{eq:bcs} and \eqref{eq:pfunctions},} 
\end{split}
\end{equation}
where $V^{i,M}$ and $V^{i,C}$ are the individual values of being married and cohabiting.

\subsection{Household Planning Problem}
The problem of the couple depends both on the type of relationship---cohabitation or marriage----and on the divorce regime, which can be either \textit{mutual consent} or \textit{unilateral divorce}. Breakup is always unilateral. Under the unilateral regime, one partner can initiate the breakup/divorce process alone, while under mutual consent the agreement of both partners is needed.
\subsubsection*{Mutual Consent Regime}
Under mutual consent regime, marriage is denoted by $\hat{M}$. Couples solve a Pareto problem where the weight of the wife is $\theta^f$ and the one of the husband is $1-\theta^f$.\footnote{Later in this section we describe how initial Pareto weight are set.} The state vector is $\Omega^{\hat{M}}_t=\{a^m_t,a^f_t,z^f_t,z^m_t,\psi_t,\theta^f\}$, while the variables over which the couple maximizes are summarized by the vector $\mathbf{q}^M_t=\{a^f_{t+1},a^m_{t+1},d_{t},c^m_{t},c^f_{t},P^f_t,D_t\}$, where $D_t$ is a dummy variable that takes value $1$ is divorce happens and $0$ otherwise. The formal problem solved by a couple who enters period $t$ as married is:
\begin{equation}\label{eq:v_mutual}
\begin{split}
V_t^{\hat{M}}(\Omega^{\hat{M}}_t)=&\max_{\mathbf{q}^M_t} (1-D_t)\{\theta^f u(c^f_t,Q_t)+(1-\theta^f)u(c^m_t,Q_t)+\psi_t+\beta E_t V^{\hat{M}}_{t+1}(\Omega^{\hat{M}}_{t+1})\}\\ &\quad\quad+D_t \{\theta^fV^{fS}_{t}(\omega^{f}_{t})+(1-\theta^f)  V^{mS}_{t}(\omega^{m}_{t}))\}
\\ &\text{if $D_t=0$:}\hspace{35pt}\text{s.t. \eqref{eq:bcm} and \eqref{eq:pfunction}
}
\\ &\text{if $D_t=1$:}\hspace{35pt}\text{s.t. \eqref{eq:bcs}, \eqref{eq:pfunctions} for $i\in\{f,m\}$,}\\ &
\hspace{80pt}a_t^m+a_t^f=\delta a_t,	\\ &\hspace{80pt}
V_{t}^{fS}(\omega^f_{t})\ge W^{f\hat{M}}_{t}(\Omega^{\hat{M}}_{t}),\\ &\hspace{80pt}
V_{t}^{mS}(\omega^m_{t})\ge W^{m\hat{M}}_{t}(\Omega^{\hat{M}}_{t}).
\end{split}
\end{equation}
The individual value of marriage conditional on $D_t=0$ is $W_{t}^{i\hat{M}}$ for $i\in\{F,M\}$, and it is defined as 
\begin{equation}
W_{t}^{i\hat{M}}=u(\tilde{c}_t^{i},\tilde{Q}_t)+\psi_t+\beta E_t V_{t+1}^{i\hat{M}}(\Omega^{\hat{M}}_{t+1}),
\end{equation}
where $\mathbf{q}^{\hat{M}}_t=\{\tilde{a}^m_{t+1},\tilde{a}^f_{t+1},\tilde{d}_{t},\tilde{c}^{m}_{t},\tilde{c}^{f}_{t},\tilde{P}^{f}_t\}$ is the $\argmax$ of problem \eqref{eq:v_mutual} conditionally on having chosen $D_t=0$. $V_{t+1}^{i\hat{M}}(\Omega^{\hat{M}}_{t+1})$ instead can be obtained by the expectation of the sum of the time utilities that the agent gets from $t+1$ to $T$, where the variables entering the utility function derive from the Pareto problem if the agent is in a relationship, otherwise they are the solution of \eqref{eq:v_single}, which represent the singles' problem.

 Under the mutual consent regime, the allocation corresponds to the Pareto efficient solution if the couple is intact. Intuitively, the fact that Pareto weights stay constant allows for a functioning risk-sharing and the female labor force participation decisions are taken cooperatively, ruling out the possibility that women over-supply labor to increase their bargaining power. In this framework, the conditions for divorce are particularly stringent: the couple splits only if both partners are better-off divorcing than staying together for a feasible allocation. More in particular, if one spouse only wishes to divorce under the divorce allocation dictated by the law where assets are split equally, she will ``bribe'' the other by offering a larger share of assets to make her indifferent between staying married and divorcing.\footnote{Note that if both partners are better off divorcing under the sharing rule dictated by the law, which corresponds to an equal division in community property states, no bribing happens.}
\subsubsection*{Unilateral Divorce Regime}
Under the unilateral divorce regime marriage is denoted by $\overline{M}$. Couples solve a Pareto problem where the weight of the wife is $\theta^f_t$ and the one of the husband is $\theta^m_t$. Note that, in opposition to the mutual consent regime, Pareto weights can vary over time. The state vector of this problem is $\Omega^{\overline{M}}_t=\{a^m_t,a^f_t,z^f_t,z^m_t,\psi_t,\theta^f_t,\theta^m_t\}$, while the variables over which the couple maximize are summarized by the vector $\mathbf{q}^M_t$. The formal problem of a couple entering $t$ as married is:
\begin{equation}\label{eq:v_uni}
\begin{split}
V_t^{\overline{M}}(\Omega^{\overline{M}}_t)=&\max_{\mathbf{q}^M_t} (1-D_t)\{\theta^f_{t} u(c^f_t,Q_t)+\theta^m_{t} u(c^m_t,Q_t)+\psi_t+\beta E_t V^{\overline{M}}_{t+1}(\Omega^{\overline{M}}_{t+1})\}\\ &\quad\quad+D_t \{\theta^f_{t}V^{fS}_{t}(\omega^{f}_{t+1})+\theta^m_{t} V^{mS}_{t}(\omega^{m}_{t}))\}
\\ &\text{if $D_t=0$:}\hspace{35pt}\text{s.t. \eqref{eq:bcm} and \eqref{eq:pfunction},}\\ &\hspace{80pt}
\theta^f_{t+1}=\theta^f_{t}+\mu^f_t,\\ &\hspace{80pt}
\theta^m_{t+1}=\theta^m_{t}+\mu^m_t,
\\ &\text{if $D_t=1$:}\hspace{35pt}\text{s.t. \eqref{eq:bcs}, \eqref{eq:pfunction} for $i\in\{f,m\}$,}\\ &
\hspace{80pt}a_t^m+a_t^f=\delta a_t,	\\ &
\hspace{80pt}a_t^m=a_t^f,
\end{split}
\end{equation}
where $\theta^f_{t+1}$ and $\theta^m_{t+1}$ adjust such that the following participation constraints are satisfied:
\begin{equation}\label{eq:p_cons_mar}
\begin{split}
&
W^{f\overline{M}}_{t}(\Omega^{\overline{M}}_{t})\geq V_{t}^{fS}(\omega^f_{t}),\\ &
W^{m\overline{M}}_{t}(\Omega^{\overline{M}}_{t})\geq V_{t}^{mS}(\omega^m_{t}). 
\end{split}
\end{equation}
Note that $\mu^i_t$ are the Lagrange multipliers associated with spouses' participation constraints. The individual value of marriage conditional on $D_t=0$ is denoted by $W_{t}^{i\overline{M}}$ and it can be obtained following the procedure described in the mutual consent regime section. 

Under the unilateral divorce regime Pareto weights vary every time one participation constraint is binding. Whenever a spouse is better off divorcing, the other member will try to convince her not to split by offering her a larger bargaining power, such that she is indifferent between divorcing and staying married. In this framework risk-sharing is less functional than under the mutual consent regime, since variations in the Pareto weight imply less smooth consumption patterns over time. Labor market specialization is also less functioning, since conditionally on having the same state variables, the risk of divorce is higher, which makes women willing to insure against this event through labor market participation. While cooperation is more effective under mutual consent than unilateral divorce, it is still possible that the individual value of being married under the latter regime is larger. This is possible because of the possibility to exit marriage without the consent of the other spouse.

%Property rights upon divorce plays a significant role when splitting is unilateral: for example under community properly the least wealthy member can bargain a higher share of resources since the threat of divorce is real. This could not happen under a title-based regime.
\subsubsection*{Cohabitation}
The problem of cohabiting couples is like the one of marriage under the unilateral divorce regime, but it differs from it for three crucial reasons. First, there is no loss in assets upon breakup. Second, the choice set of the cohabiting couple $\mathbf{q}^C_t=\{a^m_{t+1},a^f_{t+1},d_t,c^m_t,c^f_t,P^f_t,D_t,M_t,\chi_{t+1}\}$ and the state variables $\Omega^{C}_t=\{a^m_t,a^f_t,z^f_t,z^m_t,\psi_t,\theta^f_t,\theta^m_t,\chi_t\}$ are different. Note that $M_t$ is dummy that indicates the choice of marrying and $\chi_t$ is the share of assets going to the women in case of breakup.\footnote{Note that while cohabiting couples can decide to marry, married couples cannot decide to cohabit. This asymmetry is not relevant because married couples would never decide to divorce and start cohabiting, even if they could.} Third, the time utility of cohabiting couple is decreased by $\gamma$.
%Cohabiting couples, denoted by $C$, solve a pareto problem where the weight of the wife is $\theta^f_t$ and the one of the husband is $\theta^m_t$.The state vector is $\Omega^{C}_t=\{a_t,z^f_t,z^m_t,\psi_t,\theta^f_t,\theta^m_t\}$, while the variables over which the couple maximize are summarized by the vector $\mathbf{q}^C_t=\{a_{t+1},d_t,c^m_t,c^f_t,P^f_t,S_t,MA_t\}$. $S_t$ and ${M}_t$ are dummy variable that take value $1$ is the couple respectively breackup or marry\footnote{We denote marriage by $M$, which might be fall under unilateral divorce regime $\overline{M}$ or mutual consent $\hat{M}$.} and 0 otherwise. The formal problem that a cohabiting couple at $t$ solves is:
%\begin{equation}\label{eq:v_coh}
%\begin{split}
%V_t^{C}(\Omega^{C}_t)=&\max_{\mathbf{q}^C_t} (1-S_t)\{\theta^f_{t+1} u(c^f_t,Q_t)+\theta^m_{t+1} u(c^m_t,Q_t)+\psi_t-\gamma+\beta E_t V^{C}_{t+1}(\Omega^{C}_{t+1})\}
%\\ &\quad\quad+MA_t\{\theta^f_{t+1} u(c^f_t,Q_t)+\theta^m_{t+1} u(c^m_t,Q_t)+\psi_t+\beta E_t V^{M}_{t+1}(\Omega^{M}_{t+1})\}\\ &\quad\quad\quad\quad+S_t \{\theta^f_{t}V^{fS}_{t}(\omega^{f}_{t+1})+\theta^m_{t} V^{mS}_{t}(\omega^{m}_{t}))\}
%\\ &\text{if $S_t=0$:}\hspace{35pt}\text{s.t. \eqref{eq:bcm} and \eqref{eq:pfunction},}\\ &\hspace{80pt}
%\theta^f_{t+1}=\theta^f_{t}+\mu^f_t,\\ &\hspace{80pt}
%\theta^m_{t+1}=\theta^m_{t}+\mu^m_t,
%\\ &\text{if $S_t=1$:}\hspace{35pt}\text{s.t. \eqref{eq:bcs}, \eqref{eq:pfunction} for $i\in\{f,m\}$,}\\ &
%\hspace{80pt}a_t^m+a_t^f= a_t,	\\ &
%\hspace{80pt}a_t^m,a_t^f\text{ determined as in the title-based regime},
%\end{split}
%\end{equation}
%where $\theta^f_{t+1}$ and $\theta^m_{t+1}$ adjust such that the following participation constraints are satisfied:
%\begin{equation}\label{eq:p_cons_coh}
%\begin{split}
%&
%W^{fC}_{t}(\Omega^{C}_{t})\geq V_{t}^{fS}(\omega^f_{t}),\\ &
%W^{mC}_{t}(\Omega^{C}_{t})\geq V_{t}^{mS}(\omega^m_{t}). 
%\end{split}
%\end{equation}
%Note that $\mu^i_t$ are the Lagrange multipliers associated with spouses' participation constraints.
%The individual value of cohabitation conditional on $S_t=$ is  $W_{t}^{iC}$ for $i\in\{f,m\}$, and it is defined as 
%\begin{equation}
%W_{t}^{iC}=u(\tilde{c}_t^{i},\tilde{Q}_t^{i})+\psi_t-\gamma+\beta E_t V_{t+1}^{iC}(\Omega^{C}_{t+1}),
%\end{equation}
%where
%$\mathbf{\tilde{q}}^{C}_t=\{\tilde{a}_{t+1},\tilde{d}_{t},\tilde{c}^{m}_{t},\tilde{c}^{f}_{t},\tilde{P}^{f}_t\}$ is the $\argmax$ of problem \eqref{eq:v_mutual} conditionally on having chosen $S_t=0$. $V_{t+1}^{iC}(\Omega^{C}_{t+1})$ instead can be obtained by the expectation of the sum of the time utilities that the agent get from $t+1$ to $T$, where the variables entering the utility function derive from the pareto problem if the agent is in a relationship, otherwise they are the solution of \eqref{eq:v_single}.  Similarly to the unilateral divorce regime, we assume that the planner evaluates the welfare of the two members of the couple if a breakup happens with the current Pareto weights.

The fact that there is not breakup cost makes risk-sharing and cooperation less functional compared to marriage. This happens because the couple is left without a commitment-enhancing technology, which would have allowed the couple to improve its ability to commit.\footnote{Under the limit case of an infinite cost of splitting, as long as the couple stays intact the allocation under the mutual consent and unilateral divorce regimes are the same and correspond to the inter-temporal Pareto-efficient allocation.} On the other hand, assuming no cost of breakup makes cohabitation more appealing to couples whose risk of splitting is high. For example, this is the case of couples with a low match quality.

 Property rights upon divorce/breakup differ between marriage and cohabitation. In the former, assets are divided equally when the couple splits, while in the latter assets are divided according to individual property rights. We model property rights at breakup following \cite{bayot2015}, where upon divorce assets are split following the sharing rule decided by the couple in the previous period.\footnote{\cite{bayot2015} study the choice between community and separation of property in Italy. Separation of property resembles the American title-based regime, which also applies to cohabitation.} They show that this regime is always preferred to community property if outside options are invariant to property right regimes. In our framework this result implies that if the cost of breaking up was the same as the one of divorcing and there was no stigma towards cohabitation, the value of cohabitation would always be higher than the one of marriage. The benefits of having a positive cost of divorce and the stigma linked to cohabitation allows us to generate a positive number of marriages and to match the data.

\subsection{Partnership Choice and the Mating Market}\label{ssec:marriage_market}
In each period $t$ singles have a probability $\lambda_t$ to meet a potential partner. The productivity and the assets of the potential partner depends on the single agent's characteristics. Formally, the assets of the potential partner $p$ are defined as:
\begin{equation}\label{eq:mma}
\ln(a^p_t)=\ln(a^s_t)+\overline{a}^{s}+\epsilon^a,
\end{equation}
where $a^s_t$ are the assets of the individual, $\overline{a}^{s}_t$ is a number that depends on gender and $\epsilon^a$ is a normally distributed shock. Instead, the productivity of the potential partner is defined as:
\begin{equation}\label{eq:mmz}
z_t^p=f(\overline{z}^{s^*,i^*}_t,z^r_t,\epsilon^z),
\end{equation}
where $\overline{z}^{s^*,i^*}_t$ represents the average productivity of singles of gender $s^*$,  $z^r_t$ is the productivity of the agent net of the gender and education specific trend, while $\epsilon^z$ is a normally distributed shock. These assumptions capture in a reduced from fashion that people are mating assortatively both within marriage and cohabitation.
Once the meeting happened, agents must decide whether to stay in a couple and eventually decide which partnership contract to choose and they must pick a Pareto weight. We now describe how these decisions are taken. Note that for the rest of this section we will refer to marriage as $M$, where $M \ \in\{\hat{M},\overline{M}\}$ depending on the divorce regime. The decisions follow a three-steps procedure. 
\begin{enumerate}
\item The couple considers marriage $M$ (cohabitation $C$) as a viable alternative if the set of Pareto weights $\theta^f$ such that the couple prefers to marry (cohabit) is non-empty.\footnote{Without loss of generality, we impose $\theta^f_t+\theta^m_t=1$ at first meeting.} Formally, for relationship $J\in\{M,C\}$ the set is
\begin{equation}\label{eq:set_couple}
\Theta^J_t(\Omega^J_t,\omega^f_t,\omega^m_t)=\big\{\theta_t^f: V_t^{fJ}(\Omega^J_t)\geq V_t^{fS}(\omega^f_t), V_t^{mJ}(\Omega^J_t)\geq V_t^{mS}(\omega^m_t)\big\}.
\end{equation}
\item If the set for marriage (cohabitation) is non-empty, the Pareto weight for the potential marriage $\theta^{M,f}_t$ (cohabitation $\theta^{C,f}_t$) is set through symmetric Nash Bargaining.\footnote{The assumption that the initial Pareto weight is pinned down by Nash Bargaining can be found in \cite{mazzocco2007} and \cite{low2018}.} Formally $\theta^{J,f}_t$ is set to :
\begin{equation}\label{nash_couple}
\theta^{J,f}_t= \argmax_{\theta^f_t\in\Theta^J_t} \Upsilon^J(\theta^f_t,\Omega^{J-1}_t,\omega^f_t,\omega^m_t),
\end{equation}
where $\Omega^{J-1}_t$ is the state vector of the couple excluding Pareto weights and
\begin{equation}
\Upsilon^J(\theta^f_t,\Omega^{J-1}_t,\omega^f_t,\omega^m_t)=\big[V_t^{fJ}(\Omega^{J-1}_t)- V_t^{fS}(\omega^f_t)\big]\times\big[ V_t^{mJ}(\Omega^{J-1}_t)- V_t^{mS}(\omega^m_t)\big].
\end{equation}
\item Four possible situations can arise:
\begin{itemize}
\item $\Theta^M_t=\O\text{ and }\Theta^C_t=\O \Rightarrow$ stay single.
\item $\Theta^M_t\neq\O\text{ and }\Theta^C_t=\O \Rightarrow$ marry.
\item $\Theta^M_t=\O\text{ and }\Theta^C_t\neq\O \Rightarrow$ cohabit.
\item $\Theta^M_t\neq\O\text{ and }\Theta^C_t\neq\O \Rightarrow$ The couple chooses the partnership that gives the largest Nash product. Formally, if $ \Upsilon^M(\Omega^M_t,\omega^f_t,\omega^m_t)\geq\Upsilon^C(\Omega^C_t,\omega^f_t,\omega^m_t)$ the couple chooses marriage, otherwise cohabitation.
\end{itemize}
\end{enumerate}
\section{Estimation}
We estimate the structural model following a two-step procedure. The first step is to set some parameters following the literature or by matching some features of the data without the need to simulate the model. In particular, we estimate the labor income processes of men and women outside the model: this procedure is common in the literature because it reduces the burden on structural estimation.\footnote{See for example \cite{voena2015}, \cite{reynoso2019} and \cite{gourinchas2002}.}
The second step is to estimate by indirect inference the remaining parameters of the model. In this section we detail the steps of the estimation, we discuss the identification of the structural parameters and we present the results.
\subsection{Income Processes}
The income processes of men and women are estimated using the $1968$-$1993$ waves of the PSID, including people between age 20 and 65. We further restrict our sample by retaining men who are household heads or men who are married/cohabiting with the household head or who are household heads themselves. Similarly to \cite{low2018}, we drop observations where the hourly wage is less than half the minimum wage and where the hourly wage changes by more than 125\% in two subsequent years.  We compute the hourly wage rate of men and women dividing the annual labor income by the number of yearly working hours supplied. This procedure avoids considering a variation in working hours as a productivity shock. This correction is particularly relevant for the estimation of the income process of women, because their hours worked vary significantly over the life-cycle. 

The income process of men is estimated by fitting the following liner model:
\begin{equation}\label{eq:male_earn}
\ln(w^m_{i,t,s,sur})=\iota^m_0+\iota^m_1*t+\iota^m_2*t^2+\delta_s+\nu_{sur}+u_{i,t,s,sur}^m,
\end{equation}
where $i$ stands for individual, $t$ for age, $s$ for state and $sur$ for survey year. Moreover, $u_{i,t,s,sur}^m=z_t^m+e^m_{i,t,s,sur}$, where $z_t^m$ follows equation $\ref{eq:pcomp}$, while $e^m_{i,t,s,sur}$ is the measurement error. Instead, $\delta_s$ are state fixed effects and $\nu_{sur}$ are year of the survey fixed effects. The results are reported in table \ref{table:inc_men}. Then, using the residuals  $\hat{u}_{t}^m$, we estimate through GMM  1) the variance of the permanent component of income $\sigma_\zeta^{2_m}$, 2) the variance of the measurement error $\sigma_e^{2_m}$ using the following conditions:
\begin{equation}\label{eq:male_gmm}
\begin{split}
E((\Delta\hat{u}_{t}^m)^2)&=\sigma_\zeta^{2_m}+2\sigma_e^{2_m}\\
E(\Delta\hat{u}_{t}^m\Delta\hat{u}_{t-1}^m)&=-\sigma_e^{2_m}
\end{split}
\end{equation}
Results are reported in table \ref{table:income_params}.

The estimation of women's income process differs from the men's one since we need to consider the endogeneity of female labor force participation. We do so by using a two-step Heckman selection correction procedure. The first step consists in estimating a probit model where the dependent variable is female labor force participation and the independent variables includes all the regressors in equation (\ref{eq:male_earn}) plus the interaction of a dummy variable for unilateral divorce with the dummy variables for the property rights regimes upon divorce. These variables are used as exclusion restriction following the work of \cite{voena2015}, who finds that these affect female labor force participation by influencing intra-household bargaining.\footnote{\cite{voena2015} and \cite{reynoso2019} already used the interaction between grounds of divorce and division of property as an exclusion restriction for female labor force participation.} Women participate in the labor market if
\begin{equation}\label{eq:femae_part}
\gamma'\mathbf{Z}_{i,t,s,sur}+\pi_{i,t,s,sur}>0,
\end{equation}
where $\pi_{i,t,s,sur}$ is the sum of the measurement error and the permanent component of income and $\mathbf{Z}_{i,t,s,sur}$ contains the regressors. The second setup is estimating the following linear model:
\begin{equation}\label{eq:femae_earn}
\ln(w^f_{i,t,s,sur})=\iota^f_0+\iota^f_1*t+\iota^f_2*t^2+\delta_s+\nu_{sur}+\varphi_{i,t,s,sur}+u_{i,t,s,sur}^f,
\end{equation}
where $i$ stands for individual, $t$ for age, $s$ for state and $sur$ for survey year. Moreover, $u_{i,t,s,sur}^f=z_t^f+e^f_{i,t,s,sur}$. $z_t^f$ follows equation $\ref{eq:pcomp}$, while $e^f_{i,t,s,sur}$ is the measurement error. Instead, $\delta_s$ are state fixed effects and $\nu_{sur}$ are year of the survey fixed effects. The endogeneity of female labor force participation is considered by controlling for $\varphi_{i,t,s,sur}$, the inverse of the Mills ratio of the prediction obtained in the first step. The estimation results of the two steps are reported in tables \ref{table:prb_wom} and \ref{table:inc_wom}. We then use the regression residuals from the second step $\hat{u}_{t}^m$  to estimate through GMM 1) the variance of the permanent component of income $\sigma_\zeta^{2_f}$, 2) the variance of the measurement error $\sigma_e^{2_f}$ using the following conditions:\footnote{The conditions are those used by \cite{low2018}.}
\begin{equation}\label{eq:female_gmm}
\begin{split}
E(\Delta\hat{u}_{t}^f | P^f_t=1,P^f_{t-1}=1)&=\sigma_\pi^{f}\frac{\phi(\tau_t)}{1-\Phi(\tau_t)},\\
E((\Delta\hat{u}_{t}^{f})^2 | P^f_t=1,P^f_{t-1}=1)&=\sigma_\zeta^{2_f}+\sigma_\pi^{2_f}+2\sigma_e^{2_f}+\tau_t\frac{\phi(\tau_t)}{1-\Phi(\tau_t)},\\
E(\Delta\hat{u}_{t}^f\Delta\hat{u}_{t-1}^f | P^f_t=1,P^f_{t-1}=1,P^f_{t-2}=1))&=-\sigma_e^{2_f}.
\end{split}
\end{equation}
where $\phi()$ and $\Phi()$ are respectively the density and the distribution function of a standardized normal,  while $\tau_t=-\gamma'\mathbf{Z}_{i,t,s,sur}$. Results are displayed in table \ref{table:income_params}.

\begin{table}[H]
\caption{\\Parameters of the income processes} % title of Table
\centering % used for centering table
\begin{threeparttable}
\begin{tabular}{@{\extracolsep{5pt}}lccc}   % centered columns (4 columns)
\hline \hline%inserts single line
\rule{-4pt}{2.5ex}
Parameter & Symbol  & Value \\ [0.45ex] % inserts table
\hline
\rule{-4pt}{2.5ex}
$f$'s age return (constant)         & $\iota^f_0$  & -0.383 &  \\[0.45ex]
$f$'s age return (linear component) & $\iota^f_1$         & 0.0244  & \\[0.45ex]
$f$'s age return (squared component)               & $\iota^f_2$       & -0.0005 &  \\[0.45ex]
Variance of $f$'s permanent income shock                                 & $\sigma_\zeta^{2_f}$             & 0.0399  &  \\[0.45ex]
$m$'s age return (constant)         & $\iota^m_0$  & -0.342 &\\[0.45ex]
$m$'s age return (linear component) & $\iota^m_1$         & 0.0495  &  \\[0.45ex]
$m$'s age return (squared component)               & $\iota^m_2$       & -0.0009 &  \\[0.45ex]
Variance of $m$'s permanent income shock                                 & $\sigma_\zeta^{2_m}$             & 0.0417  &  \\[0.45ex]
\hline
\end{tabular}
\begin{tablenotes}[flushleft]
\footnotesize{\item \textsc{Notes}: The parameters are estimated using nonlinear least squares using single, cohabiting and married males and females from the PSID. }
\end{tablenotes}
\end{threeparttable}
\label{table:income_params}
\end{table}
\FloatBarrier
\subsection{Preset Parameters}
This section describes how we fix the set of preset parameters. Each period in the model lasts $1$ year: we chose this length balancing the benefits of having a short period, which fits the fact that cohabitation spells are particularly short, and the computational burden associated with having too many periods. We assume that men (women) start making decisions at age $20$ ($18$). Couples are always formed by men who are 2 years older than women. Agents retire at the age of 62 and the number of periods in the model if $T=62$. The discount factor $\beta$, the annual interest rate and the relative risk aversion $\sigma$ of private goods match those in \cite{attanasio2008}. Instead, the parameters relative to the production of public goods, $\nu$ and $\kappa$ match those in \cite{mcgrattan1997}. As far as the pensions are concerned, I follow \cite{heathcote2010}: they consider the progressive nature of the US system but they simplify it, assuming that only the last period before retirement is relevant for the amount of the pension that a person receives. Parameter $\phi$ is set to 0.189 to reflect the relative time that singles spend on house works relative to the time spend on the labor market.\footnote{In the PSID the average yearly time spend on house works by singles is 465.5 hours. Assuming that the yearly hours of full-time work in the labor market is 2000, we get $\phi=465.5/(465.5+2000)=0.189$. The median number of yearly hours spent in the labor market for single men is 1976, while for single women is 1848. We considered the 1940-1955 birth cohorts of the PSID for these computations because the moments that we will use in the structural estimation are based on the behavior of people born in those years.} Wages are normalized such that average log wages of male at age 30 is 0. The variance of male and female's earnings at age $20$ $\sigma_{\zeta,1}^{2_m}$ and $\sigma_{\zeta,1}^{2_f}$ are taken directly from the PSID data. The parameters regarding the mating market, contained in equations \ref{eq:mma} and \ref{eq:mmz}, are pinned down to obtain a realistic degree of assortative mating with respect to assets and wages. In particular, we target the correlation in log wages in the PSID and the share of households with family income above the median whose wealth is also above the median in the Survey of Consumer Finances (SCF). The parameters of the mating market are pinned down to respect a second condition, which is \textit{symmetry}. For example, married men at age $t$ should have on average the same wage and wealth regardless of being simulated for their life cycle, or being partners of women who are simulated for their whole life cycle.\footnote{The agents who belong to our fictional sample are simulated for their whole life cycle and they marry/cohabit with partners that they randomly meet. The behavior of these partners is followed only while they are in a relationship with the person in our fictional sample. Figure \ref{fig:symmetry} shows the mean and variance of productivity and wealth by age, both for agents belonging to the “fictional sample” and to their “partners”. The variables of interest are similar for the two groups, which means that the two groups are symmetric with respect to these variables.} Since we set these parameters before the structural estimation takes place, we cannot perfectly match the mating market moment that we targeted. The correlation in log wages of couples in the PISD is 0.58 versus 0.62 in the simulated sample,\footnote{We obtain this value by simulating the behavior of agents under the parametrization of deep parameters described later in this section.} while the share of people that have a wealth above the median, conditionally on having a family income above the median, is 0.76 in the Survey of Consumer Finances and 0.82 in the model. 
\begin{table}[H]
\caption{\\Preset parameters} % title of Table
\label{table:preset_params}
\centering % used for centering table
\begin{threeparttable}
\begin{tabular}{@{\extracolsep{5pt}}lccc}   % centered columns (4 columns)
\hline \hline 
\rule{-4pt}{2.5ex}
 Estimated Parameters & Symbol & Value & Source  \\ [0.45ex] % inserts table
\hline
\rule{-4pt}{2.5ex}
 Initial age              &          & 18-20  &  \\[0.45ex]
 Retirement age           &          & 62  &  \\[0.45ex]
 Number of time periods             &     $T$     & 62  &  \\[0.45ex]
 Years per period         &          & 1  &  \\[0.45ex]
% Retirement income        &          & 0.61 &  \\[0.45ex]
 $m'$s average earnings at $30$     &       & 1 & Normalization \\[0.45ex]
 Mating market---productivities      &       &  & PSID \\[0.45ex]
 Mating market---assets      &       &  & SCF \\[0.45ex]
 Pensions      &       &  &  \cite{heathcote2010} \\[0.45ex]
  Var. $f$'s productivity in $t=1$&$\sigma_{\zeta,1}^{2_f}$       & 0.54 & PSID \\[0.45ex]
   Var. $m$'s productivity   $t=1$  &$\sigma_{\zeta,1}^{2_m}$       & 0.54 & PSID \\[0.45ex]
    Interest rate            &$R-1$       & 1.5\% & \cite{attanasio2008}   \\[0.45ex]
 Relative Risk Aversion private good   &$\gamma$  & 1.5 &\cite{attanasio2008}  \\[0.45ex]
 Discount factor          &$\beta$   & 0.98&\cite{attanasio2008}  \\[0.45ex]
\hline \hline
\rule{-4pt}{2.5ex}
 Function &Symbol & Value & Source  \\ [0.45ex] % inserts table
 \hline 
$Q_t=[d_t^\nu+\kappa {(1-P^f_t)}^\nu]^{\frac{1}{\nu}}$& 
\begin{tabular}{@{}c@{}}$\kappa$ \\[0.25ex] $\nu$\end{tabular} &\begin{tabular}{@{}c@{}}3.76 \\[0.25ex] 0.19\end{tabular}  &  \begin{tabular}{@{}c@{}} \cite{mcgrattan1997} \\[0.25ex] \cite{mcgrattan1997}\end{tabular}  \\[0.95ex]
\hline
\end{tabular}
\end{threeparttable}
\end{table}
\FloatBarrier


\subsection{Indirect Inference}
We use the method of indirect inference \citep{gourieroux1993} to pin down the vector $\vartheta=(\alpha, \lambda, \sigma_\psi, \sigma_{\psi,I}, \delta, \mu, \xi, \gamma )$ of the 8 remaining parameters of the model. We use 31 moments and regression coefficients for the structural estimation, which capture the process of marriage and cohabitation creation and dissolution, as well as female labor supply. More precisely, we include as targets the coefficient of unilateral divorce estimated through equation \ref{eq:ols_baseline},\footnote{Note that the sample used for estimating equation \ref{eq:ols_baseline} in the empirical section and in the structural estimation is different. We will describe within this section how the sample used for structural estimation is constructed.} the hazard of divorce (6), the hazard of breakup (3), the hazard of marriage (3), the share of people ever married over time (7), the share of people that ever cohabited over time (7), female labor supply (1),\footnote{Female labor supply in the model is constructed by multiplying the indicator of female labor force participation by 2000 hours. The assumption that working full-time corresponds to 2000 hours of work in a year was also used for calibrating $\phi$. Alternatively, we could have targeted female labor force participation, picking a number of hours for full-time work such that female labor supply is also matched.  Since the amount of part-time work is very different according to the status (married, cohabiting or single) of the women, the number of hours for participating women should have been differed by status. The problem with this approach is that women would have chosen their partnership according to the artificially fixed working schedule that partnerships offer, and not only accolading to the mechanisms that our model generates.} differences in female labor force participation between marriage and cohabitation (2) and differences in log wages between married and cohabiting men (1). We use the retrospective marital history data from the NSFH wave III to construct the moments linked to partnership choice, while we all the others are computed using the PSID.\footnote{NSFH wave III is conducted in 2001/2003 following the original respondents of wave 1. This sample does not include respondents under age 45 as of January 2000 unless some particular conditions are met, but this is not an issue for us since the youngest person in our estimating sample was 44 in 2000. One possible issue with this data is that by mistake during NSFH wave II all cohabiting couples were dropped by the sample. We overcome this problem by simulating the same ``mistake" on the sample drawn from the simulated data.} The data moments are constructed selecting men and women born in 1940-1955 in community property states.

The first step for the estimation is to solve the model for a vector of parameters $\vartheta$, then simulating income, love shocks and unexpected divorce policy changes to obtain the simulated behavior for the given parametrization. The next step is to perform stratified sampling on the simulated population in order to obtain the same distribution over gender/age/regime of divorce as in the data used to construct the moments. This allows us to compare the simulated and data moments: the objective is to obtain $\vartheta$ such that this difference is the smallest possible. Formally, the problem that we solve is
\begin{equation}\label{eq:msm}
\hat{\vartheta}=\arg\min_\vartheta \quad\quad (\mathbf{m}-\mathbf{m}_\vartheta)'\mathbf{W}(\mathbf{m}-\mathbf{m}_\vartheta),
\end{equation}
where $\mathbf{m}$ is the vector of empirical moments, as described in the section about target moments, while $\mathbf{m}_\vartheta$ is the vector of the moments simulated by the model parametrized with $\vartheta$. $\mathbf{W}$ is a matrix where the diagonal contains the inverse of the variance of the data moments, while all the other entries are zeros. The minimization of this object function is performed using the global optimization algorithm TikTak, which according to \cite{arnoud2019} outperforms an array of global and local optimizers when the target is a difficult objective function. In appendix \ref{section:computation} we describe in detail how the algorithm TikTak works and how me modify it to allow for the possibility of running it in parallel.

\subsection{Identification}
This section provides a description of how the structural parameters of the model are identified  heuristically. The parameter $\alpha$ is identified by total female labor supply: when this parameter is large, the household want to produce more of public good which requires women's time. Instead $\mu$ affects the gap in female labor supply for married and cohabiting couples. When this parameter is large the gap increases, because specialization within cohabitation becomes relatively harder as this relationship lacks a commitment technology $\lambda$ is intuitively identified by the share of people in a relationship.  The parameter $\sigma_{\psi}$ has a role in identifying the stability of marriage and cohabitation by modifying the likelihood that marriage surplus becomes negative, but it is mostly identified by the share of people that are choosing marriage over cohabitation. In fact, as this parameter
grows larger, money become less important than love for total utility. This means that
agents care less about insuring against income shocks and labor specialization starts
binding less, while the risk of breakup and divorce increases. The parameter $\sigma_{\psi}$ alone is not able to generate a large enough marriage surplus, such that the number of ever married people is matched. For this reason we introduced parameter $\gamma$, thanks to which we can match the share of people ever married and that ever cohabited. Also parameter $\delta$ influences the gain of marriage with respect to cohabitation, but it does so in a non-monotone fashion. In fact, on the one hand increasing the cost of divorce enhance commitment, while on the other hand it makes more costly to end the relationship. Hence, the effect of increasing or decreasing $\delta$ depends on its initial value. Since the introduction of unilateral divorce is to a first approximation like a decrease in the cost of divorce, the parameter $\delta$ is mostly identified by the coefficient of unilateral divorce of regression \ref{eq:ols_baseline}. Instead, $\sigma_{\psi,I}$ is identified by the hazard of breakup and marriage: when this parameter is small compared to the variance of the transitory shocks, agents are not \textit{picky} about sorting into cohabitation, but they move fast to a marriage or they separate within the first periods of the relationship, according to the evolution of the love and productivity shocks. Finally, the parameter $\xi$ influences the surplus of marriage and cohabitation by wealth. In fact, when $\xi$ is small, wealthier agents find the consumption of the public good $Q$ relatively more attractive. Since marriage allows to consume a larger quantity of $Q$ because it protects women that devote time to its production, marriage becomes a relatively more interesting option for wealthier families. Hence, $\xi$ is identified by the difference in log wages of married and cohabiting men.
\subsection{Model Fit}
Table \ref{table:structural_params} reports the results of the structural estimation. The estimated standard deviation $\sigma_{\psi}$ of the transitory match quality shock is $0.76$, while standard deviation $\sigma_{\psi,I}$ of the love shock at first meeting is higher with a value of $1.67$. Instead, the probability of meeting a partner $\lambda$ is $0.38$, while the share of assets left after divorce is $0.80$. The weight on the public good $\alpha$ is $1.20$, while the loss in productivity parameter $\mu$ is $0.07$. Finally, the penalty of cohabiting $\gamma$ is $0.15$, while the coefficient of relative risk aversion for the public good $\xi$ is $1.14$.

The fit of the model is reported in table \ref{table:fit}. The model matches generally well the hazard of marriage, breakup and divorce over time, even though lie outside the 95\% confidence interval of data moments. One exception is that the hazard of divorce and breakup are not hump shaped over duration because our model abstracts from learning, which is necessary to match these pattern of the data \citep{blasutto2020} . The share of people that ever cohabited and married over time is well matched. The data about female labor supply is well matched. Instead, the differences in log wages for married and cohabiting men is lower than in the data. Finally, the coefficient of unilateral divorce estimated through equation (\ref{eq:ols_baseline}) is slightly larger than in the data, but it lies within the 95\% confidence interval.

The model is validated according to its ability to reproduce the effects of unilateral divorce on cohabitation duration, the share of income in the couple earned by married women, the average wage earned by women over their age and the ratio of hazard rates of richer over poorer men.\footnote{Richer men are those whose income is above the median, while poorer men are those whose income is below the median.} We run the same econometric models of section \ref{empirics} for cohabitation duration, both with the real data sample and with the simulated sample. The results, reported in table \ref{table:fit}, show that the size of the effects is well matched. Women's wages over age and the average share of income provided by the wife in the household match the data, which validates the selection of women into the labor force. The fact that the model matches that divorce rates are lower for richer men supports our assumptions regarding the cost of divorce, which influences both the allocation within divorce and the surplus of marriage.

A further test for our model is to check whether the effect of unilateral divorce on the propensity to cohabit is lower under title-based regime than under community property regime, as it is in the data. We solve the model assuming a title-based regime and we obtain that the coefficient of unilateral divorce of equation (\ref{eq:ols_baseline}) is -0.09, while it was -0.16 under community property.\footnote{Note that we do not expect to match exactly the empirical coefficient (\ref{eq:ols_baseline}) under the title-based regime because we did not re-estimate the model using a sample of residents in title-based states. The parameters used for this exercise are those in table \ref{table:fit}.} This result is consistent with the idea that under community property regime the shift towards cohabitation is larger because men, who are those with most decisional power, start finding cohabitation attractive when the risk of divorce increases. This is because upon divorce they would lose most of their assets, leaving a part of them to their ex-wife. This mechanism bites less under a title-based regime, because men would keep the assets of their property upon divorce.




%Table of deep parameters
\begin{table}[H]
\caption{\\Estimated structural parameters} % title of Table
\label{table:structural_params}
\centering % used for centering table
\begin{threeparttable}
\begin{tabular}{@{\extracolsep{5pt}}lccc}   % centered columns (4 columns)
\hline \hline%inserts single line
\rule{-4pt}{2.5ex}
Estimated Parameters &  & Value & \\ [0.45ex] % inserts table
\hline
\rule{-4pt}{2.5ex}
Standard deviation of match quality shock         & $\sigma_{\psi}$  & 0.76 &  \\[0.45ex]
Standard deviation of initial match quality shock & $\sigma_{\psi,I}$         & 1.67  &  \\[0.45ex]
Probability of meeting a partner               & $\lambda$       & 0.38 &\\[0.45ex]
Assets left upon divorce                                 & $\delta$             & 0.80  &  \\[0.45ex]
Weight of public good               & $\alpha$             & 1.20 & \\[0.45ex]
Loss in productivity while not working               & $\mu$             & 0.07 &  \\[0.45ex]
Relative Risk Aversion public good          &$\xi$   & 1.14& \\[0.45ex]
Penalty of Cohabiting          &$\gamma$   & 0.15&  \\[0.45ex]
\hline
\end{tabular}
\end{threeparttable}
\end{table}
\FloatBarrier


\begin{table}[H]
	\caption{\\Model fit and validation} % title of Table
	\label{table:fit} % is used to refer this table in the text
	\centering % used for centering table
\begin{threeparttable}[t]
\begin{tabular}{@{} l c c c c @{}}  % centered columns (4 columns)
\hline\hline %inserts double horizontal lines
\rule{-4pt}{2.5ex}
Estimated Moments & Model  & Data & 95\% CI \\ [0.05ex] % inserts table
%heading
\hline % inserts single horizontal line
\rule{-4pt}{2.5ex}
Hazards over Time              & fig. \ref{fig:haz} & fig. \ref{fig:haz} & fig. \ref{fig:haz} \\[0.15ex]
Share Ever Cohabited and Married         & fig. \ref{fig:erel} &  fig. \ref{fig:erel} & fig. \ref{fig:erel} \\[0.15ex]
FLS in a Couple (hours)                 & 1007 & 1016& [1002,1029] \\[0.15ex]
FLS if Married/ FLS if Cohabiting ($<$35 yrs.)                 & 1.02 & 0.86 & [0.78,0.95] \\[0.15ex]
FLS if Married/ FLS if Cohabiting ($\ge$35 yrs.)                 & 0.97 & 1.00 & [0.89,1.13] \\[0.15ex]
Log wages Marriage-Log wages Cohabitation                  & -0.08 & 0.12 & [0.04,0.12] \\[0.15ex]
Unilateral Divorce coefficient equation (\ref{eq:ols_baseline})     & -0.16  & -0.11 & [-0.21,-0.02] \\[0.15ex]
\hline \hline%inserts single line
\rule{-4pt}{2.5ex}
External Moments & Model  & Data       &      95\% CI                      \\ [0.05ex] % inserts table
\hline 
\rule{-4pt}{2.5ex}
Unilateral Divorce on the relative Risk of Marriage      & 0.75 & 0.73 &   [0.60,0.86]\\[0.15ex]
Unilateral Divorce on the relative Risk of Breakup                   & 0.81 & 0.82 & [0.75,0.89] \\[0.15ex]
Women wages by age           &  fig. \ref{fig:datasimwage} &  fig. \ref{fig:datasimwage} & fig. \ref{fig:datasimwage} \\[0.15ex]
Divorce Rate Rich/Divorce Rate Poor                  & 0.74 & 0.79 & [0.75,0.84] \\[0.15ex]
Share household income earned by women  &  0.34\% &  0.35\% & [0.36-0.38] \\[0.15ex]

\hline
\end{tabular}
	\begin{tablenotes}[flushleft]
\footnotesize{\item \textsc{Notes}: The coefficients and the relative hazard ratios in the table differs from those obtained with the same econometric model in section \ref{empirics}. The reason is that the sample used for the empirical part is different from the one used for structural estimation as explained in the section.}
\end{tablenotes}
\end{threeparttable}
\end{table}

\section{Mechanisms}
The aim of this section is to better understand the mechanisms underlying the introduction of unilateral divorce and the subsequent rise in cohabitation.

 We start by analyzing how selection and intra-household bargaining change as a result of the reform. The estimated structural model allows us to study the evolution of the match quality $\psi$ and women's Pareto weight $\theta_t$ using a standard event study. Specifically, we estimate the following regression model on simulated data
\begin{equation}\label{eq:ev_stud}
\text{Variable of Interest}_{i,a,t}=\sum_{j=-5}^{5}\beta^{Uni}_j\cdot\mathcal{I}(t=j)+\alpha_{0}+\alpha_{a}+\epsilon_{i,t}
\end{equation}
where $a$ is age, $t$ if the year relative to switching to unilateral divorce ($t=-1$ is omitted) and $i$ is a couple. We estimate the model for $\psi$ and $\theta$ using as samples 1) cohabiting couples that just met 2) married couples that just met.
Figure \ref{fig:fis2g} reports the results. We normalize the coefficient estimates $\beta^{Uni}_j$ by adding the average of the variable of interest in the year before unilateral divorce is introduced $E[\text{Variable of Interest}|t=-1]$. 

\textbf{Match quality $\psi$.} We start by analyzing panel \subref{sf:sub-first}. First, note that the average match quality of married couples is higher than for cohabitants.\footnote{A more in depth analysis reveals that the distribution of match quality at meeting of cohabiting couples dominates that of married couples. See figure \ref{fig:psidist}.} This fact is consistent with a strong selection on marriage and cohabitation with respect to match quality. Marriage guarantees a better commitment and cooperation, but when the match quality is low the best option is to opt for cohabitation because breaking up is cheaper than divorcing. The results of the event study show that upon the introduction of unilateral divorce the match quality of newly formed cohabitations increases by a value that is around 35\% percent of its structural standard deviation.\footnote{Note that the observed and structural distributions of the initial match quality are different because couples are not formed when the match quality at meeting is too low.} This result is consistent with selection of relatively high match quality couples into cohabitation after the policy change. This happens because unilateral divorce increases the risk of dissolution of marriage and the spouses' ability to cooperate. 

%This result explains why the hazard of breakup of newly formed cohabiting couples decreases after the policy reform. From panel \subref{sf:sub-first} we also observe that the match quality of newly formed marriages goes up, which happens because of the decreased gains from marriage. Overall, the match quality threshold for being in a couples increases, which increases the rate of singleness, which is consistent with the empirical evidence provided by \cite{rasul2003,rasul2006} and \cite{reynoso2019} that unilateral divorce increased the rate of singleness. 

\textbf{Women's bargaining power $\theta$.} Panel \subref{sf:sub-second} depicts the evolution of the women bargaining power $\theta$ at meeting for cohabitation and marriage around unilateral divorce. $\theta$ increases with respect to baseline for cohabitation after the policy change, while it decreases for marriage. Under Mutual consent, marriage was protecting women from ending up divorced and poor, while cohabitation was chosen only by couples where the man was not able to commit for a long term relationship and the woman had little say about the decision. After the reform, men prefer cohabitation over marriage because it avoids splitting up assets equally upon divorce, but they have to promise a higher initial $\theta$ to convince women.\footnote{Note that upon breakup men receive on average around 65\% of the couple's wealth.} Similarly, the Pareto weight of women that marry goes down because men are willing to marry instead of cohabiting only if they can control more resources within the household.

%Lastly, panel \subref{sf:sub-third} shows the evolution of female labor force participation around the reform. Consistently with \cite{fernandezw2014} we find that female labor force participation increases after the reform, as women seek to have higher productivities through experience and accumulate more wealth to smooth out consumption under the increased possibility of a divorce. Note that this effect is weaker within cohabiting couples, as their baseline participation in the labor market is already high.

\begin{figure}[H]
\begin{center}
\caption{\\ Event Studies Around the introduction of Unilateral Divorce--Simulated Data}
\label{fig:fis2g}
\begin{subfigure}{.49\textwidth}
\centering
% include first image
\caption{Love Shock $\psi$ at meeting}
\label{sf:sub-first}
\scalebox{0.5}{%% Creator: Matplotlib, PGF backend
%%
%% To include the figure in your LaTeX document, write
%%   \input{<filename>.pgf}
%%
%% Make sure the required packages are loaded in your preamble
%%   \usepackage{pgf}
%%
%% Figures using additional raster images can only be included by \input if
%% they are in the same directory as the main LaTeX file. For loading figures
%% from other directories you can use the `import` package
%%   \usepackage{import}
%% and then include the figures with
%%   \import{<path to file>}{<filename>.pgf}
%%
%% Matplotlib used the following preamble
%%
\begingroup%
\makeatletter%
\begin{pgfpicture}%
\pgfpathrectangle{\pgfpointorigin}{\pgfqpoint{5.286497in}{3.975237in}}%
\pgfusepath{use as bounding box, clip}%
\begin{pgfscope}%
\pgfsetbuttcap%
\pgfsetmiterjoin%
\definecolor{currentfill}{rgb}{1.000000,1.000000,1.000000}%
\pgfsetfillcolor{currentfill}%
\pgfsetlinewidth{0.000000pt}%
\definecolor{currentstroke}{rgb}{1.000000,1.000000,1.000000}%
\pgfsetstrokecolor{currentstroke}%
\pgfsetdash{}{0pt}%
\pgfpathmoveto{\pgfqpoint{0.000000in}{0.000000in}}%
\pgfpathlineto{\pgfqpoint{5.286497in}{0.000000in}}%
\pgfpathlineto{\pgfqpoint{5.286497in}{3.975237in}}%
\pgfpathlineto{\pgfqpoint{0.000000in}{3.975237in}}%
\pgfpathclose%
\pgfusepath{fill}%
\end{pgfscope}%
\begin{pgfscope}%
\pgfsetbuttcap%
\pgfsetmiterjoin%
\definecolor{currentfill}{rgb}{1.000000,1.000000,1.000000}%
\pgfsetfillcolor{currentfill}%
\pgfsetlinewidth{0.000000pt}%
\definecolor{currentstroke}{rgb}{0.000000,0.000000,0.000000}%
\pgfsetstrokecolor{currentstroke}%
\pgfsetstrokeopacity{0.000000}%
\pgfsetdash{}{0pt}%
\pgfpathmoveto{\pgfqpoint{0.636497in}{0.955237in}}%
\pgfpathlineto{\pgfqpoint{5.286497in}{0.955237in}}%
\pgfpathlineto{\pgfqpoint{5.286497in}{3.975237in}}%
\pgfpathlineto{\pgfqpoint{0.636497in}{3.975237in}}%
\pgfpathclose%
\pgfusepath{fill}%
\end{pgfscope}%
\begin{pgfscope}%
\pgfsetbuttcap%
\pgfsetroundjoin%
\definecolor{currentfill}{rgb}{0.000000,0.000000,0.000000}%
\pgfsetfillcolor{currentfill}%
\pgfsetlinewidth{0.803000pt}%
\definecolor{currentstroke}{rgb}{0.000000,0.000000,0.000000}%
\pgfsetstrokecolor{currentstroke}%
\pgfsetdash{}{0pt}%
\pgfsys@defobject{currentmarker}{\pgfqpoint{0.000000in}{-0.048611in}}{\pgfqpoint{0.000000in}{0.000000in}}{%
\pgfpathmoveto{\pgfqpoint{0.000000in}{0.000000in}}%
\pgfpathlineto{\pgfqpoint{0.000000in}{-0.048611in}}%
\pgfusepath{stroke,fill}%
}%
\begin{pgfscope}%
\pgfsys@transformshift{1.270588in}{0.955237in}%
\pgfsys@useobject{currentmarker}{}%
\end{pgfscope}%
\end{pgfscope}%
\begin{pgfscope}%
\definecolor{textcolor}{rgb}{0.000000,0.000000,0.000000}%
\pgfsetstrokecolor{textcolor}%
\pgfsetfillcolor{textcolor}%
\pgftext[x=1.270588in,y=0.858015in,,top]{\color{textcolor}\rmfamily\fontsize{11.000000}{13.200000}\selectfont \(\displaystyle -4\)}%
\end{pgfscope}%
\begin{pgfscope}%
\pgfsetbuttcap%
\pgfsetroundjoin%
\definecolor{currentfill}{rgb}{0.000000,0.000000,0.000000}%
\pgfsetfillcolor{currentfill}%
\pgfsetlinewidth{0.803000pt}%
\definecolor{currentstroke}{rgb}{0.000000,0.000000,0.000000}%
\pgfsetstrokecolor{currentstroke}%
\pgfsetdash{}{0pt}%
\pgfsys@defobject{currentmarker}{\pgfqpoint{0.000000in}{-0.048611in}}{\pgfqpoint{0.000000in}{0.000000in}}{%
\pgfpathmoveto{\pgfqpoint{0.000000in}{0.000000in}}%
\pgfpathlineto{\pgfqpoint{0.000000in}{-0.048611in}}%
\pgfusepath{stroke,fill}%
}%
\begin{pgfscope}%
\pgfsys@transformshift{2.116043in}{0.955237in}%
\pgfsys@useobject{currentmarker}{}%
\end{pgfscope}%
\end{pgfscope}%
\begin{pgfscope}%
\definecolor{textcolor}{rgb}{0.000000,0.000000,0.000000}%
\pgfsetstrokecolor{textcolor}%
\pgfsetfillcolor{textcolor}%
\pgftext[x=2.116043in,y=0.858015in,,top]{\color{textcolor}\rmfamily\fontsize{11.000000}{13.200000}\selectfont \(\displaystyle -2\)}%
\end{pgfscope}%
\begin{pgfscope}%
\pgfsetbuttcap%
\pgfsetroundjoin%
\definecolor{currentfill}{rgb}{0.000000,0.000000,0.000000}%
\pgfsetfillcolor{currentfill}%
\pgfsetlinewidth{0.803000pt}%
\definecolor{currentstroke}{rgb}{0.000000,0.000000,0.000000}%
\pgfsetstrokecolor{currentstroke}%
\pgfsetdash{}{0pt}%
\pgfsys@defobject{currentmarker}{\pgfqpoint{0.000000in}{-0.048611in}}{\pgfqpoint{0.000000in}{0.000000in}}{%
\pgfpathmoveto{\pgfqpoint{0.000000in}{0.000000in}}%
\pgfpathlineto{\pgfqpoint{0.000000in}{-0.048611in}}%
\pgfusepath{stroke,fill}%
}%
\begin{pgfscope}%
\pgfsys@transformshift{2.961497in}{0.955237in}%
\pgfsys@useobject{currentmarker}{}%
\end{pgfscope}%
\end{pgfscope}%
\begin{pgfscope}%
\definecolor{textcolor}{rgb}{0.000000,0.000000,0.000000}%
\pgfsetstrokecolor{textcolor}%
\pgfsetfillcolor{textcolor}%
\pgftext[x=2.961497in,y=0.858015in,,top]{\color{textcolor}\rmfamily\fontsize{11.000000}{13.200000}\selectfont \(\displaystyle 0\)}%
\end{pgfscope}%
\begin{pgfscope}%
\pgfsetbuttcap%
\pgfsetroundjoin%
\definecolor{currentfill}{rgb}{0.000000,0.000000,0.000000}%
\pgfsetfillcolor{currentfill}%
\pgfsetlinewidth{0.803000pt}%
\definecolor{currentstroke}{rgb}{0.000000,0.000000,0.000000}%
\pgfsetstrokecolor{currentstroke}%
\pgfsetdash{}{0pt}%
\pgfsys@defobject{currentmarker}{\pgfqpoint{0.000000in}{-0.048611in}}{\pgfqpoint{0.000000in}{0.000000in}}{%
\pgfpathmoveto{\pgfqpoint{0.000000in}{0.000000in}}%
\pgfpathlineto{\pgfqpoint{0.000000in}{-0.048611in}}%
\pgfusepath{stroke,fill}%
}%
\begin{pgfscope}%
\pgfsys@transformshift{3.806952in}{0.955237in}%
\pgfsys@useobject{currentmarker}{}%
\end{pgfscope}%
\end{pgfscope}%
\begin{pgfscope}%
\definecolor{textcolor}{rgb}{0.000000,0.000000,0.000000}%
\pgfsetstrokecolor{textcolor}%
\pgfsetfillcolor{textcolor}%
\pgftext[x=3.806952in,y=0.858015in,,top]{\color{textcolor}\rmfamily\fontsize{11.000000}{13.200000}\selectfont \(\displaystyle 2\)}%
\end{pgfscope}%
\begin{pgfscope}%
\pgfsetbuttcap%
\pgfsetroundjoin%
\definecolor{currentfill}{rgb}{0.000000,0.000000,0.000000}%
\pgfsetfillcolor{currentfill}%
\pgfsetlinewidth{0.803000pt}%
\definecolor{currentstroke}{rgb}{0.000000,0.000000,0.000000}%
\pgfsetstrokecolor{currentstroke}%
\pgfsetdash{}{0pt}%
\pgfsys@defobject{currentmarker}{\pgfqpoint{0.000000in}{-0.048611in}}{\pgfqpoint{0.000000in}{0.000000in}}{%
\pgfpathmoveto{\pgfqpoint{0.000000in}{0.000000in}}%
\pgfpathlineto{\pgfqpoint{0.000000in}{-0.048611in}}%
\pgfusepath{stroke,fill}%
}%
\begin{pgfscope}%
\pgfsys@transformshift{4.652406in}{0.955237in}%
\pgfsys@useobject{currentmarker}{}%
\end{pgfscope}%
\end{pgfscope}%
\begin{pgfscope}%
\definecolor{textcolor}{rgb}{0.000000,0.000000,0.000000}%
\pgfsetstrokecolor{textcolor}%
\pgfsetfillcolor{textcolor}%
\pgftext[x=4.652406in,y=0.858015in,,top]{\color{textcolor}\rmfamily\fontsize{11.000000}{13.200000}\selectfont \(\displaystyle 4\)}%
\end{pgfscope}%
\begin{pgfscope}%
\definecolor{textcolor}{rgb}{0.000000,0.000000,0.000000}%
\pgfsetstrokecolor{textcolor}%
\pgfsetfillcolor{textcolor}%
\pgftext[x=2.961497in,y=0.667275in,,top]{\color{textcolor}\rmfamily\fontsize{16.000000}{19.200000}\selectfont Event time (model years)}%
\end{pgfscope}%
\begin{pgfscope}%
\pgfsetbuttcap%
\pgfsetroundjoin%
\definecolor{currentfill}{rgb}{0.000000,0.000000,0.000000}%
\pgfsetfillcolor{currentfill}%
\pgfsetlinewidth{0.803000pt}%
\definecolor{currentstroke}{rgb}{0.000000,0.000000,0.000000}%
\pgfsetstrokecolor{currentstroke}%
\pgfsetdash{}{0pt}%
\pgfsys@defobject{currentmarker}{\pgfqpoint{-0.048611in}{0.000000in}}{\pgfqpoint{0.000000in}{0.000000in}}{%
\pgfpathmoveto{\pgfqpoint{0.000000in}{0.000000in}}%
\pgfpathlineto{\pgfqpoint{-0.048611in}{0.000000in}}%
\pgfusepath{stroke,fill}%
}%
\begin{pgfscope}%
\pgfsys@transformshift{0.636497in}{1.316023in}%
\pgfsys@useobject{currentmarker}{}%
\end{pgfscope}%
\end{pgfscope}%
\begin{pgfscope}%
\definecolor{textcolor}{rgb}{0.000000,0.000000,0.000000}%
\pgfsetstrokecolor{textcolor}%
\pgfsetfillcolor{textcolor}%
\pgftext[x=0.268904in,y=1.263216in,left,base]{\color{textcolor}\rmfamily\fontsize{11.000000}{13.200000}\selectfont \(\displaystyle 0.20\)}%
\end{pgfscope}%
\begin{pgfscope}%
\pgfsetbuttcap%
\pgfsetroundjoin%
\definecolor{currentfill}{rgb}{0.000000,0.000000,0.000000}%
\pgfsetfillcolor{currentfill}%
\pgfsetlinewidth{0.803000pt}%
\definecolor{currentstroke}{rgb}{0.000000,0.000000,0.000000}%
\pgfsetstrokecolor{currentstroke}%
\pgfsetdash{}{0pt}%
\pgfsys@defobject{currentmarker}{\pgfqpoint{-0.048611in}{0.000000in}}{\pgfqpoint{0.000000in}{0.000000in}}{%
\pgfpathmoveto{\pgfqpoint{0.000000in}{0.000000in}}%
\pgfpathlineto{\pgfqpoint{-0.048611in}{0.000000in}}%
\pgfusepath{stroke,fill}%
}%
\begin{pgfscope}%
\pgfsys@transformshift{0.636497in}{1.761450in}%
\pgfsys@useobject{currentmarker}{}%
\end{pgfscope}%
\end{pgfscope}%
\begin{pgfscope}%
\definecolor{textcolor}{rgb}{0.000000,0.000000,0.000000}%
\pgfsetstrokecolor{textcolor}%
\pgfsetfillcolor{textcolor}%
\pgftext[x=0.268904in,y=1.708643in,left,base]{\color{textcolor}\rmfamily\fontsize{11.000000}{13.200000}\selectfont \(\displaystyle 0.25\)}%
\end{pgfscope}%
\begin{pgfscope}%
\pgfsetbuttcap%
\pgfsetroundjoin%
\definecolor{currentfill}{rgb}{0.000000,0.000000,0.000000}%
\pgfsetfillcolor{currentfill}%
\pgfsetlinewidth{0.803000pt}%
\definecolor{currentstroke}{rgb}{0.000000,0.000000,0.000000}%
\pgfsetstrokecolor{currentstroke}%
\pgfsetdash{}{0pt}%
\pgfsys@defobject{currentmarker}{\pgfqpoint{-0.048611in}{0.000000in}}{\pgfqpoint{0.000000in}{0.000000in}}{%
\pgfpathmoveto{\pgfqpoint{0.000000in}{0.000000in}}%
\pgfpathlineto{\pgfqpoint{-0.048611in}{0.000000in}}%
\pgfusepath{stroke,fill}%
}%
\begin{pgfscope}%
\pgfsys@transformshift{0.636497in}{2.206877in}%
\pgfsys@useobject{currentmarker}{}%
\end{pgfscope}%
\end{pgfscope}%
\begin{pgfscope}%
\definecolor{textcolor}{rgb}{0.000000,0.000000,0.000000}%
\pgfsetstrokecolor{textcolor}%
\pgfsetfillcolor{textcolor}%
\pgftext[x=0.268904in,y=2.154070in,left,base]{\color{textcolor}\rmfamily\fontsize{11.000000}{13.200000}\selectfont \(\displaystyle 0.30\)}%
\end{pgfscope}%
\begin{pgfscope}%
\pgfsetbuttcap%
\pgfsetroundjoin%
\definecolor{currentfill}{rgb}{0.000000,0.000000,0.000000}%
\pgfsetfillcolor{currentfill}%
\pgfsetlinewidth{0.803000pt}%
\definecolor{currentstroke}{rgb}{0.000000,0.000000,0.000000}%
\pgfsetstrokecolor{currentstroke}%
\pgfsetdash{}{0pt}%
\pgfsys@defobject{currentmarker}{\pgfqpoint{-0.048611in}{0.000000in}}{\pgfqpoint{0.000000in}{0.000000in}}{%
\pgfpathmoveto{\pgfqpoint{0.000000in}{0.000000in}}%
\pgfpathlineto{\pgfqpoint{-0.048611in}{0.000000in}}%
\pgfusepath{stroke,fill}%
}%
\begin{pgfscope}%
\pgfsys@transformshift{0.636497in}{2.652304in}%
\pgfsys@useobject{currentmarker}{}%
\end{pgfscope}%
\end{pgfscope}%
\begin{pgfscope}%
\definecolor{textcolor}{rgb}{0.000000,0.000000,0.000000}%
\pgfsetstrokecolor{textcolor}%
\pgfsetfillcolor{textcolor}%
\pgftext[x=0.268904in,y=2.599497in,left,base]{\color{textcolor}\rmfamily\fontsize{11.000000}{13.200000}\selectfont \(\displaystyle 0.35\)}%
\end{pgfscope}%
\begin{pgfscope}%
\pgfsetbuttcap%
\pgfsetroundjoin%
\definecolor{currentfill}{rgb}{0.000000,0.000000,0.000000}%
\pgfsetfillcolor{currentfill}%
\pgfsetlinewidth{0.803000pt}%
\definecolor{currentstroke}{rgb}{0.000000,0.000000,0.000000}%
\pgfsetstrokecolor{currentstroke}%
\pgfsetdash{}{0pt}%
\pgfsys@defobject{currentmarker}{\pgfqpoint{-0.048611in}{0.000000in}}{\pgfqpoint{0.000000in}{0.000000in}}{%
\pgfpathmoveto{\pgfqpoint{0.000000in}{0.000000in}}%
\pgfpathlineto{\pgfqpoint{-0.048611in}{0.000000in}}%
\pgfusepath{stroke,fill}%
}%
\begin{pgfscope}%
\pgfsys@transformshift{0.636497in}{3.097731in}%
\pgfsys@useobject{currentmarker}{}%
\end{pgfscope}%
\end{pgfscope}%
\begin{pgfscope}%
\definecolor{textcolor}{rgb}{0.000000,0.000000,0.000000}%
\pgfsetstrokecolor{textcolor}%
\pgfsetfillcolor{textcolor}%
\pgftext[x=0.268904in,y=3.044924in,left,base]{\color{textcolor}\rmfamily\fontsize{11.000000}{13.200000}\selectfont \(\displaystyle 0.40\)}%
\end{pgfscope}%
\begin{pgfscope}%
\pgfsetbuttcap%
\pgfsetroundjoin%
\definecolor{currentfill}{rgb}{0.000000,0.000000,0.000000}%
\pgfsetfillcolor{currentfill}%
\pgfsetlinewidth{0.803000pt}%
\definecolor{currentstroke}{rgb}{0.000000,0.000000,0.000000}%
\pgfsetstrokecolor{currentstroke}%
\pgfsetdash{}{0pt}%
\pgfsys@defobject{currentmarker}{\pgfqpoint{-0.048611in}{0.000000in}}{\pgfqpoint{0.000000in}{0.000000in}}{%
\pgfpathmoveto{\pgfqpoint{0.000000in}{0.000000in}}%
\pgfpathlineto{\pgfqpoint{-0.048611in}{0.000000in}}%
\pgfusepath{stroke,fill}%
}%
\begin{pgfscope}%
\pgfsys@transformshift{0.636497in}{3.543158in}%
\pgfsys@useobject{currentmarker}{}%
\end{pgfscope}%
\end{pgfscope}%
\begin{pgfscope}%
\definecolor{textcolor}{rgb}{0.000000,0.000000,0.000000}%
\pgfsetstrokecolor{textcolor}%
\pgfsetfillcolor{textcolor}%
\pgftext[x=0.268904in,y=3.490351in,left,base]{\color{textcolor}\rmfamily\fontsize{11.000000}{13.200000}\selectfont \(\displaystyle 0.45\)}%
\end{pgfscope}%
\begin{pgfscope}%
\definecolor{textcolor}{rgb}{0.000000,0.000000,0.000000}%
\pgfsetstrokecolor{textcolor}%
\pgfsetfillcolor{textcolor}%
\pgftext[x=0.213349in,y=2.465237in,,bottom,rotate=90.000000]{\color{textcolor}\rmfamily\fontsize{16.000000}{19.200000}\selectfont \(\displaystyle \psi\)---variation from baseline}%
\end{pgfscope}%
\begin{pgfscope}%
\pgfpathrectangle{\pgfqpoint{0.636497in}{0.955237in}}{\pgfqpoint{4.650000in}{3.020000in}}%
\pgfusepath{clip}%
\pgfsetrectcap%
\pgfsetroundjoin%
\pgfsetlinewidth{1.505625pt}%
\definecolor{currentstroke}{rgb}{1.000000,0.000000,0.000000}%
\pgfsetstrokecolor{currentstroke}%
\pgfsetdash{}{0pt}%
\pgfpathmoveto{\pgfqpoint{0.847861in}{1.205478in}}%
\pgfpathlineto{\pgfqpoint{1.270588in}{1.353924in}}%
\pgfpathlineto{\pgfqpoint{1.693315in}{1.094850in}}%
\pgfpathlineto{\pgfqpoint{2.116043in}{1.204355in}}%
\pgfpathlineto{\pgfqpoint{2.538770in}{1.092510in}}%
\pgfpathlineto{\pgfqpoint{2.961497in}{3.636915in}}%
\pgfpathlineto{\pgfqpoint{3.384224in}{3.405590in}}%
\pgfpathlineto{\pgfqpoint{3.806952in}{3.486069in}}%
\pgfpathlineto{\pgfqpoint{4.229679in}{3.585514in}}%
\pgfpathlineto{\pgfqpoint{4.652406in}{3.614268in}}%
\pgfpathlineto{\pgfqpoint{5.075133in}{3.837965in}}%
\pgfusepath{stroke}%
\end{pgfscope}%
\begin{pgfscope}%
\pgfpathrectangle{\pgfqpoint{0.636497in}{0.955237in}}{\pgfqpoint{4.650000in}{3.020000in}}%
\pgfusepath{clip}%
\pgfsetbuttcap%
\pgfsetroundjoin%
\definecolor{currentfill}{rgb}{1.000000,0.000000,0.000000}%
\pgfsetfillcolor{currentfill}%
\pgfsetlinewidth{1.003750pt}%
\definecolor{currentstroke}{rgb}{1.000000,0.000000,0.000000}%
\pgfsetstrokecolor{currentstroke}%
\pgfsetdash{}{0pt}%
\pgfsys@defobject{currentmarker}{\pgfqpoint{-0.041667in}{-0.041667in}}{\pgfqpoint{0.041667in}{0.041667in}}{%
\pgfpathmoveto{\pgfqpoint{-0.041667in}{0.000000in}}%
\pgfpathlineto{\pgfqpoint{0.041667in}{0.000000in}}%
\pgfpathmoveto{\pgfqpoint{0.000000in}{-0.041667in}}%
\pgfpathlineto{\pgfqpoint{0.000000in}{0.041667in}}%
\pgfusepath{stroke,fill}%
}%
\begin{pgfscope}%
\pgfsys@transformshift{0.847861in}{1.205478in}%
\pgfsys@useobject{currentmarker}{}%
\end{pgfscope}%
\begin{pgfscope}%
\pgfsys@transformshift{1.270588in}{1.353924in}%
\pgfsys@useobject{currentmarker}{}%
\end{pgfscope}%
\begin{pgfscope}%
\pgfsys@transformshift{1.693315in}{1.094850in}%
\pgfsys@useobject{currentmarker}{}%
\end{pgfscope}%
\begin{pgfscope}%
\pgfsys@transformshift{2.116043in}{1.204355in}%
\pgfsys@useobject{currentmarker}{}%
\end{pgfscope}%
\begin{pgfscope}%
\pgfsys@transformshift{2.538770in}{1.092510in}%
\pgfsys@useobject{currentmarker}{}%
\end{pgfscope}%
\begin{pgfscope}%
\pgfsys@transformshift{2.961497in}{3.636915in}%
\pgfsys@useobject{currentmarker}{}%
\end{pgfscope}%
\begin{pgfscope}%
\pgfsys@transformshift{3.384224in}{3.405590in}%
\pgfsys@useobject{currentmarker}{}%
\end{pgfscope}%
\begin{pgfscope}%
\pgfsys@transformshift{3.806952in}{3.486069in}%
\pgfsys@useobject{currentmarker}{}%
\end{pgfscope}%
\begin{pgfscope}%
\pgfsys@transformshift{4.229679in}{3.585514in}%
\pgfsys@useobject{currentmarker}{}%
\end{pgfscope}%
\begin{pgfscope}%
\pgfsys@transformshift{4.652406in}{3.614268in}%
\pgfsys@useobject{currentmarker}{}%
\end{pgfscope}%
\begin{pgfscope}%
\pgfsys@transformshift{5.075133in}{3.837965in}%
\pgfsys@useobject{currentmarker}{}%
\end{pgfscope}%
\end{pgfscope}%
\begin{pgfscope}%
\pgfpathrectangle{\pgfqpoint{0.636497in}{0.955237in}}{\pgfqpoint{4.650000in}{3.020000in}}%
\pgfusepath{clip}%
\pgfsetbuttcap%
\pgfsetroundjoin%
\pgfsetlinewidth{1.505625pt}%
\definecolor{currentstroke}{rgb}{0.000000,0.000000,1.000000}%
\pgfsetstrokecolor{currentstroke}%
\pgfsetdash{{5.550000pt}{2.400000pt}}{0.000000pt}%
\pgfpathmoveto{\pgfqpoint{0.847861in}{3.293740in}}%
\pgfpathlineto{\pgfqpoint{1.270588in}{3.227712in}}%
\pgfpathlineto{\pgfqpoint{1.693315in}{3.357696in}}%
\pgfpathlineto{\pgfqpoint{2.116043in}{3.091616in}}%
\pgfpathlineto{\pgfqpoint{2.538770in}{3.117584in}}%
\pgfpathlineto{\pgfqpoint{2.961497in}{3.478322in}}%
\pgfpathlineto{\pgfqpoint{3.384224in}{3.439173in}}%
\pgfpathlineto{\pgfqpoint{3.806952in}{3.552030in}}%
\pgfpathlineto{\pgfqpoint{4.229679in}{3.811911in}}%
\pgfpathlineto{\pgfqpoint{4.652406in}{3.532477in}}%
\pgfpathlineto{\pgfqpoint{5.075133in}{3.462987in}}%
\pgfusepath{stroke}%
\end{pgfscope}%
\begin{pgfscope}%
\pgfpathrectangle{\pgfqpoint{0.636497in}{0.955237in}}{\pgfqpoint{4.650000in}{3.020000in}}%
\pgfusepath{clip}%
\pgfsetbuttcap%
\pgfsetroundjoin%
\definecolor{currentfill}{rgb}{0.000000,0.000000,1.000000}%
\pgfsetfillcolor{currentfill}%
\pgfsetlinewidth{1.003750pt}%
\definecolor{currentstroke}{rgb}{0.000000,0.000000,1.000000}%
\pgfsetstrokecolor{currentstroke}%
\pgfsetdash{}{0pt}%
\pgfsys@defobject{currentmarker}{\pgfqpoint{-0.041667in}{-0.041667in}}{\pgfqpoint{0.041667in}{0.041667in}}{%
\pgfpathmoveto{\pgfqpoint{-0.041667in}{-0.041667in}}%
\pgfpathlineto{\pgfqpoint{0.041667in}{0.041667in}}%
\pgfpathmoveto{\pgfqpoint{-0.041667in}{0.041667in}}%
\pgfpathlineto{\pgfqpoint{0.041667in}{-0.041667in}}%
\pgfusepath{stroke,fill}%
}%
\begin{pgfscope}%
\pgfsys@transformshift{0.847861in}{3.293740in}%
\pgfsys@useobject{currentmarker}{}%
\end{pgfscope}%
\begin{pgfscope}%
\pgfsys@transformshift{1.270588in}{3.227712in}%
\pgfsys@useobject{currentmarker}{}%
\end{pgfscope}%
\begin{pgfscope}%
\pgfsys@transformshift{1.693315in}{3.357696in}%
\pgfsys@useobject{currentmarker}{}%
\end{pgfscope}%
\begin{pgfscope}%
\pgfsys@transformshift{2.116043in}{3.091616in}%
\pgfsys@useobject{currentmarker}{}%
\end{pgfscope}%
\begin{pgfscope}%
\pgfsys@transformshift{2.538770in}{3.117584in}%
\pgfsys@useobject{currentmarker}{}%
\end{pgfscope}%
\begin{pgfscope}%
\pgfsys@transformshift{2.961497in}{3.478322in}%
\pgfsys@useobject{currentmarker}{}%
\end{pgfscope}%
\begin{pgfscope}%
\pgfsys@transformshift{3.384224in}{3.439173in}%
\pgfsys@useobject{currentmarker}{}%
\end{pgfscope}%
\begin{pgfscope}%
\pgfsys@transformshift{3.806952in}{3.552030in}%
\pgfsys@useobject{currentmarker}{}%
\end{pgfscope}%
\begin{pgfscope}%
\pgfsys@transformshift{4.229679in}{3.811911in}%
\pgfsys@useobject{currentmarker}{}%
\end{pgfscope}%
\begin{pgfscope}%
\pgfsys@transformshift{4.652406in}{3.532477in}%
\pgfsys@useobject{currentmarker}{}%
\end{pgfscope}%
\begin{pgfscope}%
\pgfsys@transformshift{5.075133in}{3.462987in}%
\pgfsys@useobject{currentmarker}{}%
\end{pgfscope}%
\end{pgfscope}%
\begin{pgfscope}%
\pgfsetrectcap%
\pgfsetmiterjoin%
\pgfsetlinewidth{0.803000pt}%
\definecolor{currentstroke}{rgb}{0.000000,0.000000,0.000000}%
\pgfsetstrokecolor{currentstroke}%
\pgfsetdash{}{0pt}%
\pgfpathmoveto{\pgfqpoint{0.636497in}{0.955237in}}%
\pgfpathlineto{\pgfqpoint{0.636497in}{3.975237in}}%
\pgfusepath{stroke}%
\end{pgfscope}%
\begin{pgfscope}%
\pgfsetrectcap%
\pgfsetmiterjoin%
\pgfsetlinewidth{0.803000pt}%
\definecolor{currentstroke}{rgb}{0.000000,0.000000,0.000000}%
\pgfsetstrokecolor{currentstroke}%
\pgfsetdash{}{0pt}%
\pgfpathmoveto{\pgfqpoint{5.286497in}{0.955237in}}%
\pgfpathlineto{\pgfqpoint{5.286497in}{3.975237in}}%
\pgfusepath{stroke}%
\end{pgfscope}%
\begin{pgfscope}%
\pgfsetrectcap%
\pgfsetmiterjoin%
\pgfsetlinewidth{0.803000pt}%
\definecolor{currentstroke}{rgb}{0.000000,0.000000,0.000000}%
\pgfsetstrokecolor{currentstroke}%
\pgfsetdash{}{0pt}%
\pgfpathmoveto{\pgfqpoint{0.636497in}{0.955237in}}%
\pgfpathlineto{\pgfqpoint{5.286497in}{0.955237in}}%
\pgfusepath{stroke}%
\end{pgfscope}%
\begin{pgfscope}%
\pgfsetrectcap%
\pgfsetmiterjoin%
\pgfsetlinewidth{0.803000pt}%
\definecolor{currentstroke}{rgb}{0.000000,0.000000,0.000000}%
\pgfsetstrokecolor{currentstroke}%
\pgfsetdash{}{0pt}%
\pgfpathmoveto{\pgfqpoint{0.636497in}{3.975237in}}%
\pgfpathlineto{\pgfqpoint{5.286497in}{3.975237in}}%
\pgfusepath{stroke}%
\end{pgfscope}%
\begin{pgfscope}%
\pgfsetbuttcap%
\pgfsetmiterjoin%
\definecolor{currentfill}{rgb}{0.300000,0.300000,0.300000}%
\pgfsetfillcolor{currentfill}%
\pgfsetfillopacity{0.500000}%
\pgfsetlinewidth{1.003750pt}%
\definecolor{currentstroke}{rgb}{0.300000,0.300000,0.300000}%
\pgfsetstrokecolor{currentstroke}%
\pgfsetstrokeopacity{0.500000}%
\pgfsetdash{}{0pt}%
\pgfpathmoveto{\pgfqpoint{1.040502in}{-0.027778in}}%
\pgfpathlineto{\pgfqpoint{4.938048in}{-0.027778in}}%
\pgfpathquadraticcurveto{\pgfqpoint{4.982493in}{-0.027778in}}{\pgfqpoint{4.982493in}{0.016667in}}%
\pgfpathlineto{\pgfqpoint{4.982493in}{0.318904in}}%
\pgfpathquadraticcurveto{\pgfqpoint{4.982493in}{0.363349in}}{\pgfqpoint{4.938048in}{0.363349in}}%
\pgfpathlineto{\pgfqpoint{1.040502in}{0.363349in}}%
\pgfpathquadraticcurveto{\pgfqpoint{0.996057in}{0.363349in}}{\pgfqpoint{0.996057in}{0.318904in}}%
\pgfpathlineto{\pgfqpoint{0.996057in}{0.016667in}}%
\pgfpathquadraticcurveto{\pgfqpoint{0.996057in}{-0.027778in}}{\pgfqpoint{1.040502in}{-0.027778in}}%
\pgfpathclose%
\pgfusepath{stroke,fill}%
\end{pgfscope}%
\begin{pgfscope}%
\pgfsetbuttcap%
\pgfsetmiterjoin%
\definecolor{currentfill}{rgb}{1.000000,1.000000,1.000000}%
\pgfsetfillcolor{currentfill}%
\pgfsetlinewidth{1.003750pt}%
\definecolor{currentstroke}{rgb}{0.800000,0.800000,0.800000}%
\pgfsetstrokecolor{currentstroke}%
\pgfsetdash{}{0pt}%
\pgfpathmoveto{\pgfqpoint{1.012724in}{0.000000in}}%
\pgfpathlineto{\pgfqpoint{4.910270in}{0.000000in}}%
\pgfpathquadraticcurveto{\pgfqpoint{4.954715in}{0.000000in}}{\pgfqpoint{4.954715in}{0.044444in}}%
\pgfpathlineto{\pgfqpoint{4.954715in}{0.346682in}}%
\pgfpathquadraticcurveto{\pgfqpoint{4.954715in}{0.391126in}}{\pgfqpoint{4.910270in}{0.391126in}}%
\pgfpathlineto{\pgfqpoint{1.012724in}{0.391126in}}%
\pgfpathquadraticcurveto{\pgfqpoint{0.968279in}{0.391126in}}{\pgfqpoint{0.968279in}{0.346682in}}%
\pgfpathlineto{\pgfqpoint{0.968279in}{0.044444in}}%
\pgfpathquadraticcurveto{\pgfqpoint{0.968279in}{0.000000in}}{\pgfqpoint{1.012724in}{0.000000in}}%
\pgfpathclose%
\pgfusepath{stroke,fill}%
\end{pgfscope}%
\begin{pgfscope}%
\pgfsetrectcap%
\pgfsetroundjoin%
\pgfsetlinewidth{1.505625pt}%
\definecolor{currentstroke}{rgb}{1.000000,0.000000,0.000000}%
\pgfsetstrokecolor{currentstroke}%
\pgfsetdash{}{0pt}%
\pgfpathmoveto{\pgfqpoint{1.057168in}{0.213349in}}%
\pgfpathlineto{\pgfqpoint{1.501613in}{0.213349in}}%
\pgfusepath{stroke}%
\end{pgfscope}%
\begin{pgfscope}%
\pgfsetbuttcap%
\pgfsetroundjoin%
\definecolor{currentfill}{rgb}{1.000000,0.000000,0.000000}%
\pgfsetfillcolor{currentfill}%
\pgfsetlinewidth{1.003750pt}%
\definecolor{currentstroke}{rgb}{1.000000,0.000000,0.000000}%
\pgfsetstrokecolor{currentstroke}%
\pgfsetdash{}{0pt}%
\pgfsys@defobject{currentmarker}{\pgfqpoint{-0.041667in}{-0.041667in}}{\pgfqpoint{0.041667in}{0.041667in}}{%
\pgfpathmoveto{\pgfqpoint{-0.041667in}{0.000000in}}%
\pgfpathlineto{\pgfqpoint{0.041667in}{0.000000in}}%
\pgfpathmoveto{\pgfqpoint{0.000000in}{-0.041667in}}%
\pgfpathlineto{\pgfqpoint{0.000000in}{0.041667in}}%
\pgfusepath{stroke,fill}%
}%
\begin{pgfscope}%
\pgfsys@transformshift{1.279390in}{0.213349in}%
\pgfsys@useobject{currentmarker}{}%
\end{pgfscope}%
\end{pgfscope}%
\begin{pgfscope}%
\definecolor{textcolor}{rgb}{0.000000,0.000000,0.000000}%
\pgfsetstrokecolor{textcolor}%
\pgfsetfillcolor{textcolor}%
\pgftext[x=1.679390in,y=0.135571in,left,base]{\color{textcolor}\rmfamily\fontsize{16.000000}{19.200000}\selectfont Cohabitation}%
\end{pgfscope}%
\begin{pgfscope}%
\pgfsetbuttcap%
\pgfsetroundjoin%
\pgfsetlinewidth{1.505625pt}%
\definecolor{currentstroke}{rgb}{0.000000,0.000000,1.000000}%
\pgfsetstrokecolor{currentstroke}%
\pgfsetdash{{5.550000pt}{2.400000pt}}{0.000000pt}%
\pgfpathmoveto{\pgfqpoint{3.381962in}{0.213349in}}%
\pgfpathlineto{\pgfqpoint{3.826406in}{0.213349in}}%
\pgfusepath{stroke}%
\end{pgfscope}%
\begin{pgfscope}%
\pgfsetbuttcap%
\pgfsetroundjoin%
\definecolor{currentfill}{rgb}{0.000000,0.000000,1.000000}%
\pgfsetfillcolor{currentfill}%
\pgfsetlinewidth{1.003750pt}%
\definecolor{currentstroke}{rgb}{0.000000,0.000000,1.000000}%
\pgfsetstrokecolor{currentstroke}%
\pgfsetdash{}{0pt}%
\pgfsys@defobject{currentmarker}{\pgfqpoint{-0.041667in}{-0.041667in}}{\pgfqpoint{0.041667in}{0.041667in}}{%
\pgfpathmoveto{\pgfqpoint{-0.041667in}{-0.041667in}}%
\pgfpathlineto{\pgfqpoint{0.041667in}{0.041667in}}%
\pgfpathmoveto{\pgfqpoint{-0.041667in}{0.041667in}}%
\pgfpathlineto{\pgfqpoint{0.041667in}{-0.041667in}}%
\pgfusepath{stroke,fill}%
}%
\begin{pgfscope}%
\pgfsys@transformshift{3.604184in}{0.213349in}%
\pgfsys@useobject{currentmarker}{}%
\end{pgfscope}%
\end{pgfscope}%
\begin{pgfscope}%
\definecolor{textcolor}{rgb}{0.000000,0.000000,0.000000}%
\pgfsetstrokecolor{textcolor}%
\pgfsetfillcolor{textcolor}%
\pgftext[x=4.004184in,y=0.135571in,left,base]{\color{textcolor}\rmfamily\fontsize{16.000000}{19.200000}\selectfont Marriage}%
\end{pgfscope}%
\end{pgfpicture}%
\makeatother%
\endgroup%
 } 
\end{subfigure}
\begin{subfigure}{.49\textwidth}
\centering
% include second image
\caption{Women Pareto Weight $\theta$ at meeting}
\label{sf:sub-second}
\scalebox{0.5}{%% Creator: Matplotlib, PGF backend
%%
%% To include the figure in your LaTeX document, write
%%   \input{<filename>.pgf}
%%
%% Make sure the required packages are loaded in your preamble
%%   \usepackage{pgf}
%%
%% Figures using additional raster images can only be included by \input if
%% they are in the same directory as the main LaTeX file. For loading figures
%% from other directories you can use the `import` package
%%   \usepackage{import}
%% and then include the figures with
%%   \import{<path to file>}{<filename>.pgf}
%%
%% Matplotlib used the following preamble
%%
\begingroup%
\makeatletter%
\begin{pgfpicture}%
\pgfpathrectangle{\pgfpointorigin}{\pgfqpoint{5.362539in}{3.975237in}}%
\pgfusepath{use as bounding box, clip}%
\begin{pgfscope}%
\pgfsetbuttcap%
\pgfsetmiterjoin%
\definecolor{currentfill}{rgb}{1.000000,1.000000,1.000000}%
\pgfsetfillcolor{currentfill}%
\pgfsetlinewidth{0.000000pt}%
\definecolor{currentstroke}{rgb}{1.000000,1.000000,1.000000}%
\pgfsetstrokecolor{currentstroke}%
\pgfsetdash{}{0pt}%
\pgfpathmoveto{\pgfqpoint{0.000000in}{0.000000in}}%
\pgfpathlineto{\pgfqpoint{5.362539in}{0.000000in}}%
\pgfpathlineto{\pgfqpoint{5.362539in}{3.975237in}}%
\pgfpathlineto{\pgfqpoint{0.000000in}{3.975237in}}%
\pgfpathclose%
\pgfusepath{fill}%
\end{pgfscope}%
\begin{pgfscope}%
\pgfsetbuttcap%
\pgfsetmiterjoin%
\definecolor{currentfill}{rgb}{1.000000,1.000000,1.000000}%
\pgfsetfillcolor{currentfill}%
\pgfsetlinewidth{0.000000pt}%
\definecolor{currentstroke}{rgb}{0.000000,0.000000,0.000000}%
\pgfsetstrokecolor{currentstroke}%
\pgfsetstrokeopacity{0.000000}%
\pgfsetdash{}{0pt}%
\pgfpathmoveto{\pgfqpoint{0.712539in}{0.955237in}}%
\pgfpathlineto{\pgfqpoint{5.362539in}{0.955237in}}%
\pgfpathlineto{\pgfqpoint{5.362539in}{3.975237in}}%
\pgfpathlineto{\pgfqpoint{0.712539in}{3.975237in}}%
\pgfpathclose%
\pgfusepath{fill}%
\end{pgfscope}%
\begin{pgfscope}%
\pgfsetbuttcap%
\pgfsetroundjoin%
\definecolor{currentfill}{rgb}{0.000000,0.000000,0.000000}%
\pgfsetfillcolor{currentfill}%
\pgfsetlinewidth{0.803000pt}%
\definecolor{currentstroke}{rgb}{0.000000,0.000000,0.000000}%
\pgfsetstrokecolor{currentstroke}%
\pgfsetdash{}{0pt}%
\pgfsys@defobject{currentmarker}{\pgfqpoint{0.000000in}{-0.048611in}}{\pgfqpoint{0.000000in}{0.000000in}}{%
\pgfpathmoveto{\pgfqpoint{0.000000in}{0.000000in}}%
\pgfpathlineto{\pgfqpoint{0.000000in}{-0.048611in}}%
\pgfusepath{stroke,fill}%
}%
\begin{pgfscope}%
\pgfsys@transformshift{1.346630in}{0.955237in}%
\pgfsys@useobject{currentmarker}{}%
\end{pgfscope}%
\end{pgfscope}%
\begin{pgfscope}%
\definecolor{textcolor}{rgb}{0.000000,0.000000,0.000000}%
\pgfsetstrokecolor{textcolor}%
\pgfsetfillcolor{textcolor}%
\pgftext[x=1.346630in,y=0.858015in,,top]{\color{textcolor}\rmfamily\fontsize{11.000000}{13.200000}\selectfont \(\displaystyle -4\)}%
\end{pgfscope}%
\begin{pgfscope}%
\pgfsetbuttcap%
\pgfsetroundjoin%
\definecolor{currentfill}{rgb}{0.000000,0.000000,0.000000}%
\pgfsetfillcolor{currentfill}%
\pgfsetlinewidth{0.803000pt}%
\definecolor{currentstroke}{rgb}{0.000000,0.000000,0.000000}%
\pgfsetstrokecolor{currentstroke}%
\pgfsetdash{}{0pt}%
\pgfsys@defobject{currentmarker}{\pgfqpoint{0.000000in}{-0.048611in}}{\pgfqpoint{0.000000in}{0.000000in}}{%
\pgfpathmoveto{\pgfqpoint{0.000000in}{0.000000in}}%
\pgfpathlineto{\pgfqpoint{0.000000in}{-0.048611in}}%
\pgfusepath{stroke,fill}%
}%
\begin{pgfscope}%
\pgfsys@transformshift{2.192084in}{0.955237in}%
\pgfsys@useobject{currentmarker}{}%
\end{pgfscope}%
\end{pgfscope}%
\begin{pgfscope}%
\definecolor{textcolor}{rgb}{0.000000,0.000000,0.000000}%
\pgfsetstrokecolor{textcolor}%
\pgfsetfillcolor{textcolor}%
\pgftext[x=2.192084in,y=0.858015in,,top]{\color{textcolor}\rmfamily\fontsize{11.000000}{13.200000}\selectfont \(\displaystyle -2\)}%
\end{pgfscope}%
\begin{pgfscope}%
\pgfsetbuttcap%
\pgfsetroundjoin%
\definecolor{currentfill}{rgb}{0.000000,0.000000,0.000000}%
\pgfsetfillcolor{currentfill}%
\pgfsetlinewidth{0.803000pt}%
\definecolor{currentstroke}{rgb}{0.000000,0.000000,0.000000}%
\pgfsetstrokecolor{currentstroke}%
\pgfsetdash{}{0pt}%
\pgfsys@defobject{currentmarker}{\pgfqpoint{0.000000in}{-0.048611in}}{\pgfqpoint{0.000000in}{0.000000in}}{%
\pgfpathmoveto{\pgfqpoint{0.000000in}{0.000000in}}%
\pgfpathlineto{\pgfqpoint{0.000000in}{-0.048611in}}%
\pgfusepath{stroke,fill}%
}%
\begin{pgfscope}%
\pgfsys@transformshift{3.037539in}{0.955237in}%
\pgfsys@useobject{currentmarker}{}%
\end{pgfscope}%
\end{pgfscope}%
\begin{pgfscope}%
\definecolor{textcolor}{rgb}{0.000000,0.000000,0.000000}%
\pgfsetstrokecolor{textcolor}%
\pgfsetfillcolor{textcolor}%
\pgftext[x=3.037539in,y=0.858015in,,top]{\color{textcolor}\rmfamily\fontsize{11.000000}{13.200000}\selectfont \(\displaystyle 0\)}%
\end{pgfscope}%
\begin{pgfscope}%
\pgfsetbuttcap%
\pgfsetroundjoin%
\definecolor{currentfill}{rgb}{0.000000,0.000000,0.000000}%
\pgfsetfillcolor{currentfill}%
\pgfsetlinewidth{0.803000pt}%
\definecolor{currentstroke}{rgb}{0.000000,0.000000,0.000000}%
\pgfsetstrokecolor{currentstroke}%
\pgfsetdash{}{0pt}%
\pgfsys@defobject{currentmarker}{\pgfqpoint{0.000000in}{-0.048611in}}{\pgfqpoint{0.000000in}{0.000000in}}{%
\pgfpathmoveto{\pgfqpoint{0.000000in}{0.000000in}}%
\pgfpathlineto{\pgfqpoint{0.000000in}{-0.048611in}}%
\pgfusepath{stroke,fill}%
}%
\begin{pgfscope}%
\pgfsys@transformshift{3.882993in}{0.955237in}%
\pgfsys@useobject{currentmarker}{}%
\end{pgfscope}%
\end{pgfscope}%
\begin{pgfscope}%
\definecolor{textcolor}{rgb}{0.000000,0.000000,0.000000}%
\pgfsetstrokecolor{textcolor}%
\pgfsetfillcolor{textcolor}%
\pgftext[x=3.882993in,y=0.858015in,,top]{\color{textcolor}\rmfamily\fontsize{11.000000}{13.200000}\selectfont \(\displaystyle 2\)}%
\end{pgfscope}%
\begin{pgfscope}%
\pgfsetbuttcap%
\pgfsetroundjoin%
\definecolor{currentfill}{rgb}{0.000000,0.000000,0.000000}%
\pgfsetfillcolor{currentfill}%
\pgfsetlinewidth{0.803000pt}%
\definecolor{currentstroke}{rgb}{0.000000,0.000000,0.000000}%
\pgfsetstrokecolor{currentstroke}%
\pgfsetdash{}{0pt}%
\pgfsys@defobject{currentmarker}{\pgfqpoint{0.000000in}{-0.048611in}}{\pgfqpoint{0.000000in}{0.000000in}}{%
\pgfpathmoveto{\pgfqpoint{0.000000in}{0.000000in}}%
\pgfpathlineto{\pgfqpoint{0.000000in}{-0.048611in}}%
\pgfusepath{stroke,fill}%
}%
\begin{pgfscope}%
\pgfsys@transformshift{4.728448in}{0.955237in}%
\pgfsys@useobject{currentmarker}{}%
\end{pgfscope}%
\end{pgfscope}%
\begin{pgfscope}%
\definecolor{textcolor}{rgb}{0.000000,0.000000,0.000000}%
\pgfsetstrokecolor{textcolor}%
\pgfsetfillcolor{textcolor}%
\pgftext[x=4.728448in,y=0.858015in,,top]{\color{textcolor}\rmfamily\fontsize{11.000000}{13.200000}\selectfont \(\displaystyle 4\)}%
\end{pgfscope}%
\begin{pgfscope}%
\definecolor{textcolor}{rgb}{0.000000,0.000000,0.000000}%
\pgfsetstrokecolor{textcolor}%
\pgfsetfillcolor{textcolor}%
\pgftext[x=3.037539in,y=0.667275in,,top]{\color{textcolor}\rmfamily\fontsize{16.000000}{19.200000}\selectfont Event time (model years)}%
\end{pgfscope}%
\begin{pgfscope}%
\pgfsetbuttcap%
\pgfsetroundjoin%
\definecolor{currentfill}{rgb}{0.000000,0.000000,0.000000}%
\pgfsetfillcolor{currentfill}%
\pgfsetlinewidth{0.803000pt}%
\definecolor{currentstroke}{rgb}{0.000000,0.000000,0.000000}%
\pgfsetstrokecolor{currentstroke}%
\pgfsetdash{}{0pt}%
\pgfsys@defobject{currentmarker}{\pgfqpoint{-0.048611in}{0.000000in}}{\pgfqpoint{0.000000in}{0.000000in}}{%
\pgfpathmoveto{\pgfqpoint{0.000000in}{0.000000in}}%
\pgfpathlineto{\pgfqpoint{-0.048611in}{0.000000in}}%
\pgfusepath{stroke,fill}%
}%
\begin{pgfscope}%
\pgfsys@transformshift{0.712539in}{0.959343in}%
\pgfsys@useobject{currentmarker}{}%
\end{pgfscope}%
\end{pgfscope}%
\begin{pgfscope}%
\definecolor{textcolor}{rgb}{0.000000,0.000000,0.000000}%
\pgfsetstrokecolor{textcolor}%
\pgfsetfillcolor{textcolor}%
\pgftext[x=0.268904in,y=0.906536in,left,base]{\color{textcolor}\rmfamily\fontsize{11.000000}{13.200000}\selectfont \(\displaystyle 0.250\)}%
\end{pgfscope}%
\begin{pgfscope}%
\pgfsetbuttcap%
\pgfsetroundjoin%
\definecolor{currentfill}{rgb}{0.000000,0.000000,0.000000}%
\pgfsetfillcolor{currentfill}%
\pgfsetlinewidth{0.803000pt}%
\definecolor{currentstroke}{rgb}{0.000000,0.000000,0.000000}%
\pgfsetstrokecolor{currentstroke}%
\pgfsetdash{}{0pt}%
\pgfsys@defobject{currentmarker}{\pgfqpoint{-0.048611in}{0.000000in}}{\pgfqpoint{0.000000in}{0.000000in}}{%
\pgfpathmoveto{\pgfqpoint{0.000000in}{0.000000in}}%
\pgfpathlineto{\pgfqpoint{-0.048611in}{0.000000in}}%
\pgfusepath{stroke,fill}%
}%
\begin{pgfscope}%
\pgfsys@transformshift{0.712539in}{1.346445in}%
\pgfsys@useobject{currentmarker}{}%
\end{pgfscope}%
\end{pgfscope}%
\begin{pgfscope}%
\definecolor{textcolor}{rgb}{0.000000,0.000000,0.000000}%
\pgfsetstrokecolor{textcolor}%
\pgfsetfillcolor{textcolor}%
\pgftext[x=0.268904in,y=1.293638in,left,base]{\color{textcolor}\rmfamily\fontsize{11.000000}{13.200000}\selectfont \(\displaystyle 0.275\)}%
\end{pgfscope}%
\begin{pgfscope}%
\pgfsetbuttcap%
\pgfsetroundjoin%
\definecolor{currentfill}{rgb}{0.000000,0.000000,0.000000}%
\pgfsetfillcolor{currentfill}%
\pgfsetlinewidth{0.803000pt}%
\definecolor{currentstroke}{rgb}{0.000000,0.000000,0.000000}%
\pgfsetstrokecolor{currentstroke}%
\pgfsetdash{}{0pt}%
\pgfsys@defobject{currentmarker}{\pgfqpoint{-0.048611in}{0.000000in}}{\pgfqpoint{0.000000in}{0.000000in}}{%
\pgfpathmoveto{\pgfqpoint{0.000000in}{0.000000in}}%
\pgfpathlineto{\pgfqpoint{-0.048611in}{0.000000in}}%
\pgfusepath{stroke,fill}%
}%
\begin{pgfscope}%
\pgfsys@transformshift{0.712539in}{1.733546in}%
\pgfsys@useobject{currentmarker}{}%
\end{pgfscope}%
\end{pgfscope}%
\begin{pgfscope}%
\definecolor{textcolor}{rgb}{0.000000,0.000000,0.000000}%
\pgfsetstrokecolor{textcolor}%
\pgfsetfillcolor{textcolor}%
\pgftext[x=0.268904in,y=1.680740in,left,base]{\color{textcolor}\rmfamily\fontsize{11.000000}{13.200000}\selectfont \(\displaystyle 0.300\)}%
\end{pgfscope}%
\begin{pgfscope}%
\pgfsetbuttcap%
\pgfsetroundjoin%
\definecolor{currentfill}{rgb}{0.000000,0.000000,0.000000}%
\pgfsetfillcolor{currentfill}%
\pgfsetlinewidth{0.803000pt}%
\definecolor{currentstroke}{rgb}{0.000000,0.000000,0.000000}%
\pgfsetstrokecolor{currentstroke}%
\pgfsetdash{}{0pt}%
\pgfsys@defobject{currentmarker}{\pgfqpoint{-0.048611in}{0.000000in}}{\pgfqpoint{0.000000in}{0.000000in}}{%
\pgfpathmoveto{\pgfqpoint{0.000000in}{0.000000in}}%
\pgfpathlineto{\pgfqpoint{-0.048611in}{0.000000in}}%
\pgfusepath{stroke,fill}%
}%
\begin{pgfscope}%
\pgfsys@transformshift{0.712539in}{2.120648in}%
\pgfsys@useobject{currentmarker}{}%
\end{pgfscope}%
\end{pgfscope}%
\begin{pgfscope}%
\definecolor{textcolor}{rgb}{0.000000,0.000000,0.000000}%
\pgfsetstrokecolor{textcolor}%
\pgfsetfillcolor{textcolor}%
\pgftext[x=0.268904in,y=2.067841in,left,base]{\color{textcolor}\rmfamily\fontsize{11.000000}{13.200000}\selectfont \(\displaystyle 0.325\)}%
\end{pgfscope}%
\begin{pgfscope}%
\pgfsetbuttcap%
\pgfsetroundjoin%
\definecolor{currentfill}{rgb}{0.000000,0.000000,0.000000}%
\pgfsetfillcolor{currentfill}%
\pgfsetlinewidth{0.803000pt}%
\definecolor{currentstroke}{rgb}{0.000000,0.000000,0.000000}%
\pgfsetstrokecolor{currentstroke}%
\pgfsetdash{}{0pt}%
\pgfsys@defobject{currentmarker}{\pgfqpoint{-0.048611in}{0.000000in}}{\pgfqpoint{0.000000in}{0.000000in}}{%
\pgfpathmoveto{\pgfqpoint{0.000000in}{0.000000in}}%
\pgfpathlineto{\pgfqpoint{-0.048611in}{0.000000in}}%
\pgfusepath{stroke,fill}%
}%
\begin{pgfscope}%
\pgfsys@transformshift{0.712539in}{2.507749in}%
\pgfsys@useobject{currentmarker}{}%
\end{pgfscope}%
\end{pgfscope}%
\begin{pgfscope}%
\definecolor{textcolor}{rgb}{0.000000,0.000000,0.000000}%
\pgfsetstrokecolor{textcolor}%
\pgfsetfillcolor{textcolor}%
\pgftext[x=0.268904in,y=2.454943in,left,base]{\color{textcolor}\rmfamily\fontsize{11.000000}{13.200000}\selectfont \(\displaystyle 0.350\)}%
\end{pgfscope}%
\begin{pgfscope}%
\pgfsetbuttcap%
\pgfsetroundjoin%
\definecolor{currentfill}{rgb}{0.000000,0.000000,0.000000}%
\pgfsetfillcolor{currentfill}%
\pgfsetlinewidth{0.803000pt}%
\definecolor{currentstroke}{rgb}{0.000000,0.000000,0.000000}%
\pgfsetstrokecolor{currentstroke}%
\pgfsetdash{}{0pt}%
\pgfsys@defobject{currentmarker}{\pgfqpoint{-0.048611in}{0.000000in}}{\pgfqpoint{0.000000in}{0.000000in}}{%
\pgfpathmoveto{\pgfqpoint{0.000000in}{0.000000in}}%
\pgfpathlineto{\pgfqpoint{-0.048611in}{0.000000in}}%
\pgfusepath{stroke,fill}%
}%
\begin{pgfscope}%
\pgfsys@transformshift{0.712539in}{2.894851in}%
\pgfsys@useobject{currentmarker}{}%
\end{pgfscope}%
\end{pgfscope}%
\begin{pgfscope}%
\definecolor{textcolor}{rgb}{0.000000,0.000000,0.000000}%
\pgfsetstrokecolor{textcolor}%
\pgfsetfillcolor{textcolor}%
\pgftext[x=0.268904in,y=2.842044in,left,base]{\color{textcolor}\rmfamily\fontsize{11.000000}{13.200000}\selectfont \(\displaystyle 0.375\)}%
\end{pgfscope}%
\begin{pgfscope}%
\pgfsetbuttcap%
\pgfsetroundjoin%
\definecolor{currentfill}{rgb}{0.000000,0.000000,0.000000}%
\pgfsetfillcolor{currentfill}%
\pgfsetlinewidth{0.803000pt}%
\definecolor{currentstroke}{rgb}{0.000000,0.000000,0.000000}%
\pgfsetstrokecolor{currentstroke}%
\pgfsetdash{}{0pt}%
\pgfsys@defobject{currentmarker}{\pgfqpoint{-0.048611in}{0.000000in}}{\pgfqpoint{0.000000in}{0.000000in}}{%
\pgfpathmoveto{\pgfqpoint{0.000000in}{0.000000in}}%
\pgfpathlineto{\pgfqpoint{-0.048611in}{0.000000in}}%
\pgfusepath{stroke,fill}%
}%
\begin{pgfscope}%
\pgfsys@transformshift{0.712539in}{3.281953in}%
\pgfsys@useobject{currentmarker}{}%
\end{pgfscope}%
\end{pgfscope}%
\begin{pgfscope}%
\definecolor{textcolor}{rgb}{0.000000,0.000000,0.000000}%
\pgfsetstrokecolor{textcolor}%
\pgfsetfillcolor{textcolor}%
\pgftext[x=0.268904in,y=3.229146in,left,base]{\color{textcolor}\rmfamily\fontsize{11.000000}{13.200000}\selectfont \(\displaystyle 0.400\)}%
\end{pgfscope}%
\begin{pgfscope}%
\pgfsetbuttcap%
\pgfsetroundjoin%
\definecolor{currentfill}{rgb}{0.000000,0.000000,0.000000}%
\pgfsetfillcolor{currentfill}%
\pgfsetlinewidth{0.803000pt}%
\definecolor{currentstroke}{rgb}{0.000000,0.000000,0.000000}%
\pgfsetstrokecolor{currentstroke}%
\pgfsetdash{}{0pt}%
\pgfsys@defobject{currentmarker}{\pgfqpoint{-0.048611in}{0.000000in}}{\pgfqpoint{0.000000in}{0.000000in}}{%
\pgfpathmoveto{\pgfqpoint{0.000000in}{0.000000in}}%
\pgfpathlineto{\pgfqpoint{-0.048611in}{0.000000in}}%
\pgfusepath{stroke,fill}%
}%
\begin{pgfscope}%
\pgfsys@transformshift{0.712539in}{3.669054in}%
\pgfsys@useobject{currentmarker}{}%
\end{pgfscope}%
\end{pgfscope}%
\begin{pgfscope}%
\definecolor{textcolor}{rgb}{0.000000,0.000000,0.000000}%
\pgfsetstrokecolor{textcolor}%
\pgfsetfillcolor{textcolor}%
\pgftext[x=0.268904in,y=3.616248in,left,base]{\color{textcolor}\rmfamily\fontsize{11.000000}{13.200000}\selectfont \(\displaystyle 0.425\)}%
\end{pgfscope}%
\begin{pgfscope}%
\definecolor{textcolor}{rgb}{0.000000,0.000000,0.000000}%
\pgfsetstrokecolor{textcolor}%
\pgfsetfillcolor{textcolor}%
\pgftext[x=0.213349in,y=2.465237in,,bottom,rotate=90.000000]{\color{textcolor}\rmfamily\fontsize{16.000000}{19.200000}\selectfont \(\displaystyle \theta\)---variation from baseline}%
\end{pgfscope}%
\begin{pgfscope}%
\pgfpathrectangle{\pgfqpoint{0.712539in}{0.955237in}}{\pgfqpoint{4.650000in}{3.020000in}}%
\pgfusepath{clip}%
\pgfsetrectcap%
\pgfsetroundjoin%
\pgfsetlinewidth{1.505625pt}%
\definecolor{currentstroke}{rgb}{1.000000,0.000000,0.000000}%
\pgfsetstrokecolor{currentstroke}%
\pgfsetdash{}{0pt}%
\pgfpathmoveto{\pgfqpoint{0.923903in}{1.285591in}}%
\pgfpathlineto{\pgfqpoint{1.346630in}{1.337023in}}%
\pgfpathlineto{\pgfqpoint{1.769357in}{1.139057in}}%
\pgfpathlineto{\pgfqpoint{2.192084in}{1.174817in}}%
\pgfpathlineto{\pgfqpoint{2.614812in}{1.092510in}}%
\pgfpathlineto{\pgfqpoint{3.037539in}{2.008171in}}%
\pgfpathlineto{\pgfqpoint{3.460266in}{2.102530in}}%
\pgfpathlineto{\pgfqpoint{3.882993in}{1.921258in}}%
\pgfpathlineto{\pgfqpoint{4.305721in}{2.083921in}}%
\pgfpathlineto{\pgfqpoint{4.728448in}{2.054167in}}%
\pgfpathlineto{\pgfqpoint{5.151175in}{2.182837in}}%
\pgfusepath{stroke}%
\end{pgfscope}%
\begin{pgfscope}%
\pgfpathrectangle{\pgfqpoint{0.712539in}{0.955237in}}{\pgfqpoint{4.650000in}{3.020000in}}%
\pgfusepath{clip}%
\pgfsetbuttcap%
\pgfsetroundjoin%
\definecolor{currentfill}{rgb}{1.000000,0.000000,0.000000}%
\pgfsetfillcolor{currentfill}%
\pgfsetlinewidth{1.003750pt}%
\definecolor{currentstroke}{rgb}{1.000000,0.000000,0.000000}%
\pgfsetstrokecolor{currentstroke}%
\pgfsetdash{}{0pt}%
\pgfsys@defobject{currentmarker}{\pgfqpoint{-0.041667in}{-0.041667in}}{\pgfqpoint{0.041667in}{0.041667in}}{%
\pgfpathmoveto{\pgfqpoint{-0.041667in}{0.000000in}}%
\pgfpathlineto{\pgfqpoint{0.041667in}{0.000000in}}%
\pgfpathmoveto{\pgfqpoint{0.000000in}{-0.041667in}}%
\pgfpathlineto{\pgfqpoint{0.000000in}{0.041667in}}%
\pgfusepath{stroke,fill}%
}%
\begin{pgfscope}%
\pgfsys@transformshift{0.923903in}{1.285591in}%
\pgfsys@useobject{currentmarker}{}%
\end{pgfscope}%
\begin{pgfscope}%
\pgfsys@transformshift{1.346630in}{1.337023in}%
\pgfsys@useobject{currentmarker}{}%
\end{pgfscope}%
\begin{pgfscope}%
\pgfsys@transformshift{1.769357in}{1.139057in}%
\pgfsys@useobject{currentmarker}{}%
\end{pgfscope}%
\begin{pgfscope}%
\pgfsys@transformshift{2.192084in}{1.174817in}%
\pgfsys@useobject{currentmarker}{}%
\end{pgfscope}%
\begin{pgfscope}%
\pgfsys@transformshift{2.614812in}{1.092510in}%
\pgfsys@useobject{currentmarker}{}%
\end{pgfscope}%
\begin{pgfscope}%
\pgfsys@transformshift{3.037539in}{2.008171in}%
\pgfsys@useobject{currentmarker}{}%
\end{pgfscope}%
\begin{pgfscope}%
\pgfsys@transformshift{3.460266in}{2.102530in}%
\pgfsys@useobject{currentmarker}{}%
\end{pgfscope}%
\begin{pgfscope}%
\pgfsys@transformshift{3.882993in}{1.921258in}%
\pgfsys@useobject{currentmarker}{}%
\end{pgfscope}%
\begin{pgfscope}%
\pgfsys@transformshift{4.305721in}{2.083921in}%
\pgfsys@useobject{currentmarker}{}%
\end{pgfscope}%
\begin{pgfscope}%
\pgfsys@transformshift{4.728448in}{2.054167in}%
\pgfsys@useobject{currentmarker}{}%
\end{pgfscope}%
\begin{pgfscope}%
\pgfsys@transformshift{5.151175in}{2.182837in}%
\pgfsys@useobject{currentmarker}{}%
\end{pgfscope}%
\end{pgfscope}%
\begin{pgfscope}%
\pgfpathrectangle{\pgfqpoint{0.712539in}{0.955237in}}{\pgfqpoint{4.650000in}{3.020000in}}%
\pgfusepath{clip}%
\pgfsetbuttcap%
\pgfsetroundjoin%
\pgfsetlinewidth{1.505625pt}%
\definecolor{currentstroke}{rgb}{0.000000,0.000000,1.000000}%
\pgfsetstrokecolor{currentstroke}%
\pgfsetdash{{5.550000pt}{2.400000pt}}{0.000000pt}%
\pgfpathmoveto{\pgfqpoint{0.923903in}{3.344787in}}%
\pgfpathlineto{\pgfqpoint{1.346630in}{3.344280in}}%
\pgfpathlineto{\pgfqpoint{1.769357in}{3.281604in}}%
\pgfpathlineto{\pgfqpoint{2.192084in}{3.335825in}}%
\pgfpathlineto{\pgfqpoint{2.614812in}{3.316460in}}%
\pgfpathlineto{\pgfqpoint{3.037539in}{3.603068in}}%
\pgfpathlineto{\pgfqpoint{3.460266in}{3.574834in}}%
\pgfpathlineto{\pgfqpoint{3.882993in}{3.681118in}}%
\pgfpathlineto{\pgfqpoint{4.305721in}{3.837965in}}%
\pgfpathlineto{\pgfqpoint{4.728448in}{3.721378in}}%
\pgfpathlineto{\pgfqpoint{5.151175in}{3.569668in}}%
\pgfusepath{stroke}%
\end{pgfscope}%
\begin{pgfscope}%
\pgfpathrectangle{\pgfqpoint{0.712539in}{0.955237in}}{\pgfqpoint{4.650000in}{3.020000in}}%
\pgfusepath{clip}%
\pgfsetbuttcap%
\pgfsetroundjoin%
\definecolor{currentfill}{rgb}{0.000000,0.000000,1.000000}%
\pgfsetfillcolor{currentfill}%
\pgfsetlinewidth{1.003750pt}%
\definecolor{currentstroke}{rgb}{0.000000,0.000000,1.000000}%
\pgfsetstrokecolor{currentstroke}%
\pgfsetdash{}{0pt}%
\pgfsys@defobject{currentmarker}{\pgfqpoint{-0.041667in}{-0.041667in}}{\pgfqpoint{0.041667in}{0.041667in}}{%
\pgfpathmoveto{\pgfqpoint{-0.041667in}{-0.041667in}}%
\pgfpathlineto{\pgfqpoint{0.041667in}{0.041667in}}%
\pgfpathmoveto{\pgfqpoint{-0.041667in}{0.041667in}}%
\pgfpathlineto{\pgfqpoint{0.041667in}{-0.041667in}}%
\pgfusepath{stroke,fill}%
}%
\begin{pgfscope}%
\pgfsys@transformshift{0.923903in}{3.344787in}%
\pgfsys@useobject{currentmarker}{}%
\end{pgfscope}%
\begin{pgfscope}%
\pgfsys@transformshift{1.346630in}{3.344280in}%
\pgfsys@useobject{currentmarker}{}%
\end{pgfscope}%
\begin{pgfscope}%
\pgfsys@transformshift{1.769357in}{3.281604in}%
\pgfsys@useobject{currentmarker}{}%
\end{pgfscope}%
\begin{pgfscope}%
\pgfsys@transformshift{2.192084in}{3.335825in}%
\pgfsys@useobject{currentmarker}{}%
\end{pgfscope}%
\begin{pgfscope}%
\pgfsys@transformshift{2.614812in}{3.316460in}%
\pgfsys@useobject{currentmarker}{}%
\end{pgfscope}%
\begin{pgfscope}%
\pgfsys@transformshift{3.037539in}{3.603068in}%
\pgfsys@useobject{currentmarker}{}%
\end{pgfscope}%
\begin{pgfscope}%
\pgfsys@transformshift{3.460266in}{3.574834in}%
\pgfsys@useobject{currentmarker}{}%
\end{pgfscope}%
\begin{pgfscope}%
\pgfsys@transformshift{3.882993in}{3.681118in}%
\pgfsys@useobject{currentmarker}{}%
\end{pgfscope}%
\begin{pgfscope}%
\pgfsys@transformshift{4.305721in}{3.837965in}%
\pgfsys@useobject{currentmarker}{}%
\end{pgfscope}%
\begin{pgfscope}%
\pgfsys@transformshift{4.728448in}{3.721378in}%
\pgfsys@useobject{currentmarker}{}%
\end{pgfscope}%
\begin{pgfscope}%
\pgfsys@transformshift{5.151175in}{3.569668in}%
\pgfsys@useobject{currentmarker}{}%
\end{pgfscope}%
\end{pgfscope}%
\begin{pgfscope}%
\pgfsetrectcap%
\pgfsetmiterjoin%
\pgfsetlinewidth{0.803000pt}%
\definecolor{currentstroke}{rgb}{0.000000,0.000000,0.000000}%
\pgfsetstrokecolor{currentstroke}%
\pgfsetdash{}{0pt}%
\pgfpathmoveto{\pgfqpoint{0.712539in}{0.955238in}}%
\pgfpathlineto{\pgfqpoint{0.712539in}{3.975237in}}%
\pgfusepath{stroke}%
\end{pgfscope}%
\begin{pgfscope}%
\pgfsetrectcap%
\pgfsetmiterjoin%
\pgfsetlinewidth{0.803000pt}%
\definecolor{currentstroke}{rgb}{0.000000,0.000000,0.000000}%
\pgfsetstrokecolor{currentstroke}%
\pgfsetdash{}{0pt}%
\pgfpathmoveto{\pgfqpoint{5.362539in}{0.955238in}}%
\pgfpathlineto{\pgfqpoint{5.362539in}{3.975237in}}%
\pgfusepath{stroke}%
\end{pgfscope}%
\begin{pgfscope}%
\pgfsetrectcap%
\pgfsetmiterjoin%
\pgfsetlinewidth{0.803000pt}%
\definecolor{currentstroke}{rgb}{0.000000,0.000000,0.000000}%
\pgfsetstrokecolor{currentstroke}%
\pgfsetdash{}{0pt}%
\pgfpathmoveto{\pgfqpoint{0.712539in}{0.955237in}}%
\pgfpathlineto{\pgfqpoint{5.362539in}{0.955237in}}%
\pgfusepath{stroke}%
\end{pgfscope}%
\begin{pgfscope}%
\pgfsetrectcap%
\pgfsetmiterjoin%
\pgfsetlinewidth{0.803000pt}%
\definecolor{currentstroke}{rgb}{0.000000,0.000000,0.000000}%
\pgfsetstrokecolor{currentstroke}%
\pgfsetdash{}{0pt}%
\pgfpathmoveto{\pgfqpoint{0.712539in}{3.975237in}}%
\pgfpathlineto{\pgfqpoint{5.362539in}{3.975237in}}%
\pgfusepath{stroke}%
\end{pgfscope}%
\begin{pgfscope}%
\pgfsetbuttcap%
\pgfsetmiterjoin%
\definecolor{currentfill}{rgb}{0.300000,0.300000,0.300000}%
\pgfsetfillcolor{currentfill}%
\pgfsetfillopacity{0.500000}%
\pgfsetlinewidth{1.003750pt}%
\definecolor{currentstroke}{rgb}{0.300000,0.300000,0.300000}%
\pgfsetstrokecolor{currentstroke}%
\pgfsetstrokeopacity{0.500000}%
\pgfsetdash{}{0pt}%
\pgfpathmoveto{\pgfqpoint{1.116543in}{-0.027778in}}%
\pgfpathlineto{\pgfqpoint{5.014090in}{-0.027778in}}%
\pgfpathquadraticcurveto{\pgfqpoint{5.058534in}{-0.027778in}}{\pgfqpoint{5.058534in}{0.016667in}}%
\pgfpathlineto{\pgfqpoint{5.058534in}{0.318904in}}%
\pgfpathquadraticcurveto{\pgfqpoint{5.058534in}{0.363349in}}{\pgfqpoint{5.014090in}{0.363349in}}%
\pgfpathlineto{\pgfqpoint{1.116543in}{0.363349in}}%
\pgfpathquadraticcurveto{\pgfqpoint{1.072099in}{0.363349in}}{\pgfqpoint{1.072099in}{0.318904in}}%
\pgfpathlineto{\pgfqpoint{1.072099in}{0.016667in}}%
\pgfpathquadraticcurveto{\pgfqpoint{1.072099in}{-0.027778in}}{\pgfqpoint{1.116543in}{-0.027778in}}%
\pgfpathclose%
\pgfusepath{stroke,fill}%
\end{pgfscope}%
\begin{pgfscope}%
\pgfsetbuttcap%
\pgfsetmiterjoin%
\definecolor{currentfill}{rgb}{1.000000,1.000000,1.000000}%
\pgfsetfillcolor{currentfill}%
\pgfsetlinewidth{1.003750pt}%
\definecolor{currentstroke}{rgb}{0.800000,0.800000,0.800000}%
\pgfsetstrokecolor{currentstroke}%
\pgfsetdash{}{0pt}%
\pgfpathmoveto{\pgfqpoint{1.088766in}{0.000000in}}%
\pgfpathlineto{\pgfqpoint{4.986312in}{0.000000in}}%
\pgfpathquadraticcurveto{\pgfqpoint{5.030757in}{0.000000in}}{\pgfqpoint{5.030757in}{0.044444in}}%
\pgfpathlineto{\pgfqpoint{5.030757in}{0.346682in}}%
\pgfpathquadraticcurveto{\pgfqpoint{5.030757in}{0.391126in}}{\pgfqpoint{4.986312in}{0.391126in}}%
\pgfpathlineto{\pgfqpoint{1.088766in}{0.391126in}}%
\pgfpathquadraticcurveto{\pgfqpoint{1.044321in}{0.391126in}}{\pgfqpoint{1.044321in}{0.346682in}}%
\pgfpathlineto{\pgfqpoint{1.044321in}{0.044444in}}%
\pgfpathquadraticcurveto{\pgfqpoint{1.044321in}{0.000000in}}{\pgfqpoint{1.088766in}{0.000000in}}%
\pgfpathclose%
\pgfusepath{stroke,fill}%
\end{pgfscope}%
\begin{pgfscope}%
\pgfsetrectcap%
\pgfsetroundjoin%
\pgfsetlinewidth{1.505625pt}%
\definecolor{currentstroke}{rgb}{1.000000,0.000000,0.000000}%
\pgfsetstrokecolor{currentstroke}%
\pgfsetdash{}{0pt}%
\pgfpathmoveto{\pgfqpoint{1.133210in}{0.213349in}}%
\pgfpathlineto{\pgfqpoint{1.577655in}{0.213349in}}%
\pgfusepath{stroke}%
\end{pgfscope}%
\begin{pgfscope}%
\pgfsetbuttcap%
\pgfsetroundjoin%
\definecolor{currentfill}{rgb}{1.000000,0.000000,0.000000}%
\pgfsetfillcolor{currentfill}%
\pgfsetlinewidth{1.003750pt}%
\definecolor{currentstroke}{rgb}{1.000000,0.000000,0.000000}%
\pgfsetstrokecolor{currentstroke}%
\pgfsetdash{}{0pt}%
\pgfsys@defobject{currentmarker}{\pgfqpoint{-0.041667in}{-0.041667in}}{\pgfqpoint{0.041667in}{0.041667in}}{%
\pgfpathmoveto{\pgfqpoint{-0.041667in}{0.000000in}}%
\pgfpathlineto{\pgfqpoint{0.041667in}{0.000000in}}%
\pgfpathmoveto{\pgfqpoint{0.000000in}{-0.041667in}}%
\pgfpathlineto{\pgfqpoint{0.000000in}{0.041667in}}%
\pgfusepath{stroke,fill}%
}%
\begin{pgfscope}%
\pgfsys@transformshift{1.355432in}{0.213349in}%
\pgfsys@useobject{currentmarker}{}%
\end{pgfscope}%
\end{pgfscope}%
\begin{pgfscope}%
\definecolor{textcolor}{rgb}{0.000000,0.000000,0.000000}%
\pgfsetstrokecolor{textcolor}%
\pgfsetfillcolor{textcolor}%
\pgftext[x=1.755432in,y=0.135571in,left,base]{\color{textcolor}\rmfamily\fontsize{16.000000}{19.200000}\selectfont Cohabitation}%
\end{pgfscope}%
\begin{pgfscope}%
\pgfsetbuttcap%
\pgfsetroundjoin%
\pgfsetlinewidth{1.505625pt}%
\definecolor{currentstroke}{rgb}{0.000000,0.000000,1.000000}%
\pgfsetstrokecolor{currentstroke}%
\pgfsetdash{{5.550000pt}{2.400000pt}}{0.000000pt}%
\pgfpathmoveto{\pgfqpoint{3.458004in}{0.213349in}}%
\pgfpathlineto{\pgfqpoint{3.902448in}{0.213349in}}%
\pgfusepath{stroke}%
\end{pgfscope}%
\begin{pgfscope}%
\pgfsetbuttcap%
\pgfsetroundjoin%
\definecolor{currentfill}{rgb}{0.000000,0.000000,1.000000}%
\pgfsetfillcolor{currentfill}%
\pgfsetlinewidth{1.003750pt}%
\definecolor{currentstroke}{rgb}{0.000000,0.000000,1.000000}%
\pgfsetstrokecolor{currentstroke}%
\pgfsetdash{}{0pt}%
\pgfsys@defobject{currentmarker}{\pgfqpoint{-0.041667in}{-0.041667in}}{\pgfqpoint{0.041667in}{0.041667in}}{%
\pgfpathmoveto{\pgfqpoint{-0.041667in}{-0.041667in}}%
\pgfpathlineto{\pgfqpoint{0.041667in}{0.041667in}}%
\pgfpathmoveto{\pgfqpoint{-0.041667in}{0.041667in}}%
\pgfpathlineto{\pgfqpoint{0.041667in}{-0.041667in}}%
\pgfusepath{stroke,fill}%
}%
\begin{pgfscope}%
\pgfsys@transformshift{3.680226in}{0.213349in}%
\pgfsys@useobject{currentmarker}{}%
\end{pgfscope}%
\end{pgfscope}%
\begin{pgfscope}%
\definecolor{textcolor}{rgb}{0.000000,0.000000,0.000000}%
\pgfsetstrokecolor{textcolor}%
\pgfsetfillcolor{textcolor}%
\pgftext[x=4.080226in,y=0.135571in,left,base]{\color{textcolor}\rmfamily\fontsize{16.000000}{19.200000}\selectfont Marriage}%
\end{pgfscope}%
\end{pgfpicture}%
\makeatother%
\endgroup%
 } 
\end{subfigure}
\end{center}


\begin{minipage}{0.99\textwidth} % choose width suitably
{\footnotesize \textsc{Notes} The figures display the evolution of the love shock $\psi$ and the female Pareto weight $\theta$ around the introduction on Unilateral Divorce. The displayed patterns are normalized coefficients from event studies around divorce. The graphs are relative to couples that started a relationship.  \par}
\end{minipage}
\end{figure}
\FloatBarrier

\section{Welfare}
Previous research already studied the welfare effects of the introduction of unilateral divorce: both \cite{reynoso2019} and \cite{fernandez2017} find that this policy change decreases welfare for both genders, but more so for women. While we find a similar effect, in this section we claim that accounting for cohabitation results in an even stronger difference by gender.
To study well-being under the two divorce regimes we perform an \textit{ex-ante} welfare comparison, where for each gender we compute the expected value of spending the whole life cycle under a certain regime, before the realization of productivity and love shocks. Table \ref{table:welf} reports the results, which show that welfare under a unilateral divorce regime is lower than under mutual consent for both genders. The difference is larger for women, which would need to be given almost 13,000\$ in assets in $t=0$ to be indifferent between the two regimes, while men would need only \$3,244 to be indifferent between the two. To understand the role of cohabitation for the changes in well-being, we repeat the welfare analysis assuming that cohabitation in no longer a choice.\footnote{In practice, we increase the stigma parameter towards cohabitation $\gamma$ such that cohabitation is never chosen.} For ease of exposition, we refer to the model with cohabitation as model $A$, while model $B$ is the one without cohabitation. The results in table \ref{table:welf} show that the loss of welfare related to unilateral divorce is similar under models $A$ and $B$ for women, while men lose more under model $B$. This result suggests not accounting for cohabitation overestimates the welfare losses for men when unilateral divorce is introduced. The intuition is that cohabitation is valuable for men under the unilateral divorce regime, because it avoids the risk of losing most of their assets in favor of their ex-wife upon dissolution.
\begin{table}[htbp]\centering
\caption{\\Welfare by gender and divorce regime}
\label{table:welf}
\begin{threeparttable}[t]\centering
\begin{tabular}{cccc}
    \hline\midrule
    \multicolumn{2}{c}{\textbf{Female}}& \multicolumn{2}{c}{\textbf{Male}}\\
    \cmidrule(l){1-2}\cmidrule(l){3-4}
     Mutual Consent & Unilateral Divorce & Mutual Consent & Unilateral Divorce\\
     \cmidrule(l){1-4}
    \multicolumn{4}{c}{\textit{Life-Time utilities in $t=0$}}\\[3ex]
     -364.62 &-368.13 &-351.53 &-351.88 \\
    \cmidrule(l){1-4}
    \multicolumn{4}{c}{\textit{Welfare Losses with Unilateral Divorce}}\\[3ex]
    \multicolumn{2}{c}{\Chartgirls{1.0}}& \multicolumn{2}{c}{\Chartguys{0.2508214676889376}}\\[-0.15ex]
    \multicolumn{2}{c}{12933.66 \$}& \multicolumn{2}{c}{3244.04 \$}\\
     \hline\midrule\cmidrule(l){1-4} 
      \multicolumn{4}{c}{\textit{Life-Time utilities in $t=0$ when cohabitation is not in the choice set}}\\[3ex]
    -362.49 &-365.86 &-353.11 &-354.13 \\
    \cmidrule(l){1-4}
    \multicolumn{4}{c}{\textit{Welfare Losses with Unilateral Divorce}}\\[3ex]
    \multicolumn{2}{c}{\Chartgirls{1.0}}& \multicolumn{2}{c}{\Chartguys{0.7564234326824255}}\\[-0.15ex]
    \multicolumn{2}{c}{13783.62 \$}& \multicolumn{2}{c}{10426.25 \$}\\
    \hline\hline
    \end{tabular}
    
\begin{tablenotes}[flushleft]
\footnotesize{\item Welfare losses are obtained computing the amount of wealth that must be transferred to men and women in $t=0$ such that their lifetime utility under the unilateral divorce regime equals the one under mutual consent. The wealth is measured in 1990 dollars.}
\end{tablenotes}
\end{threeparttable}
\end{table}
\FloatBarrier


\section{Counterfactual Experiments}
The aim of this section is to understand the quantitative importance of the economic mechanisms that contributed to the rise of cohabitation during the last decades. To do so, we examine the results from a series of counterfactual experiments.

\textbf{Unilateral Divorce.} The qualitative impact of unilateral divorce on the choice between marriage and cohabitation has been largely discussed throughout this paper. Here we assess its quantitative relevance by performing an experiment where unilateral divorce is never introduced. Table \ref{table:counterfactual_exp} reports the share of people that cohabited at 39 and the average years spent cohabiting\footnote{We consider the number of years spent cohabiting between the age of 20 to the age of 55.} under the baseline scenario and the counterfactual. The results show that under the counterfactual only 67\% of the people would have cohabited by the age of 39, while the years spent into cohabitation would have moved from 2.19 to 1.24. The latter effect is the strongest because it captures both changes in partnership choices of singles and in the duration of partnerships.

\textbf{Shrinking gender wage gap.} Table \ref{table:counterfactual_exp} reports the results of another scenario where the gender productivity gap is reduced by increasing women potential wages by 10\% and men's productivity is reduced by 10\%:\footnote{More specifically, we increase women's productivity, which might not be realized if they decide not to participate in the labor market.} the share of people that ever cohabited increases from 43.3\% to 47.3\%, while the number of years spent cohabiting move from 2.19 to 2.65. In the counterfactual there is less room from specialization in the couple when the two partner's wages are more similar and the opportunity cost of not working for women rises. Hence, in the counterfactual the couple's need for commitment decreases: cohabitation becomes relatively more interesting as it comes with a lower cost of breakup. This result is consistent with the work of \cite{anelli2019}, who find that exposure to robots causes both a decline in market opportunities of men with respect to women and a decrease (increase) in the likelihood of being married (cohabiting).

\textbf{Decreasing the price of home appliances.} In table \ref{table:counterfactual_exp} we report one last counterfactual experiment that explores the effects of reducing by 10\% the relative price of goods $d$, used to produce public goods $Q$. This change is to be interpreted as a result of the improvement in home production technologies, such as the dish washer or the washing machine, which freed up women's time. Previous research already showed the impact of those changes of female labor supply \citep{greenwood2005}, the decline in marriage, the rise in divorce and assortative mating  \citep{greenwood2016}. The counterfactual experiment shows that the share of people that ever cohabited  increases by moved from 43.3\% to 44.8\%, while the years spent cohabiting move from 2.19 to 2.27. Similarly to a reduction in the gender wage gap, improvements in the technology of home production decrease the need for labor specialization within the household and for a commitment technology to enforce it. Hence, improvements in the technology of home production not only caused a decline of marriage with respect to singleness, as \cite{greenwood2016} claim, but also a change in the relative convenience of partnership contracts.

\textbf{No stigma on cohabitation.} Table \ref{table:counterfactual_exp} reports the results of one last counterfactual scenario where the gains stigma towards cohabitation $\gamma$ is set to zero. In the counterfactual, over 80\% of people have ever cohabited and agents spend on average more than 11 years cohabiting. These results suggest that norms have an important role for the rise of cohabitation over time. Finally, note that many people continue marrying: in this scenario 44\% of people have ever married, which suggest that the economic incentives alone are able to generate a positive surplus of marriage with respect to cohabitation for certain individuals.

%Table of deep parameters
\begin{table}[H]
\caption{\\Counterfactual experiments} % title of Table
\label{table:counterfactual_exp}
\centering % used for centering table
\begin{threeparttable}
\begin{tabular}{@{\extracolsep{5pt}}lccc}   % centered columns (4 columns)
\hline \hline%inserts single line
\rule{-4pt}{2.5ex}
Scenario & \% people ever cohabited   & Years spent cohabiting &  \\ [0.45ex] % inserts table
\hline
\rule{-4pt}{2.5ex}
Baseline         &  43.3 & 2.19 &  \\[0.45ex]
No Unilateral Divorce &  29.1  & 1.24 &  \\[0.45ex]
$\downarrow$ gender productivity gap      & 47.3 & 2.65 &  \\[0.45ex]
$\downarrow$ 10\% Price of good $d$     & 44.8  & 2.27 &  \\[0.45ex]
No stigma on Cohabitation ($\gamma=0$)     & 82.4  & 11.40 &  \\[0.45ex]
\hline
\end{tabular}
\begin{tablenotes}[flushleft]
\footnotesize{\item \textsc{Notes}. The Baseline scenario reports the model output with the parameters reported in the previous section. The scenario ``No Unilateral divorce" assumes that all the agents live under a Mutual consent regime during all their life, while in the lower productivity gender gap scenario women's productivity is increased by 10\%, while men's productivity is decreased by 10\%. The share of people that ever cohabited is measured at the simulated age of 39, while years spent cohabiting are computed between ages 20 and 55.}
\end{tablenotes}
\end{threeparttable}
\end{table}
\FloatBarrier

\section{Conclusion}
In this paper, we show that partnership choices depend on the rights to divorce: the introduction of unilateral divorce in most US states influenced selection into marriage and cohabitation as well as the duration of these relationships and women's bargaining power. Using NSFH and NSFG data, we show that the introduction of unilateral divorce is responsible for a 5\% increase in the likelihood that singles choose cohabitation over marriage, and that newly formed cohabitations last longer. To understand the mechanisms that underlie those changes, as well its welfare effect for the two genders, we build a dynamic structural model where agents can choose to marry, cohabit and when to end these relationships. We use regression results from survey data as well as moments that describe the mating market and female labor supply to estimate our model by indirect inference. The structural estimation reveals that couples choosing cohabitation instead of marriage are those that would have had the highest risk of divorce. Since cohabiting couples had on average a lower match quality than married ones, this selection effect increases the duration of newly formed cohabitations. Moreover, in the US states where assets are split equally, men are those who wish to cohabit after the policy reform, since they would lose relatively more of their assets upon divorce. Women are convinced to enter this relationship in exchange for a higher bargaining power, even though this makes them worse off if the couple subsequently breaks up. The possibility to switch to cohabitation does not affect much men's welfare, while for women the welfare under unilateral divorce is much lower because. Finally, we show that the magnitude of the overall effect of unilateral divorce on cohabitation is large: a counterfactual experiment reveals that if the law never changed, time spent into cohabitation for the birth cohorts used in our estimation would have been 1.24 years instead of 2.19, while the share of people that eve cohabited would have moved from 43.3\% to 29.1\%.

Beyond what is studied in this paper it would be interesting to introduce explicitly fertility in our framework to understand why children born within cohabitation does not perform well later in life. A promising approach would be to follow \cite{kozlov2020}, who distinguishes between fertility as a choice and as an unplanned event. In fact, children raised by single mothers are likely outcomes of unwanted births that happen within cohabitation. This situation might happen less frequently within marriage, since it is a more stable relationship than cohabitation.

%\section{Stuff to be Done}
%\begin{itemize}
%\item Check literature of rights of cohabitors in Family Law Quarterly
%\item Read and Understand well Bayot-Voena, then provide a robustness check that incorporates their view of title-based breakup
%\item Robustness check on comomn-law marriage: cohabitations with some probability $p$ should turn into marriages. Consider p varying with female labor supply, as cohabitation is more likely to be considered as common law marriage if children are present
%\item Revise the empirical part, making the choice between cohabitation and marriage a duration model---better if a multinomial probit.
%\item Robustness checks for the empirical part: eliminate california from the sample, do an event study.
%\end{itemize}

%\bibliographystyle{achicago}
\bibliography{mybibliography}

\appendix

\section*{Appendix}
\counterwithin{figure}{section}
\counterwithin{table}{section}
\section{Net worth around divorce/breakup}\label{section:eventsa}

In this section we provide evidence about the evolution of household's net worth around the event of divorce/breakup. Using the 1997-2017 waves of the PSID, we build a sample of 1087 divorces and 1187 breakups that respect the following characteristics: 1) household wealth is observed before and after the relationship  breakdown 2) the number of adults in the household move from two to one after the relationship breakdown 3) the net worth of the household is below the 96\% of the relative distribution 4) we exclude household where the head is older than 65 years old.\footnote{We could not distinguish the net worth of the couple/individuals against the other member if we considered households with more adults.} Net worth is constructed using the PSID variables employed by \cite{blundell2016}. We analyze the evolution of net worth using a standard event study on our sample. Note that, after the relationship breakdown, we report the net worth of the household of the partner that the PSID kept interviewing. Specifically, we estimate the following regression model
\begin{equation}\label{eq:ev_studa}
\text{Net worth}_{i,a,t,y,ma}=\sum_{j=-6}^{4}\beta^{Split}_j\cdot\mathcal{I}(t=j)+\alpha_{0}+\alpha_{a}+\alpha_{y}+\alpha_{ma}+\epsilon_{i,t},
\end{equation}
where $a$ is age of the person observed after the couple chase to exist, $t$ if the year relative to switching to unilateral divorce ($t=-1$ is omitted), and $i$ is the household, $y$ is the year and $ma$ is the number of years since the start of the marriage/cohabitation. Note that we included year, years since marriage/cohabitation and age fixed effects.  We estimate this model separately for formerly married and cohabiting households and we further subdivide our sample considering wealthier/poorer households and men/women.\footnote{A household is considered wealthy if its net worth before couple disruption is above the $75^{th}$ percentile of the distribution and poor otherwise.} Figure \ref{fig:assdiv} reports the results. We normalize the coefficient estimates $\beta^{Split}_j$ by adding the average of net worth at divorce $E[\text{Net worth}|t = -1]$. In panel \subref{sf:assdiv1} we can see a decrease in net worth for richer households: the estimates indicate the year after the divorce the household is left with significantly less than half its original net worth, even though the large standard errors do not allow us to identify clearly the amount of net worth lost because of the divorce. Instead, no clear decrease in net worth can be observed for poorer household. Panel \subref{sf:assdiv3} shows that there is not clear loss in net worth for poor and rich cohabiting households. Finally, panels \subref{sf:assdiv2} and \subref{sf:assdiv4} show that no gender-related difference regarding the evolution of net worth can be detected.


%%%%%%%%%%%%%%%%%%%%%%%%%%%%%%%%%%%%%%
%Symmetry in income and assets
%%%%%%%%%%%%%%%%%%%%%%%%%%%%%%%%%%%%%%
\begin{figure}[ht]
\caption{\\ Event studies of net-worth around divorce}
\label{fig:assdiv}


\begin{subfigure}{.49\textwidth}
\centering
% include second image
\caption{Net worth---rich and poor households}
\label{sf:assdiv1}
\scalebox{0.5}{%% Creator: Matplotlib, PGF backend
%%
%% To include the figure in your LaTeX document, write
%%   \input{<filename>.pgf}
%%
%% Make sure the required packages are loaded in your preamble
%%   \usepackage{pgf}
%%
%% Figures using additional raster images can only be included by \input if
%% they are in the same directory as the main LaTeX file. For loading figures
%% from other directories you can use the `import` package
%%   \usepackage{import}
%% and then include the figures with
%%   \import{<path to file>}{<filename>.pgf}
%%
%% Matplotlib used the following preamble
%%
\begingroup%
\makeatletter%
\begin{pgfpicture}%
\pgfpathrectangle{\pgfpointorigin}{\pgfqpoint{5.270486in}{3.903555in}}%
\pgfusepath{use as bounding box, clip}%
\begin{pgfscope}%
\pgfsetbuttcap%
\pgfsetmiterjoin%
\definecolor{currentfill}{rgb}{1.000000,1.000000,1.000000}%
\pgfsetfillcolor{currentfill}%
\pgfsetlinewidth{0.000000pt}%
\definecolor{currentstroke}{rgb}{1.000000,1.000000,1.000000}%
\pgfsetstrokecolor{currentstroke}%
\pgfsetdash{}{0pt}%
\pgfpathmoveto{\pgfqpoint{0.000000in}{0.000000in}}%
\pgfpathlineto{\pgfqpoint{5.270486in}{0.000000in}}%
\pgfpathlineto{\pgfqpoint{5.270486in}{3.903555in}}%
\pgfpathlineto{\pgfqpoint{0.000000in}{3.903555in}}%
\pgfpathclose%
\pgfusepath{fill}%
\end{pgfscope}%
\begin{pgfscope}%
\pgfsetbuttcap%
\pgfsetmiterjoin%
\definecolor{currentfill}{rgb}{1.000000,1.000000,1.000000}%
\pgfsetfillcolor{currentfill}%
\pgfsetlinewidth{0.000000pt}%
\definecolor{currentstroke}{rgb}{0.000000,0.000000,0.000000}%
\pgfsetstrokecolor{currentstroke}%
\pgfsetstrokeopacity{0.000000}%
\pgfsetdash{}{0pt}%
\pgfpathmoveto{\pgfqpoint{0.620486in}{0.883555in}}%
\pgfpathlineto{\pgfqpoint{5.270486in}{0.883555in}}%
\pgfpathlineto{\pgfqpoint{5.270486in}{3.903555in}}%
\pgfpathlineto{\pgfqpoint{0.620486in}{3.903555in}}%
\pgfpathclose%
\pgfusepath{fill}%
\end{pgfscope}%
\begin{pgfscope}%
\pgfpathrectangle{\pgfqpoint{0.620486in}{0.883555in}}{\pgfqpoint{4.650000in}{3.020000in}}%
\pgfusepath{clip}%
\pgfsetbuttcap%
\pgfsetroundjoin%
\definecolor{currentfill}{rgb}{1.000000,0.000000,0.000000}%
\pgfsetfillcolor{currentfill}%
\pgfsetfillopacity{0.200000}%
\pgfsetlinewidth{0.000000pt}%
\definecolor{currentstroke}{rgb}{0.000000,0.000000,0.000000}%
\pgfsetstrokecolor{currentstroke}%
\pgfsetdash{}{0pt}%
\pgfpathmoveto{\pgfqpoint{0.831850in}{1.502624in}}%
\pgfpathlineto{\pgfqpoint{0.831850in}{1.312496in}}%
\pgfpathlineto{\pgfqpoint{1.536395in}{1.311650in}}%
\pgfpathlineto{\pgfqpoint{1.888668in}{1.392452in}}%
\pgfpathlineto{\pgfqpoint{2.240941in}{1.312073in}}%
\pgfpathlineto{\pgfqpoint{2.593213in}{1.461292in}}%
\pgfpathlineto{\pgfqpoint{2.945486in}{1.345189in}}%
\pgfpathlineto{\pgfqpoint{2.945486in}{1.345189in}}%
\pgfpathlineto{\pgfqpoint{3.297759in}{1.332301in}}%
\pgfpathlineto{\pgfqpoint{3.650031in}{1.377751in}}%
\pgfpathlineto{\pgfqpoint{4.002304in}{1.463750in}}%
\pgfpathlineto{\pgfqpoint{4.354577in}{1.468585in}}%
\pgfpathlineto{\pgfqpoint{5.059122in}{1.262479in}}%
\pgfpathlineto{\pgfqpoint{5.059122in}{1.518149in}}%
\pgfpathlineto{\pgfqpoint{5.059122in}{1.518149in}}%
\pgfpathlineto{\pgfqpoint{4.354577in}{1.690031in}}%
\pgfpathlineto{\pgfqpoint{4.002304in}{1.673623in}}%
\pgfpathlineto{\pgfqpoint{3.650031in}{1.571564in}}%
\pgfpathlineto{\pgfqpoint{3.297759in}{1.506363in}}%
\pgfpathlineto{\pgfqpoint{2.945486in}{1.516421in}}%
\pgfpathlineto{\pgfqpoint{2.945486in}{1.516421in}}%
\pgfpathlineto{\pgfqpoint{2.593213in}{1.461292in}}%
\pgfpathlineto{\pgfqpoint{2.240941in}{1.478110in}}%
\pgfpathlineto{\pgfqpoint{1.888668in}{1.578424in}}%
\pgfpathlineto{\pgfqpoint{1.536395in}{1.492203in}}%
\pgfpathlineto{\pgfqpoint{0.831850in}{1.502624in}}%
\pgfpathclose%
\pgfusepath{fill}%
\end{pgfscope}%
\begin{pgfscope}%
\pgfpathrectangle{\pgfqpoint{0.620486in}{0.883555in}}{\pgfqpoint{4.650000in}{3.020000in}}%
\pgfusepath{clip}%
\pgfsetbuttcap%
\pgfsetroundjoin%
\definecolor{currentfill}{rgb}{0.000000,0.000000,1.000000}%
\pgfsetfillcolor{currentfill}%
\pgfsetfillopacity{0.200000}%
\pgfsetlinewidth{0.000000pt}%
\definecolor{currentstroke}{rgb}{0.000000,0.000000,0.000000}%
\pgfsetstrokecolor{currentstroke}%
\pgfsetdash{}{0pt}%
\pgfpathmoveto{\pgfqpoint{0.831850in}{3.311220in}}%
\pgfpathlineto{\pgfqpoint{0.831850in}{2.374181in}}%
\pgfpathlineto{\pgfqpoint{1.536395in}{2.750564in}}%
\pgfpathlineto{\pgfqpoint{1.888668in}{2.566399in}}%
\pgfpathlineto{\pgfqpoint{2.240941in}{2.960147in}}%
\pgfpathlineto{\pgfqpoint{2.593213in}{3.283789in}}%
\pgfpathlineto{\pgfqpoint{2.945486in}{1.541689in}}%
\pgfpathlineto{\pgfqpoint{2.945486in}{1.541689in}}%
\pgfpathlineto{\pgfqpoint{3.297759in}{1.811912in}}%
\pgfpathlineto{\pgfqpoint{3.650031in}{2.030756in}}%
\pgfpathlineto{\pgfqpoint{4.002304in}{1.405646in}}%
\pgfpathlineto{\pgfqpoint{4.354577in}{1.020828in}}%
\pgfpathlineto{\pgfqpoint{5.059122in}{1.222930in}}%
\pgfpathlineto{\pgfqpoint{5.059122in}{2.452246in}}%
\pgfpathlineto{\pgfqpoint{5.059122in}{2.452246in}}%
\pgfpathlineto{\pgfqpoint{4.354577in}{2.199643in}}%
\pgfpathlineto{\pgfqpoint{4.002304in}{2.399813in}}%
\pgfpathlineto{\pgfqpoint{3.650031in}{3.004002in}}%
\pgfpathlineto{\pgfqpoint{3.297759in}{2.624758in}}%
\pgfpathlineto{\pgfqpoint{2.945486in}{2.360405in}}%
\pgfpathlineto{\pgfqpoint{2.945486in}{2.360405in}}%
\pgfpathlineto{\pgfqpoint{2.593213in}{3.283789in}}%
\pgfpathlineto{\pgfqpoint{2.240941in}{3.766282in}}%
\pgfpathlineto{\pgfqpoint{1.888668in}{3.422041in}}%
\pgfpathlineto{\pgfqpoint{1.536395in}{3.606202in}}%
\pgfpathlineto{\pgfqpoint{0.831850in}{3.311220in}}%
\pgfpathclose%
\pgfusepath{fill}%
\end{pgfscope}%
\begin{pgfscope}%
\pgfsetbuttcap%
\pgfsetroundjoin%
\definecolor{currentfill}{rgb}{0.000000,0.000000,0.000000}%
\pgfsetfillcolor{currentfill}%
\pgfsetlinewidth{0.803000pt}%
\definecolor{currentstroke}{rgb}{0.000000,0.000000,0.000000}%
\pgfsetstrokecolor{currentstroke}%
\pgfsetdash{}{0pt}%
\pgfsys@defobject{currentmarker}{\pgfqpoint{0.000000in}{-0.048611in}}{\pgfqpoint{0.000000in}{0.000000in}}{%
\pgfpathmoveto{\pgfqpoint{0.000000in}{0.000000in}}%
\pgfpathlineto{\pgfqpoint{0.000000in}{-0.048611in}}%
\pgfusepath{stroke,fill}%
}%
\begin{pgfscope}%
\pgfsys@transformshift{0.831850in}{0.883555in}%
\pgfsys@useobject{currentmarker}{}%
\end{pgfscope}%
\end{pgfscope}%
\begin{pgfscope}%
\definecolor{textcolor}{rgb}{0.000000,0.000000,0.000000}%
\pgfsetstrokecolor{textcolor}%
\pgfsetfillcolor{textcolor}%
\pgftext[x=0.831850in,y=0.786333in,,top]{\color{textcolor}\rmfamily\fontsize{11.000000}{13.200000}\selectfont \(\displaystyle -6\)}%
\end{pgfscope}%
\begin{pgfscope}%
\pgfsetbuttcap%
\pgfsetroundjoin%
\definecolor{currentfill}{rgb}{0.000000,0.000000,0.000000}%
\pgfsetfillcolor{currentfill}%
\pgfsetlinewidth{0.803000pt}%
\definecolor{currentstroke}{rgb}{0.000000,0.000000,0.000000}%
\pgfsetstrokecolor{currentstroke}%
\pgfsetdash{}{0pt}%
\pgfsys@defobject{currentmarker}{\pgfqpoint{0.000000in}{-0.048611in}}{\pgfqpoint{0.000000in}{0.000000in}}{%
\pgfpathmoveto{\pgfqpoint{0.000000in}{0.000000in}}%
\pgfpathlineto{\pgfqpoint{0.000000in}{-0.048611in}}%
\pgfusepath{stroke,fill}%
}%
\begin{pgfscope}%
\pgfsys@transformshift{1.536395in}{0.883555in}%
\pgfsys@useobject{currentmarker}{}%
\end{pgfscope}%
\end{pgfscope}%
\begin{pgfscope}%
\definecolor{textcolor}{rgb}{0.000000,0.000000,0.000000}%
\pgfsetstrokecolor{textcolor}%
\pgfsetfillcolor{textcolor}%
\pgftext[x=1.536395in,y=0.786333in,,top]{\color{textcolor}\rmfamily\fontsize{11.000000}{13.200000}\selectfont \(\displaystyle -4\)}%
\end{pgfscope}%
\begin{pgfscope}%
\pgfsetbuttcap%
\pgfsetroundjoin%
\definecolor{currentfill}{rgb}{0.000000,0.000000,0.000000}%
\pgfsetfillcolor{currentfill}%
\pgfsetlinewidth{0.803000pt}%
\definecolor{currentstroke}{rgb}{0.000000,0.000000,0.000000}%
\pgfsetstrokecolor{currentstroke}%
\pgfsetdash{}{0pt}%
\pgfsys@defobject{currentmarker}{\pgfqpoint{0.000000in}{-0.048611in}}{\pgfqpoint{0.000000in}{0.000000in}}{%
\pgfpathmoveto{\pgfqpoint{0.000000in}{0.000000in}}%
\pgfpathlineto{\pgfqpoint{0.000000in}{-0.048611in}}%
\pgfusepath{stroke,fill}%
}%
\begin{pgfscope}%
\pgfsys@transformshift{2.240941in}{0.883555in}%
\pgfsys@useobject{currentmarker}{}%
\end{pgfscope}%
\end{pgfscope}%
\begin{pgfscope}%
\definecolor{textcolor}{rgb}{0.000000,0.000000,0.000000}%
\pgfsetstrokecolor{textcolor}%
\pgfsetfillcolor{textcolor}%
\pgftext[x=2.240941in,y=0.786333in,,top]{\color{textcolor}\rmfamily\fontsize{11.000000}{13.200000}\selectfont \(\displaystyle -2\)}%
\end{pgfscope}%
\begin{pgfscope}%
\pgfsetbuttcap%
\pgfsetroundjoin%
\definecolor{currentfill}{rgb}{0.000000,0.000000,0.000000}%
\pgfsetfillcolor{currentfill}%
\pgfsetlinewidth{0.803000pt}%
\definecolor{currentstroke}{rgb}{0.000000,0.000000,0.000000}%
\pgfsetstrokecolor{currentstroke}%
\pgfsetdash{}{0pt}%
\pgfsys@defobject{currentmarker}{\pgfqpoint{0.000000in}{-0.048611in}}{\pgfqpoint{0.000000in}{0.000000in}}{%
\pgfpathmoveto{\pgfqpoint{0.000000in}{0.000000in}}%
\pgfpathlineto{\pgfqpoint{0.000000in}{-0.048611in}}%
\pgfusepath{stroke,fill}%
}%
\begin{pgfscope}%
\pgfsys@transformshift{2.945486in}{0.883555in}%
\pgfsys@useobject{currentmarker}{}%
\end{pgfscope}%
\end{pgfscope}%
\begin{pgfscope}%
\definecolor{textcolor}{rgb}{0.000000,0.000000,0.000000}%
\pgfsetstrokecolor{textcolor}%
\pgfsetfillcolor{textcolor}%
\pgftext[x=2.945486in,y=0.786333in,,top]{\color{textcolor}\rmfamily\fontsize{11.000000}{13.200000}\selectfont \(\displaystyle 0\)}%
\end{pgfscope}%
\begin{pgfscope}%
\pgfsetbuttcap%
\pgfsetroundjoin%
\definecolor{currentfill}{rgb}{0.000000,0.000000,0.000000}%
\pgfsetfillcolor{currentfill}%
\pgfsetlinewidth{0.803000pt}%
\definecolor{currentstroke}{rgb}{0.000000,0.000000,0.000000}%
\pgfsetstrokecolor{currentstroke}%
\pgfsetdash{}{0pt}%
\pgfsys@defobject{currentmarker}{\pgfqpoint{0.000000in}{-0.048611in}}{\pgfqpoint{0.000000in}{0.000000in}}{%
\pgfpathmoveto{\pgfqpoint{0.000000in}{0.000000in}}%
\pgfpathlineto{\pgfqpoint{0.000000in}{-0.048611in}}%
\pgfusepath{stroke,fill}%
}%
\begin{pgfscope}%
\pgfsys@transformshift{3.650031in}{0.883555in}%
\pgfsys@useobject{currentmarker}{}%
\end{pgfscope}%
\end{pgfscope}%
\begin{pgfscope}%
\definecolor{textcolor}{rgb}{0.000000,0.000000,0.000000}%
\pgfsetstrokecolor{textcolor}%
\pgfsetfillcolor{textcolor}%
\pgftext[x=3.650031in,y=0.786333in,,top]{\color{textcolor}\rmfamily\fontsize{11.000000}{13.200000}\selectfont \(\displaystyle 2\)}%
\end{pgfscope}%
\begin{pgfscope}%
\pgfsetbuttcap%
\pgfsetroundjoin%
\definecolor{currentfill}{rgb}{0.000000,0.000000,0.000000}%
\pgfsetfillcolor{currentfill}%
\pgfsetlinewidth{0.803000pt}%
\definecolor{currentstroke}{rgb}{0.000000,0.000000,0.000000}%
\pgfsetstrokecolor{currentstroke}%
\pgfsetdash{}{0pt}%
\pgfsys@defobject{currentmarker}{\pgfqpoint{0.000000in}{-0.048611in}}{\pgfqpoint{0.000000in}{0.000000in}}{%
\pgfpathmoveto{\pgfqpoint{0.000000in}{0.000000in}}%
\pgfpathlineto{\pgfqpoint{0.000000in}{-0.048611in}}%
\pgfusepath{stroke,fill}%
}%
\begin{pgfscope}%
\pgfsys@transformshift{4.354577in}{0.883555in}%
\pgfsys@useobject{currentmarker}{}%
\end{pgfscope}%
\end{pgfscope}%
\begin{pgfscope}%
\definecolor{textcolor}{rgb}{0.000000,0.000000,0.000000}%
\pgfsetstrokecolor{textcolor}%
\pgfsetfillcolor{textcolor}%
\pgftext[x=4.354577in,y=0.786333in,,top]{\color{textcolor}\rmfamily\fontsize{11.000000}{13.200000}\selectfont \(\displaystyle 4\)}%
\end{pgfscope}%
\begin{pgfscope}%
\pgfsetbuttcap%
\pgfsetroundjoin%
\definecolor{currentfill}{rgb}{0.000000,0.000000,0.000000}%
\pgfsetfillcolor{currentfill}%
\pgfsetlinewidth{0.803000pt}%
\definecolor{currentstroke}{rgb}{0.000000,0.000000,0.000000}%
\pgfsetstrokecolor{currentstroke}%
\pgfsetdash{}{0pt}%
\pgfsys@defobject{currentmarker}{\pgfqpoint{0.000000in}{-0.048611in}}{\pgfqpoint{0.000000in}{0.000000in}}{%
\pgfpathmoveto{\pgfqpoint{0.000000in}{0.000000in}}%
\pgfpathlineto{\pgfqpoint{0.000000in}{-0.048611in}}%
\pgfusepath{stroke,fill}%
}%
\begin{pgfscope}%
\pgfsys@transformshift{5.059122in}{0.883555in}%
\pgfsys@useobject{currentmarker}{}%
\end{pgfscope}%
\end{pgfscope}%
\begin{pgfscope}%
\definecolor{textcolor}{rgb}{0.000000,0.000000,0.000000}%
\pgfsetstrokecolor{textcolor}%
\pgfsetfillcolor{textcolor}%
\pgftext[x=5.059122in,y=0.786333in,,top]{\color{textcolor}\rmfamily\fontsize{11.000000}{13.200000}\selectfont \(\displaystyle 6\)}%
\end{pgfscope}%
\begin{pgfscope}%
\definecolor{textcolor}{rgb}{0.000000,0.000000,0.000000}%
\pgfsetstrokecolor{textcolor}%
\pgfsetfillcolor{textcolor}%
\pgftext[x=2.945486in,y=0.595592in,,top]{\color{textcolor}\rmfamily\fontsize{16.000000}{19.200000}\selectfont Event time (Years)}%
\end{pgfscope}%
\begin{pgfscope}%
\pgfsetbuttcap%
\pgfsetroundjoin%
\definecolor{currentfill}{rgb}{0.000000,0.000000,0.000000}%
\pgfsetfillcolor{currentfill}%
\pgfsetlinewidth{0.803000pt}%
\definecolor{currentstroke}{rgb}{0.000000,0.000000,0.000000}%
\pgfsetstrokecolor{currentstroke}%
\pgfsetdash{}{0pt}%
\pgfsys@defobject{currentmarker}{\pgfqpoint{-0.048611in}{0.000000in}}{\pgfqpoint{0.000000in}{0.000000in}}{%
\pgfpathmoveto{\pgfqpoint{0.000000in}{0.000000in}}%
\pgfpathlineto{\pgfqpoint{-0.048611in}{0.000000in}}%
\pgfusepath{stroke,fill}%
}%
\begin{pgfscope}%
\pgfsys@transformshift{0.620486in}{1.295359in}%
\pgfsys@useobject{currentmarker}{}%
\end{pgfscope}%
\end{pgfscope}%
\begin{pgfscope}%
\definecolor{textcolor}{rgb}{0.000000,0.000000,0.000000}%
\pgfsetstrokecolor{textcolor}%
\pgfsetfillcolor{textcolor}%
\pgftext[x=0.447222in,y=1.242552in,left,base]{\color{textcolor}\rmfamily\fontsize{11.000000}{13.200000}\selectfont \(\displaystyle 0\)}%
\end{pgfscope}%
\begin{pgfscope}%
\pgfsetbuttcap%
\pgfsetroundjoin%
\definecolor{currentfill}{rgb}{0.000000,0.000000,0.000000}%
\pgfsetfillcolor{currentfill}%
\pgfsetlinewidth{0.803000pt}%
\definecolor{currentstroke}{rgb}{0.000000,0.000000,0.000000}%
\pgfsetstrokecolor{currentstroke}%
\pgfsetdash{}{0pt}%
\pgfsys@defobject{currentmarker}{\pgfqpoint{-0.048611in}{0.000000in}}{\pgfqpoint{0.000000in}{0.000000in}}{%
\pgfpathmoveto{\pgfqpoint{0.000000in}{0.000000in}}%
\pgfpathlineto{\pgfqpoint{-0.048611in}{0.000000in}}%
\pgfusepath{stroke,fill}%
}%
\begin{pgfscope}%
\pgfsys@transformshift{0.620486in}{1.714992in}%
\pgfsys@useobject{currentmarker}{}%
\end{pgfscope}%
\end{pgfscope}%
\begin{pgfscope}%
\definecolor{textcolor}{rgb}{0.000000,0.000000,0.000000}%
\pgfsetstrokecolor{textcolor}%
\pgfsetfillcolor{textcolor}%
\pgftext[x=0.371180in,y=1.662185in,left,base]{\color{textcolor}\rmfamily\fontsize{11.000000}{13.200000}\selectfont \(\displaystyle 50\)}%
\end{pgfscope}%
\begin{pgfscope}%
\pgfsetbuttcap%
\pgfsetroundjoin%
\definecolor{currentfill}{rgb}{0.000000,0.000000,0.000000}%
\pgfsetfillcolor{currentfill}%
\pgfsetlinewidth{0.803000pt}%
\definecolor{currentstroke}{rgb}{0.000000,0.000000,0.000000}%
\pgfsetstrokecolor{currentstroke}%
\pgfsetdash{}{0pt}%
\pgfsys@defobject{currentmarker}{\pgfqpoint{-0.048611in}{0.000000in}}{\pgfqpoint{0.000000in}{0.000000in}}{%
\pgfpathmoveto{\pgfqpoint{0.000000in}{0.000000in}}%
\pgfpathlineto{\pgfqpoint{-0.048611in}{0.000000in}}%
\pgfusepath{stroke,fill}%
}%
\begin{pgfscope}%
\pgfsys@transformshift{0.620486in}{2.134625in}%
\pgfsys@useobject{currentmarker}{}%
\end{pgfscope}%
\end{pgfscope}%
\begin{pgfscope}%
\definecolor{textcolor}{rgb}{0.000000,0.000000,0.000000}%
\pgfsetstrokecolor{textcolor}%
\pgfsetfillcolor{textcolor}%
\pgftext[x=0.295138in,y=2.081819in,left,base]{\color{textcolor}\rmfamily\fontsize{11.000000}{13.200000}\selectfont \(\displaystyle 100\)}%
\end{pgfscope}%
\begin{pgfscope}%
\pgfsetbuttcap%
\pgfsetroundjoin%
\definecolor{currentfill}{rgb}{0.000000,0.000000,0.000000}%
\pgfsetfillcolor{currentfill}%
\pgfsetlinewidth{0.803000pt}%
\definecolor{currentstroke}{rgb}{0.000000,0.000000,0.000000}%
\pgfsetstrokecolor{currentstroke}%
\pgfsetdash{}{0pt}%
\pgfsys@defobject{currentmarker}{\pgfqpoint{-0.048611in}{0.000000in}}{\pgfqpoint{0.000000in}{0.000000in}}{%
\pgfpathmoveto{\pgfqpoint{0.000000in}{0.000000in}}%
\pgfpathlineto{\pgfqpoint{-0.048611in}{0.000000in}}%
\pgfusepath{stroke,fill}%
}%
\begin{pgfscope}%
\pgfsys@transformshift{0.620486in}{2.554259in}%
\pgfsys@useobject{currentmarker}{}%
\end{pgfscope}%
\end{pgfscope}%
\begin{pgfscope}%
\definecolor{textcolor}{rgb}{0.000000,0.000000,0.000000}%
\pgfsetstrokecolor{textcolor}%
\pgfsetfillcolor{textcolor}%
\pgftext[x=0.295138in,y=2.501452in,left,base]{\color{textcolor}\rmfamily\fontsize{11.000000}{13.200000}\selectfont \(\displaystyle 150\)}%
\end{pgfscope}%
\begin{pgfscope}%
\pgfsetbuttcap%
\pgfsetroundjoin%
\definecolor{currentfill}{rgb}{0.000000,0.000000,0.000000}%
\pgfsetfillcolor{currentfill}%
\pgfsetlinewidth{0.803000pt}%
\definecolor{currentstroke}{rgb}{0.000000,0.000000,0.000000}%
\pgfsetstrokecolor{currentstroke}%
\pgfsetdash{}{0pt}%
\pgfsys@defobject{currentmarker}{\pgfqpoint{-0.048611in}{0.000000in}}{\pgfqpoint{0.000000in}{0.000000in}}{%
\pgfpathmoveto{\pgfqpoint{0.000000in}{0.000000in}}%
\pgfpathlineto{\pgfqpoint{-0.048611in}{0.000000in}}%
\pgfusepath{stroke,fill}%
}%
\begin{pgfscope}%
\pgfsys@transformshift{0.620486in}{2.973892in}%
\pgfsys@useobject{currentmarker}{}%
\end{pgfscope}%
\end{pgfscope}%
\begin{pgfscope}%
\definecolor{textcolor}{rgb}{0.000000,0.000000,0.000000}%
\pgfsetstrokecolor{textcolor}%
\pgfsetfillcolor{textcolor}%
\pgftext[x=0.295138in,y=2.921085in,left,base]{\color{textcolor}\rmfamily\fontsize{11.000000}{13.200000}\selectfont \(\displaystyle 200\)}%
\end{pgfscope}%
\begin{pgfscope}%
\pgfsetbuttcap%
\pgfsetroundjoin%
\definecolor{currentfill}{rgb}{0.000000,0.000000,0.000000}%
\pgfsetfillcolor{currentfill}%
\pgfsetlinewidth{0.803000pt}%
\definecolor{currentstroke}{rgb}{0.000000,0.000000,0.000000}%
\pgfsetstrokecolor{currentstroke}%
\pgfsetdash{}{0pt}%
\pgfsys@defobject{currentmarker}{\pgfqpoint{-0.048611in}{0.000000in}}{\pgfqpoint{0.000000in}{0.000000in}}{%
\pgfpathmoveto{\pgfqpoint{0.000000in}{0.000000in}}%
\pgfpathlineto{\pgfqpoint{-0.048611in}{0.000000in}}%
\pgfusepath{stroke,fill}%
}%
\begin{pgfscope}%
\pgfsys@transformshift{0.620486in}{3.393525in}%
\pgfsys@useobject{currentmarker}{}%
\end{pgfscope}%
\end{pgfscope}%
\begin{pgfscope}%
\definecolor{textcolor}{rgb}{0.000000,0.000000,0.000000}%
\pgfsetstrokecolor{textcolor}%
\pgfsetfillcolor{textcolor}%
\pgftext[x=0.295138in,y=3.340718in,left,base]{\color{textcolor}\rmfamily\fontsize{11.000000}{13.200000}\selectfont \(\displaystyle 250\)}%
\end{pgfscope}%
\begin{pgfscope}%
\pgfsetbuttcap%
\pgfsetroundjoin%
\definecolor{currentfill}{rgb}{0.000000,0.000000,0.000000}%
\pgfsetfillcolor{currentfill}%
\pgfsetlinewidth{0.803000pt}%
\definecolor{currentstroke}{rgb}{0.000000,0.000000,0.000000}%
\pgfsetstrokecolor{currentstroke}%
\pgfsetdash{}{0pt}%
\pgfsys@defobject{currentmarker}{\pgfqpoint{-0.048611in}{0.000000in}}{\pgfqpoint{0.000000in}{0.000000in}}{%
\pgfpathmoveto{\pgfqpoint{0.000000in}{0.000000in}}%
\pgfpathlineto{\pgfqpoint{-0.048611in}{0.000000in}}%
\pgfusepath{stroke,fill}%
}%
\begin{pgfscope}%
\pgfsys@transformshift{0.620486in}{3.813158in}%
\pgfsys@useobject{currentmarker}{}%
\end{pgfscope}%
\end{pgfscope}%
\begin{pgfscope}%
\definecolor{textcolor}{rgb}{0.000000,0.000000,0.000000}%
\pgfsetstrokecolor{textcolor}%
\pgfsetfillcolor{textcolor}%
\pgftext[x=0.295138in,y=3.760352in,left,base]{\color{textcolor}\rmfamily\fontsize{11.000000}{13.200000}\selectfont \(\displaystyle 300\)}%
\end{pgfscope}%
\begin{pgfscope}%
\definecolor{textcolor}{rgb}{0.000000,0.000000,0.000000}%
\pgfsetstrokecolor{textcolor}%
\pgfsetfillcolor{textcolor}%
\pgftext[x=0.239583in,y=2.393555in,,bottom,rotate=90.000000]{\color{textcolor}\rmfamily\fontsize{16.000000}{19.200000}\selectfont Net Worth (\$ 1000s)}%
\end{pgfscope}%
\begin{pgfscope}%
\pgfpathrectangle{\pgfqpoint{0.620486in}{0.883555in}}{\pgfqpoint{4.650000in}{3.020000in}}%
\pgfusepath{clip}%
\pgfsetrectcap%
\pgfsetroundjoin%
\pgfsetlinewidth{1.505625pt}%
\definecolor{currentstroke}{rgb}{1.000000,0.000000,0.000000}%
\pgfsetstrokecolor{currentstroke}%
\pgfsetdash{}{0pt}%
\pgfpathmoveto{\pgfqpoint{0.831850in}{1.407560in}}%
\pgfpathlineto{\pgfqpoint{1.536395in}{1.401926in}}%
\pgfpathlineto{\pgfqpoint{1.888668in}{1.485438in}}%
\pgfpathlineto{\pgfqpoint{2.240941in}{1.395091in}}%
\pgfpathlineto{\pgfqpoint{2.593213in}{1.461292in}}%
\pgfpathlineto{\pgfqpoint{2.945486in}{1.430805in}}%
\pgfpathlineto{\pgfqpoint{2.945486in}{1.430805in}}%
\pgfpathlineto{\pgfqpoint{3.297759in}{1.419332in}}%
\pgfpathlineto{\pgfqpoint{3.650031in}{1.474658in}}%
\pgfpathlineto{\pgfqpoint{4.002304in}{1.568687in}}%
\pgfpathlineto{\pgfqpoint{4.354577in}{1.579308in}}%
\pgfpathlineto{\pgfqpoint{5.059122in}{1.390314in}}%
\pgfusepath{stroke}%
\end{pgfscope}%
\begin{pgfscope}%
\pgfpathrectangle{\pgfqpoint{0.620486in}{0.883555in}}{\pgfqpoint{4.650000in}{3.020000in}}%
\pgfusepath{clip}%
\pgfsetbuttcap%
\pgfsetroundjoin%
\definecolor{currentfill}{rgb}{1.000000,0.000000,0.000000}%
\pgfsetfillcolor{currentfill}%
\pgfsetlinewidth{1.003750pt}%
\definecolor{currentstroke}{rgb}{1.000000,0.000000,0.000000}%
\pgfsetstrokecolor{currentstroke}%
\pgfsetdash{}{0pt}%
\pgfsys@defobject{currentmarker}{\pgfqpoint{-0.041667in}{-0.041667in}}{\pgfqpoint{0.041667in}{0.041667in}}{%
\pgfpathmoveto{\pgfqpoint{-0.041667in}{0.000000in}}%
\pgfpathlineto{\pgfqpoint{0.041667in}{0.000000in}}%
\pgfpathmoveto{\pgfqpoint{0.000000in}{-0.041667in}}%
\pgfpathlineto{\pgfqpoint{0.000000in}{0.041667in}}%
\pgfusepath{stroke,fill}%
}%
\begin{pgfscope}%
\pgfsys@transformshift{0.831850in}{1.407560in}%
\pgfsys@useobject{currentmarker}{}%
\end{pgfscope}%
\begin{pgfscope}%
\pgfsys@transformshift{1.536395in}{1.401926in}%
\pgfsys@useobject{currentmarker}{}%
\end{pgfscope}%
\begin{pgfscope}%
\pgfsys@transformshift{1.888668in}{1.485438in}%
\pgfsys@useobject{currentmarker}{}%
\end{pgfscope}%
\begin{pgfscope}%
\pgfsys@transformshift{2.240941in}{1.395091in}%
\pgfsys@useobject{currentmarker}{}%
\end{pgfscope}%
\begin{pgfscope}%
\pgfsys@transformshift{2.593213in}{1.461292in}%
\pgfsys@useobject{currentmarker}{}%
\end{pgfscope}%
\begin{pgfscope}%
\pgfsys@transformshift{2.945486in}{1.430805in}%
\pgfsys@useobject{currentmarker}{}%
\end{pgfscope}%
\begin{pgfscope}%
\pgfsys@transformshift{2.945486in}{1.430805in}%
\pgfsys@useobject{currentmarker}{}%
\end{pgfscope}%
\begin{pgfscope}%
\pgfsys@transformshift{3.297759in}{1.419332in}%
\pgfsys@useobject{currentmarker}{}%
\end{pgfscope}%
\begin{pgfscope}%
\pgfsys@transformshift{3.650031in}{1.474658in}%
\pgfsys@useobject{currentmarker}{}%
\end{pgfscope}%
\begin{pgfscope}%
\pgfsys@transformshift{4.002304in}{1.568687in}%
\pgfsys@useobject{currentmarker}{}%
\end{pgfscope}%
\begin{pgfscope}%
\pgfsys@transformshift{4.354577in}{1.579308in}%
\pgfsys@useobject{currentmarker}{}%
\end{pgfscope}%
\begin{pgfscope}%
\pgfsys@transformshift{5.059122in}{1.390314in}%
\pgfsys@useobject{currentmarker}{}%
\end{pgfscope}%
\end{pgfscope}%
\begin{pgfscope}%
\pgfpathrectangle{\pgfqpoint{0.620486in}{0.883555in}}{\pgfqpoint{4.650000in}{3.020000in}}%
\pgfusepath{clip}%
\pgfsetbuttcap%
\pgfsetroundjoin%
\pgfsetlinewidth{1.505625pt}%
\definecolor{currentstroke}{rgb}{0.000000,0.000000,1.000000}%
\pgfsetstrokecolor{currentstroke}%
\pgfsetdash{{5.550000pt}{2.400000pt}}{0.000000pt}%
\pgfpathmoveto{\pgfqpoint{0.831850in}{2.842700in}}%
\pgfpathlineto{\pgfqpoint{1.536395in}{3.178383in}}%
\pgfpathlineto{\pgfqpoint{1.888668in}{2.994220in}}%
\pgfpathlineto{\pgfqpoint{2.240941in}{3.363215in}}%
\pgfpathlineto{\pgfqpoint{2.593213in}{3.283789in}}%
\pgfpathlineto{\pgfqpoint{2.945486in}{1.951047in}}%
\pgfpathlineto{\pgfqpoint{2.945486in}{1.951047in}}%
\pgfpathlineto{\pgfqpoint{3.297759in}{2.218335in}}%
\pgfpathlineto{\pgfqpoint{3.650031in}{2.517379in}}%
\pgfpathlineto{\pgfqpoint{4.002304in}{1.902729in}}%
\pgfpathlineto{\pgfqpoint{4.354577in}{1.610235in}}%
\pgfpathlineto{\pgfqpoint{5.059122in}{1.837588in}}%
\pgfusepath{stroke}%
\end{pgfscope}%
\begin{pgfscope}%
\pgfpathrectangle{\pgfqpoint{0.620486in}{0.883555in}}{\pgfqpoint{4.650000in}{3.020000in}}%
\pgfusepath{clip}%
\pgfsetbuttcap%
\pgfsetroundjoin%
\definecolor{currentfill}{rgb}{0.000000,0.000000,1.000000}%
\pgfsetfillcolor{currentfill}%
\pgfsetlinewidth{1.003750pt}%
\definecolor{currentstroke}{rgb}{0.000000,0.000000,1.000000}%
\pgfsetstrokecolor{currentstroke}%
\pgfsetdash{}{0pt}%
\pgfsys@defobject{currentmarker}{\pgfqpoint{-0.041667in}{-0.041667in}}{\pgfqpoint{0.041667in}{0.041667in}}{%
\pgfpathmoveto{\pgfqpoint{-0.041667in}{-0.041667in}}%
\pgfpathlineto{\pgfqpoint{0.041667in}{0.041667in}}%
\pgfpathmoveto{\pgfqpoint{-0.041667in}{0.041667in}}%
\pgfpathlineto{\pgfqpoint{0.041667in}{-0.041667in}}%
\pgfusepath{stroke,fill}%
}%
\begin{pgfscope}%
\pgfsys@transformshift{0.831850in}{2.842700in}%
\pgfsys@useobject{currentmarker}{}%
\end{pgfscope}%
\begin{pgfscope}%
\pgfsys@transformshift{1.536395in}{3.178383in}%
\pgfsys@useobject{currentmarker}{}%
\end{pgfscope}%
\begin{pgfscope}%
\pgfsys@transformshift{1.888668in}{2.994220in}%
\pgfsys@useobject{currentmarker}{}%
\end{pgfscope}%
\begin{pgfscope}%
\pgfsys@transformshift{2.240941in}{3.363215in}%
\pgfsys@useobject{currentmarker}{}%
\end{pgfscope}%
\begin{pgfscope}%
\pgfsys@transformshift{2.593213in}{3.283789in}%
\pgfsys@useobject{currentmarker}{}%
\end{pgfscope}%
\begin{pgfscope}%
\pgfsys@transformshift{2.945486in}{1.951047in}%
\pgfsys@useobject{currentmarker}{}%
\end{pgfscope}%
\begin{pgfscope}%
\pgfsys@transformshift{2.945486in}{1.951047in}%
\pgfsys@useobject{currentmarker}{}%
\end{pgfscope}%
\begin{pgfscope}%
\pgfsys@transformshift{3.297759in}{2.218335in}%
\pgfsys@useobject{currentmarker}{}%
\end{pgfscope}%
\begin{pgfscope}%
\pgfsys@transformshift{3.650031in}{2.517379in}%
\pgfsys@useobject{currentmarker}{}%
\end{pgfscope}%
\begin{pgfscope}%
\pgfsys@transformshift{4.002304in}{1.902729in}%
\pgfsys@useobject{currentmarker}{}%
\end{pgfscope}%
\begin{pgfscope}%
\pgfsys@transformshift{4.354577in}{1.610235in}%
\pgfsys@useobject{currentmarker}{}%
\end{pgfscope}%
\begin{pgfscope}%
\pgfsys@transformshift{5.059122in}{1.837588in}%
\pgfsys@useobject{currentmarker}{}%
\end{pgfscope}%
\end{pgfscope}%
\begin{pgfscope}%
\pgfpathrectangle{\pgfqpoint{0.620486in}{0.883555in}}{\pgfqpoint{4.650000in}{3.020000in}}%
\pgfusepath{clip}%
\pgfsetrectcap%
\pgfsetroundjoin%
\pgfsetlinewidth{1.505625pt}%
\definecolor{currentstroke}{rgb}{0.000000,0.000000,0.000000}%
\pgfsetstrokecolor{currentstroke}%
\pgfsetdash{}{0pt}%
\pgfpathmoveto{\pgfqpoint{2.945486in}{0.883555in}}%
\pgfpathlineto{\pgfqpoint{2.945486in}{3.903555in}}%
\pgfusepath{stroke}%
\end{pgfscope}%
\begin{pgfscope}%
\pgfsetrectcap%
\pgfsetmiterjoin%
\pgfsetlinewidth{0.803000pt}%
\definecolor{currentstroke}{rgb}{0.000000,0.000000,0.000000}%
\pgfsetstrokecolor{currentstroke}%
\pgfsetdash{}{0pt}%
\pgfpathmoveto{\pgfqpoint{0.620486in}{0.883555in}}%
\pgfpathlineto{\pgfqpoint{0.620486in}{3.903555in}}%
\pgfusepath{stroke}%
\end{pgfscope}%
\begin{pgfscope}%
\pgfsetrectcap%
\pgfsetmiterjoin%
\pgfsetlinewidth{0.803000pt}%
\definecolor{currentstroke}{rgb}{0.000000,0.000000,0.000000}%
\pgfsetstrokecolor{currentstroke}%
\pgfsetdash{}{0pt}%
\pgfpathmoveto{\pgfqpoint{5.270486in}{0.883555in}}%
\pgfpathlineto{\pgfqpoint{5.270486in}{3.903555in}}%
\pgfusepath{stroke}%
\end{pgfscope}%
\begin{pgfscope}%
\pgfsetrectcap%
\pgfsetmiterjoin%
\pgfsetlinewidth{0.803000pt}%
\definecolor{currentstroke}{rgb}{0.000000,0.000000,0.000000}%
\pgfsetstrokecolor{currentstroke}%
\pgfsetdash{}{0pt}%
\pgfpathmoveto{\pgfqpoint{0.620486in}{0.883555in}}%
\pgfpathlineto{\pgfqpoint{5.270486in}{0.883555in}}%
\pgfusepath{stroke}%
\end{pgfscope}%
\begin{pgfscope}%
\pgfsetrectcap%
\pgfsetmiterjoin%
\pgfsetlinewidth{0.803000pt}%
\definecolor{currentstroke}{rgb}{0.000000,0.000000,0.000000}%
\pgfsetstrokecolor{currentstroke}%
\pgfsetdash{}{0pt}%
\pgfpathmoveto{\pgfqpoint{0.620486in}{3.903555in}}%
\pgfpathlineto{\pgfqpoint{5.270486in}{3.903555in}}%
\pgfusepath{stroke}%
\end{pgfscope}%
\begin{pgfscope}%
\pgfsetbuttcap%
\pgfsetmiterjoin%
\definecolor{currentfill}{rgb}{0.300000,0.300000,0.300000}%
\pgfsetfillcolor{currentfill}%
\pgfsetfillopacity{0.500000}%
\pgfsetlinewidth{1.003750pt}%
\definecolor{currentstroke}{rgb}{0.300000,0.300000,0.300000}%
\pgfsetstrokecolor{currentstroke}%
\pgfsetstrokeopacity{0.500000}%
\pgfsetdash{}{0pt}%
\pgfpathmoveto{\pgfqpoint{1.798419in}{-0.027778in}}%
\pgfpathlineto{\pgfqpoint{4.148108in}{-0.027778in}}%
\pgfpathquadraticcurveto{\pgfqpoint{4.186997in}{-0.027778in}}{\pgfqpoint{4.186997in}{0.011111in}}%
\pgfpathlineto{\pgfqpoint{4.186997in}{0.266666in}}%
\pgfpathquadraticcurveto{\pgfqpoint{4.186997in}{0.305555in}}{\pgfqpoint{4.148108in}{0.305555in}}%
\pgfpathlineto{\pgfqpoint{1.798419in}{0.305555in}}%
\pgfpathquadraticcurveto{\pgfqpoint{1.759530in}{0.305555in}}{\pgfqpoint{1.759530in}{0.266666in}}%
\pgfpathlineto{\pgfqpoint{1.759530in}{0.011111in}}%
\pgfpathquadraticcurveto{\pgfqpoint{1.759530in}{-0.027778in}}{\pgfqpoint{1.798419in}{-0.027778in}}%
\pgfpathclose%
\pgfusepath{stroke,fill}%
\end{pgfscope}%
\begin{pgfscope}%
\pgfsetbuttcap%
\pgfsetmiterjoin%
\definecolor{currentfill}{rgb}{1.000000,1.000000,1.000000}%
\pgfsetfillcolor{currentfill}%
\pgfsetlinewidth{1.003750pt}%
\definecolor{currentstroke}{rgb}{0.800000,0.800000,0.800000}%
\pgfsetstrokecolor{currentstroke}%
\pgfsetdash{}{0pt}%
\pgfpathmoveto{\pgfqpoint{1.770641in}{0.000000in}}%
\pgfpathlineto{\pgfqpoint{4.120331in}{0.000000in}}%
\pgfpathquadraticcurveto{\pgfqpoint{4.159220in}{0.000000in}}{\pgfqpoint{4.159220in}{0.038889in}}%
\pgfpathlineto{\pgfqpoint{4.159220in}{0.294444in}}%
\pgfpathquadraticcurveto{\pgfqpoint{4.159220in}{0.333333in}}{\pgfqpoint{4.120331in}{0.333333in}}%
\pgfpathlineto{\pgfqpoint{1.770641in}{0.333333in}}%
\pgfpathquadraticcurveto{\pgfqpoint{1.731752in}{0.333333in}}{\pgfqpoint{1.731752in}{0.294444in}}%
\pgfpathlineto{\pgfqpoint{1.731752in}{0.038889in}}%
\pgfpathquadraticcurveto{\pgfqpoint{1.731752in}{0.000000in}}{\pgfqpoint{1.770641in}{0.000000in}}%
\pgfpathclose%
\pgfusepath{stroke,fill}%
\end{pgfscope}%
\begin{pgfscope}%
\pgfsetrectcap%
\pgfsetroundjoin%
\pgfsetlinewidth{1.505625pt}%
\definecolor{currentstroke}{rgb}{1.000000,0.000000,0.000000}%
\pgfsetstrokecolor{currentstroke}%
\pgfsetdash{}{0pt}%
\pgfpathmoveto{\pgfqpoint{1.809530in}{0.184722in}}%
\pgfpathlineto{\pgfqpoint{2.198419in}{0.184722in}}%
\pgfusepath{stroke}%
\end{pgfscope}%
\begin{pgfscope}%
\pgfsetbuttcap%
\pgfsetroundjoin%
\definecolor{currentfill}{rgb}{1.000000,0.000000,0.000000}%
\pgfsetfillcolor{currentfill}%
\pgfsetlinewidth{1.003750pt}%
\definecolor{currentstroke}{rgb}{1.000000,0.000000,0.000000}%
\pgfsetstrokecolor{currentstroke}%
\pgfsetdash{}{0pt}%
\pgfsys@defobject{currentmarker}{\pgfqpoint{-0.041667in}{-0.041667in}}{\pgfqpoint{0.041667in}{0.041667in}}{%
\pgfpathmoveto{\pgfqpoint{-0.041667in}{0.000000in}}%
\pgfpathlineto{\pgfqpoint{0.041667in}{0.000000in}}%
\pgfpathmoveto{\pgfqpoint{0.000000in}{-0.041667in}}%
\pgfpathlineto{\pgfqpoint{0.000000in}{0.041667in}}%
\pgfusepath{stroke,fill}%
}%
\begin{pgfscope}%
\pgfsys@transformshift{2.003975in}{0.184722in}%
\pgfsys@useobject{currentmarker}{}%
\end{pgfscope}%
\end{pgfscope}%
\begin{pgfscope}%
\definecolor{textcolor}{rgb}{0.000000,0.000000,0.000000}%
\pgfsetstrokecolor{textcolor}%
\pgfsetfillcolor{textcolor}%
\pgftext[x=2.353975in,y=0.116667in,left,base]{\color{textcolor}\rmfamily\fontsize{14.000000}{16.800000}\selectfont Poor}%
\end{pgfscope}%
\begin{pgfscope}%
\pgfsetbuttcap%
\pgfsetroundjoin%
\pgfsetlinewidth{1.505625pt}%
\definecolor{currentstroke}{rgb}{0.000000,0.000000,1.000000}%
\pgfsetstrokecolor{currentstroke}%
\pgfsetdash{{5.550000pt}{2.400000pt}}{0.000000pt}%
\pgfpathmoveto{\pgfqpoint{3.148090in}{0.184722in}}%
\pgfpathlineto{\pgfqpoint{3.536979in}{0.184722in}}%
\pgfusepath{stroke}%
\end{pgfscope}%
\begin{pgfscope}%
\pgfsetbuttcap%
\pgfsetroundjoin%
\definecolor{currentfill}{rgb}{0.000000,0.000000,1.000000}%
\pgfsetfillcolor{currentfill}%
\pgfsetlinewidth{1.003750pt}%
\definecolor{currentstroke}{rgb}{0.000000,0.000000,1.000000}%
\pgfsetstrokecolor{currentstroke}%
\pgfsetdash{}{0pt}%
\pgfsys@defobject{currentmarker}{\pgfqpoint{-0.041667in}{-0.041667in}}{\pgfqpoint{0.041667in}{0.041667in}}{%
\pgfpathmoveto{\pgfqpoint{-0.041667in}{-0.041667in}}%
\pgfpathlineto{\pgfqpoint{0.041667in}{0.041667in}}%
\pgfpathmoveto{\pgfqpoint{-0.041667in}{0.041667in}}%
\pgfpathlineto{\pgfqpoint{0.041667in}{-0.041667in}}%
\pgfusepath{stroke,fill}%
}%
\begin{pgfscope}%
\pgfsys@transformshift{3.342534in}{0.184722in}%
\pgfsys@useobject{currentmarker}{}%
\end{pgfscope}%
\end{pgfscope}%
\begin{pgfscope}%
\definecolor{textcolor}{rgb}{0.000000,0.000000,0.000000}%
\pgfsetstrokecolor{textcolor}%
\pgfsetfillcolor{textcolor}%
\pgftext[x=3.692534in,y=0.116667in,left,base]{\color{textcolor}\rmfamily\fontsize{14.000000}{16.800000}\selectfont Rich}%
\end{pgfscope}%
\end{pgfpicture}%
\makeatother%
\endgroup%
 } 
\end{subfigure}
\begin{subfigure}{.49\textwidth}
\centering
% include first image
\caption{Net worth---men and women}
\label{sf:assdiv2}
\scalebox{0.5}{%% Creator: Matplotlib, PGF backend
%%
%% To include the figure in your LaTeX document, write
%%   \input{<filename>.pgf}
%%
%% Make sure the required packages are loaded in your preamble
%%   \usepackage{pgf}
%%
%% Figures using additional raster images can only be included by \input if
%% they are in the same directory as the main LaTeX file. For loading figures
%% from other directories you can use the `import` package
%%   \usepackage{import}
%% and then include the figures with
%%   \import{<path to file>}{<filename>.pgf}
%%
%% Matplotlib used the following preamble
%%
\begingroup%
\makeatletter%
\begin{pgfpicture}%
\pgfpathrectangle{\pgfpointorigin}{\pgfqpoint{5.312732in}{3.903555in}}%
\pgfusepath{use as bounding box, clip}%
\begin{pgfscope}%
\pgfsetbuttcap%
\pgfsetmiterjoin%
\definecolor{currentfill}{rgb}{1.000000,1.000000,1.000000}%
\pgfsetfillcolor{currentfill}%
\pgfsetlinewidth{0.000000pt}%
\definecolor{currentstroke}{rgb}{1.000000,1.000000,1.000000}%
\pgfsetstrokecolor{currentstroke}%
\pgfsetdash{}{0pt}%
\pgfpathmoveto{\pgfqpoint{0.000000in}{0.000000in}}%
\pgfpathlineto{\pgfqpoint{5.312732in}{0.000000in}}%
\pgfpathlineto{\pgfqpoint{5.312732in}{3.903555in}}%
\pgfpathlineto{\pgfqpoint{0.000000in}{3.903555in}}%
\pgfpathclose%
\pgfusepath{fill}%
\end{pgfscope}%
\begin{pgfscope}%
\pgfsetbuttcap%
\pgfsetmiterjoin%
\definecolor{currentfill}{rgb}{1.000000,1.000000,1.000000}%
\pgfsetfillcolor{currentfill}%
\pgfsetlinewidth{0.000000pt}%
\definecolor{currentstroke}{rgb}{0.000000,0.000000,0.000000}%
\pgfsetstrokecolor{currentstroke}%
\pgfsetstrokeopacity{0.000000}%
\pgfsetdash{}{0pt}%
\pgfpathmoveto{\pgfqpoint{0.662732in}{0.883555in}}%
\pgfpathlineto{\pgfqpoint{5.312732in}{0.883555in}}%
\pgfpathlineto{\pgfqpoint{5.312732in}{3.903555in}}%
\pgfpathlineto{\pgfqpoint{0.662732in}{3.903555in}}%
\pgfpathclose%
\pgfusepath{fill}%
\end{pgfscope}%
\begin{pgfscope}%
\pgfpathrectangle{\pgfqpoint{0.662732in}{0.883555in}}{\pgfqpoint{4.650000in}{3.020000in}}%
\pgfusepath{clip}%
\pgfsetbuttcap%
\pgfsetroundjoin%
\definecolor{currentfill}{rgb}{1.000000,0.000000,0.000000}%
\pgfsetfillcolor{currentfill}%
\pgfsetfillopacity{0.200000}%
\pgfsetlinewidth{0.000000pt}%
\definecolor{currentstroke}{rgb}{0.000000,0.000000,0.000000}%
\pgfsetstrokecolor{currentstroke}%
\pgfsetdash{}{0pt}%
\pgfpathmoveto{\pgfqpoint{0.874095in}{3.473094in}}%
\pgfpathlineto{\pgfqpoint{0.874095in}{2.429581in}}%
\pgfpathlineto{\pgfqpoint{1.578641in}{2.372660in}}%
\pgfpathlineto{\pgfqpoint{1.930913in}{2.660687in}}%
\pgfpathlineto{\pgfqpoint{2.283186in}{2.579367in}}%
\pgfpathlineto{\pgfqpoint{2.635459in}{3.092804in}}%
\pgfpathlineto{\pgfqpoint{2.987732in}{1.778259in}}%
\pgfpathlineto{\pgfqpoint{2.987732in}{1.778259in}}%
\pgfpathlineto{\pgfqpoint{3.340004in}{1.846301in}}%
\pgfpathlineto{\pgfqpoint{3.692277in}{1.763238in}}%
\pgfpathlineto{\pgfqpoint{4.044550in}{1.578833in}}%
\pgfpathlineto{\pgfqpoint{4.396822in}{1.396484in}}%
\pgfpathlineto{\pgfqpoint{5.101368in}{1.213548in}}%
\pgfpathlineto{\pgfqpoint{5.101368in}{2.634711in}}%
\pgfpathlineto{\pgfqpoint{5.101368in}{2.634711in}}%
\pgfpathlineto{\pgfqpoint{4.396822in}{2.661751in}}%
\pgfpathlineto{\pgfqpoint{4.044550in}{2.783098in}}%
\pgfpathlineto{\pgfqpoint{3.692277in}{2.845186in}}%
\pgfpathlineto{\pgfqpoint{3.340004in}{2.818807in}}%
\pgfpathlineto{\pgfqpoint{2.987732in}{2.714379in}}%
\pgfpathlineto{\pgfqpoint{2.987732in}{2.714379in}}%
\pgfpathlineto{\pgfqpoint{2.635459in}{3.092804in}}%
\pgfpathlineto{\pgfqpoint{2.283186in}{3.488244in}}%
\pgfpathlineto{\pgfqpoint{1.930913in}{3.709260in}}%
\pgfpathlineto{\pgfqpoint{1.578641in}{3.369583in}}%
\pgfpathlineto{\pgfqpoint{0.874095in}{3.473094in}}%
\pgfpathclose%
\pgfusepath{fill}%
\end{pgfscope}%
\begin{pgfscope}%
\pgfpathrectangle{\pgfqpoint{0.662732in}{0.883555in}}{\pgfqpoint{4.650000in}{3.020000in}}%
\pgfusepath{clip}%
\pgfsetbuttcap%
\pgfsetroundjoin%
\definecolor{currentfill}{rgb}{0.000000,0.000000,1.000000}%
\pgfsetfillcolor{currentfill}%
\pgfsetfillopacity{0.200000}%
\pgfsetlinewidth{0.000000pt}%
\definecolor{currentstroke}{rgb}{0.000000,0.000000,0.000000}%
\pgfsetstrokecolor{currentstroke}%
\pgfsetdash{}{0pt}%
\pgfpathmoveto{\pgfqpoint{0.874095in}{3.253704in}}%
\pgfpathlineto{\pgfqpoint{0.874095in}{2.200367in}}%
\pgfpathlineto{\pgfqpoint{1.578641in}{2.454819in}}%
\pgfpathlineto{\pgfqpoint{1.930913in}{2.777730in}}%
\pgfpathlineto{\pgfqpoint{2.283186in}{2.501791in}}%
\pgfpathlineto{\pgfqpoint{2.635459in}{3.381021in}}%
\pgfpathlineto{\pgfqpoint{2.987732in}{1.785750in}}%
\pgfpathlineto{\pgfqpoint{2.987732in}{1.785750in}}%
\pgfpathlineto{\pgfqpoint{3.340004in}{2.150775in}}%
\pgfpathlineto{\pgfqpoint{3.692277in}{1.978321in}}%
\pgfpathlineto{\pgfqpoint{4.044550in}{1.972884in}}%
\pgfpathlineto{\pgfqpoint{4.396822in}{1.518296in}}%
\pgfpathlineto{\pgfqpoint{5.101368in}{1.020828in}}%
\pgfpathlineto{\pgfqpoint{5.101368in}{2.424968in}}%
\pgfpathlineto{\pgfqpoint{5.101368in}{2.424968in}}%
\pgfpathlineto{\pgfqpoint{4.396822in}{2.747456in}}%
\pgfpathlineto{\pgfqpoint{4.044550in}{3.104540in}}%
\pgfpathlineto{\pgfqpoint{3.692277in}{3.051660in}}%
\pgfpathlineto{\pgfqpoint{3.340004in}{3.095792in}}%
\pgfpathlineto{\pgfqpoint{2.987732in}{2.724233in}}%
\pgfpathlineto{\pgfqpoint{2.987732in}{2.724233in}}%
\pgfpathlineto{\pgfqpoint{2.635459in}{3.381021in}}%
\pgfpathlineto{\pgfqpoint{2.283186in}{3.411643in}}%
\pgfpathlineto{\pgfqpoint{1.930913in}{3.766282in}}%
\pgfpathlineto{\pgfqpoint{1.578641in}{3.432898in}}%
\pgfpathlineto{\pgfqpoint{0.874095in}{3.253704in}}%
\pgfpathclose%
\pgfusepath{fill}%
\end{pgfscope}%
\begin{pgfscope}%
\pgfsetbuttcap%
\pgfsetroundjoin%
\definecolor{currentfill}{rgb}{0.000000,0.000000,0.000000}%
\pgfsetfillcolor{currentfill}%
\pgfsetlinewidth{0.803000pt}%
\definecolor{currentstroke}{rgb}{0.000000,0.000000,0.000000}%
\pgfsetstrokecolor{currentstroke}%
\pgfsetdash{}{0pt}%
\pgfsys@defobject{currentmarker}{\pgfqpoint{0.000000in}{-0.048611in}}{\pgfqpoint{0.000000in}{0.000000in}}{%
\pgfpathmoveto{\pgfqpoint{0.000000in}{0.000000in}}%
\pgfpathlineto{\pgfqpoint{0.000000in}{-0.048611in}}%
\pgfusepath{stroke,fill}%
}%
\begin{pgfscope}%
\pgfsys@transformshift{0.874095in}{0.883555in}%
\pgfsys@useobject{currentmarker}{}%
\end{pgfscope}%
\end{pgfscope}%
\begin{pgfscope}%
\definecolor{textcolor}{rgb}{0.000000,0.000000,0.000000}%
\pgfsetstrokecolor{textcolor}%
\pgfsetfillcolor{textcolor}%
\pgftext[x=0.874095in,y=0.786333in,,top]{\color{textcolor}\rmfamily\fontsize{11.000000}{13.200000}\selectfont \(\displaystyle -6\)}%
\end{pgfscope}%
\begin{pgfscope}%
\pgfsetbuttcap%
\pgfsetroundjoin%
\definecolor{currentfill}{rgb}{0.000000,0.000000,0.000000}%
\pgfsetfillcolor{currentfill}%
\pgfsetlinewidth{0.803000pt}%
\definecolor{currentstroke}{rgb}{0.000000,0.000000,0.000000}%
\pgfsetstrokecolor{currentstroke}%
\pgfsetdash{}{0pt}%
\pgfsys@defobject{currentmarker}{\pgfqpoint{0.000000in}{-0.048611in}}{\pgfqpoint{0.000000in}{0.000000in}}{%
\pgfpathmoveto{\pgfqpoint{0.000000in}{0.000000in}}%
\pgfpathlineto{\pgfqpoint{0.000000in}{-0.048611in}}%
\pgfusepath{stroke,fill}%
}%
\begin{pgfscope}%
\pgfsys@transformshift{1.578641in}{0.883555in}%
\pgfsys@useobject{currentmarker}{}%
\end{pgfscope}%
\end{pgfscope}%
\begin{pgfscope}%
\definecolor{textcolor}{rgb}{0.000000,0.000000,0.000000}%
\pgfsetstrokecolor{textcolor}%
\pgfsetfillcolor{textcolor}%
\pgftext[x=1.578641in,y=0.786333in,,top]{\color{textcolor}\rmfamily\fontsize{11.000000}{13.200000}\selectfont \(\displaystyle -4\)}%
\end{pgfscope}%
\begin{pgfscope}%
\pgfsetbuttcap%
\pgfsetroundjoin%
\definecolor{currentfill}{rgb}{0.000000,0.000000,0.000000}%
\pgfsetfillcolor{currentfill}%
\pgfsetlinewidth{0.803000pt}%
\definecolor{currentstroke}{rgb}{0.000000,0.000000,0.000000}%
\pgfsetstrokecolor{currentstroke}%
\pgfsetdash{}{0pt}%
\pgfsys@defobject{currentmarker}{\pgfqpoint{0.000000in}{-0.048611in}}{\pgfqpoint{0.000000in}{0.000000in}}{%
\pgfpathmoveto{\pgfqpoint{0.000000in}{0.000000in}}%
\pgfpathlineto{\pgfqpoint{0.000000in}{-0.048611in}}%
\pgfusepath{stroke,fill}%
}%
\begin{pgfscope}%
\pgfsys@transformshift{2.283186in}{0.883555in}%
\pgfsys@useobject{currentmarker}{}%
\end{pgfscope}%
\end{pgfscope}%
\begin{pgfscope}%
\definecolor{textcolor}{rgb}{0.000000,0.000000,0.000000}%
\pgfsetstrokecolor{textcolor}%
\pgfsetfillcolor{textcolor}%
\pgftext[x=2.283186in,y=0.786333in,,top]{\color{textcolor}\rmfamily\fontsize{11.000000}{13.200000}\selectfont \(\displaystyle -2\)}%
\end{pgfscope}%
\begin{pgfscope}%
\pgfsetbuttcap%
\pgfsetroundjoin%
\definecolor{currentfill}{rgb}{0.000000,0.000000,0.000000}%
\pgfsetfillcolor{currentfill}%
\pgfsetlinewidth{0.803000pt}%
\definecolor{currentstroke}{rgb}{0.000000,0.000000,0.000000}%
\pgfsetstrokecolor{currentstroke}%
\pgfsetdash{}{0pt}%
\pgfsys@defobject{currentmarker}{\pgfqpoint{0.000000in}{-0.048611in}}{\pgfqpoint{0.000000in}{0.000000in}}{%
\pgfpathmoveto{\pgfqpoint{0.000000in}{0.000000in}}%
\pgfpathlineto{\pgfqpoint{0.000000in}{-0.048611in}}%
\pgfusepath{stroke,fill}%
}%
\begin{pgfscope}%
\pgfsys@transformshift{2.987732in}{0.883555in}%
\pgfsys@useobject{currentmarker}{}%
\end{pgfscope}%
\end{pgfscope}%
\begin{pgfscope}%
\definecolor{textcolor}{rgb}{0.000000,0.000000,0.000000}%
\pgfsetstrokecolor{textcolor}%
\pgfsetfillcolor{textcolor}%
\pgftext[x=2.987732in,y=0.786333in,,top]{\color{textcolor}\rmfamily\fontsize{11.000000}{13.200000}\selectfont \(\displaystyle 0\)}%
\end{pgfscope}%
\begin{pgfscope}%
\pgfsetbuttcap%
\pgfsetroundjoin%
\definecolor{currentfill}{rgb}{0.000000,0.000000,0.000000}%
\pgfsetfillcolor{currentfill}%
\pgfsetlinewidth{0.803000pt}%
\definecolor{currentstroke}{rgb}{0.000000,0.000000,0.000000}%
\pgfsetstrokecolor{currentstroke}%
\pgfsetdash{}{0pt}%
\pgfsys@defobject{currentmarker}{\pgfqpoint{0.000000in}{-0.048611in}}{\pgfqpoint{0.000000in}{0.000000in}}{%
\pgfpathmoveto{\pgfqpoint{0.000000in}{0.000000in}}%
\pgfpathlineto{\pgfqpoint{0.000000in}{-0.048611in}}%
\pgfusepath{stroke,fill}%
}%
\begin{pgfscope}%
\pgfsys@transformshift{3.692277in}{0.883555in}%
\pgfsys@useobject{currentmarker}{}%
\end{pgfscope}%
\end{pgfscope}%
\begin{pgfscope}%
\definecolor{textcolor}{rgb}{0.000000,0.000000,0.000000}%
\pgfsetstrokecolor{textcolor}%
\pgfsetfillcolor{textcolor}%
\pgftext[x=3.692277in,y=0.786333in,,top]{\color{textcolor}\rmfamily\fontsize{11.000000}{13.200000}\selectfont \(\displaystyle 2\)}%
\end{pgfscope}%
\begin{pgfscope}%
\pgfsetbuttcap%
\pgfsetroundjoin%
\definecolor{currentfill}{rgb}{0.000000,0.000000,0.000000}%
\pgfsetfillcolor{currentfill}%
\pgfsetlinewidth{0.803000pt}%
\definecolor{currentstroke}{rgb}{0.000000,0.000000,0.000000}%
\pgfsetstrokecolor{currentstroke}%
\pgfsetdash{}{0pt}%
\pgfsys@defobject{currentmarker}{\pgfqpoint{0.000000in}{-0.048611in}}{\pgfqpoint{0.000000in}{0.000000in}}{%
\pgfpathmoveto{\pgfqpoint{0.000000in}{0.000000in}}%
\pgfpathlineto{\pgfqpoint{0.000000in}{-0.048611in}}%
\pgfusepath{stroke,fill}%
}%
\begin{pgfscope}%
\pgfsys@transformshift{4.396822in}{0.883555in}%
\pgfsys@useobject{currentmarker}{}%
\end{pgfscope}%
\end{pgfscope}%
\begin{pgfscope}%
\definecolor{textcolor}{rgb}{0.000000,0.000000,0.000000}%
\pgfsetstrokecolor{textcolor}%
\pgfsetfillcolor{textcolor}%
\pgftext[x=4.396822in,y=0.786333in,,top]{\color{textcolor}\rmfamily\fontsize{11.000000}{13.200000}\selectfont \(\displaystyle 4\)}%
\end{pgfscope}%
\begin{pgfscope}%
\pgfsetbuttcap%
\pgfsetroundjoin%
\definecolor{currentfill}{rgb}{0.000000,0.000000,0.000000}%
\pgfsetfillcolor{currentfill}%
\pgfsetlinewidth{0.803000pt}%
\definecolor{currentstroke}{rgb}{0.000000,0.000000,0.000000}%
\pgfsetstrokecolor{currentstroke}%
\pgfsetdash{}{0pt}%
\pgfsys@defobject{currentmarker}{\pgfqpoint{0.000000in}{-0.048611in}}{\pgfqpoint{0.000000in}{0.000000in}}{%
\pgfpathmoveto{\pgfqpoint{0.000000in}{0.000000in}}%
\pgfpathlineto{\pgfqpoint{0.000000in}{-0.048611in}}%
\pgfusepath{stroke,fill}%
}%
\begin{pgfscope}%
\pgfsys@transformshift{5.101368in}{0.883555in}%
\pgfsys@useobject{currentmarker}{}%
\end{pgfscope}%
\end{pgfscope}%
\begin{pgfscope}%
\definecolor{textcolor}{rgb}{0.000000,0.000000,0.000000}%
\pgfsetstrokecolor{textcolor}%
\pgfsetfillcolor{textcolor}%
\pgftext[x=5.101368in,y=0.786333in,,top]{\color{textcolor}\rmfamily\fontsize{11.000000}{13.200000}\selectfont \(\displaystyle 6\)}%
\end{pgfscope}%
\begin{pgfscope}%
\definecolor{textcolor}{rgb}{0.000000,0.000000,0.000000}%
\pgfsetstrokecolor{textcolor}%
\pgfsetfillcolor{textcolor}%
\pgftext[x=2.987732in,y=0.595592in,,top]{\color{textcolor}\rmfamily\fontsize{16.000000}{19.200000}\selectfont Event time (Years)}%
\end{pgfscope}%
\begin{pgfscope}%
\pgfsetbuttcap%
\pgfsetroundjoin%
\definecolor{currentfill}{rgb}{0.000000,0.000000,0.000000}%
\pgfsetfillcolor{currentfill}%
\pgfsetlinewidth{0.803000pt}%
\definecolor{currentstroke}{rgb}{0.000000,0.000000,0.000000}%
\pgfsetstrokecolor{currentstroke}%
\pgfsetdash{}{0pt}%
\pgfsys@defobject{currentmarker}{\pgfqpoint{-0.048611in}{0.000000in}}{\pgfqpoint{0.000000in}{0.000000in}}{%
\pgfpathmoveto{\pgfqpoint{0.000000in}{0.000000in}}%
\pgfpathlineto{\pgfqpoint{-0.048611in}{0.000000in}}%
\pgfusepath{stroke,fill}%
}%
\begin{pgfscope}%
\pgfsys@transformshift{0.662732in}{1.010220in}%
\pgfsys@useobject{currentmarker}{}%
\end{pgfscope}%
\end{pgfscope}%
\begin{pgfscope}%
\definecolor{textcolor}{rgb}{0.000000,0.000000,0.000000}%
\pgfsetstrokecolor{textcolor}%
\pgfsetfillcolor{textcolor}%
\pgftext[x=0.295138in,y=0.957413in,left,base]{\color{textcolor}\rmfamily\fontsize{11.000000}{13.200000}\selectfont \(\displaystyle -40\)}%
\end{pgfscope}%
\begin{pgfscope}%
\pgfsetbuttcap%
\pgfsetroundjoin%
\definecolor{currentfill}{rgb}{0.000000,0.000000,0.000000}%
\pgfsetfillcolor{currentfill}%
\pgfsetlinewidth{0.803000pt}%
\definecolor{currentstroke}{rgb}{0.000000,0.000000,0.000000}%
\pgfsetstrokecolor{currentstroke}%
\pgfsetdash{}{0pt}%
\pgfsys@defobject{currentmarker}{\pgfqpoint{-0.048611in}{0.000000in}}{\pgfqpoint{0.000000in}{0.000000in}}{%
\pgfpathmoveto{\pgfqpoint{0.000000in}{0.000000in}}%
\pgfpathlineto{\pgfqpoint{-0.048611in}{0.000000in}}%
\pgfusepath{stroke,fill}%
}%
\begin{pgfscope}%
\pgfsys@transformshift{0.662732in}{1.391526in}%
\pgfsys@useobject{currentmarker}{}%
\end{pgfscope}%
\end{pgfscope}%
\begin{pgfscope}%
\definecolor{textcolor}{rgb}{0.000000,0.000000,0.000000}%
\pgfsetstrokecolor{textcolor}%
\pgfsetfillcolor{textcolor}%
\pgftext[x=0.295138in,y=1.338719in,left,base]{\color{textcolor}\rmfamily\fontsize{11.000000}{13.200000}\selectfont \(\displaystyle -20\)}%
\end{pgfscope}%
\begin{pgfscope}%
\pgfsetbuttcap%
\pgfsetroundjoin%
\definecolor{currentfill}{rgb}{0.000000,0.000000,0.000000}%
\pgfsetfillcolor{currentfill}%
\pgfsetlinewidth{0.803000pt}%
\definecolor{currentstroke}{rgb}{0.000000,0.000000,0.000000}%
\pgfsetstrokecolor{currentstroke}%
\pgfsetdash{}{0pt}%
\pgfsys@defobject{currentmarker}{\pgfqpoint{-0.048611in}{0.000000in}}{\pgfqpoint{0.000000in}{0.000000in}}{%
\pgfpathmoveto{\pgfqpoint{0.000000in}{0.000000in}}%
\pgfpathlineto{\pgfqpoint{-0.048611in}{0.000000in}}%
\pgfusepath{stroke,fill}%
}%
\begin{pgfscope}%
\pgfsys@transformshift{0.662732in}{1.772831in}%
\pgfsys@useobject{currentmarker}{}%
\end{pgfscope}%
\end{pgfscope}%
\begin{pgfscope}%
\definecolor{textcolor}{rgb}{0.000000,0.000000,0.000000}%
\pgfsetstrokecolor{textcolor}%
\pgfsetfillcolor{textcolor}%
\pgftext[x=0.489468in,y=1.720025in,left,base]{\color{textcolor}\rmfamily\fontsize{11.000000}{13.200000}\selectfont \(\displaystyle 0\)}%
\end{pgfscope}%
\begin{pgfscope}%
\pgfsetbuttcap%
\pgfsetroundjoin%
\definecolor{currentfill}{rgb}{0.000000,0.000000,0.000000}%
\pgfsetfillcolor{currentfill}%
\pgfsetlinewidth{0.803000pt}%
\definecolor{currentstroke}{rgb}{0.000000,0.000000,0.000000}%
\pgfsetstrokecolor{currentstroke}%
\pgfsetdash{}{0pt}%
\pgfsys@defobject{currentmarker}{\pgfqpoint{-0.048611in}{0.000000in}}{\pgfqpoint{0.000000in}{0.000000in}}{%
\pgfpathmoveto{\pgfqpoint{0.000000in}{0.000000in}}%
\pgfpathlineto{\pgfqpoint{-0.048611in}{0.000000in}}%
\pgfusepath{stroke,fill}%
}%
\begin{pgfscope}%
\pgfsys@transformshift{0.662732in}{2.154137in}%
\pgfsys@useobject{currentmarker}{}%
\end{pgfscope}%
\end{pgfscope}%
\begin{pgfscope}%
\definecolor{textcolor}{rgb}{0.000000,0.000000,0.000000}%
\pgfsetstrokecolor{textcolor}%
\pgfsetfillcolor{textcolor}%
\pgftext[x=0.413426in,y=2.101330in,left,base]{\color{textcolor}\rmfamily\fontsize{11.000000}{13.200000}\selectfont \(\displaystyle 20\)}%
\end{pgfscope}%
\begin{pgfscope}%
\pgfsetbuttcap%
\pgfsetroundjoin%
\definecolor{currentfill}{rgb}{0.000000,0.000000,0.000000}%
\pgfsetfillcolor{currentfill}%
\pgfsetlinewidth{0.803000pt}%
\definecolor{currentstroke}{rgb}{0.000000,0.000000,0.000000}%
\pgfsetstrokecolor{currentstroke}%
\pgfsetdash{}{0pt}%
\pgfsys@defobject{currentmarker}{\pgfqpoint{-0.048611in}{0.000000in}}{\pgfqpoint{0.000000in}{0.000000in}}{%
\pgfpathmoveto{\pgfqpoint{0.000000in}{0.000000in}}%
\pgfpathlineto{\pgfqpoint{-0.048611in}{0.000000in}}%
\pgfusepath{stroke,fill}%
}%
\begin{pgfscope}%
\pgfsys@transformshift{0.662732in}{2.535443in}%
\pgfsys@useobject{currentmarker}{}%
\end{pgfscope}%
\end{pgfscope}%
\begin{pgfscope}%
\definecolor{textcolor}{rgb}{0.000000,0.000000,0.000000}%
\pgfsetstrokecolor{textcolor}%
\pgfsetfillcolor{textcolor}%
\pgftext[x=0.413426in,y=2.482636in,left,base]{\color{textcolor}\rmfamily\fontsize{11.000000}{13.200000}\selectfont \(\displaystyle 40\)}%
\end{pgfscope}%
\begin{pgfscope}%
\pgfsetbuttcap%
\pgfsetroundjoin%
\definecolor{currentfill}{rgb}{0.000000,0.000000,0.000000}%
\pgfsetfillcolor{currentfill}%
\pgfsetlinewidth{0.803000pt}%
\definecolor{currentstroke}{rgb}{0.000000,0.000000,0.000000}%
\pgfsetstrokecolor{currentstroke}%
\pgfsetdash{}{0pt}%
\pgfsys@defobject{currentmarker}{\pgfqpoint{-0.048611in}{0.000000in}}{\pgfqpoint{0.000000in}{0.000000in}}{%
\pgfpathmoveto{\pgfqpoint{0.000000in}{0.000000in}}%
\pgfpathlineto{\pgfqpoint{-0.048611in}{0.000000in}}%
\pgfusepath{stroke,fill}%
}%
\begin{pgfscope}%
\pgfsys@transformshift{0.662732in}{2.916748in}%
\pgfsys@useobject{currentmarker}{}%
\end{pgfscope}%
\end{pgfscope}%
\begin{pgfscope}%
\definecolor{textcolor}{rgb}{0.000000,0.000000,0.000000}%
\pgfsetstrokecolor{textcolor}%
\pgfsetfillcolor{textcolor}%
\pgftext[x=0.413426in,y=2.863942in,left,base]{\color{textcolor}\rmfamily\fontsize{11.000000}{13.200000}\selectfont \(\displaystyle 60\)}%
\end{pgfscope}%
\begin{pgfscope}%
\pgfsetbuttcap%
\pgfsetroundjoin%
\definecolor{currentfill}{rgb}{0.000000,0.000000,0.000000}%
\pgfsetfillcolor{currentfill}%
\pgfsetlinewidth{0.803000pt}%
\definecolor{currentstroke}{rgb}{0.000000,0.000000,0.000000}%
\pgfsetstrokecolor{currentstroke}%
\pgfsetdash{}{0pt}%
\pgfsys@defobject{currentmarker}{\pgfqpoint{-0.048611in}{0.000000in}}{\pgfqpoint{0.000000in}{0.000000in}}{%
\pgfpathmoveto{\pgfqpoint{0.000000in}{0.000000in}}%
\pgfpathlineto{\pgfqpoint{-0.048611in}{0.000000in}}%
\pgfusepath{stroke,fill}%
}%
\begin{pgfscope}%
\pgfsys@transformshift{0.662732in}{3.298054in}%
\pgfsys@useobject{currentmarker}{}%
\end{pgfscope}%
\end{pgfscope}%
\begin{pgfscope}%
\definecolor{textcolor}{rgb}{0.000000,0.000000,0.000000}%
\pgfsetstrokecolor{textcolor}%
\pgfsetfillcolor{textcolor}%
\pgftext[x=0.413426in,y=3.245247in,left,base]{\color{textcolor}\rmfamily\fontsize{11.000000}{13.200000}\selectfont \(\displaystyle 80\)}%
\end{pgfscope}%
\begin{pgfscope}%
\pgfsetbuttcap%
\pgfsetroundjoin%
\definecolor{currentfill}{rgb}{0.000000,0.000000,0.000000}%
\pgfsetfillcolor{currentfill}%
\pgfsetlinewidth{0.803000pt}%
\definecolor{currentstroke}{rgb}{0.000000,0.000000,0.000000}%
\pgfsetstrokecolor{currentstroke}%
\pgfsetdash{}{0pt}%
\pgfsys@defobject{currentmarker}{\pgfqpoint{-0.048611in}{0.000000in}}{\pgfqpoint{0.000000in}{0.000000in}}{%
\pgfpathmoveto{\pgfqpoint{0.000000in}{0.000000in}}%
\pgfpathlineto{\pgfqpoint{-0.048611in}{0.000000in}}%
\pgfusepath{stroke,fill}%
}%
\begin{pgfscope}%
\pgfsys@transformshift{0.662732in}{3.679360in}%
\pgfsys@useobject{currentmarker}{}%
\end{pgfscope}%
\end{pgfscope}%
\begin{pgfscope}%
\definecolor{textcolor}{rgb}{0.000000,0.000000,0.000000}%
\pgfsetstrokecolor{textcolor}%
\pgfsetfillcolor{textcolor}%
\pgftext[x=0.337384in,y=3.626553in,left,base]{\color{textcolor}\rmfamily\fontsize{11.000000}{13.200000}\selectfont \(\displaystyle 100\)}%
\end{pgfscope}%
\begin{pgfscope}%
\definecolor{textcolor}{rgb}{0.000000,0.000000,0.000000}%
\pgfsetstrokecolor{textcolor}%
\pgfsetfillcolor{textcolor}%
\pgftext[x=0.239583in,y=2.393555in,,bottom,rotate=90.000000]{\color{textcolor}\rmfamily\fontsize{16.000000}{19.200000}\selectfont Net Worth (\$ 1000s)}%
\end{pgfscope}%
\begin{pgfscope}%
\pgfpathrectangle{\pgfqpoint{0.662732in}{0.883555in}}{\pgfqpoint{4.650000in}{3.020000in}}%
\pgfusepath{clip}%
\pgfsetrectcap%
\pgfsetroundjoin%
\pgfsetlinewidth{1.505625pt}%
\definecolor{currentstroke}{rgb}{1.000000,0.000000,0.000000}%
\pgfsetstrokecolor{currentstroke}%
\pgfsetdash{}{0pt}%
\pgfpathmoveto{\pgfqpoint{0.874095in}{2.951337in}}%
\pgfpathlineto{\pgfqpoint{1.578641in}{2.871121in}}%
\pgfpathlineto{\pgfqpoint{1.930913in}{3.184974in}}%
\pgfpathlineto{\pgfqpoint{2.283186in}{3.033806in}}%
\pgfpathlineto{\pgfqpoint{2.635459in}{3.092804in}}%
\pgfpathlineto{\pgfqpoint{2.987732in}{2.246319in}}%
\pgfpathlineto{\pgfqpoint{2.987732in}{2.246319in}}%
\pgfpathlineto{\pgfqpoint{3.340004in}{2.332554in}}%
\pgfpathlineto{\pgfqpoint{3.692277in}{2.304212in}}%
\pgfpathlineto{\pgfqpoint{4.044550in}{2.180966in}}%
\pgfpathlineto{\pgfqpoint{4.396822in}{2.029118in}}%
\pgfpathlineto{\pgfqpoint{5.101368in}{1.924130in}}%
\pgfusepath{stroke}%
\end{pgfscope}%
\begin{pgfscope}%
\pgfpathrectangle{\pgfqpoint{0.662732in}{0.883555in}}{\pgfqpoint{4.650000in}{3.020000in}}%
\pgfusepath{clip}%
\pgfsetbuttcap%
\pgfsetroundjoin%
\definecolor{currentfill}{rgb}{1.000000,0.000000,0.000000}%
\pgfsetfillcolor{currentfill}%
\pgfsetlinewidth{1.003750pt}%
\definecolor{currentstroke}{rgb}{1.000000,0.000000,0.000000}%
\pgfsetstrokecolor{currentstroke}%
\pgfsetdash{}{0pt}%
\pgfsys@defobject{currentmarker}{\pgfqpoint{-0.041667in}{-0.041667in}}{\pgfqpoint{0.041667in}{0.041667in}}{%
\pgfpathmoveto{\pgfqpoint{-0.041667in}{0.000000in}}%
\pgfpathlineto{\pgfqpoint{0.041667in}{0.000000in}}%
\pgfpathmoveto{\pgfqpoint{0.000000in}{-0.041667in}}%
\pgfpathlineto{\pgfqpoint{0.000000in}{0.041667in}}%
\pgfusepath{stroke,fill}%
}%
\begin{pgfscope}%
\pgfsys@transformshift{0.874095in}{2.951337in}%
\pgfsys@useobject{currentmarker}{}%
\end{pgfscope}%
\begin{pgfscope}%
\pgfsys@transformshift{1.578641in}{2.871121in}%
\pgfsys@useobject{currentmarker}{}%
\end{pgfscope}%
\begin{pgfscope}%
\pgfsys@transformshift{1.930913in}{3.184974in}%
\pgfsys@useobject{currentmarker}{}%
\end{pgfscope}%
\begin{pgfscope}%
\pgfsys@transformshift{2.283186in}{3.033806in}%
\pgfsys@useobject{currentmarker}{}%
\end{pgfscope}%
\begin{pgfscope}%
\pgfsys@transformshift{2.635459in}{3.092804in}%
\pgfsys@useobject{currentmarker}{}%
\end{pgfscope}%
\begin{pgfscope}%
\pgfsys@transformshift{2.987732in}{2.246319in}%
\pgfsys@useobject{currentmarker}{}%
\end{pgfscope}%
\begin{pgfscope}%
\pgfsys@transformshift{2.987732in}{2.246319in}%
\pgfsys@useobject{currentmarker}{}%
\end{pgfscope}%
\begin{pgfscope}%
\pgfsys@transformshift{3.340004in}{2.332554in}%
\pgfsys@useobject{currentmarker}{}%
\end{pgfscope}%
\begin{pgfscope}%
\pgfsys@transformshift{3.692277in}{2.304212in}%
\pgfsys@useobject{currentmarker}{}%
\end{pgfscope}%
\begin{pgfscope}%
\pgfsys@transformshift{4.044550in}{2.180966in}%
\pgfsys@useobject{currentmarker}{}%
\end{pgfscope}%
\begin{pgfscope}%
\pgfsys@transformshift{4.396822in}{2.029118in}%
\pgfsys@useobject{currentmarker}{}%
\end{pgfscope}%
\begin{pgfscope}%
\pgfsys@transformshift{5.101368in}{1.924130in}%
\pgfsys@useobject{currentmarker}{}%
\end{pgfscope}%
\end{pgfscope}%
\begin{pgfscope}%
\pgfpathrectangle{\pgfqpoint{0.662732in}{0.883555in}}{\pgfqpoint{4.650000in}{3.020000in}}%
\pgfusepath{clip}%
\pgfsetbuttcap%
\pgfsetroundjoin%
\pgfsetlinewidth{1.505625pt}%
\definecolor{currentstroke}{rgb}{0.000000,0.000000,1.000000}%
\pgfsetstrokecolor{currentstroke}%
\pgfsetdash{{5.550000pt}{2.400000pt}}{0.000000pt}%
\pgfpathmoveto{\pgfqpoint{0.874095in}{2.727035in}}%
\pgfpathlineto{\pgfqpoint{1.578641in}{2.943859in}}%
\pgfpathlineto{\pgfqpoint{1.930913in}{3.272006in}}%
\pgfpathlineto{\pgfqpoint{2.283186in}{2.956717in}}%
\pgfpathlineto{\pgfqpoint{2.635459in}{3.381021in}}%
\pgfpathlineto{\pgfqpoint{2.987732in}{2.254992in}}%
\pgfpathlineto{\pgfqpoint{2.987732in}{2.254992in}}%
\pgfpathlineto{\pgfqpoint{3.340004in}{2.623283in}}%
\pgfpathlineto{\pgfqpoint{3.692277in}{2.514990in}}%
\pgfpathlineto{\pgfqpoint{4.044550in}{2.538712in}}%
\pgfpathlineto{\pgfqpoint{4.396822in}{2.132876in}}%
\pgfpathlineto{\pgfqpoint{5.101368in}{1.722898in}}%
\pgfusepath{stroke}%
\end{pgfscope}%
\begin{pgfscope}%
\pgfpathrectangle{\pgfqpoint{0.662732in}{0.883555in}}{\pgfqpoint{4.650000in}{3.020000in}}%
\pgfusepath{clip}%
\pgfsetbuttcap%
\pgfsetroundjoin%
\definecolor{currentfill}{rgb}{0.000000,0.000000,1.000000}%
\pgfsetfillcolor{currentfill}%
\pgfsetlinewidth{1.003750pt}%
\definecolor{currentstroke}{rgb}{0.000000,0.000000,1.000000}%
\pgfsetstrokecolor{currentstroke}%
\pgfsetdash{}{0pt}%
\pgfsys@defobject{currentmarker}{\pgfqpoint{-0.041667in}{-0.041667in}}{\pgfqpoint{0.041667in}{0.041667in}}{%
\pgfpathmoveto{\pgfqpoint{-0.041667in}{-0.041667in}}%
\pgfpathlineto{\pgfqpoint{0.041667in}{0.041667in}}%
\pgfpathmoveto{\pgfqpoint{-0.041667in}{0.041667in}}%
\pgfpathlineto{\pgfqpoint{0.041667in}{-0.041667in}}%
\pgfusepath{stroke,fill}%
}%
\begin{pgfscope}%
\pgfsys@transformshift{0.874095in}{2.727035in}%
\pgfsys@useobject{currentmarker}{}%
\end{pgfscope}%
\begin{pgfscope}%
\pgfsys@transformshift{1.578641in}{2.943859in}%
\pgfsys@useobject{currentmarker}{}%
\end{pgfscope}%
\begin{pgfscope}%
\pgfsys@transformshift{1.930913in}{3.272006in}%
\pgfsys@useobject{currentmarker}{}%
\end{pgfscope}%
\begin{pgfscope}%
\pgfsys@transformshift{2.283186in}{2.956717in}%
\pgfsys@useobject{currentmarker}{}%
\end{pgfscope}%
\begin{pgfscope}%
\pgfsys@transformshift{2.635459in}{3.381021in}%
\pgfsys@useobject{currentmarker}{}%
\end{pgfscope}%
\begin{pgfscope}%
\pgfsys@transformshift{2.987732in}{2.254992in}%
\pgfsys@useobject{currentmarker}{}%
\end{pgfscope}%
\begin{pgfscope}%
\pgfsys@transformshift{2.987732in}{2.254992in}%
\pgfsys@useobject{currentmarker}{}%
\end{pgfscope}%
\begin{pgfscope}%
\pgfsys@transformshift{3.340004in}{2.623283in}%
\pgfsys@useobject{currentmarker}{}%
\end{pgfscope}%
\begin{pgfscope}%
\pgfsys@transformshift{3.692277in}{2.514990in}%
\pgfsys@useobject{currentmarker}{}%
\end{pgfscope}%
\begin{pgfscope}%
\pgfsys@transformshift{4.044550in}{2.538712in}%
\pgfsys@useobject{currentmarker}{}%
\end{pgfscope}%
\begin{pgfscope}%
\pgfsys@transformshift{4.396822in}{2.132876in}%
\pgfsys@useobject{currentmarker}{}%
\end{pgfscope}%
\begin{pgfscope}%
\pgfsys@transformshift{5.101368in}{1.722898in}%
\pgfsys@useobject{currentmarker}{}%
\end{pgfscope}%
\end{pgfscope}%
\begin{pgfscope}%
\pgfpathrectangle{\pgfqpoint{0.662732in}{0.883555in}}{\pgfqpoint{4.650000in}{3.020000in}}%
\pgfusepath{clip}%
\pgfsetrectcap%
\pgfsetroundjoin%
\pgfsetlinewidth{1.505625pt}%
\definecolor{currentstroke}{rgb}{0.000000,0.000000,0.000000}%
\pgfsetstrokecolor{currentstroke}%
\pgfsetdash{}{0pt}%
\pgfpathmoveto{\pgfqpoint{2.987732in}{0.883555in}}%
\pgfpathlineto{\pgfqpoint{2.987732in}{3.903555in}}%
\pgfusepath{stroke}%
\end{pgfscope}%
\begin{pgfscope}%
\pgfsetrectcap%
\pgfsetmiterjoin%
\pgfsetlinewidth{0.803000pt}%
\definecolor{currentstroke}{rgb}{0.000000,0.000000,0.000000}%
\pgfsetstrokecolor{currentstroke}%
\pgfsetdash{}{0pt}%
\pgfpathmoveto{\pgfqpoint{0.662732in}{0.883555in}}%
\pgfpathlineto{\pgfqpoint{0.662732in}{3.903555in}}%
\pgfusepath{stroke}%
\end{pgfscope}%
\begin{pgfscope}%
\pgfsetrectcap%
\pgfsetmiterjoin%
\pgfsetlinewidth{0.803000pt}%
\definecolor{currentstroke}{rgb}{0.000000,0.000000,0.000000}%
\pgfsetstrokecolor{currentstroke}%
\pgfsetdash{}{0pt}%
\pgfpathmoveto{\pgfqpoint{5.312732in}{0.883555in}}%
\pgfpathlineto{\pgfqpoint{5.312732in}{3.903555in}}%
\pgfusepath{stroke}%
\end{pgfscope}%
\begin{pgfscope}%
\pgfsetrectcap%
\pgfsetmiterjoin%
\pgfsetlinewidth{0.803000pt}%
\definecolor{currentstroke}{rgb}{0.000000,0.000000,0.000000}%
\pgfsetstrokecolor{currentstroke}%
\pgfsetdash{}{0pt}%
\pgfpathmoveto{\pgfqpoint{0.662732in}{0.883555in}}%
\pgfpathlineto{\pgfqpoint{5.312732in}{0.883555in}}%
\pgfusepath{stroke}%
\end{pgfscope}%
\begin{pgfscope}%
\pgfsetrectcap%
\pgfsetmiterjoin%
\pgfsetlinewidth{0.803000pt}%
\definecolor{currentstroke}{rgb}{0.000000,0.000000,0.000000}%
\pgfsetstrokecolor{currentstroke}%
\pgfsetdash{}{0pt}%
\pgfpathmoveto{\pgfqpoint{0.662732in}{3.903555in}}%
\pgfpathlineto{\pgfqpoint{5.312732in}{3.903555in}}%
\pgfusepath{stroke}%
\end{pgfscope}%
\begin{pgfscope}%
\pgfsetbuttcap%
\pgfsetmiterjoin%
\definecolor{currentfill}{rgb}{0.300000,0.300000,0.300000}%
\pgfsetfillcolor{currentfill}%
\pgfsetfillopacity{0.500000}%
\pgfsetlinewidth{1.003750pt}%
\definecolor{currentstroke}{rgb}{0.300000,0.300000,0.300000}%
\pgfsetstrokecolor{currentstroke}%
\pgfsetstrokeopacity{0.500000}%
\pgfsetdash{}{0pt}%
\pgfpathmoveto{\pgfqpoint{1.729184in}{-0.027778in}}%
\pgfpathlineto{\pgfqpoint{4.301834in}{-0.027778in}}%
\pgfpathquadraticcurveto{\pgfqpoint{4.340723in}{-0.027778in}}{\pgfqpoint{4.340723in}{0.011111in}}%
\pgfpathlineto{\pgfqpoint{4.340723in}{0.266666in}}%
\pgfpathquadraticcurveto{\pgfqpoint{4.340723in}{0.305555in}}{\pgfqpoint{4.301834in}{0.305555in}}%
\pgfpathlineto{\pgfqpoint{1.729184in}{0.305555in}}%
\pgfpathquadraticcurveto{\pgfqpoint{1.690295in}{0.305555in}}{\pgfqpoint{1.690295in}{0.266666in}}%
\pgfpathlineto{\pgfqpoint{1.690295in}{0.011111in}}%
\pgfpathquadraticcurveto{\pgfqpoint{1.690295in}{-0.027778in}}{\pgfqpoint{1.729184in}{-0.027778in}}%
\pgfpathclose%
\pgfusepath{stroke,fill}%
\end{pgfscope}%
\begin{pgfscope}%
\pgfsetbuttcap%
\pgfsetmiterjoin%
\definecolor{currentfill}{rgb}{1.000000,1.000000,1.000000}%
\pgfsetfillcolor{currentfill}%
\pgfsetlinewidth{1.003750pt}%
\definecolor{currentstroke}{rgb}{0.800000,0.800000,0.800000}%
\pgfsetstrokecolor{currentstroke}%
\pgfsetdash{}{0pt}%
\pgfpathmoveto{\pgfqpoint{1.701407in}{0.000000in}}%
\pgfpathlineto{\pgfqpoint{4.274057in}{0.000000in}}%
\pgfpathquadraticcurveto{\pgfqpoint{4.312945in}{0.000000in}}{\pgfqpoint{4.312945in}{0.038889in}}%
\pgfpathlineto{\pgfqpoint{4.312945in}{0.294444in}}%
\pgfpathquadraticcurveto{\pgfqpoint{4.312945in}{0.333333in}}{\pgfqpoint{4.274057in}{0.333333in}}%
\pgfpathlineto{\pgfqpoint{1.701407in}{0.333333in}}%
\pgfpathquadraticcurveto{\pgfqpoint{1.662518in}{0.333333in}}{\pgfqpoint{1.662518in}{0.294444in}}%
\pgfpathlineto{\pgfqpoint{1.662518in}{0.038889in}}%
\pgfpathquadraticcurveto{\pgfqpoint{1.662518in}{0.000000in}}{\pgfqpoint{1.701407in}{0.000000in}}%
\pgfpathclose%
\pgfusepath{stroke,fill}%
\end{pgfscope}%
\begin{pgfscope}%
\pgfsetrectcap%
\pgfsetroundjoin%
\pgfsetlinewidth{1.505625pt}%
\definecolor{currentstroke}{rgb}{1.000000,0.000000,0.000000}%
\pgfsetstrokecolor{currentstroke}%
\pgfsetdash{}{0pt}%
\pgfpathmoveto{\pgfqpoint{1.740295in}{0.184722in}}%
\pgfpathlineto{\pgfqpoint{2.129184in}{0.184722in}}%
\pgfusepath{stroke}%
\end{pgfscope}%
\begin{pgfscope}%
\pgfsetbuttcap%
\pgfsetroundjoin%
\definecolor{currentfill}{rgb}{1.000000,0.000000,0.000000}%
\pgfsetfillcolor{currentfill}%
\pgfsetlinewidth{1.003750pt}%
\definecolor{currentstroke}{rgb}{1.000000,0.000000,0.000000}%
\pgfsetstrokecolor{currentstroke}%
\pgfsetdash{}{0pt}%
\pgfsys@defobject{currentmarker}{\pgfqpoint{-0.041667in}{-0.041667in}}{\pgfqpoint{0.041667in}{0.041667in}}{%
\pgfpathmoveto{\pgfqpoint{-0.041667in}{0.000000in}}%
\pgfpathlineto{\pgfqpoint{0.041667in}{0.000000in}}%
\pgfpathmoveto{\pgfqpoint{0.000000in}{-0.041667in}}%
\pgfpathlineto{\pgfqpoint{0.000000in}{0.041667in}}%
\pgfusepath{stroke,fill}%
}%
\begin{pgfscope}%
\pgfsys@transformshift{1.934740in}{0.184722in}%
\pgfsys@useobject{currentmarker}{}%
\end{pgfscope}%
\end{pgfscope}%
\begin{pgfscope}%
\definecolor{textcolor}{rgb}{0.000000,0.000000,0.000000}%
\pgfsetstrokecolor{textcolor}%
\pgfsetfillcolor{textcolor}%
\pgftext[x=2.284740in,y=0.116667in,left,base]{\color{textcolor}\rmfamily\fontsize{14.000000}{16.800000}\selectfont Women}%
\end{pgfscope}%
\begin{pgfscope}%
\pgfsetbuttcap%
\pgfsetroundjoin%
\pgfsetlinewidth{1.505625pt}%
\definecolor{currentstroke}{rgb}{0.000000,0.000000,1.000000}%
\pgfsetstrokecolor{currentstroke}%
\pgfsetdash{{5.550000pt}{2.400000pt}}{0.000000pt}%
\pgfpathmoveto{\pgfqpoint{3.315450in}{0.184722in}}%
\pgfpathlineto{\pgfqpoint{3.704339in}{0.184722in}}%
\pgfusepath{stroke}%
\end{pgfscope}%
\begin{pgfscope}%
\pgfsetbuttcap%
\pgfsetroundjoin%
\definecolor{currentfill}{rgb}{0.000000,0.000000,1.000000}%
\pgfsetfillcolor{currentfill}%
\pgfsetlinewidth{1.003750pt}%
\definecolor{currentstroke}{rgb}{0.000000,0.000000,1.000000}%
\pgfsetstrokecolor{currentstroke}%
\pgfsetdash{}{0pt}%
\pgfsys@defobject{currentmarker}{\pgfqpoint{-0.041667in}{-0.041667in}}{\pgfqpoint{0.041667in}{0.041667in}}{%
\pgfpathmoveto{\pgfqpoint{-0.041667in}{-0.041667in}}%
\pgfpathlineto{\pgfqpoint{0.041667in}{0.041667in}}%
\pgfpathmoveto{\pgfqpoint{-0.041667in}{0.041667in}}%
\pgfpathlineto{\pgfqpoint{0.041667in}{-0.041667in}}%
\pgfusepath{stroke,fill}%
}%
\begin{pgfscope}%
\pgfsys@transformshift{3.509894in}{0.184722in}%
\pgfsys@useobject{currentmarker}{}%
\end{pgfscope}%
\end{pgfscope}%
\begin{pgfscope}%
\definecolor{textcolor}{rgb}{0.000000,0.000000,0.000000}%
\pgfsetstrokecolor{textcolor}%
\pgfsetfillcolor{textcolor}%
\pgftext[x=3.859894in,y=0.116667in,left,base]{\color{textcolor}\rmfamily\fontsize{14.000000}{16.800000}\selectfont Men}%
\end{pgfscope}%
\end{pgfpicture}%
\makeatother%
\endgroup%
 } 
\end{subfigure}

\hspace{20em}


\begin{subfigure}{.49\textwidth}
\centering
% include second image
\caption{Net worth---rich and poor households}
\label{sf:assdiv3}
\scalebox{0.5}{%% Creator: Matplotlib, PGF backend
%%
%% To include the figure in your LaTeX document, write
%%   \input{<filename>.pgf}
%%
%% Make sure the required packages are loaded in your preamble
%%   \usepackage{pgf}
%%
%% Figures using additional raster images can only be included by \input if
%% they are in the same directory as the main LaTeX file. For loading figures
%% from other directories you can use the `import` package
%%   \usepackage{import}
%% and then include the figures with
%%   \import{<path to file>}{<filename>.pgf}
%%
%% Matplotlib used the following preamble
%%
\begingroup%
\makeatletter%
\begin{pgfpicture}%
\pgfpathrectangle{\pgfpointorigin}{\pgfqpoint{5.388773in}{3.903555in}}%
\pgfusepath{use as bounding box, clip}%
\begin{pgfscope}%
\pgfsetbuttcap%
\pgfsetmiterjoin%
\definecolor{currentfill}{rgb}{1.000000,1.000000,1.000000}%
\pgfsetfillcolor{currentfill}%
\pgfsetlinewidth{0.000000pt}%
\definecolor{currentstroke}{rgb}{1.000000,1.000000,1.000000}%
\pgfsetstrokecolor{currentstroke}%
\pgfsetdash{}{0pt}%
\pgfpathmoveto{\pgfqpoint{0.000000in}{0.000000in}}%
\pgfpathlineto{\pgfqpoint{5.388773in}{0.000000in}}%
\pgfpathlineto{\pgfqpoint{5.388773in}{3.903555in}}%
\pgfpathlineto{\pgfqpoint{0.000000in}{3.903555in}}%
\pgfpathclose%
\pgfusepath{fill}%
\end{pgfscope}%
\begin{pgfscope}%
\pgfsetbuttcap%
\pgfsetmiterjoin%
\definecolor{currentfill}{rgb}{1.000000,1.000000,1.000000}%
\pgfsetfillcolor{currentfill}%
\pgfsetlinewidth{0.000000pt}%
\definecolor{currentstroke}{rgb}{0.000000,0.000000,0.000000}%
\pgfsetstrokecolor{currentstroke}%
\pgfsetstrokeopacity{0.000000}%
\pgfsetdash{}{0pt}%
\pgfpathmoveto{\pgfqpoint{0.738773in}{0.883555in}}%
\pgfpathlineto{\pgfqpoint{5.388773in}{0.883555in}}%
\pgfpathlineto{\pgfqpoint{5.388773in}{3.903555in}}%
\pgfpathlineto{\pgfqpoint{0.738773in}{3.903555in}}%
\pgfpathclose%
\pgfusepath{fill}%
\end{pgfscope}%
\begin{pgfscope}%
\pgfpathrectangle{\pgfqpoint{0.738773in}{0.883555in}}{\pgfqpoint{4.650000in}{3.020000in}}%
\pgfusepath{clip}%
\pgfsetbuttcap%
\pgfsetroundjoin%
\definecolor{currentfill}{rgb}{1.000000,0.000000,0.000000}%
\pgfsetfillcolor{currentfill}%
\pgfsetfillopacity{0.200000}%
\pgfsetlinewidth{0.000000pt}%
\definecolor{currentstroke}{rgb}{0.000000,0.000000,0.000000}%
\pgfsetstrokecolor{currentstroke}%
\pgfsetdash{}{0pt}%
\pgfpathmoveto{\pgfqpoint{0.950137in}{2.102043in}}%
\pgfpathlineto{\pgfqpoint{0.950137in}{1.925995in}}%
\pgfpathlineto{\pgfqpoint{1.795592in}{1.940758in}}%
\pgfpathlineto{\pgfqpoint{2.641046in}{1.943732in}}%
\pgfpathlineto{\pgfqpoint{3.486501in}{1.954709in}}%
\pgfpathlineto{\pgfqpoint{4.331955in}{1.958944in}}%
\pgfpathlineto{\pgfqpoint{5.177410in}{1.939869in}}%
\pgfpathlineto{\pgfqpoint{5.177410in}{2.101676in}}%
\pgfpathlineto{\pgfqpoint{5.177410in}{2.101676in}}%
\pgfpathlineto{\pgfqpoint{4.331955in}{2.122458in}}%
\pgfpathlineto{\pgfqpoint{3.486501in}{2.102841in}}%
\pgfpathlineto{\pgfqpoint{2.641046in}{2.100710in}}%
\pgfpathlineto{\pgfqpoint{1.795592in}{2.105237in}}%
\pgfpathlineto{\pgfqpoint{0.950137in}{2.102043in}}%
\pgfpathclose%
\pgfusepath{fill}%
\end{pgfscope}%
\begin{pgfscope}%
\pgfpathrectangle{\pgfqpoint{0.738773in}{0.883555in}}{\pgfqpoint{4.650000in}{3.020000in}}%
\pgfusepath{clip}%
\pgfsetbuttcap%
\pgfsetroundjoin%
\definecolor{currentfill}{rgb}{0.000000,0.000000,1.000000}%
\pgfsetfillcolor{currentfill}%
\pgfsetfillopacity{0.200000}%
\pgfsetlinewidth{0.000000pt}%
\definecolor{currentstroke}{rgb}{0.000000,0.000000,0.000000}%
\pgfsetstrokecolor{currentstroke}%
\pgfsetdash{}{0pt}%
\pgfpathmoveto{\pgfqpoint{0.950137in}{3.386788in}}%
\pgfpathlineto{\pgfqpoint{0.950137in}{1.315709in}}%
\pgfpathlineto{\pgfqpoint{1.795592in}{1.309778in}}%
\pgfpathlineto{\pgfqpoint{2.641046in}{1.560150in}}%
\pgfpathlineto{\pgfqpoint{3.486501in}{1.055083in}}%
\pgfpathlineto{\pgfqpoint{4.331955in}{1.514938in}}%
\pgfpathlineto{\pgfqpoint{5.177410in}{1.020828in}}%
\pgfpathlineto{\pgfqpoint{5.177410in}{3.289330in}}%
\pgfpathlineto{\pgfqpoint{5.177410in}{3.289330in}}%
\pgfpathlineto{\pgfqpoint{4.331955in}{3.766282in}}%
\pgfpathlineto{\pgfqpoint{3.486501in}{3.057333in}}%
\pgfpathlineto{\pgfqpoint{2.641046in}{3.525075in}}%
\pgfpathlineto{\pgfqpoint{1.795592in}{3.318099in}}%
\pgfpathlineto{\pgfqpoint{0.950137in}{3.386788in}}%
\pgfpathclose%
\pgfusepath{fill}%
\end{pgfscope}%
\begin{pgfscope}%
\pgfsetbuttcap%
\pgfsetroundjoin%
\definecolor{currentfill}{rgb}{0.000000,0.000000,0.000000}%
\pgfsetfillcolor{currentfill}%
\pgfsetlinewidth{0.803000pt}%
\definecolor{currentstroke}{rgb}{0.000000,0.000000,0.000000}%
\pgfsetstrokecolor{currentstroke}%
\pgfsetdash{}{0pt}%
\pgfsys@defobject{currentmarker}{\pgfqpoint{0.000000in}{-0.048611in}}{\pgfqpoint{0.000000in}{0.000000in}}{%
\pgfpathmoveto{\pgfqpoint{0.000000in}{0.000000in}}%
\pgfpathlineto{\pgfqpoint{0.000000in}{-0.048611in}}%
\pgfusepath{stroke,fill}%
}%
\begin{pgfscope}%
\pgfsys@transformshift{0.950137in}{0.883555in}%
\pgfsys@useobject{currentmarker}{}%
\end{pgfscope}%
\end{pgfscope}%
\begin{pgfscope}%
\definecolor{textcolor}{rgb}{0.000000,0.000000,0.000000}%
\pgfsetstrokecolor{textcolor}%
\pgfsetfillcolor{textcolor}%
\pgftext[x=0.950137in,y=0.786333in,,top]{\color{textcolor}\rmfamily\fontsize{11.000000}{13.200000}\selectfont \(\displaystyle -6\)}%
\end{pgfscope}%
\begin{pgfscope}%
\pgfsetbuttcap%
\pgfsetroundjoin%
\definecolor{currentfill}{rgb}{0.000000,0.000000,0.000000}%
\pgfsetfillcolor{currentfill}%
\pgfsetlinewidth{0.803000pt}%
\definecolor{currentstroke}{rgb}{0.000000,0.000000,0.000000}%
\pgfsetstrokecolor{currentstroke}%
\pgfsetdash{}{0pt}%
\pgfsys@defobject{currentmarker}{\pgfqpoint{0.000000in}{-0.048611in}}{\pgfqpoint{0.000000in}{0.000000in}}{%
\pgfpathmoveto{\pgfqpoint{0.000000in}{0.000000in}}%
\pgfpathlineto{\pgfqpoint{0.000000in}{-0.048611in}}%
\pgfusepath{stroke,fill}%
}%
\begin{pgfscope}%
\pgfsys@transformshift{1.795592in}{0.883555in}%
\pgfsys@useobject{currentmarker}{}%
\end{pgfscope}%
\end{pgfscope}%
\begin{pgfscope}%
\definecolor{textcolor}{rgb}{0.000000,0.000000,0.000000}%
\pgfsetstrokecolor{textcolor}%
\pgfsetfillcolor{textcolor}%
\pgftext[x=1.795592in,y=0.786333in,,top]{\color{textcolor}\rmfamily\fontsize{11.000000}{13.200000}\selectfont \(\displaystyle -4\)}%
\end{pgfscope}%
\begin{pgfscope}%
\pgfsetbuttcap%
\pgfsetroundjoin%
\definecolor{currentfill}{rgb}{0.000000,0.000000,0.000000}%
\pgfsetfillcolor{currentfill}%
\pgfsetlinewidth{0.803000pt}%
\definecolor{currentstroke}{rgb}{0.000000,0.000000,0.000000}%
\pgfsetstrokecolor{currentstroke}%
\pgfsetdash{}{0pt}%
\pgfsys@defobject{currentmarker}{\pgfqpoint{0.000000in}{-0.048611in}}{\pgfqpoint{0.000000in}{0.000000in}}{%
\pgfpathmoveto{\pgfqpoint{0.000000in}{0.000000in}}%
\pgfpathlineto{\pgfqpoint{0.000000in}{-0.048611in}}%
\pgfusepath{stroke,fill}%
}%
\begin{pgfscope}%
\pgfsys@transformshift{2.641046in}{0.883555in}%
\pgfsys@useobject{currentmarker}{}%
\end{pgfscope}%
\end{pgfscope}%
\begin{pgfscope}%
\definecolor{textcolor}{rgb}{0.000000,0.000000,0.000000}%
\pgfsetstrokecolor{textcolor}%
\pgfsetfillcolor{textcolor}%
\pgftext[x=2.641046in,y=0.786333in,,top]{\color{textcolor}\rmfamily\fontsize{11.000000}{13.200000}\selectfont \(\displaystyle -2\)}%
\end{pgfscope}%
\begin{pgfscope}%
\pgfsetbuttcap%
\pgfsetroundjoin%
\definecolor{currentfill}{rgb}{0.000000,0.000000,0.000000}%
\pgfsetfillcolor{currentfill}%
\pgfsetlinewidth{0.803000pt}%
\definecolor{currentstroke}{rgb}{0.000000,0.000000,0.000000}%
\pgfsetstrokecolor{currentstroke}%
\pgfsetdash{}{0pt}%
\pgfsys@defobject{currentmarker}{\pgfqpoint{0.000000in}{-0.048611in}}{\pgfqpoint{0.000000in}{0.000000in}}{%
\pgfpathmoveto{\pgfqpoint{0.000000in}{0.000000in}}%
\pgfpathlineto{\pgfqpoint{0.000000in}{-0.048611in}}%
\pgfusepath{stroke,fill}%
}%
\begin{pgfscope}%
\pgfsys@transformshift{3.486501in}{0.883555in}%
\pgfsys@useobject{currentmarker}{}%
\end{pgfscope}%
\end{pgfscope}%
\begin{pgfscope}%
\definecolor{textcolor}{rgb}{0.000000,0.000000,0.000000}%
\pgfsetstrokecolor{textcolor}%
\pgfsetfillcolor{textcolor}%
\pgftext[x=3.486501in,y=0.786333in,,top]{\color{textcolor}\rmfamily\fontsize{11.000000}{13.200000}\selectfont \(\displaystyle 0\)}%
\end{pgfscope}%
\begin{pgfscope}%
\pgfsetbuttcap%
\pgfsetroundjoin%
\definecolor{currentfill}{rgb}{0.000000,0.000000,0.000000}%
\pgfsetfillcolor{currentfill}%
\pgfsetlinewidth{0.803000pt}%
\definecolor{currentstroke}{rgb}{0.000000,0.000000,0.000000}%
\pgfsetstrokecolor{currentstroke}%
\pgfsetdash{}{0pt}%
\pgfsys@defobject{currentmarker}{\pgfqpoint{0.000000in}{-0.048611in}}{\pgfqpoint{0.000000in}{0.000000in}}{%
\pgfpathmoveto{\pgfqpoint{0.000000in}{0.000000in}}%
\pgfpathlineto{\pgfqpoint{0.000000in}{-0.048611in}}%
\pgfusepath{stroke,fill}%
}%
\begin{pgfscope}%
\pgfsys@transformshift{4.331955in}{0.883555in}%
\pgfsys@useobject{currentmarker}{}%
\end{pgfscope}%
\end{pgfscope}%
\begin{pgfscope}%
\definecolor{textcolor}{rgb}{0.000000,0.000000,0.000000}%
\pgfsetstrokecolor{textcolor}%
\pgfsetfillcolor{textcolor}%
\pgftext[x=4.331955in,y=0.786333in,,top]{\color{textcolor}\rmfamily\fontsize{11.000000}{13.200000}\selectfont \(\displaystyle 2\)}%
\end{pgfscope}%
\begin{pgfscope}%
\pgfsetbuttcap%
\pgfsetroundjoin%
\definecolor{currentfill}{rgb}{0.000000,0.000000,0.000000}%
\pgfsetfillcolor{currentfill}%
\pgfsetlinewidth{0.803000pt}%
\definecolor{currentstroke}{rgb}{0.000000,0.000000,0.000000}%
\pgfsetstrokecolor{currentstroke}%
\pgfsetdash{}{0pt}%
\pgfsys@defobject{currentmarker}{\pgfqpoint{0.000000in}{-0.048611in}}{\pgfqpoint{0.000000in}{0.000000in}}{%
\pgfpathmoveto{\pgfqpoint{0.000000in}{0.000000in}}%
\pgfpathlineto{\pgfqpoint{0.000000in}{-0.048611in}}%
\pgfusepath{stroke,fill}%
}%
\begin{pgfscope}%
\pgfsys@transformshift{5.177410in}{0.883555in}%
\pgfsys@useobject{currentmarker}{}%
\end{pgfscope}%
\end{pgfscope}%
\begin{pgfscope}%
\definecolor{textcolor}{rgb}{0.000000,0.000000,0.000000}%
\pgfsetstrokecolor{textcolor}%
\pgfsetfillcolor{textcolor}%
\pgftext[x=5.177410in,y=0.786333in,,top]{\color{textcolor}\rmfamily\fontsize{11.000000}{13.200000}\selectfont \(\displaystyle 4\)}%
\end{pgfscope}%
\begin{pgfscope}%
\definecolor{textcolor}{rgb}{0.000000,0.000000,0.000000}%
\pgfsetstrokecolor{textcolor}%
\pgfsetfillcolor{textcolor}%
\pgftext[x=3.063773in,y=0.595592in,,top]{\color{textcolor}\rmfamily\fontsize{16.000000}{19.200000}\selectfont Event time (Years)}%
\end{pgfscope}%
\begin{pgfscope}%
\pgfsetbuttcap%
\pgfsetroundjoin%
\definecolor{currentfill}{rgb}{0.000000,0.000000,0.000000}%
\pgfsetfillcolor{currentfill}%
\pgfsetlinewidth{0.803000pt}%
\definecolor{currentstroke}{rgb}{0.000000,0.000000,0.000000}%
\pgfsetstrokecolor{currentstroke}%
\pgfsetdash{}{0pt}%
\pgfsys@defobject{currentmarker}{\pgfqpoint{-0.048611in}{0.000000in}}{\pgfqpoint{0.000000in}{0.000000in}}{%
\pgfpathmoveto{\pgfqpoint{0.000000in}{0.000000in}}%
\pgfpathlineto{\pgfqpoint{-0.048611in}{0.000000in}}%
\pgfusepath{stroke,fill}%
}%
\begin{pgfscope}%
\pgfsys@transformshift{0.738773in}{1.239280in}%
\pgfsys@useobject{currentmarker}{}%
\end{pgfscope}%
\end{pgfscope}%
\begin{pgfscope}%
\definecolor{textcolor}{rgb}{0.000000,0.000000,0.000000}%
\pgfsetstrokecolor{textcolor}%
\pgfsetfillcolor{textcolor}%
\pgftext[x=0.295138in,y=1.186474in,left,base]{\color{textcolor}\rmfamily\fontsize{11.000000}{13.200000}\selectfont \(\displaystyle -100\)}%
\end{pgfscope}%
\begin{pgfscope}%
\pgfsetbuttcap%
\pgfsetroundjoin%
\definecolor{currentfill}{rgb}{0.000000,0.000000,0.000000}%
\pgfsetfillcolor{currentfill}%
\pgfsetlinewidth{0.803000pt}%
\definecolor{currentstroke}{rgb}{0.000000,0.000000,0.000000}%
\pgfsetstrokecolor{currentstroke}%
\pgfsetdash{}{0pt}%
\pgfsys@defobject{currentmarker}{\pgfqpoint{-0.048611in}{0.000000in}}{\pgfqpoint{0.000000in}{0.000000in}}{%
\pgfpathmoveto{\pgfqpoint{0.000000in}{0.000000in}}%
\pgfpathlineto{\pgfqpoint{-0.048611in}{0.000000in}}%
\pgfusepath{stroke,fill}%
}%
\begin{pgfscope}%
\pgfsys@transformshift{0.738773in}{1.605136in}%
\pgfsys@useobject{currentmarker}{}%
\end{pgfscope}%
\end{pgfscope}%
\begin{pgfscope}%
\definecolor{textcolor}{rgb}{0.000000,0.000000,0.000000}%
\pgfsetstrokecolor{textcolor}%
\pgfsetfillcolor{textcolor}%
\pgftext[x=0.371180in,y=1.552330in,left,base]{\color{textcolor}\rmfamily\fontsize{11.000000}{13.200000}\selectfont \(\displaystyle -50\)}%
\end{pgfscope}%
\begin{pgfscope}%
\pgfsetbuttcap%
\pgfsetroundjoin%
\definecolor{currentfill}{rgb}{0.000000,0.000000,0.000000}%
\pgfsetfillcolor{currentfill}%
\pgfsetlinewidth{0.803000pt}%
\definecolor{currentstroke}{rgb}{0.000000,0.000000,0.000000}%
\pgfsetstrokecolor{currentstroke}%
\pgfsetdash{}{0pt}%
\pgfsys@defobject{currentmarker}{\pgfqpoint{-0.048611in}{0.000000in}}{\pgfqpoint{0.000000in}{0.000000in}}{%
\pgfpathmoveto{\pgfqpoint{0.000000in}{0.000000in}}%
\pgfpathlineto{\pgfqpoint{-0.048611in}{0.000000in}}%
\pgfusepath{stroke,fill}%
}%
\begin{pgfscope}%
\pgfsys@transformshift{0.738773in}{1.970993in}%
\pgfsys@useobject{currentmarker}{}%
\end{pgfscope}%
\end{pgfscope}%
\begin{pgfscope}%
\definecolor{textcolor}{rgb}{0.000000,0.000000,0.000000}%
\pgfsetstrokecolor{textcolor}%
\pgfsetfillcolor{textcolor}%
\pgftext[x=0.565509in,y=1.918186in,left,base]{\color{textcolor}\rmfamily\fontsize{11.000000}{13.200000}\selectfont \(\displaystyle 0\)}%
\end{pgfscope}%
\begin{pgfscope}%
\pgfsetbuttcap%
\pgfsetroundjoin%
\definecolor{currentfill}{rgb}{0.000000,0.000000,0.000000}%
\pgfsetfillcolor{currentfill}%
\pgfsetlinewidth{0.803000pt}%
\definecolor{currentstroke}{rgb}{0.000000,0.000000,0.000000}%
\pgfsetstrokecolor{currentstroke}%
\pgfsetdash{}{0pt}%
\pgfsys@defobject{currentmarker}{\pgfqpoint{-0.048611in}{0.000000in}}{\pgfqpoint{0.000000in}{0.000000in}}{%
\pgfpathmoveto{\pgfqpoint{0.000000in}{0.000000in}}%
\pgfpathlineto{\pgfqpoint{-0.048611in}{0.000000in}}%
\pgfusepath{stroke,fill}%
}%
\begin{pgfscope}%
\pgfsys@transformshift{0.738773in}{2.336849in}%
\pgfsys@useobject{currentmarker}{}%
\end{pgfscope}%
\end{pgfscope}%
\begin{pgfscope}%
\definecolor{textcolor}{rgb}{0.000000,0.000000,0.000000}%
\pgfsetstrokecolor{textcolor}%
\pgfsetfillcolor{textcolor}%
\pgftext[x=0.489468in,y=2.284042in,left,base]{\color{textcolor}\rmfamily\fontsize{11.000000}{13.200000}\selectfont \(\displaystyle 50\)}%
\end{pgfscope}%
\begin{pgfscope}%
\pgfsetbuttcap%
\pgfsetroundjoin%
\definecolor{currentfill}{rgb}{0.000000,0.000000,0.000000}%
\pgfsetfillcolor{currentfill}%
\pgfsetlinewidth{0.803000pt}%
\definecolor{currentstroke}{rgb}{0.000000,0.000000,0.000000}%
\pgfsetstrokecolor{currentstroke}%
\pgfsetdash{}{0pt}%
\pgfsys@defobject{currentmarker}{\pgfqpoint{-0.048611in}{0.000000in}}{\pgfqpoint{0.000000in}{0.000000in}}{%
\pgfpathmoveto{\pgfqpoint{0.000000in}{0.000000in}}%
\pgfpathlineto{\pgfqpoint{-0.048611in}{0.000000in}}%
\pgfusepath{stroke,fill}%
}%
\begin{pgfscope}%
\pgfsys@transformshift{0.738773in}{2.702705in}%
\pgfsys@useobject{currentmarker}{}%
\end{pgfscope}%
\end{pgfscope}%
\begin{pgfscope}%
\definecolor{textcolor}{rgb}{0.000000,0.000000,0.000000}%
\pgfsetstrokecolor{textcolor}%
\pgfsetfillcolor{textcolor}%
\pgftext[x=0.413426in,y=2.649898in,left,base]{\color{textcolor}\rmfamily\fontsize{11.000000}{13.200000}\selectfont \(\displaystyle 100\)}%
\end{pgfscope}%
\begin{pgfscope}%
\pgfsetbuttcap%
\pgfsetroundjoin%
\definecolor{currentfill}{rgb}{0.000000,0.000000,0.000000}%
\pgfsetfillcolor{currentfill}%
\pgfsetlinewidth{0.803000pt}%
\definecolor{currentstroke}{rgb}{0.000000,0.000000,0.000000}%
\pgfsetstrokecolor{currentstroke}%
\pgfsetdash{}{0pt}%
\pgfsys@defobject{currentmarker}{\pgfqpoint{-0.048611in}{0.000000in}}{\pgfqpoint{0.000000in}{0.000000in}}{%
\pgfpathmoveto{\pgfqpoint{0.000000in}{0.000000in}}%
\pgfpathlineto{\pgfqpoint{-0.048611in}{0.000000in}}%
\pgfusepath{stroke,fill}%
}%
\begin{pgfscope}%
\pgfsys@transformshift{0.738773in}{3.068561in}%
\pgfsys@useobject{currentmarker}{}%
\end{pgfscope}%
\end{pgfscope}%
\begin{pgfscope}%
\definecolor{textcolor}{rgb}{0.000000,0.000000,0.000000}%
\pgfsetstrokecolor{textcolor}%
\pgfsetfillcolor{textcolor}%
\pgftext[x=0.413426in,y=3.015754in,left,base]{\color{textcolor}\rmfamily\fontsize{11.000000}{13.200000}\selectfont \(\displaystyle 150\)}%
\end{pgfscope}%
\begin{pgfscope}%
\pgfsetbuttcap%
\pgfsetroundjoin%
\definecolor{currentfill}{rgb}{0.000000,0.000000,0.000000}%
\pgfsetfillcolor{currentfill}%
\pgfsetlinewidth{0.803000pt}%
\definecolor{currentstroke}{rgb}{0.000000,0.000000,0.000000}%
\pgfsetstrokecolor{currentstroke}%
\pgfsetdash{}{0pt}%
\pgfsys@defobject{currentmarker}{\pgfqpoint{-0.048611in}{0.000000in}}{\pgfqpoint{0.000000in}{0.000000in}}{%
\pgfpathmoveto{\pgfqpoint{0.000000in}{0.000000in}}%
\pgfpathlineto{\pgfqpoint{-0.048611in}{0.000000in}}%
\pgfusepath{stroke,fill}%
}%
\begin{pgfscope}%
\pgfsys@transformshift{0.738773in}{3.434417in}%
\pgfsys@useobject{currentmarker}{}%
\end{pgfscope}%
\end{pgfscope}%
\begin{pgfscope}%
\definecolor{textcolor}{rgb}{0.000000,0.000000,0.000000}%
\pgfsetstrokecolor{textcolor}%
\pgfsetfillcolor{textcolor}%
\pgftext[x=0.413426in,y=3.381611in,left,base]{\color{textcolor}\rmfamily\fontsize{11.000000}{13.200000}\selectfont \(\displaystyle 200\)}%
\end{pgfscope}%
\begin{pgfscope}%
\pgfsetbuttcap%
\pgfsetroundjoin%
\definecolor{currentfill}{rgb}{0.000000,0.000000,0.000000}%
\pgfsetfillcolor{currentfill}%
\pgfsetlinewidth{0.803000pt}%
\definecolor{currentstroke}{rgb}{0.000000,0.000000,0.000000}%
\pgfsetstrokecolor{currentstroke}%
\pgfsetdash{}{0pt}%
\pgfsys@defobject{currentmarker}{\pgfqpoint{-0.048611in}{0.000000in}}{\pgfqpoint{0.000000in}{0.000000in}}{%
\pgfpathmoveto{\pgfqpoint{0.000000in}{0.000000in}}%
\pgfpathlineto{\pgfqpoint{-0.048611in}{0.000000in}}%
\pgfusepath{stroke,fill}%
}%
\begin{pgfscope}%
\pgfsys@transformshift{0.738773in}{3.800273in}%
\pgfsys@useobject{currentmarker}{}%
\end{pgfscope}%
\end{pgfscope}%
\begin{pgfscope}%
\definecolor{textcolor}{rgb}{0.000000,0.000000,0.000000}%
\pgfsetstrokecolor{textcolor}%
\pgfsetfillcolor{textcolor}%
\pgftext[x=0.413426in,y=3.747467in,left,base]{\color{textcolor}\rmfamily\fontsize{11.000000}{13.200000}\selectfont \(\displaystyle 250\)}%
\end{pgfscope}%
\begin{pgfscope}%
\definecolor{textcolor}{rgb}{0.000000,0.000000,0.000000}%
\pgfsetstrokecolor{textcolor}%
\pgfsetfillcolor{textcolor}%
\pgftext[x=0.239583in,y=2.393555in,,bottom,rotate=90.000000]{\color{textcolor}\rmfamily\fontsize{16.000000}{19.200000}\selectfont Net Worth (\$ 1000s)}%
\end{pgfscope}%
\begin{pgfscope}%
\pgfpathrectangle{\pgfqpoint{0.738773in}{0.883555in}}{\pgfqpoint{4.650000in}{3.020000in}}%
\pgfusepath{clip}%
\pgfsetrectcap%
\pgfsetroundjoin%
\pgfsetlinewidth{1.505625pt}%
\definecolor{currentstroke}{rgb}{1.000000,0.000000,0.000000}%
\pgfsetstrokecolor{currentstroke}%
\pgfsetdash{}{0pt}%
\pgfpathmoveto{\pgfqpoint{0.950137in}{2.014019in}}%
\pgfpathlineto{\pgfqpoint{1.795592in}{2.022997in}}%
\pgfpathlineto{\pgfqpoint{2.641046in}{2.022221in}}%
\pgfpathlineto{\pgfqpoint{3.486501in}{2.028775in}}%
\pgfpathlineto{\pgfqpoint{4.331955in}{2.040701in}}%
\pgfpathlineto{\pgfqpoint{5.177410in}{2.020772in}}%
\pgfusepath{stroke}%
\end{pgfscope}%
\begin{pgfscope}%
\pgfpathrectangle{\pgfqpoint{0.738773in}{0.883555in}}{\pgfqpoint{4.650000in}{3.020000in}}%
\pgfusepath{clip}%
\pgfsetbuttcap%
\pgfsetroundjoin%
\definecolor{currentfill}{rgb}{1.000000,0.000000,0.000000}%
\pgfsetfillcolor{currentfill}%
\pgfsetlinewidth{1.003750pt}%
\definecolor{currentstroke}{rgb}{1.000000,0.000000,0.000000}%
\pgfsetstrokecolor{currentstroke}%
\pgfsetdash{}{0pt}%
\pgfsys@defobject{currentmarker}{\pgfqpoint{-0.041667in}{-0.041667in}}{\pgfqpoint{0.041667in}{0.041667in}}{%
\pgfpathmoveto{\pgfqpoint{-0.041667in}{0.000000in}}%
\pgfpathlineto{\pgfqpoint{0.041667in}{0.000000in}}%
\pgfpathmoveto{\pgfqpoint{0.000000in}{-0.041667in}}%
\pgfpathlineto{\pgfqpoint{0.000000in}{0.041667in}}%
\pgfusepath{stroke,fill}%
}%
\begin{pgfscope}%
\pgfsys@transformshift{0.950137in}{2.014019in}%
\pgfsys@useobject{currentmarker}{}%
\end{pgfscope}%
\begin{pgfscope}%
\pgfsys@transformshift{1.795592in}{2.022997in}%
\pgfsys@useobject{currentmarker}{}%
\end{pgfscope}%
\begin{pgfscope}%
\pgfsys@transformshift{2.641046in}{2.022221in}%
\pgfsys@useobject{currentmarker}{}%
\end{pgfscope}%
\begin{pgfscope}%
\pgfsys@transformshift{3.486501in}{2.028775in}%
\pgfsys@useobject{currentmarker}{}%
\end{pgfscope}%
\begin{pgfscope}%
\pgfsys@transformshift{4.331955in}{2.040701in}%
\pgfsys@useobject{currentmarker}{}%
\end{pgfscope}%
\begin{pgfscope}%
\pgfsys@transformshift{5.177410in}{2.020772in}%
\pgfsys@useobject{currentmarker}{}%
\end{pgfscope}%
\end{pgfscope}%
\begin{pgfscope}%
\pgfpathrectangle{\pgfqpoint{0.738773in}{0.883555in}}{\pgfqpoint{4.650000in}{3.020000in}}%
\pgfusepath{clip}%
\pgfsetbuttcap%
\pgfsetroundjoin%
\pgfsetlinewidth{1.505625pt}%
\definecolor{currentstroke}{rgb}{0.000000,0.000000,1.000000}%
\pgfsetstrokecolor{currentstroke}%
\pgfsetdash{{5.550000pt}{2.400000pt}}{0.000000pt}%
\pgfpathmoveto{\pgfqpoint{0.950137in}{2.351249in}}%
\pgfpathlineto{\pgfqpoint{1.795592in}{2.313939in}}%
\pgfpathlineto{\pgfqpoint{2.641046in}{2.542613in}}%
\pgfpathlineto{\pgfqpoint{3.486501in}{2.056208in}}%
\pgfpathlineto{\pgfqpoint{4.331955in}{2.640610in}}%
\pgfpathlineto{\pgfqpoint{5.177410in}{2.155079in}}%
\pgfusepath{stroke}%
\end{pgfscope}%
\begin{pgfscope}%
\pgfpathrectangle{\pgfqpoint{0.738773in}{0.883555in}}{\pgfqpoint{4.650000in}{3.020000in}}%
\pgfusepath{clip}%
\pgfsetbuttcap%
\pgfsetroundjoin%
\definecolor{currentfill}{rgb}{0.000000,0.000000,1.000000}%
\pgfsetfillcolor{currentfill}%
\pgfsetlinewidth{1.003750pt}%
\definecolor{currentstroke}{rgb}{0.000000,0.000000,1.000000}%
\pgfsetstrokecolor{currentstroke}%
\pgfsetdash{}{0pt}%
\pgfsys@defobject{currentmarker}{\pgfqpoint{-0.041667in}{-0.041667in}}{\pgfqpoint{0.041667in}{0.041667in}}{%
\pgfpathmoveto{\pgfqpoint{-0.041667in}{-0.041667in}}%
\pgfpathlineto{\pgfqpoint{0.041667in}{0.041667in}}%
\pgfpathmoveto{\pgfqpoint{-0.041667in}{0.041667in}}%
\pgfpathlineto{\pgfqpoint{0.041667in}{-0.041667in}}%
\pgfusepath{stroke,fill}%
}%
\begin{pgfscope}%
\pgfsys@transformshift{0.950137in}{2.351249in}%
\pgfsys@useobject{currentmarker}{}%
\end{pgfscope}%
\begin{pgfscope}%
\pgfsys@transformshift{1.795592in}{2.313939in}%
\pgfsys@useobject{currentmarker}{}%
\end{pgfscope}%
\begin{pgfscope}%
\pgfsys@transformshift{2.641046in}{2.542613in}%
\pgfsys@useobject{currentmarker}{}%
\end{pgfscope}%
\begin{pgfscope}%
\pgfsys@transformshift{3.486501in}{2.056208in}%
\pgfsys@useobject{currentmarker}{}%
\end{pgfscope}%
\begin{pgfscope}%
\pgfsys@transformshift{4.331955in}{2.640610in}%
\pgfsys@useobject{currentmarker}{}%
\end{pgfscope}%
\begin{pgfscope}%
\pgfsys@transformshift{5.177410in}{2.155079in}%
\pgfsys@useobject{currentmarker}{}%
\end{pgfscope}%
\end{pgfscope}%
\begin{pgfscope}%
\pgfpathrectangle{\pgfqpoint{0.738773in}{0.883555in}}{\pgfqpoint{4.650000in}{3.020000in}}%
\pgfusepath{clip}%
\pgfsetrectcap%
\pgfsetroundjoin%
\pgfsetlinewidth{1.505625pt}%
\definecolor{currentstroke}{rgb}{0.000000,0.000000,0.000000}%
\pgfsetstrokecolor{currentstroke}%
\pgfsetdash{}{0pt}%
\pgfpathmoveto{\pgfqpoint{3.486501in}{0.883555in}}%
\pgfpathlineto{\pgfqpoint{3.486501in}{3.903555in}}%
\pgfusepath{stroke}%
\end{pgfscope}%
\begin{pgfscope}%
\pgfsetrectcap%
\pgfsetmiterjoin%
\pgfsetlinewidth{0.803000pt}%
\definecolor{currentstroke}{rgb}{0.000000,0.000000,0.000000}%
\pgfsetstrokecolor{currentstroke}%
\pgfsetdash{}{0pt}%
\pgfpathmoveto{\pgfqpoint{0.738773in}{0.883555in}}%
\pgfpathlineto{\pgfqpoint{0.738773in}{3.903555in}}%
\pgfusepath{stroke}%
\end{pgfscope}%
\begin{pgfscope}%
\pgfsetrectcap%
\pgfsetmiterjoin%
\pgfsetlinewidth{0.803000pt}%
\definecolor{currentstroke}{rgb}{0.000000,0.000000,0.000000}%
\pgfsetstrokecolor{currentstroke}%
\pgfsetdash{}{0pt}%
\pgfpathmoveto{\pgfqpoint{5.388773in}{0.883555in}}%
\pgfpathlineto{\pgfqpoint{5.388773in}{3.903555in}}%
\pgfusepath{stroke}%
\end{pgfscope}%
\begin{pgfscope}%
\pgfsetrectcap%
\pgfsetmiterjoin%
\pgfsetlinewidth{0.803000pt}%
\definecolor{currentstroke}{rgb}{0.000000,0.000000,0.000000}%
\pgfsetstrokecolor{currentstroke}%
\pgfsetdash{}{0pt}%
\pgfpathmoveto{\pgfqpoint{0.738773in}{0.883555in}}%
\pgfpathlineto{\pgfqpoint{5.388773in}{0.883555in}}%
\pgfusepath{stroke}%
\end{pgfscope}%
\begin{pgfscope}%
\pgfsetrectcap%
\pgfsetmiterjoin%
\pgfsetlinewidth{0.803000pt}%
\definecolor{currentstroke}{rgb}{0.000000,0.000000,0.000000}%
\pgfsetstrokecolor{currentstroke}%
\pgfsetdash{}{0pt}%
\pgfpathmoveto{\pgfqpoint{0.738773in}{3.903555in}}%
\pgfpathlineto{\pgfqpoint{5.388773in}{3.903555in}}%
\pgfusepath{stroke}%
\end{pgfscope}%
\begin{pgfscope}%
\pgfsetbuttcap%
\pgfsetmiterjoin%
\definecolor{currentfill}{rgb}{0.300000,0.300000,0.300000}%
\pgfsetfillcolor{currentfill}%
\pgfsetfillopacity{0.500000}%
\pgfsetlinewidth{1.003750pt}%
\definecolor{currentstroke}{rgb}{0.300000,0.300000,0.300000}%
\pgfsetstrokecolor{currentstroke}%
\pgfsetstrokeopacity{0.500000}%
\pgfsetdash{}{0pt}%
\pgfpathmoveto{\pgfqpoint{1.916706in}{-0.027778in}}%
\pgfpathlineto{\pgfqpoint{4.266396in}{-0.027778in}}%
\pgfpathquadraticcurveto{\pgfqpoint{4.305285in}{-0.027778in}}{\pgfqpoint{4.305285in}{0.011111in}}%
\pgfpathlineto{\pgfqpoint{4.305285in}{0.266666in}}%
\pgfpathquadraticcurveto{\pgfqpoint{4.305285in}{0.305555in}}{\pgfqpoint{4.266396in}{0.305555in}}%
\pgfpathlineto{\pgfqpoint{1.916706in}{0.305555in}}%
\pgfpathquadraticcurveto{\pgfqpoint{1.877818in}{0.305555in}}{\pgfqpoint{1.877818in}{0.266666in}}%
\pgfpathlineto{\pgfqpoint{1.877818in}{0.011111in}}%
\pgfpathquadraticcurveto{\pgfqpoint{1.877818in}{-0.027778in}}{\pgfqpoint{1.916706in}{-0.027778in}}%
\pgfpathclose%
\pgfusepath{stroke,fill}%
\end{pgfscope}%
\begin{pgfscope}%
\pgfsetbuttcap%
\pgfsetmiterjoin%
\definecolor{currentfill}{rgb}{1.000000,1.000000,1.000000}%
\pgfsetfillcolor{currentfill}%
\pgfsetlinewidth{1.003750pt}%
\definecolor{currentstroke}{rgb}{0.800000,0.800000,0.800000}%
\pgfsetstrokecolor{currentstroke}%
\pgfsetdash{}{0pt}%
\pgfpathmoveto{\pgfqpoint{1.888929in}{0.000000in}}%
\pgfpathlineto{\pgfqpoint{4.238618in}{0.000000in}}%
\pgfpathquadraticcurveto{\pgfqpoint{4.277507in}{0.000000in}}{\pgfqpoint{4.277507in}{0.038889in}}%
\pgfpathlineto{\pgfqpoint{4.277507in}{0.294444in}}%
\pgfpathquadraticcurveto{\pgfqpoint{4.277507in}{0.333333in}}{\pgfqpoint{4.238618in}{0.333333in}}%
\pgfpathlineto{\pgfqpoint{1.888929in}{0.333333in}}%
\pgfpathquadraticcurveto{\pgfqpoint{1.850040in}{0.333333in}}{\pgfqpoint{1.850040in}{0.294444in}}%
\pgfpathlineto{\pgfqpoint{1.850040in}{0.038889in}}%
\pgfpathquadraticcurveto{\pgfqpoint{1.850040in}{0.000000in}}{\pgfqpoint{1.888929in}{0.000000in}}%
\pgfpathclose%
\pgfusepath{stroke,fill}%
\end{pgfscope}%
\begin{pgfscope}%
\pgfsetrectcap%
\pgfsetroundjoin%
\pgfsetlinewidth{1.505625pt}%
\definecolor{currentstroke}{rgb}{1.000000,0.000000,0.000000}%
\pgfsetstrokecolor{currentstroke}%
\pgfsetdash{}{0pt}%
\pgfpathmoveto{\pgfqpoint{1.927818in}{0.184722in}}%
\pgfpathlineto{\pgfqpoint{2.316706in}{0.184722in}}%
\pgfusepath{stroke}%
\end{pgfscope}%
\begin{pgfscope}%
\pgfsetbuttcap%
\pgfsetroundjoin%
\definecolor{currentfill}{rgb}{1.000000,0.000000,0.000000}%
\pgfsetfillcolor{currentfill}%
\pgfsetlinewidth{1.003750pt}%
\definecolor{currentstroke}{rgb}{1.000000,0.000000,0.000000}%
\pgfsetstrokecolor{currentstroke}%
\pgfsetdash{}{0pt}%
\pgfsys@defobject{currentmarker}{\pgfqpoint{-0.041667in}{-0.041667in}}{\pgfqpoint{0.041667in}{0.041667in}}{%
\pgfpathmoveto{\pgfqpoint{-0.041667in}{0.000000in}}%
\pgfpathlineto{\pgfqpoint{0.041667in}{0.000000in}}%
\pgfpathmoveto{\pgfqpoint{0.000000in}{-0.041667in}}%
\pgfpathlineto{\pgfqpoint{0.000000in}{0.041667in}}%
\pgfusepath{stroke,fill}%
}%
\begin{pgfscope}%
\pgfsys@transformshift{2.122262in}{0.184722in}%
\pgfsys@useobject{currentmarker}{}%
\end{pgfscope}%
\end{pgfscope}%
\begin{pgfscope}%
\definecolor{textcolor}{rgb}{0.000000,0.000000,0.000000}%
\pgfsetstrokecolor{textcolor}%
\pgfsetfillcolor{textcolor}%
\pgftext[x=2.472262in,y=0.116667in,left,base]{\color{textcolor}\rmfamily\fontsize{14.000000}{16.800000}\selectfont Poor}%
\end{pgfscope}%
\begin{pgfscope}%
\pgfsetbuttcap%
\pgfsetroundjoin%
\pgfsetlinewidth{1.505625pt}%
\definecolor{currentstroke}{rgb}{0.000000,0.000000,1.000000}%
\pgfsetstrokecolor{currentstroke}%
\pgfsetdash{{5.550000pt}{2.400000pt}}{0.000000pt}%
\pgfpathmoveto{\pgfqpoint{3.266377in}{0.184722in}}%
\pgfpathlineto{\pgfqpoint{3.655266in}{0.184722in}}%
\pgfusepath{stroke}%
\end{pgfscope}%
\begin{pgfscope}%
\pgfsetbuttcap%
\pgfsetroundjoin%
\definecolor{currentfill}{rgb}{0.000000,0.000000,1.000000}%
\pgfsetfillcolor{currentfill}%
\pgfsetlinewidth{1.003750pt}%
\definecolor{currentstroke}{rgb}{0.000000,0.000000,1.000000}%
\pgfsetstrokecolor{currentstroke}%
\pgfsetdash{}{0pt}%
\pgfsys@defobject{currentmarker}{\pgfqpoint{-0.041667in}{-0.041667in}}{\pgfqpoint{0.041667in}{0.041667in}}{%
\pgfpathmoveto{\pgfqpoint{-0.041667in}{-0.041667in}}%
\pgfpathlineto{\pgfqpoint{0.041667in}{0.041667in}}%
\pgfpathmoveto{\pgfqpoint{-0.041667in}{0.041667in}}%
\pgfpathlineto{\pgfqpoint{0.041667in}{-0.041667in}}%
\pgfusepath{stroke,fill}%
}%
\begin{pgfscope}%
\pgfsys@transformshift{3.460822in}{0.184722in}%
\pgfsys@useobject{currentmarker}{}%
\end{pgfscope}%
\end{pgfscope}%
\begin{pgfscope}%
\definecolor{textcolor}{rgb}{0.000000,0.000000,0.000000}%
\pgfsetstrokecolor{textcolor}%
\pgfsetfillcolor{textcolor}%
\pgftext[x=3.810822in,y=0.116667in,left,base]{\color{textcolor}\rmfamily\fontsize{14.000000}{16.800000}\selectfont Rich}%
\end{pgfscope}%
\end{pgfpicture}%
\makeatother%
\endgroup%
 } 
\end{subfigure}
\begin{subfigure}{.49\textwidth}
\centering
% include first image
\caption{Net worth---men and women}
\label{sf:assdiv4}
\scalebox{0.5}{%% Creator: Matplotlib, PGF backend
%%
%% To include the figure in your LaTeX document, write
%%   \input{<filename>.pgf}
%%
%% Make sure the required packages are loaded in your preamble
%%   \usepackage{pgf}
%%
%% Figures using additional raster images can only be included by \input if
%% they are in the same directory as the main LaTeX file. For loading figures
%% from other directories you can use the `import` package
%%   \usepackage{import}
%% and then include the figures with
%%   \import{<path to file>}{<filename>.pgf}
%%
%% Matplotlib used the following preamble
%%
\begingroup%
\makeatletter%
\begin{pgfpicture}%
\pgfpathrectangle{\pgfpointorigin}{\pgfqpoint{5.312732in}{3.903555in}}%
\pgfusepath{use as bounding box, clip}%
\begin{pgfscope}%
\pgfsetbuttcap%
\pgfsetmiterjoin%
\definecolor{currentfill}{rgb}{1.000000,1.000000,1.000000}%
\pgfsetfillcolor{currentfill}%
\pgfsetlinewidth{0.000000pt}%
\definecolor{currentstroke}{rgb}{1.000000,1.000000,1.000000}%
\pgfsetstrokecolor{currentstroke}%
\pgfsetdash{}{0pt}%
\pgfpathmoveto{\pgfqpoint{0.000000in}{0.000000in}}%
\pgfpathlineto{\pgfqpoint{5.312732in}{0.000000in}}%
\pgfpathlineto{\pgfqpoint{5.312732in}{3.903555in}}%
\pgfpathlineto{\pgfqpoint{0.000000in}{3.903555in}}%
\pgfpathclose%
\pgfusepath{fill}%
\end{pgfscope}%
\begin{pgfscope}%
\pgfsetbuttcap%
\pgfsetmiterjoin%
\definecolor{currentfill}{rgb}{1.000000,1.000000,1.000000}%
\pgfsetfillcolor{currentfill}%
\pgfsetlinewidth{0.000000pt}%
\definecolor{currentstroke}{rgb}{0.000000,0.000000,0.000000}%
\pgfsetstrokecolor{currentstroke}%
\pgfsetstrokeopacity{0.000000}%
\pgfsetdash{}{0pt}%
\pgfpathmoveto{\pgfqpoint{0.662732in}{0.883555in}}%
\pgfpathlineto{\pgfqpoint{5.312732in}{0.883555in}}%
\pgfpathlineto{\pgfqpoint{5.312732in}{3.903555in}}%
\pgfpathlineto{\pgfqpoint{0.662732in}{3.903555in}}%
\pgfpathclose%
\pgfusepath{fill}%
\end{pgfscope}%
\begin{pgfscope}%
\pgfpathrectangle{\pgfqpoint{0.662732in}{0.883555in}}{\pgfqpoint{4.650000in}{3.020000in}}%
\pgfusepath{clip}%
\pgfsetbuttcap%
\pgfsetroundjoin%
\definecolor{currentfill}{rgb}{1.000000,0.000000,0.000000}%
\pgfsetfillcolor{currentfill}%
\pgfsetfillopacity{0.200000}%
\pgfsetlinewidth{0.000000pt}%
\definecolor{currentstroke}{rgb}{0.000000,0.000000,0.000000}%
\pgfsetstrokecolor{currentstroke}%
\pgfsetdash{}{0pt}%
\pgfpathmoveto{\pgfqpoint{0.874095in}{3.700597in}}%
\pgfpathlineto{\pgfqpoint{0.874095in}{2.112947in}}%
\pgfpathlineto{\pgfqpoint{1.719550in}{2.021671in}}%
\pgfpathlineto{\pgfqpoint{2.565004in}{2.024431in}}%
\pgfpathlineto{\pgfqpoint{3.410459in}{1.454345in}}%
\pgfpathlineto{\pgfqpoint{4.255913in}{1.508776in}}%
\pgfpathlineto{\pgfqpoint{5.101368in}{1.020828in}}%
\pgfpathlineto{\pgfqpoint{5.101368in}{2.868366in}}%
\pgfpathlineto{\pgfqpoint{5.101368in}{2.868366in}}%
\pgfpathlineto{\pgfqpoint{4.255913in}{3.211577in}}%
\pgfpathlineto{\pgfqpoint{3.410459in}{2.899207in}}%
\pgfpathlineto{\pgfqpoint{2.565004in}{3.510254in}}%
\pgfpathlineto{\pgfqpoint{1.719550in}{3.541742in}}%
\pgfpathlineto{\pgfqpoint{0.874095in}{3.700597in}}%
\pgfpathclose%
\pgfusepath{fill}%
\end{pgfscope}%
\begin{pgfscope}%
\pgfpathrectangle{\pgfqpoint{0.662732in}{0.883555in}}{\pgfqpoint{4.650000in}{3.020000in}}%
\pgfusepath{clip}%
\pgfsetbuttcap%
\pgfsetroundjoin%
\definecolor{currentfill}{rgb}{0.000000,0.000000,1.000000}%
\pgfsetfillcolor{currentfill}%
\pgfsetfillopacity{0.200000}%
\pgfsetlinewidth{0.000000pt}%
\definecolor{currentstroke}{rgb}{0.000000,0.000000,0.000000}%
\pgfsetstrokecolor{currentstroke}%
\pgfsetdash{}{0pt}%
\pgfpathmoveto{\pgfqpoint{0.874095in}{3.482755in}}%
\pgfpathlineto{\pgfqpoint{0.874095in}{2.173686in}}%
\pgfpathlineto{\pgfqpoint{1.719550in}{2.397128in}}%
\pgfpathlineto{\pgfqpoint{2.565004in}{2.632937in}}%
\pgfpathlineto{\pgfqpoint{3.410459in}{2.248698in}}%
\pgfpathlineto{\pgfqpoint{4.255913in}{2.388931in}}%
\pgfpathlineto{\pgfqpoint{5.101368in}{2.374195in}}%
\pgfpathlineto{\pgfqpoint{5.101368in}{3.479301in}}%
\pgfpathlineto{\pgfqpoint{5.101368in}{3.479301in}}%
\pgfpathlineto{\pgfqpoint{4.255913in}{3.528106in}}%
\pgfpathlineto{\pgfqpoint{3.410459in}{3.372629in}}%
\pgfpathlineto{\pgfqpoint{2.565004in}{3.766282in}}%
\pgfpathlineto{\pgfqpoint{1.719550in}{3.594256in}}%
\pgfpathlineto{\pgfqpoint{0.874095in}{3.482755in}}%
\pgfpathclose%
\pgfusepath{fill}%
\end{pgfscope}%
\begin{pgfscope}%
\pgfsetbuttcap%
\pgfsetroundjoin%
\definecolor{currentfill}{rgb}{0.000000,0.000000,0.000000}%
\pgfsetfillcolor{currentfill}%
\pgfsetlinewidth{0.803000pt}%
\definecolor{currentstroke}{rgb}{0.000000,0.000000,0.000000}%
\pgfsetstrokecolor{currentstroke}%
\pgfsetdash{}{0pt}%
\pgfsys@defobject{currentmarker}{\pgfqpoint{0.000000in}{-0.048611in}}{\pgfqpoint{0.000000in}{0.000000in}}{%
\pgfpathmoveto{\pgfqpoint{0.000000in}{0.000000in}}%
\pgfpathlineto{\pgfqpoint{0.000000in}{-0.048611in}}%
\pgfusepath{stroke,fill}%
}%
\begin{pgfscope}%
\pgfsys@transformshift{0.874095in}{0.883555in}%
\pgfsys@useobject{currentmarker}{}%
\end{pgfscope}%
\end{pgfscope}%
\begin{pgfscope}%
\definecolor{textcolor}{rgb}{0.000000,0.000000,0.000000}%
\pgfsetstrokecolor{textcolor}%
\pgfsetfillcolor{textcolor}%
\pgftext[x=0.874095in,y=0.786333in,,top]{\color{textcolor}\rmfamily\fontsize{11.000000}{13.200000}\selectfont \(\displaystyle -6\)}%
\end{pgfscope}%
\begin{pgfscope}%
\pgfsetbuttcap%
\pgfsetroundjoin%
\definecolor{currentfill}{rgb}{0.000000,0.000000,0.000000}%
\pgfsetfillcolor{currentfill}%
\pgfsetlinewidth{0.803000pt}%
\definecolor{currentstroke}{rgb}{0.000000,0.000000,0.000000}%
\pgfsetstrokecolor{currentstroke}%
\pgfsetdash{}{0pt}%
\pgfsys@defobject{currentmarker}{\pgfqpoint{0.000000in}{-0.048611in}}{\pgfqpoint{0.000000in}{0.000000in}}{%
\pgfpathmoveto{\pgfqpoint{0.000000in}{0.000000in}}%
\pgfpathlineto{\pgfqpoint{0.000000in}{-0.048611in}}%
\pgfusepath{stroke,fill}%
}%
\begin{pgfscope}%
\pgfsys@transformshift{1.719550in}{0.883555in}%
\pgfsys@useobject{currentmarker}{}%
\end{pgfscope}%
\end{pgfscope}%
\begin{pgfscope}%
\definecolor{textcolor}{rgb}{0.000000,0.000000,0.000000}%
\pgfsetstrokecolor{textcolor}%
\pgfsetfillcolor{textcolor}%
\pgftext[x=1.719550in,y=0.786333in,,top]{\color{textcolor}\rmfamily\fontsize{11.000000}{13.200000}\selectfont \(\displaystyle -4\)}%
\end{pgfscope}%
\begin{pgfscope}%
\pgfsetbuttcap%
\pgfsetroundjoin%
\definecolor{currentfill}{rgb}{0.000000,0.000000,0.000000}%
\pgfsetfillcolor{currentfill}%
\pgfsetlinewidth{0.803000pt}%
\definecolor{currentstroke}{rgb}{0.000000,0.000000,0.000000}%
\pgfsetstrokecolor{currentstroke}%
\pgfsetdash{}{0pt}%
\pgfsys@defobject{currentmarker}{\pgfqpoint{0.000000in}{-0.048611in}}{\pgfqpoint{0.000000in}{0.000000in}}{%
\pgfpathmoveto{\pgfqpoint{0.000000in}{0.000000in}}%
\pgfpathlineto{\pgfqpoint{0.000000in}{-0.048611in}}%
\pgfusepath{stroke,fill}%
}%
\begin{pgfscope}%
\pgfsys@transformshift{2.565004in}{0.883555in}%
\pgfsys@useobject{currentmarker}{}%
\end{pgfscope}%
\end{pgfscope}%
\begin{pgfscope}%
\definecolor{textcolor}{rgb}{0.000000,0.000000,0.000000}%
\pgfsetstrokecolor{textcolor}%
\pgfsetfillcolor{textcolor}%
\pgftext[x=2.565004in,y=0.786333in,,top]{\color{textcolor}\rmfamily\fontsize{11.000000}{13.200000}\selectfont \(\displaystyle -2\)}%
\end{pgfscope}%
\begin{pgfscope}%
\pgfsetbuttcap%
\pgfsetroundjoin%
\definecolor{currentfill}{rgb}{0.000000,0.000000,0.000000}%
\pgfsetfillcolor{currentfill}%
\pgfsetlinewidth{0.803000pt}%
\definecolor{currentstroke}{rgb}{0.000000,0.000000,0.000000}%
\pgfsetstrokecolor{currentstroke}%
\pgfsetdash{}{0pt}%
\pgfsys@defobject{currentmarker}{\pgfqpoint{0.000000in}{-0.048611in}}{\pgfqpoint{0.000000in}{0.000000in}}{%
\pgfpathmoveto{\pgfqpoint{0.000000in}{0.000000in}}%
\pgfpathlineto{\pgfqpoint{0.000000in}{-0.048611in}}%
\pgfusepath{stroke,fill}%
}%
\begin{pgfscope}%
\pgfsys@transformshift{3.410459in}{0.883555in}%
\pgfsys@useobject{currentmarker}{}%
\end{pgfscope}%
\end{pgfscope}%
\begin{pgfscope}%
\definecolor{textcolor}{rgb}{0.000000,0.000000,0.000000}%
\pgfsetstrokecolor{textcolor}%
\pgfsetfillcolor{textcolor}%
\pgftext[x=3.410459in,y=0.786333in,,top]{\color{textcolor}\rmfamily\fontsize{11.000000}{13.200000}\selectfont \(\displaystyle 0\)}%
\end{pgfscope}%
\begin{pgfscope}%
\pgfsetbuttcap%
\pgfsetroundjoin%
\definecolor{currentfill}{rgb}{0.000000,0.000000,0.000000}%
\pgfsetfillcolor{currentfill}%
\pgfsetlinewidth{0.803000pt}%
\definecolor{currentstroke}{rgb}{0.000000,0.000000,0.000000}%
\pgfsetstrokecolor{currentstroke}%
\pgfsetdash{}{0pt}%
\pgfsys@defobject{currentmarker}{\pgfqpoint{0.000000in}{-0.048611in}}{\pgfqpoint{0.000000in}{0.000000in}}{%
\pgfpathmoveto{\pgfqpoint{0.000000in}{0.000000in}}%
\pgfpathlineto{\pgfqpoint{0.000000in}{-0.048611in}}%
\pgfusepath{stroke,fill}%
}%
\begin{pgfscope}%
\pgfsys@transformshift{4.255913in}{0.883555in}%
\pgfsys@useobject{currentmarker}{}%
\end{pgfscope}%
\end{pgfscope}%
\begin{pgfscope}%
\definecolor{textcolor}{rgb}{0.000000,0.000000,0.000000}%
\pgfsetstrokecolor{textcolor}%
\pgfsetfillcolor{textcolor}%
\pgftext[x=4.255913in,y=0.786333in,,top]{\color{textcolor}\rmfamily\fontsize{11.000000}{13.200000}\selectfont \(\displaystyle 2\)}%
\end{pgfscope}%
\begin{pgfscope}%
\pgfsetbuttcap%
\pgfsetroundjoin%
\definecolor{currentfill}{rgb}{0.000000,0.000000,0.000000}%
\pgfsetfillcolor{currentfill}%
\pgfsetlinewidth{0.803000pt}%
\definecolor{currentstroke}{rgb}{0.000000,0.000000,0.000000}%
\pgfsetstrokecolor{currentstroke}%
\pgfsetdash{}{0pt}%
\pgfsys@defobject{currentmarker}{\pgfqpoint{0.000000in}{-0.048611in}}{\pgfqpoint{0.000000in}{0.000000in}}{%
\pgfpathmoveto{\pgfqpoint{0.000000in}{0.000000in}}%
\pgfpathlineto{\pgfqpoint{0.000000in}{-0.048611in}}%
\pgfusepath{stroke,fill}%
}%
\begin{pgfscope}%
\pgfsys@transformshift{5.101368in}{0.883555in}%
\pgfsys@useobject{currentmarker}{}%
\end{pgfscope}%
\end{pgfscope}%
\begin{pgfscope}%
\definecolor{textcolor}{rgb}{0.000000,0.000000,0.000000}%
\pgfsetstrokecolor{textcolor}%
\pgfsetfillcolor{textcolor}%
\pgftext[x=5.101368in,y=0.786333in,,top]{\color{textcolor}\rmfamily\fontsize{11.000000}{13.200000}\selectfont \(\displaystyle 4\)}%
\end{pgfscope}%
\begin{pgfscope}%
\definecolor{textcolor}{rgb}{0.000000,0.000000,0.000000}%
\pgfsetstrokecolor{textcolor}%
\pgfsetfillcolor{textcolor}%
\pgftext[x=2.987732in,y=0.595592in,,top]{\color{textcolor}\rmfamily\fontsize{16.000000}{19.200000}\selectfont Event time (Years)}%
\end{pgfscope}%
\begin{pgfscope}%
\pgfsetbuttcap%
\pgfsetroundjoin%
\definecolor{currentfill}{rgb}{0.000000,0.000000,0.000000}%
\pgfsetfillcolor{currentfill}%
\pgfsetlinewidth{0.803000pt}%
\definecolor{currentstroke}{rgb}{0.000000,0.000000,0.000000}%
\pgfsetstrokecolor{currentstroke}%
\pgfsetdash{}{0pt}%
\pgfsys@defobject{currentmarker}{\pgfqpoint{-0.048611in}{0.000000in}}{\pgfqpoint{0.000000in}{0.000000in}}{%
\pgfpathmoveto{\pgfqpoint{0.000000in}{0.000000in}}%
\pgfpathlineto{\pgfqpoint{-0.048611in}{0.000000in}}%
\pgfusepath{stroke,fill}%
}%
\begin{pgfscope}%
\pgfsys@transformshift{0.662732in}{1.102673in}%
\pgfsys@useobject{currentmarker}{}%
\end{pgfscope}%
\end{pgfscope}%
\begin{pgfscope}%
\definecolor{textcolor}{rgb}{0.000000,0.000000,0.000000}%
\pgfsetstrokecolor{textcolor}%
\pgfsetfillcolor{textcolor}%
\pgftext[x=0.295138in,y=1.049866in,left,base]{\color{textcolor}\rmfamily\fontsize{11.000000}{13.200000}\selectfont \(\displaystyle -40\)}%
\end{pgfscope}%
\begin{pgfscope}%
\pgfsetbuttcap%
\pgfsetroundjoin%
\definecolor{currentfill}{rgb}{0.000000,0.000000,0.000000}%
\pgfsetfillcolor{currentfill}%
\pgfsetlinewidth{0.803000pt}%
\definecolor{currentstroke}{rgb}{0.000000,0.000000,0.000000}%
\pgfsetstrokecolor{currentstroke}%
\pgfsetdash{}{0pt}%
\pgfsys@defobject{currentmarker}{\pgfqpoint{-0.048611in}{0.000000in}}{\pgfqpoint{0.000000in}{0.000000in}}{%
\pgfpathmoveto{\pgfqpoint{0.000000in}{0.000000in}}%
\pgfpathlineto{\pgfqpoint{-0.048611in}{0.000000in}}%
\pgfusepath{stroke,fill}%
}%
\begin{pgfscope}%
\pgfsys@transformshift{0.662732in}{1.481519in}%
\pgfsys@useobject{currentmarker}{}%
\end{pgfscope}%
\end{pgfscope}%
\begin{pgfscope}%
\definecolor{textcolor}{rgb}{0.000000,0.000000,0.000000}%
\pgfsetstrokecolor{textcolor}%
\pgfsetfillcolor{textcolor}%
\pgftext[x=0.295138in,y=1.428712in,left,base]{\color{textcolor}\rmfamily\fontsize{11.000000}{13.200000}\selectfont \(\displaystyle -20\)}%
\end{pgfscope}%
\begin{pgfscope}%
\pgfsetbuttcap%
\pgfsetroundjoin%
\definecolor{currentfill}{rgb}{0.000000,0.000000,0.000000}%
\pgfsetfillcolor{currentfill}%
\pgfsetlinewidth{0.803000pt}%
\definecolor{currentstroke}{rgb}{0.000000,0.000000,0.000000}%
\pgfsetstrokecolor{currentstroke}%
\pgfsetdash{}{0pt}%
\pgfsys@defobject{currentmarker}{\pgfqpoint{-0.048611in}{0.000000in}}{\pgfqpoint{0.000000in}{0.000000in}}{%
\pgfpathmoveto{\pgfqpoint{0.000000in}{0.000000in}}%
\pgfpathlineto{\pgfqpoint{-0.048611in}{0.000000in}}%
\pgfusepath{stroke,fill}%
}%
\begin{pgfscope}%
\pgfsys@transformshift{0.662732in}{1.860365in}%
\pgfsys@useobject{currentmarker}{}%
\end{pgfscope}%
\end{pgfscope}%
\begin{pgfscope}%
\definecolor{textcolor}{rgb}{0.000000,0.000000,0.000000}%
\pgfsetstrokecolor{textcolor}%
\pgfsetfillcolor{textcolor}%
\pgftext[x=0.489468in,y=1.807559in,left,base]{\color{textcolor}\rmfamily\fontsize{11.000000}{13.200000}\selectfont \(\displaystyle 0\)}%
\end{pgfscope}%
\begin{pgfscope}%
\pgfsetbuttcap%
\pgfsetroundjoin%
\definecolor{currentfill}{rgb}{0.000000,0.000000,0.000000}%
\pgfsetfillcolor{currentfill}%
\pgfsetlinewidth{0.803000pt}%
\definecolor{currentstroke}{rgb}{0.000000,0.000000,0.000000}%
\pgfsetstrokecolor{currentstroke}%
\pgfsetdash{}{0pt}%
\pgfsys@defobject{currentmarker}{\pgfqpoint{-0.048611in}{0.000000in}}{\pgfqpoint{0.000000in}{0.000000in}}{%
\pgfpathmoveto{\pgfqpoint{0.000000in}{0.000000in}}%
\pgfpathlineto{\pgfqpoint{-0.048611in}{0.000000in}}%
\pgfusepath{stroke,fill}%
}%
\begin{pgfscope}%
\pgfsys@transformshift{0.662732in}{2.239211in}%
\pgfsys@useobject{currentmarker}{}%
\end{pgfscope}%
\end{pgfscope}%
\begin{pgfscope}%
\definecolor{textcolor}{rgb}{0.000000,0.000000,0.000000}%
\pgfsetstrokecolor{textcolor}%
\pgfsetfillcolor{textcolor}%
\pgftext[x=0.413426in,y=2.186405in,left,base]{\color{textcolor}\rmfamily\fontsize{11.000000}{13.200000}\selectfont \(\displaystyle 20\)}%
\end{pgfscope}%
\begin{pgfscope}%
\pgfsetbuttcap%
\pgfsetroundjoin%
\definecolor{currentfill}{rgb}{0.000000,0.000000,0.000000}%
\pgfsetfillcolor{currentfill}%
\pgfsetlinewidth{0.803000pt}%
\definecolor{currentstroke}{rgb}{0.000000,0.000000,0.000000}%
\pgfsetstrokecolor{currentstroke}%
\pgfsetdash{}{0pt}%
\pgfsys@defobject{currentmarker}{\pgfqpoint{-0.048611in}{0.000000in}}{\pgfqpoint{0.000000in}{0.000000in}}{%
\pgfpathmoveto{\pgfqpoint{0.000000in}{0.000000in}}%
\pgfpathlineto{\pgfqpoint{-0.048611in}{0.000000in}}%
\pgfusepath{stroke,fill}%
}%
\begin{pgfscope}%
\pgfsys@transformshift{0.662732in}{2.618057in}%
\pgfsys@useobject{currentmarker}{}%
\end{pgfscope}%
\end{pgfscope}%
\begin{pgfscope}%
\definecolor{textcolor}{rgb}{0.000000,0.000000,0.000000}%
\pgfsetstrokecolor{textcolor}%
\pgfsetfillcolor{textcolor}%
\pgftext[x=0.413426in,y=2.565251in,left,base]{\color{textcolor}\rmfamily\fontsize{11.000000}{13.200000}\selectfont \(\displaystyle 40\)}%
\end{pgfscope}%
\begin{pgfscope}%
\pgfsetbuttcap%
\pgfsetroundjoin%
\definecolor{currentfill}{rgb}{0.000000,0.000000,0.000000}%
\pgfsetfillcolor{currentfill}%
\pgfsetlinewidth{0.803000pt}%
\definecolor{currentstroke}{rgb}{0.000000,0.000000,0.000000}%
\pgfsetstrokecolor{currentstroke}%
\pgfsetdash{}{0pt}%
\pgfsys@defobject{currentmarker}{\pgfqpoint{-0.048611in}{0.000000in}}{\pgfqpoint{0.000000in}{0.000000in}}{%
\pgfpathmoveto{\pgfqpoint{0.000000in}{0.000000in}}%
\pgfpathlineto{\pgfqpoint{-0.048611in}{0.000000in}}%
\pgfusepath{stroke,fill}%
}%
\begin{pgfscope}%
\pgfsys@transformshift{0.662732in}{2.996903in}%
\pgfsys@useobject{currentmarker}{}%
\end{pgfscope}%
\end{pgfscope}%
\begin{pgfscope}%
\definecolor{textcolor}{rgb}{0.000000,0.000000,0.000000}%
\pgfsetstrokecolor{textcolor}%
\pgfsetfillcolor{textcolor}%
\pgftext[x=0.413426in,y=2.944097in,left,base]{\color{textcolor}\rmfamily\fontsize{11.000000}{13.200000}\selectfont \(\displaystyle 60\)}%
\end{pgfscope}%
\begin{pgfscope}%
\pgfsetbuttcap%
\pgfsetroundjoin%
\definecolor{currentfill}{rgb}{0.000000,0.000000,0.000000}%
\pgfsetfillcolor{currentfill}%
\pgfsetlinewidth{0.803000pt}%
\definecolor{currentstroke}{rgb}{0.000000,0.000000,0.000000}%
\pgfsetstrokecolor{currentstroke}%
\pgfsetdash{}{0pt}%
\pgfsys@defobject{currentmarker}{\pgfqpoint{-0.048611in}{0.000000in}}{\pgfqpoint{0.000000in}{0.000000in}}{%
\pgfpathmoveto{\pgfqpoint{0.000000in}{0.000000in}}%
\pgfpathlineto{\pgfqpoint{-0.048611in}{0.000000in}}%
\pgfusepath{stroke,fill}%
}%
\begin{pgfscope}%
\pgfsys@transformshift{0.662732in}{3.375749in}%
\pgfsys@useobject{currentmarker}{}%
\end{pgfscope}%
\end{pgfscope}%
\begin{pgfscope}%
\definecolor{textcolor}{rgb}{0.000000,0.000000,0.000000}%
\pgfsetstrokecolor{textcolor}%
\pgfsetfillcolor{textcolor}%
\pgftext[x=0.413426in,y=3.322943in,left,base]{\color{textcolor}\rmfamily\fontsize{11.000000}{13.200000}\selectfont \(\displaystyle 80\)}%
\end{pgfscope}%
\begin{pgfscope}%
\pgfsetbuttcap%
\pgfsetroundjoin%
\definecolor{currentfill}{rgb}{0.000000,0.000000,0.000000}%
\pgfsetfillcolor{currentfill}%
\pgfsetlinewidth{0.803000pt}%
\definecolor{currentstroke}{rgb}{0.000000,0.000000,0.000000}%
\pgfsetstrokecolor{currentstroke}%
\pgfsetdash{}{0pt}%
\pgfsys@defobject{currentmarker}{\pgfqpoint{-0.048611in}{0.000000in}}{\pgfqpoint{0.000000in}{0.000000in}}{%
\pgfpathmoveto{\pgfqpoint{0.000000in}{0.000000in}}%
\pgfpathlineto{\pgfqpoint{-0.048611in}{0.000000in}}%
\pgfusepath{stroke,fill}%
}%
\begin{pgfscope}%
\pgfsys@transformshift{0.662732in}{3.754595in}%
\pgfsys@useobject{currentmarker}{}%
\end{pgfscope}%
\end{pgfscope}%
\begin{pgfscope}%
\definecolor{textcolor}{rgb}{0.000000,0.000000,0.000000}%
\pgfsetstrokecolor{textcolor}%
\pgfsetfillcolor{textcolor}%
\pgftext[x=0.337384in,y=3.701789in,left,base]{\color{textcolor}\rmfamily\fontsize{11.000000}{13.200000}\selectfont \(\displaystyle 100\)}%
\end{pgfscope}%
\begin{pgfscope}%
\definecolor{textcolor}{rgb}{0.000000,0.000000,0.000000}%
\pgfsetstrokecolor{textcolor}%
\pgfsetfillcolor{textcolor}%
\pgftext[x=0.239583in,y=2.393555in,,bottom,rotate=90.000000]{\color{textcolor}\rmfamily\fontsize{16.000000}{19.200000}\selectfont Net Worth (\$ 1000s)}%
\end{pgfscope}%
\begin{pgfscope}%
\pgfpathrectangle{\pgfqpoint{0.662732in}{0.883555in}}{\pgfqpoint{4.650000in}{3.020000in}}%
\pgfusepath{clip}%
\pgfsetrectcap%
\pgfsetroundjoin%
\pgfsetlinewidth{1.505625pt}%
\definecolor{currentstroke}{rgb}{1.000000,0.000000,0.000000}%
\pgfsetstrokecolor{currentstroke}%
\pgfsetdash{}{0pt}%
\pgfpathmoveto{\pgfqpoint{0.874095in}{2.906772in}}%
\pgfpathlineto{\pgfqpoint{1.719550in}{2.781707in}}%
\pgfpathlineto{\pgfqpoint{2.565004in}{2.767343in}}%
\pgfpathlineto{\pgfqpoint{3.410459in}{2.176776in}}%
\pgfpathlineto{\pgfqpoint{4.255913in}{2.360177in}}%
\pgfpathlineto{\pgfqpoint{5.101368in}{1.944597in}}%
\pgfusepath{stroke}%
\end{pgfscope}%
\begin{pgfscope}%
\pgfpathrectangle{\pgfqpoint{0.662732in}{0.883555in}}{\pgfqpoint{4.650000in}{3.020000in}}%
\pgfusepath{clip}%
\pgfsetbuttcap%
\pgfsetroundjoin%
\definecolor{currentfill}{rgb}{1.000000,0.000000,0.000000}%
\pgfsetfillcolor{currentfill}%
\pgfsetlinewidth{1.003750pt}%
\definecolor{currentstroke}{rgb}{1.000000,0.000000,0.000000}%
\pgfsetstrokecolor{currentstroke}%
\pgfsetdash{}{0pt}%
\pgfsys@defobject{currentmarker}{\pgfqpoint{-0.041667in}{-0.041667in}}{\pgfqpoint{0.041667in}{0.041667in}}{%
\pgfpathmoveto{\pgfqpoint{-0.041667in}{0.000000in}}%
\pgfpathlineto{\pgfqpoint{0.041667in}{0.000000in}}%
\pgfpathmoveto{\pgfqpoint{0.000000in}{-0.041667in}}%
\pgfpathlineto{\pgfqpoint{0.000000in}{0.041667in}}%
\pgfusepath{stroke,fill}%
}%
\begin{pgfscope}%
\pgfsys@transformshift{0.874095in}{2.906772in}%
\pgfsys@useobject{currentmarker}{}%
\end{pgfscope}%
\begin{pgfscope}%
\pgfsys@transformshift{1.719550in}{2.781707in}%
\pgfsys@useobject{currentmarker}{}%
\end{pgfscope}%
\begin{pgfscope}%
\pgfsys@transformshift{2.565004in}{2.767343in}%
\pgfsys@useobject{currentmarker}{}%
\end{pgfscope}%
\begin{pgfscope}%
\pgfsys@transformshift{3.410459in}{2.176776in}%
\pgfsys@useobject{currentmarker}{}%
\end{pgfscope}%
\begin{pgfscope}%
\pgfsys@transformshift{4.255913in}{2.360177in}%
\pgfsys@useobject{currentmarker}{}%
\end{pgfscope}%
\begin{pgfscope}%
\pgfsys@transformshift{5.101368in}{1.944597in}%
\pgfsys@useobject{currentmarker}{}%
\end{pgfscope}%
\end{pgfscope}%
\begin{pgfscope}%
\pgfpathrectangle{\pgfqpoint{0.662732in}{0.883555in}}{\pgfqpoint{4.650000in}{3.020000in}}%
\pgfusepath{clip}%
\pgfsetbuttcap%
\pgfsetroundjoin%
\pgfsetlinewidth{1.505625pt}%
\definecolor{currentstroke}{rgb}{0.000000,0.000000,1.000000}%
\pgfsetstrokecolor{currentstroke}%
\pgfsetdash{{5.550000pt}{2.400000pt}}{0.000000pt}%
\pgfpathmoveto{\pgfqpoint{0.874095in}{2.828221in}}%
\pgfpathlineto{\pgfqpoint{1.719550in}{2.995692in}}%
\pgfpathlineto{\pgfqpoint{2.565004in}{3.199610in}}%
\pgfpathlineto{\pgfqpoint{3.410459in}{2.810663in}}%
\pgfpathlineto{\pgfqpoint{4.255913in}{2.958519in}}%
\pgfpathlineto{\pgfqpoint{5.101368in}{2.926748in}}%
\pgfusepath{stroke}%
\end{pgfscope}%
\begin{pgfscope}%
\pgfpathrectangle{\pgfqpoint{0.662732in}{0.883555in}}{\pgfqpoint{4.650000in}{3.020000in}}%
\pgfusepath{clip}%
\pgfsetbuttcap%
\pgfsetroundjoin%
\definecolor{currentfill}{rgb}{0.000000,0.000000,1.000000}%
\pgfsetfillcolor{currentfill}%
\pgfsetlinewidth{1.003750pt}%
\definecolor{currentstroke}{rgb}{0.000000,0.000000,1.000000}%
\pgfsetstrokecolor{currentstroke}%
\pgfsetdash{}{0pt}%
\pgfsys@defobject{currentmarker}{\pgfqpoint{-0.041667in}{-0.041667in}}{\pgfqpoint{0.041667in}{0.041667in}}{%
\pgfpathmoveto{\pgfqpoint{-0.041667in}{-0.041667in}}%
\pgfpathlineto{\pgfqpoint{0.041667in}{0.041667in}}%
\pgfpathmoveto{\pgfqpoint{-0.041667in}{0.041667in}}%
\pgfpathlineto{\pgfqpoint{0.041667in}{-0.041667in}}%
\pgfusepath{stroke,fill}%
}%
\begin{pgfscope}%
\pgfsys@transformshift{0.874095in}{2.828221in}%
\pgfsys@useobject{currentmarker}{}%
\end{pgfscope}%
\begin{pgfscope}%
\pgfsys@transformshift{1.719550in}{2.995692in}%
\pgfsys@useobject{currentmarker}{}%
\end{pgfscope}%
\begin{pgfscope}%
\pgfsys@transformshift{2.565004in}{3.199610in}%
\pgfsys@useobject{currentmarker}{}%
\end{pgfscope}%
\begin{pgfscope}%
\pgfsys@transformshift{3.410459in}{2.810663in}%
\pgfsys@useobject{currentmarker}{}%
\end{pgfscope}%
\begin{pgfscope}%
\pgfsys@transformshift{4.255913in}{2.958519in}%
\pgfsys@useobject{currentmarker}{}%
\end{pgfscope}%
\begin{pgfscope}%
\pgfsys@transformshift{5.101368in}{2.926748in}%
\pgfsys@useobject{currentmarker}{}%
\end{pgfscope}%
\end{pgfscope}%
\begin{pgfscope}%
\pgfpathrectangle{\pgfqpoint{0.662732in}{0.883555in}}{\pgfqpoint{4.650000in}{3.020000in}}%
\pgfusepath{clip}%
\pgfsetrectcap%
\pgfsetroundjoin%
\pgfsetlinewidth{1.505625pt}%
\definecolor{currentstroke}{rgb}{0.000000,0.000000,0.000000}%
\pgfsetstrokecolor{currentstroke}%
\pgfsetdash{}{0pt}%
\pgfpathmoveto{\pgfqpoint{3.410459in}{0.883555in}}%
\pgfpathlineto{\pgfqpoint{3.410459in}{3.903555in}}%
\pgfusepath{stroke}%
\end{pgfscope}%
\begin{pgfscope}%
\pgfsetrectcap%
\pgfsetmiterjoin%
\pgfsetlinewidth{0.803000pt}%
\definecolor{currentstroke}{rgb}{0.000000,0.000000,0.000000}%
\pgfsetstrokecolor{currentstroke}%
\pgfsetdash{}{0pt}%
\pgfpathmoveto{\pgfqpoint{0.662732in}{0.883555in}}%
\pgfpathlineto{\pgfqpoint{0.662732in}{3.903555in}}%
\pgfusepath{stroke}%
\end{pgfscope}%
\begin{pgfscope}%
\pgfsetrectcap%
\pgfsetmiterjoin%
\pgfsetlinewidth{0.803000pt}%
\definecolor{currentstroke}{rgb}{0.000000,0.000000,0.000000}%
\pgfsetstrokecolor{currentstroke}%
\pgfsetdash{}{0pt}%
\pgfpathmoveto{\pgfqpoint{5.312732in}{0.883555in}}%
\pgfpathlineto{\pgfqpoint{5.312732in}{3.903555in}}%
\pgfusepath{stroke}%
\end{pgfscope}%
\begin{pgfscope}%
\pgfsetrectcap%
\pgfsetmiterjoin%
\pgfsetlinewidth{0.803000pt}%
\definecolor{currentstroke}{rgb}{0.000000,0.000000,0.000000}%
\pgfsetstrokecolor{currentstroke}%
\pgfsetdash{}{0pt}%
\pgfpathmoveto{\pgfqpoint{0.662732in}{0.883555in}}%
\pgfpathlineto{\pgfqpoint{5.312732in}{0.883555in}}%
\pgfusepath{stroke}%
\end{pgfscope}%
\begin{pgfscope}%
\pgfsetrectcap%
\pgfsetmiterjoin%
\pgfsetlinewidth{0.803000pt}%
\definecolor{currentstroke}{rgb}{0.000000,0.000000,0.000000}%
\pgfsetstrokecolor{currentstroke}%
\pgfsetdash{}{0pt}%
\pgfpathmoveto{\pgfqpoint{0.662732in}{3.903555in}}%
\pgfpathlineto{\pgfqpoint{5.312732in}{3.903555in}}%
\pgfusepath{stroke}%
\end{pgfscope}%
\begin{pgfscope}%
\pgfsetbuttcap%
\pgfsetmiterjoin%
\definecolor{currentfill}{rgb}{0.300000,0.300000,0.300000}%
\pgfsetfillcolor{currentfill}%
\pgfsetfillopacity{0.500000}%
\pgfsetlinewidth{1.003750pt}%
\definecolor{currentstroke}{rgb}{0.300000,0.300000,0.300000}%
\pgfsetstrokecolor{currentstroke}%
\pgfsetstrokeopacity{0.500000}%
\pgfsetdash{}{0pt}%
\pgfpathmoveto{\pgfqpoint{1.729184in}{-0.027778in}}%
\pgfpathlineto{\pgfqpoint{4.301834in}{-0.027778in}}%
\pgfpathquadraticcurveto{\pgfqpoint{4.340723in}{-0.027778in}}{\pgfqpoint{4.340723in}{0.011111in}}%
\pgfpathlineto{\pgfqpoint{4.340723in}{0.266666in}}%
\pgfpathquadraticcurveto{\pgfqpoint{4.340723in}{0.305555in}}{\pgfqpoint{4.301834in}{0.305555in}}%
\pgfpathlineto{\pgfqpoint{1.729184in}{0.305555in}}%
\pgfpathquadraticcurveto{\pgfqpoint{1.690295in}{0.305555in}}{\pgfqpoint{1.690295in}{0.266666in}}%
\pgfpathlineto{\pgfqpoint{1.690295in}{0.011111in}}%
\pgfpathquadraticcurveto{\pgfqpoint{1.690295in}{-0.027778in}}{\pgfqpoint{1.729184in}{-0.027778in}}%
\pgfpathclose%
\pgfusepath{stroke,fill}%
\end{pgfscope}%
\begin{pgfscope}%
\pgfsetbuttcap%
\pgfsetmiterjoin%
\definecolor{currentfill}{rgb}{1.000000,1.000000,1.000000}%
\pgfsetfillcolor{currentfill}%
\pgfsetlinewidth{1.003750pt}%
\definecolor{currentstroke}{rgb}{0.800000,0.800000,0.800000}%
\pgfsetstrokecolor{currentstroke}%
\pgfsetdash{}{0pt}%
\pgfpathmoveto{\pgfqpoint{1.701407in}{0.000000in}}%
\pgfpathlineto{\pgfqpoint{4.274057in}{0.000000in}}%
\pgfpathquadraticcurveto{\pgfqpoint{4.312945in}{0.000000in}}{\pgfqpoint{4.312945in}{0.038889in}}%
\pgfpathlineto{\pgfqpoint{4.312945in}{0.294444in}}%
\pgfpathquadraticcurveto{\pgfqpoint{4.312945in}{0.333333in}}{\pgfqpoint{4.274057in}{0.333333in}}%
\pgfpathlineto{\pgfqpoint{1.701407in}{0.333333in}}%
\pgfpathquadraticcurveto{\pgfqpoint{1.662518in}{0.333333in}}{\pgfqpoint{1.662518in}{0.294444in}}%
\pgfpathlineto{\pgfqpoint{1.662518in}{0.038889in}}%
\pgfpathquadraticcurveto{\pgfqpoint{1.662518in}{0.000000in}}{\pgfqpoint{1.701407in}{0.000000in}}%
\pgfpathclose%
\pgfusepath{stroke,fill}%
\end{pgfscope}%
\begin{pgfscope}%
\pgfsetrectcap%
\pgfsetroundjoin%
\pgfsetlinewidth{1.505625pt}%
\definecolor{currentstroke}{rgb}{1.000000,0.000000,0.000000}%
\pgfsetstrokecolor{currentstroke}%
\pgfsetdash{}{0pt}%
\pgfpathmoveto{\pgfqpoint{1.740295in}{0.184722in}}%
\pgfpathlineto{\pgfqpoint{2.129184in}{0.184722in}}%
\pgfusepath{stroke}%
\end{pgfscope}%
\begin{pgfscope}%
\pgfsetbuttcap%
\pgfsetroundjoin%
\definecolor{currentfill}{rgb}{1.000000,0.000000,0.000000}%
\pgfsetfillcolor{currentfill}%
\pgfsetlinewidth{1.003750pt}%
\definecolor{currentstroke}{rgb}{1.000000,0.000000,0.000000}%
\pgfsetstrokecolor{currentstroke}%
\pgfsetdash{}{0pt}%
\pgfsys@defobject{currentmarker}{\pgfqpoint{-0.041667in}{-0.041667in}}{\pgfqpoint{0.041667in}{0.041667in}}{%
\pgfpathmoveto{\pgfqpoint{-0.041667in}{0.000000in}}%
\pgfpathlineto{\pgfqpoint{0.041667in}{0.000000in}}%
\pgfpathmoveto{\pgfqpoint{0.000000in}{-0.041667in}}%
\pgfpathlineto{\pgfqpoint{0.000000in}{0.041667in}}%
\pgfusepath{stroke,fill}%
}%
\begin{pgfscope}%
\pgfsys@transformshift{1.934740in}{0.184722in}%
\pgfsys@useobject{currentmarker}{}%
\end{pgfscope}%
\end{pgfscope}%
\begin{pgfscope}%
\definecolor{textcolor}{rgb}{0.000000,0.000000,0.000000}%
\pgfsetstrokecolor{textcolor}%
\pgfsetfillcolor{textcolor}%
\pgftext[x=2.284740in,y=0.116667in,left,base]{\color{textcolor}\rmfamily\fontsize{14.000000}{16.800000}\selectfont Women}%
\end{pgfscope}%
\begin{pgfscope}%
\pgfsetbuttcap%
\pgfsetroundjoin%
\pgfsetlinewidth{1.505625pt}%
\definecolor{currentstroke}{rgb}{0.000000,0.000000,1.000000}%
\pgfsetstrokecolor{currentstroke}%
\pgfsetdash{{5.550000pt}{2.400000pt}}{0.000000pt}%
\pgfpathmoveto{\pgfqpoint{3.315450in}{0.184722in}}%
\pgfpathlineto{\pgfqpoint{3.704339in}{0.184722in}}%
\pgfusepath{stroke}%
\end{pgfscope}%
\begin{pgfscope}%
\pgfsetbuttcap%
\pgfsetroundjoin%
\definecolor{currentfill}{rgb}{0.000000,0.000000,1.000000}%
\pgfsetfillcolor{currentfill}%
\pgfsetlinewidth{1.003750pt}%
\definecolor{currentstroke}{rgb}{0.000000,0.000000,1.000000}%
\pgfsetstrokecolor{currentstroke}%
\pgfsetdash{}{0pt}%
\pgfsys@defobject{currentmarker}{\pgfqpoint{-0.041667in}{-0.041667in}}{\pgfqpoint{0.041667in}{0.041667in}}{%
\pgfpathmoveto{\pgfqpoint{-0.041667in}{-0.041667in}}%
\pgfpathlineto{\pgfqpoint{0.041667in}{0.041667in}}%
\pgfpathmoveto{\pgfqpoint{-0.041667in}{0.041667in}}%
\pgfpathlineto{\pgfqpoint{0.041667in}{-0.041667in}}%
\pgfusepath{stroke,fill}%
}%
\begin{pgfscope}%
\pgfsys@transformshift{3.509894in}{0.184722in}%
\pgfsys@useobject{currentmarker}{}%
\end{pgfscope}%
\end{pgfscope}%
\begin{pgfscope}%
\definecolor{textcolor}{rgb}{0.000000,0.000000,0.000000}%
\pgfsetstrokecolor{textcolor}%
\pgfsetfillcolor{textcolor}%
\pgftext[x=3.859894in,y=0.116667in,left,base]{\color{textcolor}\rmfamily\fontsize{14.000000}{16.800000}\selectfont Men}%
\end{pgfscope}%
\end{pgfpicture}%
\makeatother%
\endgroup%
 } 
\end{subfigure}






\begin{minipage}{0.99\textwidth} % choose width suitably

\hspace{50em}

{\footnotesize \textsc{Notes.} The figures display the evolution of net-worth (measured in 1997\$). The displayed patterns are normalized coefficients from event studies around divorce. Rich household are defined as those whose net-worth is above the median in the first period they were observed. Poor households are those whose net-worth is below the $75^{th}$ percentile of the distribution. Net worth is constructed using the same PSID variables that \cite{blundell2016} use.)\par}
\end{minipage}
\end{figure}
\FloatBarrier


\section{Computational Appendix}\label{section:computation}
\cite{arnoud2019} compares an array of local and global optimizers, which are given the task of finding the global optimum of difficult objective functions. They find that the multi-start algorithm that they propose, called TikTak, outperforms the others in terms of time required to reach the solution and the probability that the algorithm finds the optimum. In light of these findings, we decided to use TikTak for solving problem (\ref{eq:msm}). A description of the TikTak algorithm follows:
\begin{enumerate}
\item Determine the bounds for each parameter and generate a sequence of Sobol points with length $N$. Then evaluate the function value at each Sobol point.
\item Sort the $N$ Sobol points ($s_1$ ,..., $s_N$), with $f(s_1)\le$ ··· $\le f(s_N)$ and keep the first $N^*$ with $N^*<N$. Note that $f()$ is the objective function. We set $N^*$ such that $N^*/N=0.15.$ Set the global iteration number $j$ to 1, then run a local minimizer starting from $s_1$. Call $z^*_j$ the fit resulting from the local minimization,\footnote{We use the local minimization algorithm provided by \cite{cartis2019}, which is a derivative-free optimization (DFO) for nonlinear Least-Squares (LS) problems. This algorithm is robust to noise, which might arise because of the errors coming from the approximation of continuous problems on a discrete grid.} and define the set $Z^*_1=\min \{z^*_1\}=z^*_1 $.
\item Define a new starting point $\hat{s}_{j+1}$ defined as
$$\hat{s}_{j+1}=(1-\theta_j)s_{j+1}+\theta_j Z^*_j, $$
where
$$\theta_j=\min \big[\max[0,1,(j/N^*)^\frac{1}{2}],0.995\big]. $$
Run a local minimizer starting from $\hat{s}_{j+1}$ and call the local minimum found $z^*_{j+1}$. Then, define $Z^*_{j+1}=\min \{z^*_1,...,z^*_{j+1} \} $. Update the global iteration number: $j=j+1$. Repeat step 3 until $j=N^*$.

\item  Return $Z^*_{N^*}$.
\end{enumerate}
We adapt the original algorithm such that it can be run in parallel using $M$ nodes. Other than evaluating more points at the same time on different nodes, the only difference is the step 3. In the parallel version of TikTak, $Z^*_{j}$ is defined as the minimum among the outcomes  of the local minimizers  that already converged, while at the end of step 3 the global iteration number is updated to $j^*$, which stands for the number of global minimization that already started, without necessarily having converged already.
\section{Estimation of Income Processes}

%MARRUAGE TO DIVORCE
{	
	\def\onepc{$^{\ast\ast}$} \def\fivepc{$^{\ast}$}
	\def\tenpc{$^{\dag}$}
	\def\legend{\multicolumn{1}{l}{\footnotesize{Significance levels
				:\hspace{1em} $\ast$ : 10\% \hspace{1em}
				$\ast\ast$ : 5\% \hspace{1em} $\ast\ast\ast$ : 1\% \normalsize}}}
	\begin{table}[H]\centering
	\caption{\\OLS Regression. Observation: males in year $t$.}
	\label{table:inc_men}
	\begin{threeparttable}[t]\centering
	{\def\sym#1{\ifmmode^{#1}\else\(^{#1}\)\fi}              \begin{tabular}{l*{1}{c}}                          \toprule             &\multicolumn{1}{c}{(1)}       \\             \midrule             \textsc{Dep. Variable:} &  \\\textsc{Male Log Earnings} & \\ & \\
$\iota^m_1$         &        0.05\\
$\iota^m_2$         &       -0.00\\
$\iota^m_0$           &       -0.34\\
Survey Year Fixed Effects  & \checkmark   \\         State Fixed Effects  & \checkmark  \\                         \hline
Observations        &       98118\\
\(R^{2}\)           &       0.152\\
\hline

	\end{tabular}}
\begin{tablenotes}[flushleft]
\footnotesize{\item \textsc{Notes}: Standard errors are obtained through bootstrapping and they are reported in summary table \ref{table:income_params}.}
\end{tablenotes}
\end{threeparttable}
\end{table}
}
\FloatBarrier
%MARRUAGE TO DIVORCE
{	
	\def\onepc{$^{\ast\ast}$} \def\fivepc{$^{\ast}$}
	\def\tenpc{$^{\dag}$}
	\def\legend{\multicolumn{1}{l}{\footnotesize{Significance levels
				:\hspace{1em} $\ast$ : 10\% \hspace{1em}
				$\ast\ast$ : 5\% \hspace{1em} $\ast\ast\ast$ : 1\% \normalsize}}}
	\begin{table}[H]\centering
	\caption{\\OLS Regression. Observation: Females in Year $t$.}
	\label{table:inc_wom}
	\begin{threeparttable}[t]\centering
	{\def\sym#1{\ifmmode^{#1}\else\(^{#1}\)\fi}              \begin{tabular}{l*{1}{c}}                          \toprule             &\multicolumn{1}{c}{(1)}       \\             \midrule             \textsc{Dep. Variable:} &  \\\textsc{Female Labor Earnings} & \\ & \\
$\iota^f_1$         &        0.02\\
$\iota^f_2$         &       -0.00\\
$\iota^f_0$            &       -0.38\\
Survey Year Fixed Effects  & \checkmark   \\         State Fixed Effects  & \checkmark  \\                         \hline
Observations        &       86891\\
\(R^{2}\)           &       0.085\\
\hline

	\end{tabular}}
\begin{tablenotes}[flushleft]
\footnotesize{\item \textsc{Notes}: Standard errors are obtained through bootstrapping and they are reported in summary table \ref{table:income_params}.}
\end{tablenotes}
\end{threeparttable}
\end{table}
}
\FloatBarrier

%MARRUAGE TO DIVORCE
{	
	\def\onepc{$^{\ast\ast}$} \def\fivepc{$^{\ast}$}
	\def\tenpc{$^{\dag}$}
	\def\legend{\multicolumn{1}{l}{\footnotesize{Significance levels
				:\hspace{1em} $\ast$ : 10\% \hspace{1em}
				$\ast\ast$ : 5\% \hspace{1em} $\ast\ast\ast$ : 1\% \normalsize}}}
	\begin{table}[H]\centering
	\caption{\\Probit Regression. Observation: Females in Year $t$.}
	\label{table:prb_wom}
	\begin{threeparttable}[t]\centering
	{\def\sym#1{\ifmmode^{#1}\else\(^{#1}\)\fi}              \begin{tabular}{l*{1}{c}}                          \toprule             &\multicolumn{1}{c}{(1)}       \\             \midrule             \textsc{Dep. Variable:} &  \\\textsc{Female Labor Force Participation} & \\ & \\
Unilateral Divorce$*$Community Property&       -0.18\sym{***}\\
Unilateral Divorce$*$Title Based&       -0.08         \\
Unilateral Divorce$*$Equitable Distribution&       -0.06         \\
Equitable Distribution&       -0.00         \\
$\iota^f_1$         &        0.01\sym{***}\\
$\iota^f_2$         &       -0.00\sym{***}\\
$\iota^f_0$            &        1.95         \\
Survey Year Fixed Effects  & \checkmark   \\         State Fixed Effects  & \checkmark  \\                         \hline
Observations        &      127728         \\
\hline

	\end{tabular}}
\begin{tablenotes}[flushleft]
\footnotesize{\item \textsc{Notes}: standard errors are clustered at the state level.
	Coefficients that are significantly different from zero are denoted by the following system: *10\%, **5\%  and ***1\%. \ref{table:income_params}.}
\end{tablenotes}
\end{threeparttable}
\end{table}
}
\FloatBarrier
\section{Robustness Checks - Empirical Analysis}
\subsubsection*{Relationship Choice - Linear State Trends}

%RELATIONSHIP CHOICE
	\begin{table}[H]\centering
	\caption{\\OLS Regression. Observation: first and second relationships}
	\label{table:wmarlin}
	\begin{threeparttable}[t]\centering
	
% Table created by stargazer v.5.2.2 by Marek Hlavac, Harvard University. E-mail: hlavac at fas.harvard.edu
% Date and time: mer, feb 12, 2020 - 14:37:19
\begingroup 
\footnotesize 
\begin{tabular}{@{\extracolsep{5pt}}lcccc} 
\\[-1.8ex]\hline 
\hline \\[-1.8ex] 
 & \multicolumn{4}{c}{\textit{Dependent variable:}} \\ 
\cline{2-5} 
\\[-1.8ex] & \multicolumn{4}{c}{Married (0/1)} \\ 
 & Full Sample & Resident & NSFH & NSFG \\ 
\\[-1.8ex] & (1) & (2) & (3) & (4)\\ 
\hline \\[-1.8ex] 
 Unilateral Divorce & $-$0.060$^{***}$ & $-$0.071$^{***}$ & $-$0.071$^{***}$ & 0.005 \\ 
  & (0.020) & (0.024) & (0.022) & (0.053) \\ 
 \hline \\[-1.8ex] 
State Fixed effects & Yes & Yes & Yes & Yes \\ 
Age Polynomials & Yes & Yes & Yes & Yes \\ 
Year started Fixed Effect & Yes & Yes & Yes & Yes \\ 
Linear trend by State & Yes & Yes & Yes & Yes \\ 
Demographic Controls & Yes & Yes & Yes & Yes \\ 
Observations & 13,627 & 8,357 & 10,830 & 2,797 \\ 
R$^{2}$ & 0.208 & 0.227 & 0.232 & 0.152 \\ 
\hline 
\hline \\[-1.8ex] 
\end{tabular} 
\endgroup 

	\begin{tablenotes}[flushleft]
	\footnotesize{\item \textsc{Notes}: standard errors are clustered at the state level.
		Coefficients that are significantly different from zero are denoted by the following system: *10\%, **5\%  and ***1\%.}
	\end{tablenotes}
	\end{threeparttable}
	\end{table}
\FloatBarrier
\subsubsection*{Relationship Choice - Heterogeneity by Property regime and linear state trends}

%RELATIONSHIP CHOICE

	\begin{table}[H]\centering
	\caption{\\OLS Regression. Observation: first and second relationships}
	\label{table:wmarlinc}
	\begin{threeparttable}[t]\centering
	
% Table created by stargazer v.5.2.2 by Marek Hlavac, Harvard University. E-mail: hlavac at fas.harvard.edu
% Date and time: mer, feb 12, 2020 - 14:42:40
\begingroup 
\footnotesize 
\begin{tabular}{@{\extracolsep{5pt}}lcccc} 
\\[-1.8ex]\hline 
\hline \\[-1.8ex] 
 & \multicolumn{4}{c}{\textit{Dependent variable:}} \\ 
\cline{2-5} 
\\[-1.8ex] & \multicolumn{4}{c}{Married (0/1)} \\ 
 & Full Sample & Resident & NSFH & NSFG \\ 
\\[-1.8ex] & (1) & (2) & (3) & (4)\\ 
\hline \\[-1.8ex] 
 UnDiv*NoTit & $-$0.062$^{***}$ & $-$0.074$^{***}$ & $-$0.072$^{***}$ & 0.004 \\ 
  & (0.021) & (0.025) & (0.022) & (0.056) \\ 
  UnDiv*Tit & $-$0.015 & $-$0.045 & $-$0.023 & 0.012 \\ 
  & (0.026) & (0.046) & (0.040) & (0.051) \\ 
  Tit & $-$0.034$^{**}$ & $-$0.036 & $-$0.030$^{*}$ & $-$0.054 \\ 
  & (0.017) & (0.024) & (0.017) & (0.049) \\ 
 \hline \\[-1.8ex] 
State Fixed effects & Yes & Yes & Yes & Yes \\ 
Age Polynomials & Yes & Yes & Yes & Yes \\ 
Year started Fixed Effect & Yes & Yes & Yes & Yes \\ 
Linear trend by State & Yes & Yes & Yes & Yes \\ 
Demographic Controls & Yes & Yes & Yes & Yes \\ 
Observations & 13,627 & 8,357 & 10,830 & 2,797 \\ 
R$^{2}$ & 0.192 & 0.227 & 0.209 & 0.128 \\ 
\hline 
\hline \\[-1.8ex] 
\end{tabular} 
\endgroup 

	\begin{tablenotes}[flushleft]
	\footnotesize{\item \textsc{Notes}: standard errors are clustered at the state level.
		Coefficients that are significantly different from zero are denoted by the following system: *10\%, **5\%  and ***1\%.}
	\end{tablenotes}
	\end{threeparttable}
	\end{table}
\FloatBarrier

\FloatBarrier
\subsubsection*{Relationship Choice - Logit}

%RELATIONSHIP CHOICE
	\begin{table}[H]\centering
	\caption{\\Logitstic regression. Observation: first and second relationships}
	\label{table:wmar_logit}
	\begin{threeparttable}[t]\centering
	
% Table created by stargazer v.5.2.2 by Marek Hlavac, Harvard University. E-mail: hlavac at fas.harvard.edu
% Date and time: mer, feb 12, 2020 - 15:22:49
\begingroup 
\footnotesize 
\begin{tabular}{@{\extracolsep{5pt}}lcccc} 
\\[-1.8ex]\hline 
\hline \\[-1.8ex] 
 & \multicolumn{4}{c}{\textit{Dependent variable:}} \\ 
\cline{2-5} 
\\[-1.8ex] & \multicolumn{4}{c}{Married (0/1)} \\ 
 & Full Sample & Resident & NSFH & NSFG \\ 
\\[-1.8ex] & (1) & (2) & (3) & (4)\\ 
\hline \\[-1.8ex] 
 Unilateral Divorce & $-$0.307$^{***}$ & $-$0.387$^{***}$ & $-$0.354$^{***}$ & $-$0.317 \\ 
  & (0.095) & (0.127) & (0.107) & (0.229) \\ 
 \hline \\[-1.8ex] 
State Fixed effects & Yes & Yes & Yes & Yes \\ 
Age Polynomials & Yes & Yes & Yes & Yes \\ 
Year started Fixed Effect & Yes & Yes & Yes & Yes \\ 
Demographic Controls & Yes & Yes & Yes & Yes \\ 
\textbf{Average Marginal Effects} & -0.051 & -0.062 & -0.051 & -0.062 \\ 
Observations & 13,627 & 8,357 & 10,830 & 2,797 \\ 
\hline 
\hline \\[-1.8ex] 
\end{tabular} 
\endgroup 

	\begin{tablenotes}[flushleft]
	\footnotesize{\item \textsc{Notes}: standard errors are clustered at the state level.
		Coefficients that are significantly different from zero are denoted by the following system: *10\%, **5\%  and ***1\%.}
	\end{tablenotes}
	\end{threeparttable}
	\end{table}
\FloatBarrier

\newpage

\section{Figures}

%%%%%%%%%%%%%%%%%%%%%%%%5
% MODE FIT
%%%%%%%%%%%%%%%%%%%%%%%%5

\subsection{Model Fit}


\FloatBarrier
\begin{figure}[H]
\caption{\\Hazards by duration of spells: data and simulations}
\label{fig:haz}
\begin{center}
\begin{subfigure}{.49\textwidth}
\centering
\caption{Hazard of Divorce}
\label{fig:hazd}
\scalebox{0.49}{%% Creator: Matplotlib, PGF backend
%%
%% To include the figure in your LaTeX document, write
%%   \input{<filename>.pgf}
%%
%% Make sure the required packages are loaded in your preamble
%%   \usepackage{pgf}
%%
%% and, on pdftex
%%   \usepackage[utf8]{inputenc}\DeclareUnicodeCharacter{2212}{-}
%%
%% or, on luatex and xetex
%%   \usepackage{unicode-math}
%%
%% Figures using additional raster images can only be included by \input if
%% they are in the same directory as the main LaTeX file. For loading figures
%% from other directories you can use the `import` package
%%   \usepackage{import}
%%
%% and then include the figures with
%%   \import{<path to file>}{<filename>.pgf}
%%
%% Matplotlib used the following preamble
%%
\begingroup%
\makeatletter%
\begin{pgfpicture}%
\pgfpathrectangle{\pgfpointorigin}{\pgfqpoint{5.286497in}{3.903555in}}%
\pgfusepath{use as bounding box, clip}%
\begin{pgfscope}%
\pgfsetbuttcap%
\pgfsetmiterjoin%
\definecolor{currentfill}{rgb}{1.000000,1.000000,1.000000}%
\pgfsetfillcolor{currentfill}%
\pgfsetlinewidth{0.000000pt}%
\definecolor{currentstroke}{rgb}{1.000000,1.000000,1.000000}%
\pgfsetstrokecolor{currentstroke}%
\pgfsetdash{}{0pt}%
\pgfpathmoveto{\pgfqpoint{0.000000in}{0.000000in}}%
\pgfpathlineto{\pgfqpoint{5.286497in}{0.000000in}}%
\pgfpathlineto{\pgfqpoint{5.286497in}{3.903555in}}%
\pgfpathlineto{\pgfqpoint{0.000000in}{3.903555in}}%
\pgfpathclose%
\pgfusepath{fill}%
\end{pgfscope}%
\begin{pgfscope}%
\pgfsetbuttcap%
\pgfsetmiterjoin%
\definecolor{currentfill}{rgb}{1.000000,1.000000,1.000000}%
\pgfsetfillcolor{currentfill}%
\pgfsetlinewidth{0.000000pt}%
\definecolor{currentstroke}{rgb}{0.000000,0.000000,0.000000}%
\pgfsetstrokecolor{currentstroke}%
\pgfsetstrokeopacity{0.000000}%
\pgfsetdash{}{0pt}%
\pgfpathmoveto{\pgfqpoint{0.636497in}{0.883555in}}%
\pgfpathlineto{\pgfqpoint{5.286497in}{0.883555in}}%
\pgfpathlineto{\pgfqpoint{5.286497in}{3.903555in}}%
\pgfpathlineto{\pgfqpoint{0.636497in}{3.903555in}}%
\pgfpathclose%
\pgfusepath{fill}%
\end{pgfscope}%
\begin{pgfscope}%
\pgfpathrectangle{\pgfqpoint{0.636497in}{0.883555in}}{\pgfqpoint{4.650000in}{3.020000in}}%
\pgfusepath{clip}%
\pgfsetbuttcap%
\pgfsetroundjoin%
\definecolor{currentfill}{rgb}{0.000000,0.000000,1.000000}%
\pgfsetfillcolor{currentfill}%
\pgfsetfillopacity{0.200000}%
\pgfsetlinewidth{0.000000pt}%
\definecolor{currentstroke}{rgb}{0.000000,0.000000,0.000000}%
\pgfsetstrokecolor{currentstroke}%
\pgfsetdash{}{0pt}%
\pgfpathmoveto{\pgfqpoint{0.847861in}{2.777698in}}%
\pgfpathlineto{\pgfqpoint{0.847861in}{2.077589in}}%
\pgfpathlineto{\pgfqpoint{1.693315in}{2.471151in}}%
\pgfpathlineto{\pgfqpoint{2.538770in}{2.287934in}}%
\pgfpathlineto{\pgfqpoint{3.384224in}{2.395377in}}%
\pgfpathlineto{\pgfqpoint{4.229679in}{2.111876in}}%
\pgfpathlineto{\pgfqpoint{5.075133in}{1.859216in}}%
\pgfpathlineto{\pgfqpoint{5.075133in}{2.947720in}}%
\pgfpathlineto{\pgfqpoint{5.075133in}{2.947720in}}%
\pgfpathlineto{\pgfqpoint{4.229679in}{3.206604in}}%
\pgfpathlineto{\pgfqpoint{3.384224in}{3.377565in}}%
\pgfpathlineto{\pgfqpoint{2.538770in}{3.132646in}}%
\pgfpathlineto{\pgfqpoint{1.693315in}{3.643765in}}%
\pgfpathlineto{\pgfqpoint{0.847861in}{2.777698in}}%
\pgfpathclose%
\pgfusepath{fill}%
\end{pgfscope}%
\begin{pgfscope}%
\pgfsetbuttcap%
\pgfsetroundjoin%
\definecolor{currentfill}{rgb}{0.000000,0.000000,0.000000}%
\pgfsetfillcolor{currentfill}%
\pgfsetlinewidth{0.803000pt}%
\definecolor{currentstroke}{rgb}{0.000000,0.000000,0.000000}%
\pgfsetstrokecolor{currentstroke}%
\pgfsetdash{}{0pt}%
\pgfsys@defobject{currentmarker}{\pgfqpoint{0.000000in}{-0.048611in}}{\pgfqpoint{0.000000in}{0.000000in}}{%
\pgfpathmoveto{\pgfqpoint{0.000000in}{0.000000in}}%
\pgfpathlineto{\pgfqpoint{0.000000in}{-0.048611in}}%
\pgfusepath{stroke,fill}%
}%
\begin{pgfscope}%
\pgfsys@transformshift{0.847861in}{0.883555in}%
\pgfsys@useobject{currentmarker}{}%
\end{pgfscope}%
\end{pgfscope}%
\begin{pgfscope}%
\definecolor{textcolor}{rgb}{0.000000,0.000000,0.000000}%
\pgfsetstrokecolor{textcolor}%
\pgfsetfillcolor{textcolor}%
\pgftext[x=0.847861in,y=0.786333in,,top]{\color{textcolor}\rmfamily\fontsize{11.000000}{13.200000}\selectfont 1-2}%
\end{pgfscope}%
\begin{pgfscope}%
\pgfsetbuttcap%
\pgfsetroundjoin%
\definecolor{currentfill}{rgb}{0.000000,0.000000,0.000000}%
\pgfsetfillcolor{currentfill}%
\pgfsetlinewidth{0.803000pt}%
\definecolor{currentstroke}{rgb}{0.000000,0.000000,0.000000}%
\pgfsetstrokecolor{currentstroke}%
\pgfsetdash{}{0pt}%
\pgfsys@defobject{currentmarker}{\pgfqpoint{0.000000in}{-0.048611in}}{\pgfqpoint{0.000000in}{0.000000in}}{%
\pgfpathmoveto{\pgfqpoint{0.000000in}{0.000000in}}%
\pgfpathlineto{\pgfqpoint{0.000000in}{-0.048611in}}%
\pgfusepath{stroke,fill}%
}%
\begin{pgfscope}%
\pgfsys@transformshift{1.693315in}{0.883555in}%
\pgfsys@useobject{currentmarker}{}%
\end{pgfscope}%
\end{pgfscope}%
\begin{pgfscope}%
\definecolor{textcolor}{rgb}{0.000000,0.000000,0.000000}%
\pgfsetstrokecolor{textcolor}%
\pgfsetfillcolor{textcolor}%
\pgftext[x=1.693315in,y=0.786333in,,top]{\color{textcolor}\rmfamily\fontsize{11.000000}{13.200000}\selectfont 3-4}%
\end{pgfscope}%
\begin{pgfscope}%
\pgfsetbuttcap%
\pgfsetroundjoin%
\definecolor{currentfill}{rgb}{0.000000,0.000000,0.000000}%
\pgfsetfillcolor{currentfill}%
\pgfsetlinewidth{0.803000pt}%
\definecolor{currentstroke}{rgb}{0.000000,0.000000,0.000000}%
\pgfsetstrokecolor{currentstroke}%
\pgfsetdash{}{0pt}%
\pgfsys@defobject{currentmarker}{\pgfqpoint{0.000000in}{-0.048611in}}{\pgfqpoint{0.000000in}{0.000000in}}{%
\pgfpathmoveto{\pgfqpoint{0.000000in}{0.000000in}}%
\pgfpathlineto{\pgfqpoint{0.000000in}{-0.048611in}}%
\pgfusepath{stroke,fill}%
}%
\begin{pgfscope}%
\pgfsys@transformshift{2.538770in}{0.883555in}%
\pgfsys@useobject{currentmarker}{}%
\end{pgfscope}%
\end{pgfscope}%
\begin{pgfscope}%
\definecolor{textcolor}{rgb}{0.000000,0.000000,0.000000}%
\pgfsetstrokecolor{textcolor}%
\pgfsetfillcolor{textcolor}%
\pgftext[x=2.538770in,y=0.786333in,,top]{\color{textcolor}\rmfamily\fontsize{11.000000}{13.200000}\selectfont 5-6}%
\end{pgfscope}%
\begin{pgfscope}%
\pgfsetbuttcap%
\pgfsetroundjoin%
\definecolor{currentfill}{rgb}{0.000000,0.000000,0.000000}%
\pgfsetfillcolor{currentfill}%
\pgfsetlinewidth{0.803000pt}%
\definecolor{currentstroke}{rgb}{0.000000,0.000000,0.000000}%
\pgfsetstrokecolor{currentstroke}%
\pgfsetdash{}{0pt}%
\pgfsys@defobject{currentmarker}{\pgfqpoint{0.000000in}{-0.048611in}}{\pgfqpoint{0.000000in}{0.000000in}}{%
\pgfpathmoveto{\pgfqpoint{0.000000in}{0.000000in}}%
\pgfpathlineto{\pgfqpoint{0.000000in}{-0.048611in}}%
\pgfusepath{stroke,fill}%
}%
\begin{pgfscope}%
\pgfsys@transformshift{3.384224in}{0.883555in}%
\pgfsys@useobject{currentmarker}{}%
\end{pgfscope}%
\end{pgfscope}%
\begin{pgfscope}%
\definecolor{textcolor}{rgb}{0.000000,0.000000,0.000000}%
\pgfsetstrokecolor{textcolor}%
\pgfsetfillcolor{textcolor}%
\pgftext[x=3.384224in,y=0.786333in,,top]{\color{textcolor}\rmfamily\fontsize{11.000000}{13.200000}\selectfont 7-8}%
\end{pgfscope}%
\begin{pgfscope}%
\pgfsetbuttcap%
\pgfsetroundjoin%
\definecolor{currentfill}{rgb}{0.000000,0.000000,0.000000}%
\pgfsetfillcolor{currentfill}%
\pgfsetlinewidth{0.803000pt}%
\definecolor{currentstroke}{rgb}{0.000000,0.000000,0.000000}%
\pgfsetstrokecolor{currentstroke}%
\pgfsetdash{}{0pt}%
\pgfsys@defobject{currentmarker}{\pgfqpoint{0.000000in}{-0.048611in}}{\pgfqpoint{0.000000in}{0.000000in}}{%
\pgfpathmoveto{\pgfqpoint{0.000000in}{0.000000in}}%
\pgfpathlineto{\pgfqpoint{0.000000in}{-0.048611in}}%
\pgfusepath{stroke,fill}%
}%
\begin{pgfscope}%
\pgfsys@transformshift{4.229679in}{0.883555in}%
\pgfsys@useobject{currentmarker}{}%
\end{pgfscope}%
\end{pgfscope}%
\begin{pgfscope}%
\definecolor{textcolor}{rgb}{0.000000,0.000000,0.000000}%
\pgfsetstrokecolor{textcolor}%
\pgfsetfillcolor{textcolor}%
\pgftext[x=4.229679in,y=0.786333in,,top]{\color{textcolor}\rmfamily\fontsize{11.000000}{13.200000}\selectfont 9-10}%
\end{pgfscope}%
\begin{pgfscope}%
\pgfsetbuttcap%
\pgfsetroundjoin%
\definecolor{currentfill}{rgb}{0.000000,0.000000,0.000000}%
\pgfsetfillcolor{currentfill}%
\pgfsetlinewidth{0.803000pt}%
\definecolor{currentstroke}{rgb}{0.000000,0.000000,0.000000}%
\pgfsetstrokecolor{currentstroke}%
\pgfsetdash{}{0pt}%
\pgfsys@defobject{currentmarker}{\pgfqpoint{0.000000in}{-0.048611in}}{\pgfqpoint{0.000000in}{0.000000in}}{%
\pgfpathmoveto{\pgfqpoint{0.000000in}{0.000000in}}%
\pgfpathlineto{\pgfqpoint{0.000000in}{-0.048611in}}%
\pgfusepath{stroke,fill}%
}%
\begin{pgfscope}%
\pgfsys@transformshift{5.075133in}{0.883555in}%
\pgfsys@useobject{currentmarker}{}%
\end{pgfscope}%
\end{pgfscope}%
\begin{pgfscope}%
\definecolor{textcolor}{rgb}{0.000000,0.000000,0.000000}%
\pgfsetstrokecolor{textcolor}%
\pgfsetfillcolor{textcolor}%
\pgftext[x=5.075133in,y=0.786333in,,top]{\color{textcolor}\rmfamily\fontsize{11.000000}{13.200000}\selectfont 11-12}%
\end{pgfscope}%
\begin{pgfscope}%
\definecolor{textcolor}{rgb}{0.000000,0.000000,0.000000}%
\pgfsetstrokecolor{textcolor}%
\pgfsetfillcolor{textcolor}%
\pgftext[x=2.961497in,y=0.595592in,,top]{\color{textcolor}\rmfamily\fontsize{16.000000}{19.200000}\selectfont Duration - years}%
\end{pgfscope}%
\begin{pgfscope}%
\pgfsetbuttcap%
\pgfsetroundjoin%
\definecolor{currentfill}{rgb}{0.000000,0.000000,0.000000}%
\pgfsetfillcolor{currentfill}%
\pgfsetlinewidth{0.803000pt}%
\definecolor{currentstroke}{rgb}{0.000000,0.000000,0.000000}%
\pgfsetstrokecolor{currentstroke}%
\pgfsetdash{}{0pt}%
\pgfsys@defobject{currentmarker}{\pgfqpoint{-0.048611in}{0.000000in}}{\pgfqpoint{0.000000in}{0.000000in}}{%
\pgfpathmoveto{\pgfqpoint{0.000000in}{0.000000in}}%
\pgfpathlineto{\pgfqpoint{-0.048611in}{0.000000in}}%
\pgfusepath{stroke,fill}%
}%
\begin{pgfscope}%
\pgfsys@transformshift{0.636497in}{0.883555in}%
\pgfsys@useobject{currentmarker}{}%
\end{pgfscope}%
\end{pgfscope}%
\begin{pgfscope}%
\definecolor{textcolor}{rgb}{0.000000,0.000000,0.000000}%
\pgfsetstrokecolor{textcolor}%
\pgfsetfillcolor{textcolor}%
\pgftext[x=0.268904in, y=0.830748in, left, base]{\color{textcolor}\rmfamily\fontsize{11.000000}{13.200000}\selectfont \(\displaystyle {0.00}\)}%
\end{pgfscope}%
\begin{pgfscope}%
\pgfsetbuttcap%
\pgfsetroundjoin%
\definecolor{currentfill}{rgb}{0.000000,0.000000,0.000000}%
\pgfsetfillcolor{currentfill}%
\pgfsetlinewidth{0.803000pt}%
\definecolor{currentstroke}{rgb}{0.000000,0.000000,0.000000}%
\pgfsetstrokecolor{currentstroke}%
\pgfsetdash{}{0pt}%
\pgfsys@defobject{currentmarker}{\pgfqpoint{-0.048611in}{0.000000in}}{\pgfqpoint{0.000000in}{0.000000in}}{%
\pgfpathmoveto{\pgfqpoint{0.000000in}{0.000000in}}%
\pgfpathlineto{\pgfqpoint{-0.048611in}{0.000000in}}%
\pgfusepath{stroke,fill}%
}%
\begin{pgfscope}%
\pgfsys@transformshift{0.636497in}{1.369318in}%
\pgfsys@useobject{currentmarker}{}%
\end{pgfscope}%
\end{pgfscope}%
\begin{pgfscope}%
\definecolor{textcolor}{rgb}{0.000000,0.000000,0.000000}%
\pgfsetstrokecolor{textcolor}%
\pgfsetfillcolor{textcolor}%
\pgftext[x=0.268904in, y=1.316511in, left, base]{\color{textcolor}\rmfamily\fontsize{11.000000}{13.200000}\selectfont \(\displaystyle {0.02}\)}%
\end{pgfscope}%
\begin{pgfscope}%
\pgfsetbuttcap%
\pgfsetroundjoin%
\definecolor{currentfill}{rgb}{0.000000,0.000000,0.000000}%
\pgfsetfillcolor{currentfill}%
\pgfsetlinewidth{0.803000pt}%
\definecolor{currentstroke}{rgb}{0.000000,0.000000,0.000000}%
\pgfsetstrokecolor{currentstroke}%
\pgfsetdash{}{0pt}%
\pgfsys@defobject{currentmarker}{\pgfqpoint{-0.048611in}{0.000000in}}{\pgfqpoint{0.000000in}{0.000000in}}{%
\pgfpathmoveto{\pgfqpoint{0.000000in}{0.000000in}}%
\pgfpathlineto{\pgfqpoint{-0.048611in}{0.000000in}}%
\pgfusepath{stroke,fill}%
}%
\begin{pgfscope}%
\pgfsys@transformshift{0.636497in}{1.855081in}%
\pgfsys@useobject{currentmarker}{}%
\end{pgfscope}%
\end{pgfscope}%
\begin{pgfscope}%
\definecolor{textcolor}{rgb}{0.000000,0.000000,0.000000}%
\pgfsetstrokecolor{textcolor}%
\pgfsetfillcolor{textcolor}%
\pgftext[x=0.268904in, y=1.802274in, left, base]{\color{textcolor}\rmfamily\fontsize{11.000000}{13.200000}\selectfont \(\displaystyle {0.04}\)}%
\end{pgfscope}%
\begin{pgfscope}%
\pgfsetbuttcap%
\pgfsetroundjoin%
\definecolor{currentfill}{rgb}{0.000000,0.000000,0.000000}%
\pgfsetfillcolor{currentfill}%
\pgfsetlinewidth{0.803000pt}%
\definecolor{currentstroke}{rgb}{0.000000,0.000000,0.000000}%
\pgfsetstrokecolor{currentstroke}%
\pgfsetdash{}{0pt}%
\pgfsys@defobject{currentmarker}{\pgfqpoint{-0.048611in}{0.000000in}}{\pgfqpoint{0.000000in}{0.000000in}}{%
\pgfpathmoveto{\pgfqpoint{0.000000in}{0.000000in}}%
\pgfpathlineto{\pgfqpoint{-0.048611in}{0.000000in}}%
\pgfusepath{stroke,fill}%
}%
\begin{pgfscope}%
\pgfsys@transformshift{0.636497in}{2.340844in}%
\pgfsys@useobject{currentmarker}{}%
\end{pgfscope}%
\end{pgfscope}%
\begin{pgfscope}%
\definecolor{textcolor}{rgb}{0.000000,0.000000,0.000000}%
\pgfsetstrokecolor{textcolor}%
\pgfsetfillcolor{textcolor}%
\pgftext[x=0.268904in, y=2.288037in, left, base]{\color{textcolor}\rmfamily\fontsize{11.000000}{13.200000}\selectfont \(\displaystyle {0.06}\)}%
\end{pgfscope}%
\begin{pgfscope}%
\pgfsetbuttcap%
\pgfsetroundjoin%
\definecolor{currentfill}{rgb}{0.000000,0.000000,0.000000}%
\pgfsetfillcolor{currentfill}%
\pgfsetlinewidth{0.803000pt}%
\definecolor{currentstroke}{rgb}{0.000000,0.000000,0.000000}%
\pgfsetstrokecolor{currentstroke}%
\pgfsetdash{}{0pt}%
\pgfsys@defobject{currentmarker}{\pgfqpoint{-0.048611in}{0.000000in}}{\pgfqpoint{0.000000in}{0.000000in}}{%
\pgfpathmoveto{\pgfqpoint{0.000000in}{0.000000in}}%
\pgfpathlineto{\pgfqpoint{-0.048611in}{0.000000in}}%
\pgfusepath{stroke,fill}%
}%
\begin{pgfscope}%
\pgfsys@transformshift{0.636497in}{2.826607in}%
\pgfsys@useobject{currentmarker}{}%
\end{pgfscope}%
\end{pgfscope}%
\begin{pgfscope}%
\definecolor{textcolor}{rgb}{0.000000,0.000000,0.000000}%
\pgfsetstrokecolor{textcolor}%
\pgfsetfillcolor{textcolor}%
\pgftext[x=0.268904in, y=2.773800in, left, base]{\color{textcolor}\rmfamily\fontsize{11.000000}{13.200000}\selectfont \(\displaystyle {0.08}\)}%
\end{pgfscope}%
\begin{pgfscope}%
\pgfsetbuttcap%
\pgfsetroundjoin%
\definecolor{currentfill}{rgb}{0.000000,0.000000,0.000000}%
\pgfsetfillcolor{currentfill}%
\pgfsetlinewidth{0.803000pt}%
\definecolor{currentstroke}{rgb}{0.000000,0.000000,0.000000}%
\pgfsetstrokecolor{currentstroke}%
\pgfsetdash{}{0pt}%
\pgfsys@defobject{currentmarker}{\pgfqpoint{-0.048611in}{0.000000in}}{\pgfqpoint{0.000000in}{0.000000in}}{%
\pgfpathmoveto{\pgfqpoint{0.000000in}{0.000000in}}%
\pgfpathlineto{\pgfqpoint{-0.048611in}{0.000000in}}%
\pgfusepath{stroke,fill}%
}%
\begin{pgfscope}%
\pgfsys@transformshift{0.636497in}{3.312369in}%
\pgfsys@useobject{currentmarker}{}%
\end{pgfscope}%
\end{pgfscope}%
\begin{pgfscope}%
\definecolor{textcolor}{rgb}{0.000000,0.000000,0.000000}%
\pgfsetstrokecolor{textcolor}%
\pgfsetfillcolor{textcolor}%
\pgftext[x=0.268904in, y=3.259563in, left, base]{\color{textcolor}\rmfamily\fontsize{11.000000}{13.200000}\selectfont \(\displaystyle {0.10}\)}%
\end{pgfscope}%
\begin{pgfscope}%
\pgfsetbuttcap%
\pgfsetroundjoin%
\definecolor{currentfill}{rgb}{0.000000,0.000000,0.000000}%
\pgfsetfillcolor{currentfill}%
\pgfsetlinewidth{0.803000pt}%
\definecolor{currentstroke}{rgb}{0.000000,0.000000,0.000000}%
\pgfsetstrokecolor{currentstroke}%
\pgfsetdash{}{0pt}%
\pgfsys@defobject{currentmarker}{\pgfqpoint{-0.048611in}{0.000000in}}{\pgfqpoint{0.000000in}{0.000000in}}{%
\pgfpathmoveto{\pgfqpoint{0.000000in}{0.000000in}}%
\pgfpathlineto{\pgfqpoint{-0.048611in}{0.000000in}}%
\pgfusepath{stroke,fill}%
}%
\begin{pgfscope}%
\pgfsys@transformshift{0.636497in}{3.798132in}%
\pgfsys@useobject{currentmarker}{}%
\end{pgfscope}%
\end{pgfscope}%
\begin{pgfscope}%
\definecolor{textcolor}{rgb}{0.000000,0.000000,0.000000}%
\pgfsetstrokecolor{textcolor}%
\pgfsetfillcolor{textcolor}%
\pgftext[x=0.268904in, y=3.745326in, left, base]{\color{textcolor}\rmfamily\fontsize{11.000000}{13.200000}\selectfont \(\displaystyle {0.12}\)}%
\end{pgfscope}%
\begin{pgfscope}%
\definecolor{textcolor}{rgb}{0.000000,0.000000,0.000000}%
\pgfsetstrokecolor{textcolor}%
\pgfsetfillcolor{textcolor}%
\pgftext[x=0.213349in,y=2.393555in,,bottom,rotate=90.000000]{\color{textcolor}\rmfamily\fontsize{16.000000}{19.200000}\selectfont Hazard}%
\end{pgfscope}%
\begin{pgfscope}%
\pgfpathrectangle{\pgfqpoint{0.636497in}{0.883555in}}{\pgfqpoint{4.650000in}{3.020000in}}%
\pgfusepath{clip}%
\pgfsetbuttcap%
\pgfsetroundjoin%
\pgfsetlinewidth{1.505625pt}%
\definecolor{currentstroke}{rgb}{1.000000,0.000000,0.000000}%
\pgfsetstrokecolor{currentstroke}%
\pgfsetdash{{5.550000pt}{2.400000pt}}{0.000000pt}%
\pgfpathmoveto{\pgfqpoint{0.847861in}{3.806206in}}%
\pgfpathlineto{\pgfqpoint{1.693315in}{3.376472in}}%
\pgfpathlineto{\pgfqpoint{2.538770in}{3.006015in}}%
\pgfpathlineto{\pgfqpoint{3.384224in}{2.671297in}}%
\pgfpathlineto{\pgfqpoint{4.229679in}{2.454069in}}%
\pgfpathlineto{\pgfqpoint{5.075133in}{2.407961in}}%
\pgfusepath{stroke}%
\end{pgfscope}%
\begin{pgfscope}%
\pgfpathrectangle{\pgfqpoint{0.636497in}{0.883555in}}{\pgfqpoint{4.650000in}{3.020000in}}%
\pgfusepath{clip}%
\pgfsetrectcap%
\pgfsetroundjoin%
\pgfsetlinewidth{1.505625pt}%
\definecolor{currentstroke}{rgb}{0.000000,0.000000,1.000000}%
\pgfsetstrokecolor{currentstroke}%
\pgfsetdash{}{0pt}%
\pgfpathmoveto{\pgfqpoint{0.847861in}{2.481447in}}%
\pgfpathlineto{\pgfqpoint{1.693315in}{3.107290in}}%
\pgfpathlineto{\pgfqpoint{2.538770in}{2.738328in}}%
\pgfpathlineto{\pgfqpoint{3.384224in}{2.852613in}}%
\pgfpathlineto{\pgfqpoint{4.229679in}{2.703349in}}%
\pgfpathlineto{\pgfqpoint{5.075133in}{2.414998in}}%
\pgfusepath{stroke}%
\end{pgfscope}%
\begin{pgfscope}%
\pgfsetrectcap%
\pgfsetmiterjoin%
\pgfsetlinewidth{0.803000pt}%
\definecolor{currentstroke}{rgb}{0.000000,0.000000,0.000000}%
\pgfsetstrokecolor{currentstroke}%
\pgfsetdash{}{0pt}%
\pgfpathmoveto{\pgfqpoint{0.636497in}{0.883555in}}%
\pgfpathlineto{\pgfqpoint{0.636497in}{3.903555in}}%
\pgfusepath{stroke}%
\end{pgfscope}%
\begin{pgfscope}%
\pgfsetrectcap%
\pgfsetmiterjoin%
\pgfsetlinewidth{0.803000pt}%
\definecolor{currentstroke}{rgb}{0.000000,0.000000,0.000000}%
\pgfsetstrokecolor{currentstroke}%
\pgfsetdash{}{0pt}%
\pgfpathmoveto{\pgfqpoint{5.286497in}{0.883555in}}%
\pgfpathlineto{\pgfqpoint{5.286497in}{3.903555in}}%
\pgfusepath{stroke}%
\end{pgfscope}%
\begin{pgfscope}%
\pgfsetrectcap%
\pgfsetmiterjoin%
\pgfsetlinewidth{0.803000pt}%
\definecolor{currentstroke}{rgb}{0.000000,0.000000,0.000000}%
\pgfsetstrokecolor{currentstroke}%
\pgfsetdash{}{0pt}%
\pgfpathmoveto{\pgfqpoint{0.636497in}{0.883555in}}%
\pgfpathlineto{\pgfqpoint{5.286497in}{0.883555in}}%
\pgfusepath{stroke}%
\end{pgfscope}%
\begin{pgfscope}%
\pgfsetrectcap%
\pgfsetmiterjoin%
\pgfsetlinewidth{0.803000pt}%
\definecolor{currentstroke}{rgb}{0.000000,0.000000,0.000000}%
\pgfsetstrokecolor{currentstroke}%
\pgfsetdash{}{0pt}%
\pgfpathmoveto{\pgfqpoint{0.636497in}{3.903555in}}%
\pgfpathlineto{\pgfqpoint{5.286497in}{3.903555in}}%
\pgfusepath{stroke}%
\end{pgfscope}%
\begin{pgfscope}%
\pgfsetbuttcap%
\pgfsetmiterjoin%
\definecolor{currentfill}{rgb}{0.300000,0.300000,0.300000}%
\pgfsetfillcolor{currentfill}%
\pgfsetfillopacity{0.500000}%
\pgfsetlinewidth{1.003750pt}%
\definecolor{currentstroke}{rgb}{0.300000,0.300000,0.300000}%
\pgfsetstrokecolor{currentstroke}%
\pgfsetstrokeopacity{0.500000}%
\pgfsetdash{}{0pt}%
\pgfpathmoveto{\pgfqpoint{1.541066in}{-0.027778in}}%
\pgfpathlineto{\pgfqpoint{4.437484in}{-0.027778in}}%
\pgfpathquadraticcurveto{\pgfqpoint{4.476373in}{-0.027778in}}{\pgfqpoint{4.476373in}{0.011111in}}%
\pgfpathlineto{\pgfqpoint{4.476373in}{0.266666in}}%
\pgfpathquadraticcurveto{\pgfqpoint{4.476373in}{0.305555in}}{\pgfqpoint{4.437484in}{0.305555in}}%
\pgfpathlineto{\pgfqpoint{1.541066in}{0.305555in}}%
\pgfpathquadraticcurveto{\pgfqpoint{1.502177in}{0.305555in}}{\pgfqpoint{1.502177in}{0.266666in}}%
\pgfpathlineto{\pgfqpoint{1.502177in}{0.011111in}}%
\pgfpathquadraticcurveto{\pgfqpoint{1.502177in}{-0.027778in}}{\pgfqpoint{1.541066in}{-0.027778in}}%
\pgfpathclose%
\pgfusepath{stroke,fill}%
\end{pgfscope}%
\begin{pgfscope}%
\pgfsetbuttcap%
\pgfsetmiterjoin%
\definecolor{currentfill}{rgb}{1.000000,1.000000,1.000000}%
\pgfsetfillcolor{currentfill}%
\pgfsetlinewidth{1.003750pt}%
\definecolor{currentstroke}{rgb}{0.800000,0.800000,0.800000}%
\pgfsetstrokecolor{currentstroke}%
\pgfsetdash{}{0pt}%
\pgfpathmoveto{\pgfqpoint{1.513288in}{0.000000in}}%
\pgfpathlineto{\pgfqpoint{4.409706in}{0.000000in}}%
\pgfpathquadraticcurveto{\pgfqpoint{4.448595in}{0.000000in}}{\pgfqpoint{4.448595in}{0.038889in}}%
\pgfpathlineto{\pgfqpoint{4.448595in}{0.294444in}}%
\pgfpathquadraticcurveto{\pgfqpoint{4.448595in}{0.333333in}}{\pgfqpoint{4.409706in}{0.333333in}}%
\pgfpathlineto{\pgfqpoint{1.513288in}{0.333333in}}%
\pgfpathquadraticcurveto{\pgfqpoint{1.474399in}{0.333333in}}{\pgfqpoint{1.474399in}{0.294444in}}%
\pgfpathlineto{\pgfqpoint{1.474399in}{0.038889in}}%
\pgfpathquadraticcurveto{\pgfqpoint{1.474399in}{0.000000in}}{\pgfqpoint{1.513288in}{0.000000in}}%
\pgfpathclose%
\pgfusepath{stroke,fill}%
\end{pgfscope}%
\begin{pgfscope}%
\pgfsetbuttcap%
\pgfsetroundjoin%
\pgfsetlinewidth{1.505625pt}%
\definecolor{currentstroke}{rgb}{1.000000,0.000000,0.000000}%
\pgfsetstrokecolor{currentstroke}%
\pgfsetdash{{5.550000pt}{2.400000pt}}{0.000000pt}%
\pgfpathmoveto{\pgfqpoint{1.552177in}{0.184722in}}%
\pgfpathlineto{\pgfqpoint{1.941066in}{0.184722in}}%
\pgfusepath{stroke}%
\end{pgfscope}%
\begin{pgfscope}%
\definecolor{textcolor}{rgb}{0.000000,0.000000,0.000000}%
\pgfsetstrokecolor{textcolor}%
\pgfsetfillcolor{textcolor}%
\pgftext[x=2.096621in,y=0.116667in,left,base]{\color{textcolor}\rmfamily\fontsize{14.000000}{16.800000}\selectfont Simulation}%
\end{pgfscope}%
\begin{pgfscope}%
\pgfsetrectcap%
\pgfsetroundjoin%
\pgfsetlinewidth{1.505625pt}%
\definecolor{currentstroke}{rgb}{0.000000,0.000000,1.000000}%
\pgfsetstrokecolor{currentstroke}%
\pgfsetdash{}{0pt}%
\pgfpathmoveto{\pgfqpoint{3.404827in}{0.184722in}}%
\pgfpathlineto{\pgfqpoint{3.793716in}{0.184722in}}%
\pgfusepath{stroke}%
\end{pgfscope}%
\begin{pgfscope}%
\definecolor{textcolor}{rgb}{0.000000,0.000000,0.000000}%
\pgfsetstrokecolor{textcolor}%
\pgfsetfillcolor{textcolor}%
\pgftext[x=3.949272in,y=0.116667in,left,base]{\color{textcolor}\rmfamily\fontsize{14.000000}{16.800000}\selectfont Data}%
\end{pgfscope}%
\end{pgfpicture}%
\makeatother%
\endgroup%
} 
\end{subfigure}
\begin{subfigure}{.49\textwidth}
\centering
\caption{Hazard of Breakup}
\label{fig:hazs}
\scalebox{0.49}{%% Creator: Matplotlib, PGF backend
%%
%% To include the figure in your LaTeX document, write
%%   \input{<filename>.pgf}
%%
%% Make sure the required packages are loaded in your preamble
%%   \usepackage{pgf}
%%
%% Figures using additional raster images can only be included by \input if
%% they are in the same directory as the main LaTeX file. For loading figures
%% from other directories you can use the `import` package
%%   \usepackage{import}
%% and then include the figures with
%%   \import{<path to file>}{<filename>.pgf}
%%
%% Matplotlib used the following preamble
%%
\begingroup%
\makeatletter%
\begin{pgfpicture}%
\pgfpathrectangle{\pgfpointorigin}{\pgfqpoint{5.210455in}{3.903555in}}%
\pgfusepath{use as bounding box, clip}%
\begin{pgfscope}%
\pgfsetbuttcap%
\pgfsetmiterjoin%
\definecolor{currentfill}{rgb}{1.000000,1.000000,1.000000}%
\pgfsetfillcolor{currentfill}%
\pgfsetlinewidth{0.000000pt}%
\definecolor{currentstroke}{rgb}{1.000000,1.000000,1.000000}%
\pgfsetstrokecolor{currentstroke}%
\pgfsetdash{}{0pt}%
\pgfpathmoveto{\pgfqpoint{0.000000in}{0.000000in}}%
\pgfpathlineto{\pgfqpoint{5.210455in}{0.000000in}}%
\pgfpathlineto{\pgfqpoint{5.210455in}{3.903555in}}%
\pgfpathlineto{\pgfqpoint{0.000000in}{3.903555in}}%
\pgfpathclose%
\pgfusepath{fill}%
\end{pgfscope}%
\begin{pgfscope}%
\pgfsetbuttcap%
\pgfsetmiterjoin%
\definecolor{currentfill}{rgb}{1.000000,1.000000,1.000000}%
\pgfsetfillcolor{currentfill}%
\pgfsetlinewidth{0.000000pt}%
\definecolor{currentstroke}{rgb}{0.000000,0.000000,0.000000}%
\pgfsetstrokecolor{currentstroke}%
\pgfsetstrokeopacity{0.000000}%
\pgfsetdash{}{0pt}%
\pgfpathmoveto{\pgfqpoint{0.560455in}{0.883555in}}%
\pgfpathlineto{\pgfqpoint{5.210455in}{0.883555in}}%
\pgfpathlineto{\pgfqpoint{5.210455in}{3.903555in}}%
\pgfpathlineto{\pgfqpoint{0.560455in}{3.903555in}}%
\pgfpathclose%
\pgfusepath{fill}%
\end{pgfscope}%
\begin{pgfscope}%
\pgfpathrectangle{\pgfqpoint{0.560455in}{0.883555in}}{\pgfqpoint{4.650000in}{3.020000in}}%
\pgfusepath{clip}%
\pgfsetbuttcap%
\pgfsetroundjoin%
\definecolor{currentfill}{rgb}{0.000000,0.000000,1.000000}%
\pgfsetfillcolor{currentfill}%
\pgfsetfillopacity{0.200000}%
\pgfsetlinewidth{0.000000pt}%
\definecolor{currentstroke}{rgb}{0.000000,0.000000,0.000000}%
\pgfsetstrokecolor{currentstroke}%
\pgfsetdash{}{0pt}%
\pgfpathmoveto{\pgfqpoint{0.771819in}{2.401856in}}%
\pgfpathlineto{\pgfqpoint{0.771819in}{1.537270in}}%
\pgfpathlineto{\pgfqpoint{2.885455in}{1.613835in}}%
\pgfpathlineto{\pgfqpoint{4.999092in}{1.221529in}}%
\pgfpathlineto{\pgfqpoint{4.999092in}{3.775840in}}%
\pgfpathlineto{\pgfqpoint{4.999092in}{3.775840in}}%
\pgfpathlineto{\pgfqpoint{2.885455in}{2.723079in}}%
\pgfpathlineto{\pgfqpoint{0.771819in}{2.401856in}}%
\pgfpathclose%
\pgfusepath{fill}%
\end{pgfscope}%
\begin{pgfscope}%
\pgfsetbuttcap%
\pgfsetroundjoin%
\definecolor{currentfill}{rgb}{0.000000,0.000000,0.000000}%
\pgfsetfillcolor{currentfill}%
\pgfsetlinewidth{0.803000pt}%
\definecolor{currentstroke}{rgb}{0.000000,0.000000,0.000000}%
\pgfsetstrokecolor{currentstroke}%
\pgfsetdash{}{0pt}%
\pgfsys@defobject{currentmarker}{\pgfqpoint{0.000000in}{-0.048611in}}{\pgfqpoint{0.000000in}{0.000000in}}{%
\pgfpathmoveto{\pgfqpoint{0.000000in}{0.000000in}}%
\pgfpathlineto{\pgfqpoint{0.000000in}{-0.048611in}}%
\pgfusepath{stroke,fill}%
}%
\begin{pgfscope}%
\pgfsys@transformshift{0.771819in}{0.883555in}%
\pgfsys@useobject{currentmarker}{}%
\end{pgfscope}%
\end{pgfscope}%
\begin{pgfscope}%
\definecolor{textcolor}{rgb}{0.000000,0.000000,0.000000}%
\pgfsetstrokecolor{textcolor}%
\pgfsetfillcolor{textcolor}%
\pgftext[x=0.771819in,y=0.786333in,,top]{\color{textcolor}\rmfamily\fontsize{11.000000}{13.200000}\selectfont 1-2}%
\end{pgfscope}%
\begin{pgfscope}%
\pgfsetbuttcap%
\pgfsetroundjoin%
\definecolor{currentfill}{rgb}{0.000000,0.000000,0.000000}%
\pgfsetfillcolor{currentfill}%
\pgfsetlinewidth{0.803000pt}%
\definecolor{currentstroke}{rgb}{0.000000,0.000000,0.000000}%
\pgfsetstrokecolor{currentstroke}%
\pgfsetdash{}{0pt}%
\pgfsys@defobject{currentmarker}{\pgfqpoint{0.000000in}{-0.048611in}}{\pgfqpoint{0.000000in}{0.000000in}}{%
\pgfpathmoveto{\pgfqpoint{0.000000in}{0.000000in}}%
\pgfpathlineto{\pgfqpoint{0.000000in}{-0.048611in}}%
\pgfusepath{stroke,fill}%
}%
\begin{pgfscope}%
\pgfsys@transformshift{2.885455in}{0.883555in}%
\pgfsys@useobject{currentmarker}{}%
\end{pgfscope}%
\end{pgfscope}%
\begin{pgfscope}%
\definecolor{textcolor}{rgb}{0.000000,0.000000,0.000000}%
\pgfsetstrokecolor{textcolor}%
\pgfsetfillcolor{textcolor}%
\pgftext[x=2.885455in,y=0.786333in,,top]{\color{textcolor}\rmfamily\fontsize{11.000000}{13.200000}\selectfont 3-4}%
\end{pgfscope}%
\begin{pgfscope}%
\pgfsetbuttcap%
\pgfsetroundjoin%
\definecolor{currentfill}{rgb}{0.000000,0.000000,0.000000}%
\pgfsetfillcolor{currentfill}%
\pgfsetlinewidth{0.803000pt}%
\definecolor{currentstroke}{rgb}{0.000000,0.000000,0.000000}%
\pgfsetstrokecolor{currentstroke}%
\pgfsetdash{}{0pt}%
\pgfsys@defobject{currentmarker}{\pgfqpoint{0.000000in}{-0.048611in}}{\pgfqpoint{0.000000in}{0.000000in}}{%
\pgfpathmoveto{\pgfqpoint{0.000000in}{0.000000in}}%
\pgfpathlineto{\pgfqpoint{0.000000in}{-0.048611in}}%
\pgfusepath{stroke,fill}%
}%
\begin{pgfscope}%
\pgfsys@transformshift{4.999092in}{0.883555in}%
\pgfsys@useobject{currentmarker}{}%
\end{pgfscope}%
\end{pgfscope}%
\begin{pgfscope}%
\definecolor{textcolor}{rgb}{0.000000,0.000000,0.000000}%
\pgfsetstrokecolor{textcolor}%
\pgfsetfillcolor{textcolor}%
\pgftext[x=4.999092in,y=0.786333in,,top]{\color{textcolor}\rmfamily\fontsize{11.000000}{13.200000}\selectfont 5-6}%
\end{pgfscope}%
\begin{pgfscope}%
\definecolor{textcolor}{rgb}{0.000000,0.000000,0.000000}%
\pgfsetstrokecolor{textcolor}%
\pgfsetfillcolor{textcolor}%
\pgftext[x=2.885455in,y=0.595592in,,top]{\color{textcolor}\rmfamily\fontsize{16.000000}{19.200000}\selectfont Duration - years}%
\end{pgfscope}%
\begin{pgfscope}%
\pgfsetbuttcap%
\pgfsetroundjoin%
\definecolor{currentfill}{rgb}{0.000000,0.000000,0.000000}%
\pgfsetfillcolor{currentfill}%
\pgfsetlinewidth{0.803000pt}%
\definecolor{currentstroke}{rgb}{0.000000,0.000000,0.000000}%
\pgfsetstrokecolor{currentstroke}%
\pgfsetdash{}{0pt}%
\pgfsys@defobject{currentmarker}{\pgfqpoint{-0.048611in}{0.000000in}}{\pgfqpoint{0.000000in}{0.000000in}}{%
\pgfpathmoveto{\pgfqpoint{0.000000in}{0.000000in}}%
\pgfpathlineto{\pgfqpoint{-0.048611in}{0.000000in}}%
\pgfusepath{stroke,fill}%
}%
\begin{pgfscope}%
\pgfsys@transformshift{0.560455in}{0.883555in}%
\pgfsys@useobject{currentmarker}{}%
\end{pgfscope}%
\end{pgfscope}%
\begin{pgfscope}%
\definecolor{textcolor}{rgb}{0.000000,0.000000,0.000000}%
\pgfsetstrokecolor{textcolor}%
\pgfsetfillcolor{textcolor}%
\pgftext[x=0.268904in,y=0.830748in,left,base]{\color{textcolor}\rmfamily\fontsize{11.000000}{13.200000}\selectfont \(\displaystyle 0.0\)}%
\end{pgfscope}%
\begin{pgfscope}%
\pgfsetbuttcap%
\pgfsetroundjoin%
\definecolor{currentfill}{rgb}{0.000000,0.000000,0.000000}%
\pgfsetfillcolor{currentfill}%
\pgfsetlinewidth{0.803000pt}%
\definecolor{currentstroke}{rgb}{0.000000,0.000000,0.000000}%
\pgfsetstrokecolor{currentstroke}%
\pgfsetdash{}{0pt}%
\pgfsys@defobject{currentmarker}{\pgfqpoint{-0.048611in}{0.000000in}}{\pgfqpoint{0.000000in}{0.000000in}}{%
\pgfpathmoveto{\pgfqpoint{0.000000in}{0.000000in}}%
\pgfpathlineto{\pgfqpoint{-0.048611in}{0.000000in}}%
\pgfusepath{stroke,fill}%
}%
\begin{pgfscope}%
\pgfsys@transformshift{0.560455in}{1.503499in}%
\pgfsys@useobject{currentmarker}{}%
\end{pgfscope}%
\end{pgfscope}%
\begin{pgfscope}%
\definecolor{textcolor}{rgb}{0.000000,0.000000,0.000000}%
\pgfsetstrokecolor{textcolor}%
\pgfsetfillcolor{textcolor}%
\pgftext[x=0.268904in,y=1.450693in,left,base]{\color{textcolor}\rmfamily\fontsize{11.000000}{13.200000}\selectfont \(\displaystyle 0.1\)}%
\end{pgfscope}%
\begin{pgfscope}%
\pgfsetbuttcap%
\pgfsetroundjoin%
\definecolor{currentfill}{rgb}{0.000000,0.000000,0.000000}%
\pgfsetfillcolor{currentfill}%
\pgfsetlinewidth{0.803000pt}%
\definecolor{currentstroke}{rgb}{0.000000,0.000000,0.000000}%
\pgfsetstrokecolor{currentstroke}%
\pgfsetdash{}{0pt}%
\pgfsys@defobject{currentmarker}{\pgfqpoint{-0.048611in}{0.000000in}}{\pgfqpoint{0.000000in}{0.000000in}}{%
\pgfpathmoveto{\pgfqpoint{0.000000in}{0.000000in}}%
\pgfpathlineto{\pgfqpoint{-0.048611in}{0.000000in}}%
\pgfusepath{stroke,fill}%
}%
\begin{pgfscope}%
\pgfsys@transformshift{0.560455in}{2.123443in}%
\pgfsys@useobject{currentmarker}{}%
\end{pgfscope}%
\end{pgfscope}%
\begin{pgfscope}%
\definecolor{textcolor}{rgb}{0.000000,0.000000,0.000000}%
\pgfsetstrokecolor{textcolor}%
\pgfsetfillcolor{textcolor}%
\pgftext[x=0.268904in,y=2.070637in,left,base]{\color{textcolor}\rmfamily\fontsize{11.000000}{13.200000}\selectfont \(\displaystyle 0.2\)}%
\end{pgfscope}%
\begin{pgfscope}%
\pgfsetbuttcap%
\pgfsetroundjoin%
\definecolor{currentfill}{rgb}{0.000000,0.000000,0.000000}%
\pgfsetfillcolor{currentfill}%
\pgfsetlinewidth{0.803000pt}%
\definecolor{currentstroke}{rgb}{0.000000,0.000000,0.000000}%
\pgfsetstrokecolor{currentstroke}%
\pgfsetdash{}{0pt}%
\pgfsys@defobject{currentmarker}{\pgfqpoint{-0.048611in}{0.000000in}}{\pgfqpoint{0.000000in}{0.000000in}}{%
\pgfpathmoveto{\pgfqpoint{0.000000in}{0.000000in}}%
\pgfpathlineto{\pgfqpoint{-0.048611in}{0.000000in}}%
\pgfusepath{stroke,fill}%
}%
\begin{pgfscope}%
\pgfsys@transformshift{0.560455in}{2.743387in}%
\pgfsys@useobject{currentmarker}{}%
\end{pgfscope}%
\end{pgfscope}%
\begin{pgfscope}%
\definecolor{textcolor}{rgb}{0.000000,0.000000,0.000000}%
\pgfsetstrokecolor{textcolor}%
\pgfsetfillcolor{textcolor}%
\pgftext[x=0.268904in,y=2.690581in,left,base]{\color{textcolor}\rmfamily\fontsize{11.000000}{13.200000}\selectfont \(\displaystyle 0.3\)}%
\end{pgfscope}%
\begin{pgfscope}%
\pgfsetbuttcap%
\pgfsetroundjoin%
\definecolor{currentfill}{rgb}{0.000000,0.000000,0.000000}%
\pgfsetfillcolor{currentfill}%
\pgfsetlinewidth{0.803000pt}%
\definecolor{currentstroke}{rgb}{0.000000,0.000000,0.000000}%
\pgfsetstrokecolor{currentstroke}%
\pgfsetdash{}{0pt}%
\pgfsys@defobject{currentmarker}{\pgfqpoint{-0.048611in}{0.000000in}}{\pgfqpoint{0.000000in}{0.000000in}}{%
\pgfpathmoveto{\pgfqpoint{0.000000in}{0.000000in}}%
\pgfpathlineto{\pgfqpoint{-0.048611in}{0.000000in}}%
\pgfusepath{stroke,fill}%
}%
\begin{pgfscope}%
\pgfsys@transformshift{0.560455in}{3.363332in}%
\pgfsys@useobject{currentmarker}{}%
\end{pgfscope}%
\end{pgfscope}%
\begin{pgfscope}%
\definecolor{textcolor}{rgb}{0.000000,0.000000,0.000000}%
\pgfsetstrokecolor{textcolor}%
\pgfsetfillcolor{textcolor}%
\pgftext[x=0.268904in,y=3.310525in,left,base]{\color{textcolor}\rmfamily\fontsize{11.000000}{13.200000}\selectfont \(\displaystyle 0.4\)}%
\end{pgfscope}%
\begin{pgfscope}%
\definecolor{textcolor}{rgb}{0.000000,0.000000,0.000000}%
\pgfsetstrokecolor{textcolor}%
\pgfsetfillcolor{textcolor}%
\pgftext[x=0.213349in,y=2.393555in,,bottom,rotate=90.000000]{\color{textcolor}\rmfamily\fontsize{16.000000}{19.200000}\selectfont Hazard}%
\end{pgfscope}%
\begin{pgfscope}%
\pgfpathrectangle{\pgfqpoint{0.560455in}{0.883555in}}{\pgfqpoint{4.650000in}{3.020000in}}%
\pgfusepath{clip}%
\pgfsetbuttcap%
\pgfsetroundjoin%
\pgfsetlinewidth{1.505625pt}%
\definecolor{currentstroke}{rgb}{1.000000,0.000000,0.000000}%
\pgfsetstrokecolor{currentstroke}%
\pgfsetdash{{5.550000pt}{2.400000pt}}{0.000000pt}%
\pgfpathmoveto{\pgfqpoint{0.771819in}{3.690852in}}%
\pgfpathlineto{\pgfqpoint{2.885455in}{2.116444in}}%
\pgfpathlineto{\pgfqpoint{4.999092in}{2.006558in}}%
\pgfusepath{stroke}%
\end{pgfscope}%
\begin{pgfscope}%
\pgfpathrectangle{\pgfqpoint{0.560455in}{0.883555in}}{\pgfqpoint{4.650000in}{3.020000in}}%
\pgfusepath{clip}%
\pgfsetrectcap%
\pgfsetroundjoin%
\pgfsetlinewidth{1.505625pt}%
\definecolor{currentstroke}{rgb}{0.000000,0.000000,1.000000}%
\pgfsetstrokecolor{currentstroke}%
\pgfsetdash{}{0pt}%
\pgfpathmoveto{\pgfqpoint{0.771819in}{2.045338in}}%
\pgfpathlineto{\pgfqpoint{2.885455in}{2.190689in}}%
\pgfpathlineto{\pgfqpoint{4.999092in}{2.233908in}}%
\pgfusepath{stroke}%
\end{pgfscope}%
\begin{pgfscope}%
\pgfsetrectcap%
\pgfsetmiterjoin%
\pgfsetlinewidth{0.803000pt}%
\definecolor{currentstroke}{rgb}{0.000000,0.000000,0.000000}%
\pgfsetstrokecolor{currentstroke}%
\pgfsetdash{}{0pt}%
\pgfpathmoveto{\pgfqpoint{0.560455in}{0.883555in}}%
\pgfpathlineto{\pgfqpoint{0.560455in}{3.903555in}}%
\pgfusepath{stroke}%
\end{pgfscope}%
\begin{pgfscope}%
\pgfsetrectcap%
\pgfsetmiterjoin%
\pgfsetlinewidth{0.803000pt}%
\definecolor{currentstroke}{rgb}{0.000000,0.000000,0.000000}%
\pgfsetstrokecolor{currentstroke}%
\pgfsetdash{}{0pt}%
\pgfpathmoveto{\pgfqpoint{5.210455in}{0.883555in}}%
\pgfpathlineto{\pgfqpoint{5.210455in}{3.903555in}}%
\pgfusepath{stroke}%
\end{pgfscope}%
\begin{pgfscope}%
\pgfsetrectcap%
\pgfsetmiterjoin%
\pgfsetlinewidth{0.803000pt}%
\definecolor{currentstroke}{rgb}{0.000000,0.000000,0.000000}%
\pgfsetstrokecolor{currentstroke}%
\pgfsetdash{}{0pt}%
\pgfpathmoveto{\pgfqpoint{0.560455in}{0.883555in}}%
\pgfpathlineto{\pgfqpoint{5.210455in}{0.883555in}}%
\pgfusepath{stroke}%
\end{pgfscope}%
\begin{pgfscope}%
\pgfsetrectcap%
\pgfsetmiterjoin%
\pgfsetlinewidth{0.803000pt}%
\definecolor{currentstroke}{rgb}{0.000000,0.000000,0.000000}%
\pgfsetstrokecolor{currentstroke}%
\pgfsetdash{}{0pt}%
\pgfpathmoveto{\pgfqpoint{0.560455in}{3.903555in}}%
\pgfpathlineto{\pgfqpoint{5.210455in}{3.903555in}}%
\pgfusepath{stroke}%
\end{pgfscope}%
\begin{pgfscope}%
\pgfsetbuttcap%
\pgfsetmiterjoin%
\definecolor{currentfill}{rgb}{0.300000,0.300000,0.300000}%
\pgfsetfillcolor{currentfill}%
\pgfsetfillopacity{0.500000}%
\pgfsetlinewidth{1.003750pt}%
\definecolor{currentstroke}{rgb}{0.300000,0.300000,0.300000}%
\pgfsetstrokecolor{currentstroke}%
\pgfsetstrokeopacity{0.500000}%
\pgfsetdash{}{0pt}%
\pgfpathmoveto{\pgfqpoint{1.465024in}{-0.027778in}}%
\pgfpathlineto{\pgfqpoint{4.361442in}{-0.027778in}}%
\pgfpathquadraticcurveto{\pgfqpoint{4.400331in}{-0.027778in}}{\pgfqpoint{4.400331in}{0.011111in}}%
\pgfpathlineto{\pgfqpoint{4.400331in}{0.266666in}}%
\pgfpathquadraticcurveto{\pgfqpoint{4.400331in}{0.305555in}}{\pgfqpoint{4.361442in}{0.305555in}}%
\pgfpathlineto{\pgfqpoint{1.465024in}{0.305555in}}%
\pgfpathquadraticcurveto{\pgfqpoint{1.426135in}{0.305555in}}{\pgfqpoint{1.426135in}{0.266666in}}%
\pgfpathlineto{\pgfqpoint{1.426135in}{0.011111in}}%
\pgfpathquadraticcurveto{\pgfqpoint{1.426135in}{-0.027778in}}{\pgfqpoint{1.465024in}{-0.027778in}}%
\pgfpathclose%
\pgfusepath{stroke,fill}%
\end{pgfscope}%
\begin{pgfscope}%
\pgfsetbuttcap%
\pgfsetmiterjoin%
\definecolor{currentfill}{rgb}{1.000000,1.000000,1.000000}%
\pgfsetfillcolor{currentfill}%
\pgfsetlinewidth{1.003750pt}%
\definecolor{currentstroke}{rgb}{0.800000,0.800000,0.800000}%
\pgfsetstrokecolor{currentstroke}%
\pgfsetdash{}{0pt}%
\pgfpathmoveto{\pgfqpoint{1.437246in}{0.000000in}}%
\pgfpathlineto{\pgfqpoint{4.333665in}{0.000000in}}%
\pgfpathquadraticcurveto{\pgfqpoint{4.372553in}{0.000000in}}{\pgfqpoint{4.372553in}{0.038889in}}%
\pgfpathlineto{\pgfqpoint{4.372553in}{0.294444in}}%
\pgfpathquadraticcurveto{\pgfqpoint{4.372553in}{0.333333in}}{\pgfqpoint{4.333665in}{0.333333in}}%
\pgfpathlineto{\pgfqpoint{1.437246in}{0.333333in}}%
\pgfpathquadraticcurveto{\pgfqpoint{1.398357in}{0.333333in}}{\pgfqpoint{1.398357in}{0.294444in}}%
\pgfpathlineto{\pgfqpoint{1.398357in}{0.038889in}}%
\pgfpathquadraticcurveto{\pgfqpoint{1.398357in}{0.000000in}}{\pgfqpoint{1.437246in}{0.000000in}}%
\pgfpathclose%
\pgfusepath{stroke,fill}%
\end{pgfscope}%
\begin{pgfscope}%
\pgfsetbuttcap%
\pgfsetroundjoin%
\pgfsetlinewidth{1.505625pt}%
\definecolor{currentstroke}{rgb}{1.000000,0.000000,0.000000}%
\pgfsetstrokecolor{currentstroke}%
\pgfsetdash{{5.550000pt}{2.400000pt}}{0.000000pt}%
\pgfpathmoveto{\pgfqpoint{1.476135in}{0.184722in}}%
\pgfpathlineto{\pgfqpoint{1.865024in}{0.184722in}}%
\pgfusepath{stroke}%
\end{pgfscope}%
\begin{pgfscope}%
\definecolor{textcolor}{rgb}{0.000000,0.000000,0.000000}%
\pgfsetstrokecolor{textcolor}%
\pgfsetfillcolor{textcolor}%
\pgftext[x=2.020579in,y=0.116667in,left,base]{\color{textcolor}\rmfamily\fontsize{14.000000}{16.800000}\selectfont Simulation}%
\end{pgfscope}%
\begin{pgfscope}%
\pgfsetrectcap%
\pgfsetroundjoin%
\pgfsetlinewidth{1.505625pt}%
\definecolor{currentstroke}{rgb}{0.000000,0.000000,1.000000}%
\pgfsetstrokecolor{currentstroke}%
\pgfsetdash{}{0pt}%
\pgfpathmoveto{\pgfqpoint{3.328785in}{0.184722in}}%
\pgfpathlineto{\pgfqpoint{3.717674in}{0.184722in}}%
\pgfusepath{stroke}%
\end{pgfscope}%
\begin{pgfscope}%
\definecolor{textcolor}{rgb}{0.000000,0.000000,0.000000}%
\pgfsetstrokecolor{textcolor}%
\pgfsetfillcolor{textcolor}%
\pgftext[x=3.873230in,y=0.116667in,left,base]{\color{textcolor}\rmfamily\fontsize{14.000000}{16.800000}\selectfont Data}%
\end{pgfscope}%
\end{pgfpicture}%
\makeatother%
\endgroup%
} 
\end{subfigure}
\end{center}

\hspace{20em}

\begin{center}
\begin{subfigure}{.49\textwidth}
\centering
\caption{Hazard of Marriage}
\label{fig:hazm}
\scalebox{0.49}{%% Creator: Matplotlib, PGF backend
%%
%% To include the figure in your LaTeX document, write
%%   \input{<filename>.pgf}
%%
%% Make sure the required packages are loaded in your preamble
%%   \usepackage{pgf}
%%
%% and, on pdftex
%%   \usepackage[utf8]{inputenc}\DeclareUnicodeCharacter{2212}{-}
%%
%% or, on luatex and xetex
%%   \usepackage{unicode-math}
%%
%% Figures using additional raster images can only be included by \input if
%% they are in the same directory as the main LaTeX file. For loading figures
%% from other directories you can use the `import` package
%%   \usepackage{import}
%%
%% and then include the figures with
%%   \import{<path to file>}{<filename>.pgf}
%%
%% Matplotlib used the following preamble
%%
\begingroup%
\makeatletter%
\begin{pgfpicture}%
\pgfpathrectangle{\pgfpointorigin}{\pgfqpoint{5.210455in}{3.903555in}}%
\pgfusepath{use as bounding box, clip}%
\begin{pgfscope}%
\pgfsetbuttcap%
\pgfsetmiterjoin%
\definecolor{currentfill}{rgb}{1.000000,1.000000,1.000000}%
\pgfsetfillcolor{currentfill}%
\pgfsetlinewidth{0.000000pt}%
\definecolor{currentstroke}{rgb}{1.000000,1.000000,1.000000}%
\pgfsetstrokecolor{currentstroke}%
\pgfsetdash{}{0pt}%
\pgfpathmoveto{\pgfqpoint{0.000000in}{0.000000in}}%
\pgfpathlineto{\pgfqpoint{5.210455in}{0.000000in}}%
\pgfpathlineto{\pgfqpoint{5.210455in}{3.903555in}}%
\pgfpathlineto{\pgfqpoint{0.000000in}{3.903555in}}%
\pgfpathclose%
\pgfusepath{fill}%
\end{pgfscope}%
\begin{pgfscope}%
\pgfsetbuttcap%
\pgfsetmiterjoin%
\definecolor{currentfill}{rgb}{1.000000,1.000000,1.000000}%
\pgfsetfillcolor{currentfill}%
\pgfsetlinewidth{0.000000pt}%
\definecolor{currentstroke}{rgb}{0.000000,0.000000,0.000000}%
\pgfsetstrokecolor{currentstroke}%
\pgfsetstrokeopacity{0.000000}%
\pgfsetdash{}{0pt}%
\pgfpathmoveto{\pgfqpoint{0.560455in}{0.883555in}}%
\pgfpathlineto{\pgfqpoint{5.210455in}{0.883555in}}%
\pgfpathlineto{\pgfqpoint{5.210455in}{3.903555in}}%
\pgfpathlineto{\pgfqpoint{0.560455in}{3.903555in}}%
\pgfpathclose%
\pgfusepath{fill}%
\end{pgfscope}%
\begin{pgfscope}%
\pgfpathrectangle{\pgfqpoint{0.560455in}{0.883555in}}{\pgfqpoint{4.650000in}{3.020000in}}%
\pgfusepath{clip}%
\pgfsetbuttcap%
\pgfsetroundjoin%
\definecolor{currentfill}{rgb}{0.000000,0.000000,1.000000}%
\pgfsetfillcolor{currentfill}%
\pgfsetfillopacity{0.200000}%
\pgfsetlinewidth{0.000000pt}%
\definecolor{currentstroke}{rgb}{0.000000,0.000000,0.000000}%
\pgfsetstrokecolor{currentstroke}%
\pgfsetdash{}{0pt}%
\pgfpathmoveto{\pgfqpoint{0.771819in}{3.759759in}}%
\pgfpathlineto{\pgfqpoint{0.771819in}{2.736190in}}%
\pgfpathlineto{\pgfqpoint{2.885455in}{1.991788in}}%
\pgfpathlineto{\pgfqpoint{4.999092in}{0.883840in}}%
\pgfpathlineto{\pgfqpoint{4.999092in}{1.666105in}}%
\pgfpathlineto{\pgfqpoint{4.999092in}{1.666105in}}%
\pgfpathlineto{\pgfqpoint{2.885455in}{3.148609in}}%
\pgfpathlineto{\pgfqpoint{0.771819in}{3.759759in}}%
\pgfpathclose%
\pgfusepath{fill}%
\end{pgfscope}%
\begin{pgfscope}%
\pgfsetbuttcap%
\pgfsetroundjoin%
\definecolor{currentfill}{rgb}{0.000000,0.000000,0.000000}%
\pgfsetfillcolor{currentfill}%
\pgfsetlinewidth{0.803000pt}%
\definecolor{currentstroke}{rgb}{0.000000,0.000000,0.000000}%
\pgfsetstrokecolor{currentstroke}%
\pgfsetdash{}{0pt}%
\pgfsys@defobject{currentmarker}{\pgfqpoint{0.000000in}{-0.048611in}}{\pgfqpoint{0.000000in}{0.000000in}}{%
\pgfpathmoveto{\pgfqpoint{0.000000in}{0.000000in}}%
\pgfpathlineto{\pgfqpoint{0.000000in}{-0.048611in}}%
\pgfusepath{stroke,fill}%
}%
\begin{pgfscope}%
\pgfsys@transformshift{0.771819in}{0.883555in}%
\pgfsys@useobject{currentmarker}{}%
\end{pgfscope}%
\end{pgfscope}%
\begin{pgfscope}%
\definecolor{textcolor}{rgb}{0.000000,0.000000,0.000000}%
\pgfsetstrokecolor{textcolor}%
\pgfsetfillcolor{textcolor}%
\pgftext[x=0.771819in,y=0.786333in,,top]{\color{textcolor}\rmfamily\fontsize{11.000000}{13.200000}\selectfont 1-2}%
\end{pgfscope}%
\begin{pgfscope}%
\pgfsetbuttcap%
\pgfsetroundjoin%
\definecolor{currentfill}{rgb}{0.000000,0.000000,0.000000}%
\pgfsetfillcolor{currentfill}%
\pgfsetlinewidth{0.803000pt}%
\definecolor{currentstroke}{rgb}{0.000000,0.000000,0.000000}%
\pgfsetstrokecolor{currentstroke}%
\pgfsetdash{}{0pt}%
\pgfsys@defobject{currentmarker}{\pgfqpoint{0.000000in}{-0.048611in}}{\pgfqpoint{0.000000in}{0.000000in}}{%
\pgfpathmoveto{\pgfqpoint{0.000000in}{0.000000in}}%
\pgfpathlineto{\pgfqpoint{0.000000in}{-0.048611in}}%
\pgfusepath{stroke,fill}%
}%
\begin{pgfscope}%
\pgfsys@transformshift{2.885455in}{0.883555in}%
\pgfsys@useobject{currentmarker}{}%
\end{pgfscope}%
\end{pgfscope}%
\begin{pgfscope}%
\definecolor{textcolor}{rgb}{0.000000,0.000000,0.000000}%
\pgfsetstrokecolor{textcolor}%
\pgfsetfillcolor{textcolor}%
\pgftext[x=2.885455in,y=0.786333in,,top]{\color{textcolor}\rmfamily\fontsize{11.000000}{13.200000}\selectfont 3-4}%
\end{pgfscope}%
\begin{pgfscope}%
\pgfsetbuttcap%
\pgfsetroundjoin%
\definecolor{currentfill}{rgb}{0.000000,0.000000,0.000000}%
\pgfsetfillcolor{currentfill}%
\pgfsetlinewidth{0.803000pt}%
\definecolor{currentstroke}{rgb}{0.000000,0.000000,0.000000}%
\pgfsetstrokecolor{currentstroke}%
\pgfsetdash{}{0pt}%
\pgfsys@defobject{currentmarker}{\pgfqpoint{0.000000in}{-0.048611in}}{\pgfqpoint{0.000000in}{0.000000in}}{%
\pgfpathmoveto{\pgfqpoint{0.000000in}{0.000000in}}%
\pgfpathlineto{\pgfqpoint{0.000000in}{-0.048611in}}%
\pgfusepath{stroke,fill}%
}%
\begin{pgfscope}%
\pgfsys@transformshift{4.999092in}{0.883555in}%
\pgfsys@useobject{currentmarker}{}%
\end{pgfscope}%
\end{pgfscope}%
\begin{pgfscope}%
\definecolor{textcolor}{rgb}{0.000000,0.000000,0.000000}%
\pgfsetstrokecolor{textcolor}%
\pgfsetfillcolor{textcolor}%
\pgftext[x=4.999092in,y=0.786333in,,top]{\color{textcolor}\rmfamily\fontsize{11.000000}{13.200000}\selectfont 5-6}%
\end{pgfscope}%
\begin{pgfscope}%
\definecolor{textcolor}{rgb}{0.000000,0.000000,0.000000}%
\pgfsetstrokecolor{textcolor}%
\pgfsetfillcolor{textcolor}%
\pgftext[x=2.885455in,y=0.595592in,,top]{\color{textcolor}\rmfamily\fontsize{16.000000}{19.200000}\selectfont Duration - years}%
\end{pgfscope}%
\begin{pgfscope}%
\pgfsetbuttcap%
\pgfsetroundjoin%
\definecolor{currentfill}{rgb}{0.000000,0.000000,0.000000}%
\pgfsetfillcolor{currentfill}%
\pgfsetlinewidth{0.803000pt}%
\definecolor{currentstroke}{rgb}{0.000000,0.000000,0.000000}%
\pgfsetstrokecolor{currentstroke}%
\pgfsetdash{}{0pt}%
\pgfsys@defobject{currentmarker}{\pgfqpoint{-0.048611in}{0.000000in}}{\pgfqpoint{0.000000in}{0.000000in}}{%
\pgfpathmoveto{\pgfqpoint{0.000000in}{0.000000in}}%
\pgfpathlineto{\pgfqpoint{-0.048611in}{0.000000in}}%
\pgfusepath{stroke,fill}%
}%
\begin{pgfscope}%
\pgfsys@transformshift{0.560455in}{0.883555in}%
\pgfsys@useobject{currentmarker}{}%
\end{pgfscope}%
\end{pgfscope}%
\begin{pgfscope}%
\definecolor{textcolor}{rgb}{0.000000,0.000000,0.000000}%
\pgfsetstrokecolor{textcolor}%
\pgfsetfillcolor{textcolor}%
\pgftext[x=0.268904in, y=0.830748in, left, base]{\color{textcolor}\rmfamily\fontsize{11.000000}{13.200000}\selectfont \(\displaystyle {0.0}\)}%
\end{pgfscope}%
\begin{pgfscope}%
\pgfsetbuttcap%
\pgfsetroundjoin%
\definecolor{currentfill}{rgb}{0.000000,0.000000,0.000000}%
\pgfsetfillcolor{currentfill}%
\pgfsetlinewidth{0.803000pt}%
\definecolor{currentstroke}{rgb}{0.000000,0.000000,0.000000}%
\pgfsetstrokecolor{currentstroke}%
\pgfsetdash{}{0pt}%
\pgfsys@defobject{currentmarker}{\pgfqpoint{-0.048611in}{0.000000in}}{\pgfqpoint{0.000000in}{0.000000in}}{%
\pgfpathmoveto{\pgfqpoint{0.000000in}{0.000000in}}%
\pgfpathlineto{\pgfqpoint{-0.048611in}{0.000000in}}%
\pgfusepath{stroke,fill}%
}%
\begin{pgfscope}%
\pgfsys@transformshift{0.560455in}{1.447538in}%
\pgfsys@useobject{currentmarker}{}%
\end{pgfscope}%
\end{pgfscope}%
\begin{pgfscope}%
\definecolor{textcolor}{rgb}{0.000000,0.000000,0.000000}%
\pgfsetstrokecolor{textcolor}%
\pgfsetfillcolor{textcolor}%
\pgftext[x=0.268904in, y=1.394732in, left, base]{\color{textcolor}\rmfamily\fontsize{11.000000}{13.200000}\selectfont \(\displaystyle {0.1}\)}%
\end{pgfscope}%
\begin{pgfscope}%
\pgfsetbuttcap%
\pgfsetroundjoin%
\definecolor{currentfill}{rgb}{0.000000,0.000000,0.000000}%
\pgfsetfillcolor{currentfill}%
\pgfsetlinewidth{0.803000pt}%
\definecolor{currentstroke}{rgb}{0.000000,0.000000,0.000000}%
\pgfsetstrokecolor{currentstroke}%
\pgfsetdash{}{0pt}%
\pgfsys@defobject{currentmarker}{\pgfqpoint{-0.048611in}{0.000000in}}{\pgfqpoint{0.000000in}{0.000000in}}{%
\pgfpathmoveto{\pgfqpoint{0.000000in}{0.000000in}}%
\pgfpathlineto{\pgfqpoint{-0.048611in}{0.000000in}}%
\pgfusepath{stroke,fill}%
}%
\begin{pgfscope}%
\pgfsys@transformshift{0.560455in}{2.011521in}%
\pgfsys@useobject{currentmarker}{}%
\end{pgfscope}%
\end{pgfscope}%
\begin{pgfscope}%
\definecolor{textcolor}{rgb}{0.000000,0.000000,0.000000}%
\pgfsetstrokecolor{textcolor}%
\pgfsetfillcolor{textcolor}%
\pgftext[x=0.268904in, y=1.958715in, left, base]{\color{textcolor}\rmfamily\fontsize{11.000000}{13.200000}\selectfont \(\displaystyle {0.2}\)}%
\end{pgfscope}%
\begin{pgfscope}%
\pgfsetbuttcap%
\pgfsetroundjoin%
\definecolor{currentfill}{rgb}{0.000000,0.000000,0.000000}%
\pgfsetfillcolor{currentfill}%
\pgfsetlinewidth{0.803000pt}%
\definecolor{currentstroke}{rgb}{0.000000,0.000000,0.000000}%
\pgfsetstrokecolor{currentstroke}%
\pgfsetdash{}{0pt}%
\pgfsys@defobject{currentmarker}{\pgfqpoint{-0.048611in}{0.000000in}}{\pgfqpoint{0.000000in}{0.000000in}}{%
\pgfpathmoveto{\pgfqpoint{0.000000in}{0.000000in}}%
\pgfpathlineto{\pgfqpoint{-0.048611in}{0.000000in}}%
\pgfusepath{stroke,fill}%
}%
\begin{pgfscope}%
\pgfsys@transformshift{0.560455in}{2.575504in}%
\pgfsys@useobject{currentmarker}{}%
\end{pgfscope}%
\end{pgfscope}%
\begin{pgfscope}%
\definecolor{textcolor}{rgb}{0.000000,0.000000,0.000000}%
\pgfsetstrokecolor{textcolor}%
\pgfsetfillcolor{textcolor}%
\pgftext[x=0.268904in, y=2.522698in, left, base]{\color{textcolor}\rmfamily\fontsize{11.000000}{13.200000}\selectfont \(\displaystyle {0.3}\)}%
\end{pgfscope}%
\begin{pgfscope}%
\pgfsetbuttcap%
\pgfsetroundjoin%
\definecolor{currentfill}{rgb}{0.000000,0.000000,0.000000}%
\pgfsetfillcolor{currentfill}%
\pgfsetlinewidth{0.803000pt}%
\definecolor{currentstroke}{rgb}{0.000000,0.000000,0.000000}%
\pgfsetstrokecolor{currentstroke}%
\pgfsetdash{}{0pt}%
\pgfsys@defobject{currentmarker}{\pgfqpoint{-0.048611in}{0.000000in}}{\pgfqpoint{0.000000in}{0.000000in}}{%
\pgfpathmoveto{\pgfqpoint{0.000000in}{0.000000in}}%
\pgfpathlineto{\pgfqpoint{-0.048611in}{0.000000in}}%
\pgfusepath{stroke,fill}%
}%
\begin{pgfscope}%
\pgfsys@transformshift{0.560455in}{3.139487in}%
\pgfsys@useobject{currentmarker}{}%
\end{pgfscope}%
\end{pgfscope}%
\begin{pgfscope}%
\definecolor{textcolor}{rgb}{0.000000,0.000000,0.000000}%
\pgfsetstrokecolor{textcolor}%
\pgfsetfillcolor{textcolor}%
\pgftext[x=0.268904in, y=3.086681in, left, base]{\color{textcolor}\rmfamily\fontsize{11.000000}{13.200000}\selectfont \(\displaystyle {0.4}\)}%
\end{pgfscope}%
\begin{pgfscope}%
\pgfsetbuttcap%
\pgfsetroundjoin%
\definecolor{currentfill}{rgb}{0.000000,0.000000,0.000000}%
\pgfsetfillcolor{currentfill}%
\pgfsetlinewidth{0.803000pt}%
\definecolor{currentstroke}{rgb}{0.000000,0.000000,0.000000}%
\pgfsetstrokecolor{currentstroke}%
\pgfsetdash{}{0pt}%
\pgfsys@defobject{currentmarker}{\pgfqpoint{-0.048611in}{0.000000in}}{\pgfqpoint{0.000000in}{0.000000in}}{%
\pgfpathmoveto{\pgfqpoint{0.000000in}{0.000000in}}%
\pgfpathlineto{\pgfqpoint{-0.048611in}{0.000000in}}%
\pgfusepath{stroke,fill}%
}%
\begin{pgfscope}%
\pgfsys@transformshift{0.560455in}{3.703470in}%
\pgfsys@useobject{currentmarker}{}%
\end{pgfscope}%
\end{pgfscope}%
\begin{pgfscope}%
\definecolor{textcolor}{rgb}{0.000000,0.000000,0.000000}%
\pgfsetstrokecolor{textcolor}%
\pgfsetfillcolor{textcolor}%
\pgftext[x=0.268904in, y=3.650664in, left, base]{\color{textcolor}\rmfamily\fontsize{11.000000}{13.200000}\selectfont \(\displaystyle {0.5}\)}%
\end{pgfscope}%
\begin{pgfscope}%
\definecolor{textcolor}{rgb}{0.000000,0.000000,0.000000}%
\pgfsetstrokecolor{textcolor}%
\pgfsetfillcolor{textcolor}%
\pgftext[x=0.213349in,y=2.393555in,,bottom,rotate=90.000000]{\color{textcolor}\rmfamily\fontsize{16.000000}{19.200000}\selectfont Hazard}%
\end{pgfscope}%
\begin{pgfscope}%
\pgfpathrectangle{\pgfqpoint{0.560455in}{0.883555in}}{\pgfqpoint{4.650000in}{3.020000in}}%
\pgfusepath{clip}%
\pgfsetbuttcap%
\pgfsetroundjoin%
\pgfsetlinewidth{1.505625pt}%
\definecolor{currentstroke}{rgb}{1.000000,0.000000,0.000000}%
\pgfsetstrokecolor{currentstroke}%
\pgfsetdash{{5.550000pt}{2.400000pt}}{0.000000pt}%
\pgfpathmoveto{\pgfqpoint{0.771819in}{2.186336in}}%
\pgfpathlineto{\pgfqpoint{2.885455in}{2.320608in}}%
\pgfpathlineto{\pgfqpoint{4.999092in}{2.061184in}}%
\pgfusepath{stroke}%
\end{pgfscope}%
\begin{pgfscope}%
\pgfpathrectangle{\pgfqpoint{0.560455in}{0.883555in}}{\pgfqpoint{4.650000in}{3.020000in}}%
\pgfusepath{clip}%
\pgfsetrectcap%
\pgfsetroundjoin%
\pgfsetlinewidth{1.505625pt}%
\definecolor{currentstroke}{rgb}{0.000000,0.000000,1.000000}%
\pgfsetstrokecolor{currentstroke}%
\pgfsetdash{}{0pt}%
\pgfpathmoveto{\pgfqpoint{0.771819in}{3.359296in}}%
\pgfpathlineto{\pgfqpoint{2.885455in}{2.506528in}}%
\pgfpathlineto{\pgfqpoint{4.999092in}{1.265299in}}%
\pgfusepath{stroke}%
\end{pgfscope}%
\begin{pgfscope}%
\pgfsetrectcap%
\pgfsetmiterjoin%
\pgfsetlinewidth{0.803000pt}%
\definecolor{currentstroke}{rgb}{0.000000,0.000000,0.000000}%
\pgfsetstrokecolor{currentstroke}%
\pgfsetdash{}{0pt}%
\pgfpathmoveto{\pgfqpoint{0.560455in}{0.883555in}}%
\pgfpathlineto{\pgfqpoint{0.560455in}{3.903555in}}%
\pgfusepath{stroke}%
\end{pgfscope}%
\begin{pgfscope}%
\pgfsetrectcap%
\pgfsetmiterjoin%
\pgfsetlinewidth{0.803000pt}%
\definecolor{currentstroke}{rgb}{0.000000,0.000000,0.000000}%
\pgfsetstrokecolor{currentstroke}%
\pgfsetdash{}{0pt}%
\pgfpathmoveto{\pgfqpoint{5.210455in}{0.883555in}}%
\pgfpathlineto{\pgfqpoint{5.210455in}{3.903555in}}%
\pgfusepath{stroke}%
\end{pgfscope}%
\begin{pgfscope}%
\pgfsetrectcap%
\pgfsetmiterjoin%
\pgfsetlinewidth{0.803000pt}%
\definecolor{currentstroke}{rgb}{0.000000,0.000000,0.000000}%
\pgfsetstrokecolor{currentstroke}%
\pgfsetdash{}{0pt}%
\pgfpathmoveto{\pgfqpoint{0.560455in}{0.883555in}}%
\pgfpathlineto{\pgfqpoint{5.210455in}{0.883555in}}%
\pgfusepath{stroke}%
\end{pgfscope}%
\begin{pgfscope}%
\pgfsetrectcap%
\pgfsetmiterjoin%
\pgfsetlinewidth{0.803000pt}%
\definecolor{currentstroke}{rgb}{0.000000,0.000000,0.000000}%
\pgfsetstrokecolor{currentstroke}%
\pgfsetdash{}{0pt}%
\pgfpathmoveto{\pgfqpoint{0.560455in}{3.903555in}}%
\pgfpathlineto{\pgfqpoint{5.210455in}{3.903555in}}%
\pgfusepath{stroke}%
\end{pgfscope}%
\begin{pgfscope}%
\pgfsetbuttcap%
\pgfsetmiterjoin%
\definecolor{currentfill}{rgb}{0.300000,0.300000,0.300000}%
\pgfsetfillcolor{currentfill}%
\pgfsetfillopacity{0.500000}%
\pgfsetlinewidth{1.003750pt}%
\definecolor{currentstroke}{rgb}{0.300000,0.300000,0.300000}%
\pgfsetstrokecolor{currentstroke}%
\pgfsetstrokeopacity{0.500000}%
\pgfsetdash{}{0pt}%
\pgfpathmoveto{\pgfqpoint{1.465024in}{-0.027778in}}%
\pgfpathlineto{\pgfqpoint{4.361442in}{-0.027778in}}%
\pgfpathquadraticcurveto{\pgfqpoint{4.400331in}{-0.027778in}}{\pgfqpoint{4.400331in}{0.011111in}}%
\pgfpathlineto{\pgfqpoint{4.400331in}{0.266666in}}%
\pgfpathquadraticcurveto{\pgfqpoint{4.400331in}{0.305555in}}{\pgfqpoint{4.361442in}{0.305555in}}%
\pgfpathlineto{\pgfqpoint{1.465024in}{0.305555in}}%
\pgfpathquadraticcurveto{\pgfqpoint{1.426135in}{0.305555in}}{\pgfqpoint{1.426135in}{0.266666in}}%
\pgfpathlineto{\pgfqpoint{1.426135in}{0.011111in}}%
\pgfpathquadraticcurveto{\pgfqpoint{1.426135in}{-0.027778in}}{\pgfqpoint{1.465024in}{-0.027778in}}%
\pgfpathclose%
\pgfusepath{stroke,fill}%
\end{pgfscope}%
\begin{pgfscope}%
\pgfsetbuttcap%
\pgfsetmiterjoin%
\definecolor{currentfill}{rgb}{1.000000,1.000000,1.000000}%
\pgfsetfillcolor{currentfill}%
\pgfsetlinewidth{1.003750pt}%
\definecolor{currentstroke}{rgb}{0.800000,0.800000,0.800000}%
\pgfsetstrokecolor{currentstroke}%
\pgfsetdash{}{0pt}%
\pgfpathmoveto{\pgfqpoint{1.437246in}{0.000000in}}%
\pgfpathlineto{\pgfqpoint{4.333665in}{0.000000in}}%
\pgfpathquadraticcurveto{\pgfqpoint{4.372553in}{0.000000in}}{\pgfqpoint{4.372553in}{0.038889in}}%
\pgfpathlineto{\pgfqpoint{4.372553in}{0.294444in}}%
\pgfpathquadraticcurveto{\pgfqpoint{4.372553in}{0.333333in}}{\pgfqpoint{4.333665in}{0.333333in}}%
\pgfpathlineto{\pgfqpoint{1.437246in}{0.333333in}}%
\pgfpathquadraticcurveto{\pgfqpoint{1.398357in}{0.333333in}}{\pgfqpoint{1.398357in}{0.294444in}}%
\pgfpathlineto{\pgfqpoint{1.398357in}{0.038889in}}%
\pgfpathquadraticcurveto{\pgfqpoint{1.398357in}{0.000000in}}{\pgfqpoint{1.437246in}{0.000000in}}%
\pgfpathclose%
\pgfusepath{stroke,fill}%
\end{pgfscope}%
\begin{pgfscope}%
\pgfsetbuttcap%
\pgfsetroundjoin%
\pgfsetlinewidth{1.505625pt}%
\definecolor{currentstroke}{rgb}{1.000000,0.000000,0.000000}%
\pgfsetstrokecolor{currentstroke}%
\pgfsetdash{{5.550000pt}{2.400000pt}}{0.000000pt}%
\pgfpathmoveto{\pgfqpoint{1.476135in}{0.184722in}}%
\pgfpathlineto{\pgfqpoint{1.865024in}{0.184722in}}%
\pgfusepath{stroke}%
\end{pgfscope}%
\begin{pgfscope}%
\definecolor{textcolor}{rgb}{0.000000,0.000000,0.000000}%
\pgfsetstrokecolor{textcolor}%
\pgfsetfillcolor{textcolor}%
\pgftext[x=2.020579in,y=0.116667in,left,base]{\color{textcolor}\rmfamily\fontsize{14.000000}{16.800000}\selectfont Simulation}%
\end{pgfscope}%
\begin{pgfscope}%
\pgfsetrectcap%
\pgfsetroundjoin%
\pgfsetlinewidth{1.505625pt}%
\definecolor{currentstroke}{rgb}{0.000000,0.000000,1.000000}%
\pgfsetstrokecolor{currentstroke}%
\pgfsetdash{}{0pt}%
\pgfpathmoveto{\pgfqpoint{3.328785in}{0.184722in}}%
\pgfpathlineto{\pgfqpoint{3.717674in}{0.184722in}}%
\pgfusepath{stroke}%
\end{pgfscope}%
\begin{pgfscope}%
\definecolor{textcolor}{rgb}{0.000000,0.000000,0.000000}%
\pgfsetstrokecolor{textcolor}%
\pgfsetfillcolor{textcolor}%
\pgftext[x=3.873230in,y=0.116667in,left,base]{\color{textcolor}\rmfamily\fontsize{14.000000}{16.800000}\selectfont Data}%
\end{pgfscope}%
\end{pgfpicture}%
\makeatother%
\endgroup%
} 
\end{subfigure}
\end{center}
\end{figure}
\FloatBarrier

\begin{figure}
\begin{center}
\caption{\\Share ever cohabited and married: data and simulations}
\label{fig:erel}
%\begin{subfigure}
\hspace*{-1.3cm} 
\scalebox{0.99}{%% Creator: Matplotlib, PGF backend
%%
%% To include the figure in your LaTeX document, write
%%   \input{<filename>.pgf}
%%
%% Make sure the required packages are loaded in your preamble
%%   \usepackage{pgf}
%%
%% and, on pdftex
%%   \usepackage[utf8]{inputenc}\DeclareUnicodeCharacter{2212}{-}
%%
%% or, on luatex and xetex
%%   \usepackage{unicode-math}
%%
%% Figures using additional raster images can only be included by \input if
%% they are in the same directory as the main LaTeX file. For loading figures
%% from other directories you can use the `import` package
%%   \usepackage{import}
%%
%% and then include the figures with
%%   \import{<path to file>}{<filename>.pgf}
%%
%% Matplotlib used the following preamble
%%
\begingroup%
\makeatletter%
\begin{pgfpicture}%
\pgfpathrectangle{\pgfpointorigin}{\pgfqpoint{5.208334in}{2.225733in}}%
\pgfusepath{use as bounding box, clip}%
\begin{pgfscope}%
\pgfsetbuttcap%
\pgfsetmiterjoin%
\definecolor{currentfill}{rgb}{1.000000,1.000000,1.000000}%
\pgfsetfillcolor{currentfill}%
\pgfsetlinewidth{0.000000pt}%
\definecolor{currentstroke}{rgb}{1.000000,1.000000,1.000000}%
\pgfsetstrokecolor{currentstroke}%
\pgfsetdash{}{0pt}%
\pgfpathmoveto{\pgfqpoint{0.000000in}{0.000000in}}%
\pgfpathlineto{\pgfqpoint{5.208334in}{0.000000in}}%
\pgfpathlineto{\pgfqpoint{5.208334in}{2.225733in}}%
\pgfpathlineto{\pgfqpoint{0.000000in}{2.225733in}}%
\pgfpathclose%
\pgfusepath{fill}%
\end{pgfscope}%
\begin{pgfscope}%
\pgfsetbuttcap%
\pgfsetmiterjoin%
\definecolor{currentfill}{rgb}{1.000000,1.000000,1.000000}%
\pgfsetfillcolor{currentfill}%
\pgfsetlinewidth{0.000000pt}%
\definecolor{currentstroke}{rgb}{0.000000,0.000000,0.000000}%
\pgfsetstrokecolor{currentstroke}%
\pgfsetstrokeopacity{0.000000}%
\pgfsetdash{}{0pt}%
\pgfpathmoveto{\pgfqpoint{0.558334in}{0.800199in}}%
\pgfpathlineto{\pgfqpoint{5.208334in}{0.800199in}}%
\pgfpathlineto{\pgfqpoint{5.208334in}{2.172927in}}%
\pgfpathlineto{\pgfqpoint{0.558334in}{2.172927in}}%
\pgfpathclose%
\pgfusepath{fill}%
\end{pgfscope}%
\begin{pgfscope}%
\pgfpathrectangle{\pgfqpoint{0.558334in}{0.800199in}}{\pgfqpoint{4.650000in}{1.372727in}}%
\pgfusepath{clip}%
\pgfsetbuttcap%
\pgfsetroundjoin%
\definecolor{currentfill}{rgb}{0.000000,0.500000,0.000000}%
\pgfsetfillcolor{currentfill}%
\pgfsetfillopacity{0.200000}%
\pgfsetlinewidth{0.000000pt}%
\definecolor{currentstroke}{rgb}{0.000000,0.000000,0.000000}%
\pgfsetstrokecolor{currentstroke}%
\pgfsetdash{}{0pt}%
\pgfpathmoveto{\pgfqpoint{0.558334in}{1.601226in}}%
\pgfpathlineto{\pgfqpoint{0.558334in}{1.495959in}}%
\pgfpathlineto{\pgfqpoint{1.333334in}{1.782628in}}%
\pgfpathlineto{\pgfqpoint{2.108334in}{1.941117in}}%
\pgfpathlineto{\pgfqpoint{2.883334in}{2.018430in}}%
\pgfpathlineto{\pgfqpoint{3.658334in}{2.055091in}}%
\pgfpathlineto{\pgfqpoint{4.433334in}{2.078658in}}%
\pgfpathlineto{\pgfqpoint{5.208334in}{2.087757in}}%
\pgfpathlineto{\pgfqpoint{5.208334in}{2.132404in}}%
\pgfpathlineto{\pgfqpoint{5.208334in}{2.132404in}}%
\pgfpathlineto{\pgfqpoint{4.433334in}{2.123174in}}%
\pgfpathlineto{\pgfqpoint{3.658334in}{2.103600in}}%
\pgfpathlineto{\pgfqpoint{2.883334in}{2.073421in}}%
\pgfpathlineto{\pgfqpoint{2.108334in}{2.013324in}}%
\pgfpathlineto{\pgfqpoint{1.333334in}{1.885014in}}%
\pgfpathlineto{\pgfqpoint{0.558334in}{1.601226in}}%
\pgfpathclose%
\pgfusepath{fill}%
\end{pgfscope}%
\begin{pgfscope}%
\pgfpathrectangle{\pgfqpoint{0.558334in}{0.800199in}}{\pgfqpoint{4.650000in}{1.372727in}}%
\pgfusepath{clip}%
\pgfsetbuttcap%
\pgfsetroundjoin%
\definecolor{currentfill}{rgb}{1.000000,0.000000,0.000000}%
\pgfsetfillcolor{currentfill}%
\pgfsetfillopacity{0.200000}%
\pgfsetlinewidth{0.000000pt}%
\definecolor{currentstroke}{rgb}{0.000000,0.000000,0.000000}%
\pgfsetstrokecolor{currentstroke}%
\pgfsetdash{}{0pt}%
\pgfpathmoveto{\pgfqpoint{0.558334in}{0.923865in}}%
\pgfpathlineto{\pgfqpoint{0.558334in}{0.863638in}}%
\pgfpathlineto{\pgfqpoint{1.333334in}{0.935583in}}%
\pgfpathlineto{\pgfqpoint{2.108334in}{1.025989in}}%
\pgfpathlineto{\pgfqpoint{2.883334in}{1.122877in}}%
\pgfpathlineto{\pgfqpoint{3.658334in}{1.175117in}}%
\pgfpathlineto{\pgfqpoint{4.433334in}{1.244575in}}%
\pgfpathlineto{\pgfqpoint{5.208334in}{1.277373in}}%
\pgfpathlineto{\pgfqpoint{5.208334in}{1.386109in}}%
\pgfpathlineto{\pgfqpoint{5.208334in}{1.386109in}}%
\pgfpathlineto{\pgfqpoint{4.433334in}{1.359924in}}%
\pgfpathlineto{\pgfqpoint{3.658334in}{1.294459in}}%
\pgfpathlineto{\pgfqpoint{2.883334in}{1.235607in}}%
\pgfpathlineto{\pgfqpoint{2.108334in}{1.128245in}}%
\pgfpathlineto{\pgfqpoint{1.333334in}{1.011653in}}%
\pgfpathlineto{\pgfqpoint{0.558334in}{0.923865in}}%
\pgfpathclose%
\pgfusepath{fill}%
\end{pgfscope}%
\begin{pgfscope}%
\pgfsetbuttcap%
\pgfsetroundjoin%
\definecolor{currentfill}{rgb}{0.000000,0.000000,0.000000}%
\pgfsetfillcolor{currentfill}%
\pgfsetlinewidth{0.803000pt}%
\definecolor{currentstroke}{rgb}{0.000000,0.000000,0.000000}%
\pgfsetstrokecolor{currentstroke}%
\pgfsetdash{}{0pt}%
\pgfsys@defobject{currentmarker}{\pgfqpoint{0.000000in}{-0.048611in}}{\pgfqpoint{0.000000in}{0.000000in}}{%
\pgfpathmoveto{\pgfqpoint{0.000000in}{0.000000in}}%
\pgfpathlineto{\pgfqpoint{0.000000in}{-0.048611in}}%
\pgfusepath{stroke,fill}%
}%
\begin{pgfscope}%
\pgfsys@transformshift{0.816667in}{0.800199in}%
\pgfsys@useobject{currentmarker}{}%
\end{pgfscope}%
\end{pgfscope}%
\begin{pgfscope}%
\definecolor{textcolor}{rgb}{0.000000,0.000000,0.000000}%
\pgfsetstrokecolor{textcolor}%
\pgfsetfillcolor{textcolor}%
\pgftext[x=0.816667in,y=0.702977in,,top]{\color{textcolor}\rmfamily\fontsize{11.000000}{13.200000}\selectfont \(\displaystyle {22}\)}%
\end{pgfscope}%
\begin{pgfscope}%
\pgfsetbuttcap%
\pgfsetroundjoin%
\definecolor{currentfill}{rgb}{0.000000,0.000000,0.000000}%
\pgfsetfillcolor{currentfill}%
\pgfsetlinewidth{0.803000pt}%
\definecolor{currentstroke}{rgb}{0.000000,0.000000,0.000000}%
\pgfsetstrokecolor{currentstroke}%
\pgfsetdash{}{0pt}%
\pgfsys@defobject{currentmarker}{\pgfqpoint{0.000000in}{-0.048611in}}{\pgfqpoint{0.000000in}{0.000000in}}{%
\pgfpathmoveto{\pgfqpoint{0.000000in}{0.000000in}}%
\pgfpathlineto{\pgfqpoint{0.000000in}{-0.048611in}}%
\pgfusepath{stroke,fill}%
}%
\begin{pgfscope}%
\pgfsys@transformshift{1.333334in}{0.800199in}%
\pgfsys@useobject{currentmarker}{}%
\end{pgfscope}%
\end{pgfscope}%
\begin{pgfscope}%
\definecolor{textcolor}{rgb}{0.000000,0.000000,0.000000}%
\pgfsetstrokecolor{textcolor}%
\pgfsetfillcolor{textcolor}%
\pgftext[x=1.333334in,y=0.702977in,,top]{\color{textcolor}\rmfamily\fontsize{11.000000}{13.200000}\selectfont \(\displaystyle {24}\)}%
\end{pgfscope}%
\begin{pgfscope}%
\pgfsetbuttcap%
\pgfsetroundjoin%
\definecolor{currentfill}{rgb}{0.000000,0.000000,0.000000}%
\pgfsetfillcolor{currentfill}%
\pgfsetlinewidth{0.803000pt}%
\definecolor{currentstroke}{rgb}{0.000000,0.000000,0.000000}%
\pgfsetstrokecolor{currentstroke}%
\pgfsetdash{}{0pt}%
\pgfsys@defobject{currentmarker}{\pgfqpoint{0.000000in}{-0.048611in}}{\pgfqpoint{0.000000in}{0.000000in}}{%
\pgfpathmoveto{\pgfqpoint{0.000000in}{0.000000in}}%
\pgfpathlineto{\pgfqpoint{0.000000in}{-0.048611in}}%
\pgfusepath{stroke,fill}%
}%
\begin{pgfscope}%
\pgfsys@transformshift{1.850000in}{0.800199in}%
\pgfsys@useobject{currentmarker}{}%
\end{pgfscope}%
\end{pgfscope}%
\begin{pgfscope}%
\definecolor{textcolor}{rgb}{0.000000,0.000000,0.000000}%
\pgfsetstrokecolor{textcolor}%
\pgfsetfillcolor{textcolor}%
\pgftext[x=1.850000in,y=0.702977in,,top]{\color{textcolor}\rmfamily\fontsize{11.000000}{13.200000}\selectfont \(\displaystyle {26}\)}%
\end{pgfscope}%
\begin{pgfscope}%
\pgfsetbuttcap%
\pgfsetroundjoin%
\definecolor{currentfill}{rgb}{0.000000,0.000000,0.000000}%
\pgfsetfillcolor{currentfill}%
\pgfsetlinewidth{0.803000pt}%
\definecolor{currentstroke}{rgb}{0.000000,0.000000,0.000000}%
\pgfsetstrokecolor{currentstroke}%
\pgfsetdash{}{0pt}%
\pgfsys@defobject{currentmarker}{\pgfqpoint{0.000000in}{-0.048611in}}{\pgfqpoint{0.000000in}{0.000000in}}{%
\pgfpathmoveto{\pgfqpoint{0.000000in}{0.000000in}}%
\pgfpathlineto{\pgfqpoint{0.000000in}{-0.048611in}}%
\pgfusepath{stroke,fill}%
}%
\begin{pgfscope}%
\pgfsys@transformshift{2.366667in}{0.800199in}%
\pgfsys@useobject{currentmarker}{}%
\end{pgfscope}%
\end{pgfscope}%
\begin{pgfscope}%
\definecolor{textcolor}{rgb}{0.000000,0.000000,0.000000}%
\pgfsetstrokecolor{textcolor}%
\pgfsetfillcolor{textcolor}%
\pgftext[x=2.366667in,y=0.702977in,,top]{\color{textcolor}\rmfamily\fontsize{11.000000}{13.200000}\selectfont \(\displaystyle {28}\)}%
\end{pgfscope}%
\begin{pgfscope}%
\pgfsetbuttcap%
\pgfsetroundjoin%
\definecolor{currentfill}{rgb}{0.000000,0.000000,0.000000}%
\pgfsetfillcolor{currentfill}%
\pgfsetlinewidth{0.803000pt}%
\definecolor{currentstroke}{rgb}{0.000000,0.000000,0.000000}%
\pgfsetstrokecolor{currentstroke}%
\pgfsetdash{}{0pt}%
\pgfsys@defobject{currentmarker}{\pgfqpoint{0.000000in}{-0.048611in}}{\pgfqpoint{0.000000in}{0.000000in}}{%
\pgfpathmoveto{\pgfqpoint{0.000000in}{0.000000in}}%
\pgfpathlineto{\pgfqpoint{0.000000in}{-0.048611in}}%
\pgfusepath{stroke,fill}%
}%
\begin{pgfscope}%
\pgfsys@transformshift{2.883334in}{0.800199in}%
\pgfsys@useobject{currentmarker}{}%
\end{pgfscope}%
\end{pgfscope}%
\begin{pgfscope}%
\definecolor{textcolor}{rgb}{0.000000,0.000000,0.000000}%
\pgfsetstrokecolor{textcolor}%
\pgfsetfillcolor{textcolor}%
\pgftext[x=2.883334in,y=0.702977in,,top]{\color{textcolor}\rmfamily\fontsize{11.000000}{13.200000}\selectfont \(\displaystyle {30}\)}%
\end{pgfscope}%
\begin{pgfscope}%
\pgfsetbuttcap%
\pgfsetroundjoin%
\definecolor{currentfill}{rgb}{0.000000,0.000000,0.000000}%
\pgfsetfillcolor{currentfill}%
\pgfsetlinewidth{0.803000pt}%
\definecolor{currentstroke}{rgb}{0.000000,0.000000,0.000000}%
\pgfsetstrokecolor{currentstroke}%
\pgfsetdash{}{0pt}%
\pgfsys@defobject{currentmarker}{\pgfqpoint{0.000000in}{-0.048611in}}{\pgfqpoint{0.000000in}{0.000000in}}{%
\pgfpathmoveto{\pgfqpoint{0.000000in}{0.000000in}}%
\pgfpathlineto{\pgfqpoint{0.000000in}{-0.048611in}}%
\pgfusepath{stroke,fill}%
}%
\begin{pgfscope}%
\pgfsys@transformshift{3.400000in}{0.800199in}%
\pgfsys@useobject{currentmarker}{}%
\end{pgfscope}%
\end{pgfscope}%
\begin{pgfscope}%
\definecolor{textcolor}{rgb}{0.000000,0.000000,0.000000}%
\pgfsetstrokecolor{textcolor}%
\pgfsetfillcolor{textcolor}%
\pgftext[x=3.400000in,y=0.702977in,,top]{\color{textcolor}\rmfamily\fontsize{11.000000}{13.200000}\selectfont \(\displaystyle {32}\)}%
\end{pgfscope}%
\begin{pgfscope}%
\pgfsetbuttcap%
\pgfsetroundjoin%
\definecolor{currentfill}{rgb}{0.000000,0.000000,0.000000}%
\pgfsetfillcolor{currentfill}%
\pgfsetlinewidth{0.803000pt}%
\definecolor{currentstroke}{rgb}{0.000000,0.000000,0.000000}%
\pgfsetstrokecolor{currentstroke}%
\pgfsetdash{}{0pt}%
\pgfsys@defobject{currentmarker}{\pgfqpoint{0.000000in}{-0.048611in}}{\pgfqpoint{0.000000in}{0.000000in}}{%
\pgfpathmoveto{\pgfqpoint{0.000000in}{0.000000in}}%
\pgfpathlineto{\pgfqpoint{0.000000in}{-0.048611in}}%
\pgfusepath{stroke,fill}%
}%
\begin{pgfscope}%
\pgfsys@transformshift{3.916667in}{0.800199in}%
\pgfsys@useobject{currentmarker}{}%
\end{pgfscope}%
\end{pgfscope}%
\begin{pgfscope}%
\definecolor{textcolor}{rgb}{0.000000,0.000000,0.000000}%
\pgfsetstrokecolor{textcolor}%
\pgfsetfillcolor{textcolor}%
\pgftext[x=3.916667in,y=0.702977in,,top]{\color{textcolor}\rmfamily\fontsize{11.000000}{13.200000}\selectfont \(\displaystyle {34}\)}%
\end{pgfscope}%
\begin{pgfscope}%
\pgfsetbuttcap%
\pgfsetroundjoin%
\definecolor{currentfill}{rgb}{0.000000,0.000000,0.000000}%
\pgfsetfillcolor{currentfill}%
\pgfsetlinewidth{0.803000pt}%
\definecolor{currentstroke}{rgb}{0.000000,0.000000,0.000000}%
\pgfsetstrokecolor{currentstroke}%
\pgfsetdash{}{0pt}%
\pgfsys@defobject{currentmarker}{\pgfqpoint{0.000000in}{-0.048611in}}{\pgfqpoint{0.000000in}{0.000000in}}{%
\pgfpathmoveto{\pgfqpoint{0.000000in}{0.000000in}}%
\pgfpathlineto{\pgfqpoint{0.000000in}{-0.048611in}}%
\pgfusepath{stroke,fill}%
}%
\begin{pgfscope}%
\pgfsys@transformshift{4.433334in}{0.800199in}%
\pgfsys@useobject{currentmarker}{}%
\end{pgfscope}%
\end{pgfscope}%
\begin{pgfscope}%
\definecolor{textcolor}{rgb}{0.000000,0.000000,0.000000}%
\pgfsetstrokecolor{textcolor}%
\pgfsetfillcolor{textcolor}%
\pgftext[x=4.433334in,y=0.702977in,,top]{\color{textcolor}\rmfamily\fontsize{11.000000}{13.200000}\selectfont \(\displaystyle {36}\)}%
\end{pgfscope}%
\begin{pgfscope}%
\pgfsetbuttcap%
\pgfsetroundjoin%
\definecolor{currentfill}{rgb}{0.000000,0.000000,0.000000}%
\pgfsetfillcolor{currentfill}%
\pgfsetlinewidth{0.803000pt}%
\definecolor{currentstroke}{rgb}{0.000000,0.000000,0.000000}%
\pgfsetstrokecolor{currentstroke}%
\pgfsetdash{}{0pt}%
\pgfsys@defobject{currentmarker}{\pgfqpoint{0.000000in}{-0.048611in}}{\pgfqpoint{0.000000in}{0.000000in}}{%
\pgfpathmoveto{\pgfqpoint{0.000000in}{0.000000in}}%
\pgfpathlineto{\pgfqpoint{0.000000in}{-0.048611in}}%
\pgfusepath{stroke,fill}%
}%
\begin{pgfscope}%
\pgfsys@transformshift{4.950000in}{0.800199in}%
\pgfsys@useobject{currentmarker}{}%
\end{pgfscope}%
\end{pgfscope}%
\begin{pgfscope}%
\definecolor{textcolor}{rgb}{0.000000,0.000000,0.000000}%
\pgfsetstrokecolor{textcolor}%
\pgfsetfillcolor{textcolor}%
\pgftext[x=4.950000in,y=0.702977in,,top]{\color{textcolor}\rmfamily\fontsize{11.000000}{13.200000}\selectfont \(\displaystyle {38}\)}%
\end{pgfscope}%
\begin{pgfscope}%
\definecolor{textcolor}{rgb}{0.000000,0.000000,0.000000}%
\pgfsetstrokecolor{textcolor}%
\pgfsetfillcolor{textcolor}%
\pgftext[x=2.883334in,y=0.512236in,,top]{\color{textcolor}\rmfamily\fontsize{11.000000}{13.200000}\selectfont Age}%
\end{pgfscope}%
\begin{pgfscope}%
\pgfsetbuttcap%
\pgfsetroundjoin%
\definecolor{currentfill}{rgb}{0.000000,0.000000,0.000000}%
\pgfsetfillcolor{currentfill}%
\pgfsetlinewidth{0.803000pt}%
\definecolor{currentstroke}{rgb}{0.000000,0.000000,0.000000}%
\pgfsetstrokecolor{currentstroke}%
\pgfsetdash{}{0pt}%
\pgfsys@defobject{currentmarker}{\pgfqpoint{-0.048611in}{0.000000in}}{\pgfqpoint{0.000000in}{0.000000in}}{%
\pgfpathmoveto{\pgfqpoint{0.000000in}{0.000000in}}%
\pgfpathlineto{\pgfqpoint{-0.048611in}{0.000000in}}%
\pgfusepath{stroke,fill}%
}%
\begin{pgfscope}%
\pgfsys@transformshift{0.558334in}{1.108475in}%
\pgfsys@useobject{currentmarker}{}%
\end{pgfscope}%
\end{pgfscope}%
\begin{pgfscope}%
\definecolor{textcolor}{rgb}{0.000000,0.000000,0.000000}%
\pgfsetstrokecolor{textcolor}%
\pgfsetfillcolor{textcolor}%
\pgftext[x=0.190741in, y=1.055668in, left, base]{\color{textcolor}\rmfamily\fontsize{11.000000}{13.200000}\selectfont \(\displaystyle {0.25}\)}%
\end{pgfscope}%
\begin{pgfscope}%
\pgfsetbuttcap%
\pgfsetroundjoin%
\definecolor{currentfill}{rgb}{0.000000,0.000000,0.000000}%
\pgfsetfillcolor{currentfill}%
\pgfsetlinewidth{0.803000pt}%
\definecolor{currentstroke}{rgb}{0.000000,0.000000,0.000000}%
\pgfsetstrokecolor{currentstroke}%
\pgfsetdash{}{0pt}%
\pgfsys@defobject{currentmarker}{\pgfqpoint{-0.048611in}{0.000000in}}{\pgfqpoint{0.000000in}{0.000000in}}{%
\pgfpathmoveto{\pgfqpoint{0.000000in}{0.000000in}}%
\pgfpathlineto{\pgfqpoint{-0.048611in}{0.000000in}}%
\pgfusepath{stroke,fill}%
}%
\begin{pgfscope}%
\pgfsys@transformshift{0.558334in}{1.463292in}%
\pgfsys@useobject{currentmarker}{}%
\end{pgfscope}%
\end{pgfscope}%
\begin{pgfscope}%
\definecolor{textcolor}{rgb}{0.000000,0.000000,0.000000}%
\pgfsetstrokecolor{textcolor}%
\pgfsetfillcolor{textcolor}%
\pgftext[x=0.190741in, y=1.410485in, left, base]{\color{textcolor}\rmfamily\fontsize{11.000000}{13.200000}\selectfont \(\displaystyle {0.50}\)}%
\end{pgfscope}%
\begin{pgfscope}%
\pgfsetbuttcap%
\pgfsetroundjoin%
\definecolor{currentfill}{rgb}{0.000000,0.000000,0.000000}%
\pgfsetfillcolor{currentfill}%
\pgfsetlinewidth{0.803000pt}%
\definecolor{currentstroke}{rgb}{0.000000,0.000000,0.000000}%
\pgfsetstrokecolor{currentstroke}%
\pgfsetdash{}{0pt}%
\pgfsys@defobject{currentmarker}{\pgfqpoint{-0.048611in}{0.000000in}}{\pgfqpoint{0.000000in}{0.000000in}}{%
\pgfpathmoveto{\pgfqpoint{0.000000in}{0.000000in}}%
\pgfpathlineto{\pgfqpoint{-0.048611in}{0.000000in}}%
\pgfusepath{stroke,fill}%
}%
\begin{pgfscope}%
\pgfsys@transformshift{0.558334in}{1.818109in}%
\pgfsys@useobject{currentmarker}{}%
\end{pgfscope}%
\end{pgfscope}%
\begin{pgfscope}%
\definecolor{textcolor}{rgb}{0.000000,0.000000,0.000000}%
\pgfsetstrokecolor{textcolor}%
\pgfsetfillcolor{textcolor}%
\pgftext[x=0.190741in, y=1.765303in, left, base]{\color{textcolor}\rmfamily\fontsize{11.000000}{13.200000}\selectfont \(\displaystyle {0.75}\)}%
\end{pgfscope}%
\begin{pgfscope}%
\pgfsetbuttcap%
\pgfsetroundjoin%
\definecolor{currentfill}{rgb}{0.000000,0.000000,0.000000}%
\pgfsetfillcolor{currentfill}%
\pgfsetlinewidth{0.803000pt}%
\definecolor{currentstroke}{rgb}{0.000000,0.000000,0.000000}%
\pgfsetstrokecolor{currentstroke}%
\pgfsetdash{}{0pt}%
\pgfsys@defobject{currentmarker}{\pgfqpoint{-0.048611in}{0.000000in}}{\pgfqpoint{0.000000in}{0.000000in}}{%
\pgfpathmoveto{\pgfqpoint{0.000000in}{0.000000in}}%
\pgfpathlineto{\pgfqpoint{-0.048611in}{0.000000in}}%
\pgfusepath{stroke,fill}%
}%
\begin{pgfscope}%
\pgfsys@transformshift{0.558334in}{2.172927in}%
\pgfsys@useobject{currentmarker}{}%
\end{pgfscope}%
\end{pgfscope}%
\begin{pgfscope}%
\definecolor{textcolor}{rgb}{0.000000,0.000000,0.000000}%
\pgfsetstrokecolor{textcolor}%
\pgfsetfillcolor{textcolor}%
\pgftext[x=0.190741in, y=2.120120in, left, base]{\color{textcolor}\rmfamily\fontsize{11.000000}{13.200000}\selectfont \(\displaystyle {1.00}\)}%
\end{pgfscope}%
\begin{pgfscope}%
\definecolor{textcolor}{rgb}{0.000000,0.000000,0.000000}%
\pgfsetstrokecolor{textcolor}%
\pgfsetfillcolor{textcolor}%
\pgftext[x=0.135185in,y=1.486563in,,bottom,rotate=90.000000]{\color{textcolor}\rmfamily\fontsize{11.000000}{13.200000}\selectfont Share}%
\end{pgfscope}%
\begin{pgfscope}%
\pgfpathrectangle{\pgfqpoint{0.558334in}{0.800199in}}{\pgfqpoint{4.650000in}{1.372727in}}%
\pgfusepath{clip}%
\pgfsetrectcap%
\pgfsetroundjoin%
\pgfsetlinewidth{1.505625pt}%
\definecolor{currentstroke}{rgb}{0.000000,0.500000,0.000000}%
\pgfsetstrokecolor{currentstroke}%
\pgfsetdash{}{0pt}%
\pgfpathmoveto{\pgfqpoint{0.558334in}{1.552324in}}%
\pgfpathlineto{\pgfqpoint{1.333334in}{1.829893in}}%
\pgfpathlineto{\pgfqpoint{2.108334in}{1.968677in}}%
\pgfpathlineto{\pgfqpoint{2.883334in}{2.044616in}}%
\pgfpathlineto{\pgfqpoint{3.658334in}{2.076039in}}%
\pgfpathlineto{\pgfqpoint{4.433334in}{2.096988in}}%
\pgfpathlineto{\pgfqpoint{5.208334in}{2.107462in}}%
\pgfusepath{stroke}%
\end{pgfscope}%
\begin{pgfscope}%
\pgfpathrectangle{\pgfqpoint{0.558334in}{0.800199in}}{\pgfqpoint{4.650000in}{1.372727in}}%
\pgfusepath{clip}%
\pgfsetbuttcap%
\pgfsetroundjoin%
\pgfsetlinewidth{1.505625pt}%
\definecolor{currentstroke}{rgb}{0.000000,0.500000,0.000000}%
\pgfsetstrokecolor{currentstroke}%
\pgfsetdash{{5.550000pt}{2.400000pt}}{0.000000pt}%
\pgfpathmoveto{\pgfqpoint{0.558334in}{1.420896in}}%
\pgfpathlineto{\pgfqpoint{1.333334in}{1.756100in}}%
\pgfpathlineto{\pgfqpoint{2.108334in}{1.913797in}}%
\pgfpathlineto{\pgfqpoint{2.883334in}{2.000702in}}%
\pgfpathlineto{\pgfqpoint{3.658334in}{2.054852in}}%
\pgfpathlineto{\pgfqpoint{4.433334in}{2.090248in}}%
\pgfpathlineto{\pgfqpoint{5.208334in}{2.113493in}}%
\pgfusepath{stroke}%
\end{pgfscope}%
\begin{pgfscope}%
\pgfpathrectangle{\pgfqpoint{0.558334in}{0.800199in}}{\pgfqpoint{4.650000in}{1.372727in}}%
\pgfusepath{clip}%
\pgfsetrectcap%
\pgfsetroundjoin%
\pgfsetlinewidth{1.505625pt}%
\definecolor{currentstroke}{rgb}{1.000000,0.000000,0.000000}%
\pgfsetstrokecolor{currentstroke}%
\pgfsetdash{}{0pt}%
\pgfpathmoveto{\pgfqpoint{0.558334in}{0.897679in}}%
\pgfpathlineto{\pgfqpoint{1.333334in}{0.976236in}}%
\pgfpathlineto{\pgfqpoint{2.108334in}{1.078361in}}%
\pgfpathlineto{\pgfqpoint{2.883334in}{1.180486in}}%
\pgfpathlineto{\pgfqpoint{3.658334in}{1.240713in}}%
\pgfpathlineto{\pgfqpoint{4.433334in}{1.303559in}}%
\pgfpathlineto{\pgfqpoint{5.208334in}{1.334982in}}%
\pgfusepath{stroke}%
\end{pgfscope}%
\begin{pgfscope}%
\pgfpathrectangle{\pgfqpoint{0.558334in}{0.800199in}}{\pgfqpoint{4.650000in}{1.372727in}}%
\pgfusepath{clip}%
\pgfsetbuttcap%
\pgfsetroundjoin%
\pgfsetlinewidth{1.505625pt}%
\definecolor{currentstroke}{rgb}{1.000000,0.000000,0.000000}%
\pgfsetstrokecolor{currentstroke}%
\pgfsetdash{{5.550000pt}{2.400000pt}}{0.000000pt}%
\pgfpathmoveto{\pgfqpoint{0.558334in}{0.875958in}}%
\pgfpathlineto{\pgfqpoint{1.333334in}{0.965768in}}%
\pgfpathlineto{\pgfqpoint{2.108334in}{1.065088in}}%
\pgfpathlineto{\pgfqpoint{2.883334in}{1.167842in}}%
\pgfpathlineto{\pgfqpoint{3.658334in}{1.260030in}}%
\pgfpathlineto{\pgfqpoint{4.433334in}{1.324218in}}%
\pgfpathlineto{\pgfqpoint{5.208334in}{1.369123in}}%
\pgfusepath{stroke}%
\end{pgfscope}%
\begin{pgfscope}%
\pgfsetrectcap%
\pgfsetmiterjoin%
\pgfsetlinewidth{0.803000pt}%
\definecolor{currentstroke}{rgb}{0.000000,0.000000,0.000000}%
\pgfsetstrokecolor{currentstroke}%
\pgfsetdash{}{0pt}%
\pgfpathmoveto{\pgfqpoint{0.558334in}{0.800199in}}%
\pgfpathlineto{\pgfqpoint{0.558334in}{2.172927in}}%
\pgfusepath{stroke}%
\end{pgfscope}%
\begin{pgfscope}%
\pgfsetrectcap%
\pgfsetmiterjoin%
\pgfsetlinewidth{0.803000pt}%
\definecolor{currentstroke}{rgb}{0.000000,0.000000,0.000000}%
\pgfsetstrokecolor{currentstroke}%
\pgfsetdash{}{0pt}%
\pgfpathmoveto{\pgfqpoint{5.208334in}{0.800199in}}%
\pgfpathlineto{\pgfqpoint{5.208334in}{2.172927in}}%
\pgfusepath{stroke}%
\end{pgfscope}%
\begin{pgfscope}%
\pgfsetrectcap%
\pgfsetmiterjoin%
\pgfsetlinewidth{0.803000pt}%
\definecolor{currentstroke}{rgb}{0.000000,0.000000,0.000000}%
\pgfsetstrokecolor{currentstroke}%
\pgfsetdash{}{0pt}%
\pgfpathmoveto{\pgfqpoint{0.558334in}{0.800199in}}%
\pgfpathlineto{\pgfqpoint{5.208334in}{0.800199in}}%
\pgfusepath{stroke}%
\end{pgfscope}%
\begin{pgfscope}%
\pgfsetrectcap%
\pgfsetmiterjoin%
\pgfsetlinewidth{0.803000pt}%
\definecolor{currentstroke}{rgb}{0.000000,0.000000,0.000000}%
\pgfsetstrokecolor{currentstroke}%
\pgfsetdash{}{0pt}%
\pgfpathmoveto{\pgfqpoint{0.558334in}{2.172927in}}%
\pgfpathlineto{\pgfqpoint{5.208334in}{2.172927in}}%
\pgfusepath{stroke}%
\end{pgfscope}%
\begin{pgfscope}%
\pgfsetbuttcap%
\pgfsetmiterjoin%
\definecolor{currentfill}{rgb}{0.300000,0.300000,0.300000}%
\pgfsetfillcolor{currentfill}%
\pgfsetfillopacity{0.500000}%
\pgfsetlinewidth{1.003750pt}%
\definecolor{currentstroke}{rgb}{0.300000,0.300000,0.300000}%
\pgfsetstrokecolor{currentstroke}%
\pgfsetstrokeopacity{0.500000}%
\pgfsetdash{}{0pt}%
\pgfpathmoveto{\pgfqpoint{1.788514in}{-0.027778in}}%
\pgfpathlineto{\pgfqpoint{4.033709in}{-0.027778in}}%
\pgfpathquadraticcurveto{\pgfqpoint{4.054914in}{-0.027778in}}{\pgfqpoint{4.054914in}{-0.006572in}}%
\pgfpathlineto{\pgfqpoint{4.054914in}{0.286384in}}%
\pgfpathquadraticcurveto{\pgfqpoint{4.054914in}{0.307589in}}{\pgfqpoint{4.033709in}{0.307589in}}%
\pgfpathlineto{\pgfqpoint{1.788514in}{0.307589in}}%
\pgfpathquadraticcurveto{\pgfqpoint{1.767309in}{0.307589in}}{\pgfqpoint{1.767309in}{0.286384in}}%
\pgfpathlineto{\pgfqpoint{1.767309in}{-0.006572in}}%
\pgfpathquadraticcurveto{\pgfqpoint{1.767309in}{-0.027778in}}{\pgfqpoint{1.788514in}{-0.027778in}}%
\pgfpathclose%
\pgfusepath{stroke,fill}%
\end{pgfscope}%
\begin{pgfscope}%
\pgfsetbuttcap%
\pgfsetmiterjoin%
\definecolor{currentfill}{rgb}{1.000000,1.000000,1.000000}%
\pgfsetfillcolor{currentfill}%
\pgfsetlinewidth{1.003750pt}%
\definecolor{currentstroke}{rgb}{0.800000,0.800000,0.800000}%
\pgfsetstrokecolor{currentstroke}%
\pgfsetdash{}{0pt}%
\pgfpathmoveto{\pgfqpoint{1.760736in}{0.000000in}}%
\pgfpathlineto{\pgfqpoint{4.005931in}{0.000000in}}%
\pgfpathquadraticcurveto{\pgfqpoint{4.027136in}{0.000000in}}{\pgfqpoint{4.027136in}{0.021206in}}%
\pgfpathlineto{\pgfqpoint{4.027136in}{0.314162in}}%
\pgfpathquadraticcurveto{\pgfqpoint{4.027136in}{0.335367in}}{\pgfqpoint{4.005931in}{0.335367in}}%
\pgfpathlineto{\pgfqpoint{1.760736in}{0.335367in}}%
\pgfpathquadraticcurveto{\pgfqpoint{1.739531in}{0.335367in}}{\pgfqpoint{1.739531in}{0.314162in}}%
\pgfpathlineto{\pgfqpoint{1.739531in}{0.021206in}}%
\pgfpathquadraticcurveto{\pgfqpoint{1.739531in}{0.000000in}}{\pgfqpoint{1.760736in}{0.000000in}}%
\pgfpathclose%
\pgfusepath{stroke,fill}%
\end{pgfscope}%
\begin{pgfscope}%
\pgfsetrectcap%
\pgfsetroundjoin%
\pgfsetlinewidth{1.505625pt}%
\definecolor{currentstroke}{rgb}{0.000000,0.500000,0.000000}%
\pgfsetstrokecolor{currentstroke}%
\pgfsetdash{}{0pt}%
\pgfpathmoveto{\pgfqpoint{1.781942in}{0.252905in}}%
\pgfpathlineto{\pgfqpoint{1.993998in}{0.252905in}}%
\pgfusepath{stroke}%
\end{pgfscope}%
\begin{pgfscope}%
\definecolor{textcolor}{rgb}{0.000000,0.000000,0.000000}%
\pgfsetstrokecolor{textcolor}%
\pgfsetfillcolor{textcolor}%
\pgftext[x=2.078820in,y=0.215796in,left,base]{\color{textcolor}\rmfamily\fontsize{7.634000}{9.160800}\selectfont Married - D}%
\end{pgfscope}%
\begin{pgfscope}%
\pgfsetbuttcap%
\pgfsetroundjoin%
\pgfsetlinewidth{1.505625pt}%
\definecolor{currentstroke}{rgb}{0.000000,0.500000,0.000000}%
\pgfsetstrokecolor{currentstroke}%
\pgfsetdash{{5.550000pt}{2.400000pt}}{0.000000pt}%
\pgfpathmoveto{\pgfqpoint{1.781942in}{0.101126in}}%
\pgfpathlineto{\pgfqpoint{1.993998in}{0.101126in}}%
\pgfusepath{stroke}%
\end{pgfscope}%
\begin{pgfscope}%
\definecolor{textcolor}{rgb}{0.000000,0.000000,0.000000}%
\pgfsetstrokecolor{textcolor}%
\pgfsetfillcolor{textcolor}%
\pgftext[x=2.078820in,y=0.064016in,left,base]{\color{textcolor}\rmfamily\fontsize{7.634000}{9.160800}\selectfont Married - S}%
\end{pgfscope}%
\begin{pgfscope}%
\pgfsetrectcap%
\pgfsetroundjoin%
\pgfsetlinewidth{1.505625pt}%
\definecolor{currentstroke}{rgb}{1.000000,0.000000,0.000000}%
\pgfsetstrokecolor{currentstroke}%
\pgfsetdash{}{0pt}%
\pgfpathmoveto{\pgfqpoint{2.909055in}{0.252905in}}%
\pgfpathlineto{\pgfqpoint{3.121111in}{0.252905in}}%
\pgfusepath{stroke}%
\end{pgfscope}%
\begin{pgfscope}%
\definecolor{textcolor}{rgb}{0.000000,0.000000,0.000000}%
\pgfsetstrokecolor{textcolor}%
\pgfsetfillcolor{textcolor}%
\pgftext[x=3.205933in,y=0.215796in,left,base]{\color{textcolor}\rmfamily\fontsize{7.634000}{9.160800}\selectfont Cohabiting - D}%
\end{pgfscope}%
\begin{pgfscope}%
\pgfsetbuttcap%
\pgfsetroundjoin%
\pgfsetlinewidth{1.505625pt}%
\definecolor{currentstroke}{rgb}{1.000000,0.000000,0.000000}%
\pgfsetstrokecolor{currentstroke}%
\pgfsetdash{{5.550000pt}{2.400000pt}}{0.000000pt}%
\pgfpathmoveto{\pgfqpoint{2.909055in}{0.101126in}}%
\pgfpathlineto{\pgfqpoint{3.121111in}{0.101126in}}%
\pgfusepath{stroke}%
\end{pgfscope}%
\begin{pgfscope}%
\definecolor{textcolor}{rgb}{0.000000,0.000000,0.000000}%
\pgfsetstrokecolor{textcolor}%
\pgfsetfillcolor{textcolor}%
\pgftext[x=3.205933in,y=0.064016in,left,base]{\color{textcolor}\rmfamily\fontsize{7.634000}{9.160800}\selectfont Cohabiting - S}%
\end{pgfscope}%
\end{pgfpicture}%
\makeatother%
\endgroup%
} 
%\end{subfigure}
\end{center}
\end{figure}


%%%%%%%%%%%%%%%%%%%%%%%%%%%%%%%%%%%%%%
%Symmetry in income and assets
%%%%%%%%%%%%%%%%%%%%%%%%%%%%%%%%%%%%%%
\begin{figure}[ht]
	\begin{center}
		\caption{---Low wages over the life cycle: simulations and data}
		\label{fig:datasimwage}
		
		\begin{subfigure}{.49\textwidth}
			\centering
			% include first image
			\caption{Women}
			\label{fig:em}
			\scalebox{0.5}{%% Creator: Matplotlib, PGF backend
%%
%% To include the figure in your LaTeX document, write
%%   \input{<filename>.pgf}
%%
%% Make sure the required packages are loaded in your preamble
%%   \usepackage{pgf}
%%
%% and, on pdftex
%%   \usepackage[utf8]{inputenc}\DeclareUnicodeCharacter{2212}{-}
%%
%% or, on luatex and xetex
%%   \usepackage{unicode-math}
%%
%% Figures using additional raster images can only be included by \input if
%% they are in the same directory as the main LaTeX file. For loading figures
%% from other directories you can use the `import` package
%%   \usepackage{import}
%%
%% and then include the figures with
%%   \import{<path to file>}{<filename>.pgf}
%%
%% Matplotlib used the following preamble
%%
\begingroup%
\makeatletter%
\begin{pgfpicture}%
\pgfpathrectangle{\pgfpointorigin}{\pgfqpoint{5.328743in}{3.903555in}}%
\pgfusepath{use as bounding box, clip}%
\begin{pgfscope}%
\pgfsetbuttcap%
\pgfsetmiterjoin%
\definecolor{currentfill}{rgb}{1.000000,1.000000,1.000000}%
\pgfsetfillcolor{currentfill}%
\pgfsetlinewidth{0.000000pt}%
\definecolor{currentstroke}{rgb}{1.000000,1.000000,1.000000}%
\pgfsetstrokecolor{currentstroke}%
\pgfsetdash{}{0pt}%
\pgfpathmoveto{\pgfqpoint{0.000000in}{0.000000in}}%
\pgfpathlineto{\pgfqpoint{5.328743in}{0.000000in}}%
\pgfpathlineto{\pgfqpoint{5.328743in}{3.903555in}}%
\pgfpathlineto{\pgfqpoint{0.000000in}{3.903555in}}%
\pgfpathclose%
\pgfusepath{fill}%
\end{pgfscope}%
\begin{pgfscope}%
\pgfsetbuttcap%
\pgfsetmiterjoin%
\definecolor{currentfill}{rgb}{1.000000,1.000000,1.000000}%
\pgfsetfillcolor{currentfill}%
\pgfsetlinewidth{0.000000pt}%
\definecolor{currentstroke}{rgb}{0.000000,0.000000,0.000000}%
\pgfsetstrokecolor{currentstroke}%
\pgfsetstrokeopacity{0.000000}%
\pgfsetdash{}{0pt}%
\pgfpathmoveto{\pgfqpoint{0.678743in}{0.883555in}}%
\pgfpathlineto{\pgfqpoint{5.328743in}{0.883555in}}%
\pgfpathlineto{\pgfqpoint{5.328743in}{3.903555in}}%
\pgfpathlineto{\pgfqpoint{0.678743in}{3.903555in}}%
\pgfpathclose%
\pgfusepath{fill}%
\end{pgfscope}%
\begin{pgfscope}%
\pgfpathrectangle{\pgfqpoint{0.678743in}{0.883555in}}{\pgfqpoint{4.650000in}{3.020000in}}%
\pgfusepath{clip}%
\pgfsetbuttcap%
\pgfsetroundjoin%
\pgfsetlinewidth{1.003750pt}%
\definecolor{currentstroke}{rgb}{0.000000,0.000000,1.000000}%
\pgfsetstrokecolor{currentstroke}%
\pgfsetdash{}{0pt}%
\pgfpathmoveto{\pgfqpoint{0.890106in}{1.519223in}}%
\pgfpathcurveto{\pgfqpoint{0.904372in}{1.519223in}}{\pgfqpoint{0.918055in}{1.524891in}}{\pgfqpoint{0.928143in}{1.534978in}}%
\pgfpathcurveto{\pgfqpoint{0.938230in}{1.545066in}}{\pgfqpoint{0.943898in}{1.558749in}}{\pgfqpoint{0.943898in}{1.573015in}}%
\pgfpathcurveto{\pgfqpoint{0.943898in}{1.587280in}}{\pgfqpoint{0.938230in}{1.600964in}}{\pgfqpoint{0.928143in}{1.611051in}}%
\pgfpathcurveto{\pgfqpoint{0.918055in}{1.621138in}}{\pgfqpoint{0.904372in}{1.626806in}}{\pgfqpoint{0.890106in}{1.626806in}}%
\pgfpathcurveto{\pgfqpoint{0.875841in}{1.626806in}}{\pgfqpoint{0.862157in}{1.621138in}}{\pgfqpoint{0.852070in}{1.611051in}}%
\pgfpathcurveto{\pgfqpoint{0.841983in}{1.600964in}}{\pgfqpoint{0.836315in}{1.587280in}}{\pgfqpoint{0.836315in}{1.573015in}}%
\pgfpathcurveto{\pgfqpoint{0.836315in}{1.558749in}}{\pgfqpoint{0.841983in}{1.545066in}}{\pgfqpoint{0.852070in}{1.534978in}}%
\pgfpathcurveto{\pgfqpoint{0.862157in}{1.524891in}}{\pgfqpoint{0.875841in}{1.519223in}}{\pgfqpoint{0.890106in}{1.519223in}}%
\pgfpathclose%
\pgfusepath{stroke}%
\end{pgfscope}%
\begin{pgfscope}%
\pgfpathrectangle{\pgfqpoint{0.678743in}{0.883555in}}{\pgfqpoint{4.650000in}{3.020000in}}%
\pgfusepath{clip}%
\pgfsetbuttcap%
\pgfsetroundjoin%
\pgfsetlinewidth{1.003750pt}%
\definecolor{currentstroke}{rgb}{0.000000,0.000000,1.000000}%
\pgfsetstrokecolor{currentstroke}%
\pgfsetdash{}{0pt}%
\pgfpathmoveto{\pgfqpoint{0.995788in}{1.660154in}}%
\pgfpathcurveto{\pgfqpoint{1.010054in}{1.660154in}}{\pgfqpoint{1.023737in}{1.665822in}}{\pgfqpoint{1.033825in}{1.675909in}}%
\pgfpathcurveto{\pgfqpoint{1.043912in}{1.685996in}}{\pgfqpoint{1.049580in}{1.699680in}}{\pgfqpoint{1.049580in}{1.713945in}}%
\pgfpathcurveto{\pgfqpoint{1.049580in}{1.728211in}}{\pgfqpoint{1.043912in}{1.741894in}}{\pgfqpoint{1.033825in}{1.751982in}}%
\pgfpathcurveto{\pgfqpoint{1.023737in}{1.762069in}}{\pgfqpoint{1.010054in}{1.767737in}}{\pgfqpoint{0.995788in}{1.767737in}}%
\pgfpathcurveto{\pgfqpoint{0.981523in}{1.767737in}}{\pgfqpoint{0.967839in}{1.762069in}}{\pgfqpoint{0.957752in}{1.751982in}}%
\pgfpathcurveto{\pgfqpoint{0.947665in}{1.741894in}}{\pgfqpoint{0.941997in}{1.728211in}}{\pgfqpoint{0.941997in}{1.713945in}}%
\pgfpathcurveto{\pgfqpoint{0.941997in}{1.699680in}}{\pgfqpoint{0.947665in}{1.685996in}}{\pgfqpoint{0.957752in}{1.675909in}}%
\pgfpathcurveto{\pgfqpoint{0.967839in}{1.665822in}}{\pgfqpoint{0.981523in}{1.660154in}}{\pgfqpoint{0.995788in}{1.660154in}}%
\pgfpathclose%
\pgfusepath{stroke}%
\end{pgfscope}%
\begin{pgfscope}%
\pgfpathrectangle{\pgfqpoint{0.678743in}{0.883555in}}{\pgfqpoint{4.650000in}{3.020000in}}%
\pgfusepath{clip}%
\pgfsetbuttcap%
\pgfsetroundjoin%
\pgfsetlinewidth{1.003750pt}%
\definecolor{currentstroke}{rgb}{0.000000,0.000000,1.000000}%
\pgfsetstrokecolor{currentstroke}%
\pgfsetdash{}{0pt}%
\pgfpathmoveto{\pgfqpoint{1.101470in}{1.732567in}}%
\pgfpathcurveto{\pgfqpoint{1.115736in}{1.732567in}}{\pgfqpoint{1.129419in}{1.738235in}}{\pgfqpoint{1.139506in}{1.748322in}}%
\pgfpathcurveto{\pgfqpoint{1.149594in}{1.758410in}}{\pgfqpoint{1.155261in}{1.772093in}}{\pgfqpoint{1.155261in}{1.786358in}}%
\pgfpathcurveto{\pgfqpoint{1.155261in}{1.800624in}}{\pgfqpoint{1.149594in}{1.814307in}}{\pgfqpoint{1.139506in}{1.824395in}}%
\pgfpathcurveto{\pgfqpoint{1.129419in}{1.834482in}}{\pgfqpoint{1.115736in}{1.840150in}}{\pgfqpoint{1.101470in}{1.840150in}}%
\pgfpathcurveto{\pgfqpoint{1.087204in}{1.840150in}}{\pgfqpoint{1.073521in}{1.834482in}}{\pgfqpoint{1.063434in}{1.824395in}}%
\pgfpathcurveto{\pgfqpoint{1.053346in}{1.814307in}}{\pgfqpoint{1.047679in}{1.800624in}}{\pgfqpoint{1.047679in}{1.786358in}}%
\pgfpathcurveto{\pgfqpoint{1.047679in}{1.772093in}}{\pgfqpoint{1.053346in}{1.758410in}}{\pgfqpoint{1.063434in}{1.748322in}}%
\pgfpathcurveto{\pgfqpoint{1.073521in}{1.738235in}}{\pgfqpoint{1.087204in}{1.732567in}}{\pgfqpoint{1.101470in}{1.732567in}}%
\pgfpathclose%
\pgfusepath{stroke}%
\end{pgfscope}%
\begin{pgfscope}%
\pgfpathrectangle{\pgfqpoint{0.678743in}{0.883555in}}{\pgfqpoint{4.650000in}{3.020000in}}%
\pgfusepath{clip}%
\pgfsetbuttcap%
\pgfsetroundjoin%
\pgfsetlinewidth{1.003750pt}%
\definecolor{currentstroke}{rgb}{0.000000,0.000000,1.000000}%
\pgfsetstrokecolor{currentstroke}%
\pgfsetdash{}{0pt}%
\pgfpathmoveto{\pgfqpoint{1.207152in}{1.851616in}}%
\pgfpathcurveto{\pgfqpoint{1.221418in}{1.851616in}}{\pgfqpoint{1.235101in}{1.857284in}}{\pgfqpoint{1.245188in}{1.867371in}}%
\pgfpathcurveto{\pgfqpoint{1.255275in}{1.877459in}}{\pgfqpoint{1.260943in}{1.891142in}}{\pgfqpoint{1.260943in}{1.905408in}}%
\pgfpathcurveto{\pgfqpoint{1.260943in}{1.919673in}}{\pgfqpoint{1.255275in}{1.933357in}}{\pgfqpoint{1.245188in}{1.943444in}}%
\pgfpathcurveto{\pgfqpoint{1.235101in}{1.953531in}}{\pgfqpoint{1.221418in}{1.959199in}}{\pgfqpoint{1.207152in}{1.959199in}}%
\pgfpathcurveto{\pgfqpoint{1.192886in}{1.959199in}}{\pgfqpoint{1.179203in}{1.953531in}}{\pgfqpoint{1.169116in}{1.943444in}}%
\pgfpathcurveto{\pgfqpoint{1.159028in}{1.933357in}}{\pgfqpoint{1.153360in}{1.919673in}}{\pgfqpoint{1.153360in}{1.905408in}}%
\pgfpathcurveto{\pgfqpoint{1.153360in}{1.891142in}}{\pgfqpoint{1.159028in}{1.877459in}}{\pgfqpoint{1.169116in}{1.867371in}}%
\pgfpathcurveto{\pgfqpoint{1.179203in}{1.857284in}}{\pgfqpoint{1.192886in}{1.851616in}}{\pgfqpoint{1.207152in}{1.851616in}}%
\pgfpathclose%
\pgfusepath{stroke}%
\end{pgfscope}%
\begin{pgfscope}%
\pgfpathrectangle{\pgfqpoint{0.678743in}{0.883555in}}{\pgfqpoint{4.650000in}{3.020000in}}%
\pgfusepath{clip}%
\pgfsetbuttcap%
\pgfsetroundjoin%
\pgfsetlinewidth{1.003750pt}%
\definecolor{currentstroke}{rgb}{0.000000,0.000000,1.000000}%
\pgfsetstrokecolor{currentstroke}%
\pgfsetdash{}{0pt}%
\pgfpathmoveto{\pgfqpoint{1.312834in}{1.926874in}}%
\pgfpathcurveto{\pgfqpoint{1.327099in}{1.926874in}}{\pgfqpoint{1.340783in}{1.932542in}}{\pgfqpoint{1.350870in}{1.942629in}}%
\pgfpathcurveto{\pgfqpoint{1.360957in}{1.952716in}}{\pgfqpoint{1.366625in}{1.966400in}}{\pgfqpoint{1.366625in}{1.980665in}}%
\pgfpathcurveto{\pgfqpoint{1.366625in}{1.994931in}}{\pgfqpoint{1.360957in}{2.008614in}}{\pgfqpoint{1.350870in}{2.018701in}}%
\pgfpathcurveto{\pgfqpoint{1.340783in}{2.028789in}}{\pgfqpoint{1.327099in}{2.034457in}}{\pgfqpoint{1.312834in}{2.034457in}}%
\pgfpathcurveto{\pgfqpoint{1.298568in}{2.034457in}}{\pgfqpoint{1.284885in}{2.028789in}}{\pgfqpoint{1.274797in}{2.018701in}}%
\pgfpathcurveto{\pgfqpoint{1.264710in}{2.008614in}}{\pgfqpoint{1.259042in}{1.994931in}}{\pgfqpoint{1.259042in}{1.980665in}}%
\pgfpathcurveto{\pgfqpoint{1.259042in}{1.966400in}}{\pgfqpoint{1.264710in}{1.952716in}}{\pgfqpoint{1.274797in}{1.942629in}}%
\pgfpathcurveto{\pgfqpoint{1.284885in}{1.932542in}}{\pgfqpoint{1.298568in}{1.926874in}}{\pgfqpoint{1.312834in}{1.926874in}}%
\pgfpathclose%
\pgfusepath{stroke}%
\end{pgfscope}%
\begin{pgfscope}%
\pgfpathrectangle{\pgfqpoint{0.678743in}{0.883555in}}{\pgfqpoint{4.650000in}{3.020000in}}%
\pgfusepath{clip}%
\pgfsetbuttcap%
\pgfsetroundjoin%
\pgfsetlinewidth{1.003750pt}%
\definecolor{currentstroke}{rgb}{0.000000,0.000000,1.000000}%
\pgfsetstrokecolor{currentstroke}%
\pgfsetdash{}{0pt}%
\pgfpathmoveto{\pgfqpoint{1.418516in}{1.995994in}}%
\pgfpathcurveto{\pgfqpoint{1.432781in}{1.995994in}}{\pgfqpoint{1.446464in}{2.001662in}}{\pgfqpoint{1.456552in}{2.011749in}}%
\pgfpathcurveto{\pgfqpoint{1.466639in}{2.021836in}}{\pgfqpoint{1.472307in}{2.035520in}}{\pgfqpoint{1.472307in}{2.049785in}}%
\pgfpathcurveto{\pgfqpoint{1.472307in}{2.064051in}}{\pgfqpoint{1.466639in}{2.077734in}}{\pgfqpoint{1.456552in}{2.087822in}}%
\pgfpathcurveto{\pgfqpoint{1.446464in}{2.097909in}}{\pgfqpoint{1.432781in}{2.103577in}}{\pgfqpoint{1.418516in}{2.103577in}}%
\pgfpathcurveto{\pgfqpoint{1.404250in}{2.103577in}}{\pgfqpoint{1.390567in}{2.097909in}}{\pgfqpoint{1.380479in}{2.087822in}}%
\pgfpathcurveto{\pgfqpoint{1.370392in}{2.077734in}}{\pgfqpoint{1.364724in}{2.064051in}}{\pgfqpoint{1.364724in}{2.049785in}}%
\pgfpathcurveto{\pgfqpoint{1.364724in}{2.035520in}}{\pgfqpoint{1.370392in}{2.021836in}}{\pgfqpoint{1.380479in}{2.011749in}}%
\pgfpathcurveto{\pgfqpoint{1.390567in}{2.001662in}}{\pgfqpoint{1.404250in}{1.995994in}}{\pgfqpoint{1.418516in}{1.995994in}}%
\pgfpathclose%
\pgfusepath{stroke}%
\end{pgfscope}%
\begin{pgfscope}%
\pgfpathrectangle{\pgfqpoint{0.678743in}{0.883555in}}{\pgfqpoint{4.650000in}{3.020000in}}%
\pgfusepath{clip}%
\pgfsetbuttcap%
\pgfsetroundjoin%
\pgfsetlinewidth{1.003750pt}%
\definecolor{currentstroke}{rgb}{0.000000,0.000000,1.000000}%
\pgfsetstrokecolor{currentstroke}%
\pgfsetdash{}{0pt}%
\pgfpathmoveto{\pgfqpoint{1.524197in}{2.041330in}}%
\pgfpathcurveto{\pgfqpoint{1.538463in}{2.041330in}}{\pgfqpoint{1.552146in}{2.046998in}}{\pgfqpoint{1.562234in}{2.057085in}}%
\pgfpathcurveto{\pgfqpoint{1.572321in}{2.067172in}}{\pgfqpoint{1.577989in}{2.080856in}}{\pgfqpoint{1.577989in}{2.095121in}}%
\pgfpathcurveto{\pgfqpoint{1.577989in}{2.109387in}}{\pgfqpoint{1.572321in}{2.123070in}}{\pgfqpoint{1.562234in}{2.133158in}}%
\pgfpathcurveto{\pgfqpoint{1.552146in}{2.143245in}}{\pgfqpoint{1.538463in}{2.148913in}}{\pgfqpoint{1.524197in}{2.148913in}}%
\pgfpathcurveto{\pgfqpoint{1.509932in}{2.148913in}}{\pgfqpoint{1.496248in}{2.143245in}}{\pgfqpoint{1.486161in}{2.133158in}}%
\pgfpathcurveto{\pgfqpoint{1.476074in}{2.123070in}}{\pgfqpoint{1.470406in}{2.109387in}}{\pgfqpoint{1.470406in}{2.095121in}}%
\pgfpathcurveto{\pgfqpoint{1.470406in}{2.080856in}}{\pgfqpoint{1.476074in}{2.067172in}}{\pgfqpoint{1.486161in}{2.057085in}}%
\pgfpathcurveto{\pgfqpoint{1.496248in}{2.046998in}}{\pgfqpoint{1.509932in}{2.041330in}}{\pgfqpoint{1.524197in}{2.041330in}}%
\pgfpathclose%
\pgfusepath{stroke}%
\end{pgfscope}%
\begin{pgfscope}%
\pgfpathrectangle{\pgfqpoint{0.678743in}{0.883555in}}{\pgfqpoint{4.650000in}{3.020000in}}%
\pgfusepath{clip}%
\pgfsetbuttcap%
\pgfsetroundjoin%
\pgfsetlinewidth{1.003750pt}%
\definecolor{currentstroke}{rgb}{0.000000,0.000000,1.000000}%
\pgfsetstrokecolor{currentstroke}%
\pgfsetdash{}{0pt}%
\pgfpathmoveto{\pgfqpoint{1.629879in}{2.121261in}}%
\pgfpathcurveto{\pgfqpoint{1.644145in}{2.121261in}}{\pgfqpoint{1.657828in}{2.126929in}}{\pgfqpoint{1.667915in}{2.137016in}}%
\pgfpathcurveto{\pgfqpoint{1.678003in}{2.147104in}}{\pgfqpoint{1.683671in}{2.160787in}}{\pgfqpoint{1.683671in}{2.175053in}}%
\pgfpathcurveto{\pgfqpoint{1.683671in}{2.189318in}}{\pgfqpoint{1.678003in}{2.203002in}}{\pgfqpoint{1.667915in}{2.213089in}}%
\pgfpathcurveto{\pgfqpoint{1.657828in}{2.223176in}}{\pgfqpoint{1.644145in}{2.228844in}}{\pgfqpoint{1.629879in}{2.228844in}}%
\pgfpathcurveto{\pgfqpoint{1.615613in}{2.228844in}}{\pgfqpoint{1.601930in}{2.223176in}}{\pgfqpoint{1.591843in}{2.213089in}}%
\pgfpathcurveto{\pgfqpoint{1.581756in}{2.203002in}}{\pgfqpoint{1.576088in}{2.189318in}}{\pgfqpoint{1.576088in}{2.175053in}}%
\pgfpathcurveto{\pgfqpoint{1.576088in}{2.160787in}}{\pgfqpoint{1.581756in}{2.147104in}}{\pgfqpoint{1.591843in}{2.137016in}}%
\pgfpathcurveto{\pgfqpoint{1.601930in}{2.126929in}}{\pgfqpoint{1.615613in}{2.121261in}}{\pgfqpoint{1.629879in}{2.121261in}}%
\pgfpathclose%
\pgfusepath{stroke}%
\end{pgfscope}%
\begin{pgfscope}%
\pgfpathrectangle{\pgfqpoint{0.678743in}{0.883555in}}{\pgfqpoint{4.650000in}{3.020000in}}%
\pgfusepath{clip}%
\pgfsetbuttcap%
\pgfsetroundjoin%
\pgfsetlinewidth{1.003750pt}%
\definecolor{currentstroke}{rgb}{0.000000,0.000000,1.000000}%
\pgfsetstrokecolor{currentstroke}%
\pgfsetdash{}{0pt}%
\pgfpathmoveto{\pgfqpoint{1.735561in}{2.148397in}}%
\pgfpathcurveto{\pgfqpoint{1.749827in}{2.148397in}}{\pgfqpoint{1.763510in}{2.154064in}}{\pgfqpoint{1.773597in}{2.164152in}}%
\pgfpathcurveto{\pgfqpoint{1.783685in}{2.174239in}}{\pgfqpoint{1.789352in}{2.187922in}}{\pgfqpoint{1.789352in}{2.202188in}}%
\pgfpathcurveto{\pgfqpoint{1.789352in}{2.216454in}}{\pgfqpoint{1.783685in}{2.230137in}}{\pgfqpoint{1.773597in}{2.240224in}}%
\pgfpathcurveto{\pgfqpoint{1.763510in}{2.250312in}}{\pgfqpoint{1.749827in}{2.255979in}}{\pgfqpoint{1.735561in}{2.255979in}}%
\pgfpathcurveto{\pgfqpoint{1.721295in}{2.255979in}}{\pgfqpoint{1.707612in}{2.250312in}}{\pgfqpoint{1.697525in}{2.240224in}}%
\pgfpathcurveto{\pgfqpoint{1.687437in}{2.230137in}}{\pgfqpoint{1.681770in}{2.216454in}}{\pgfqpoint{1.681770in}{2.202188in}}%
\pgfpathcurveto{\pgfqpoint{1.681770in}{2.187922in}}{\pgfqpoint{1.687437in}{2.174239in}}{\pgfqpoint{1.697525in}{2.164152in}}%
\pgfpathcurveto{\pgfqpoint{1.707612in}{2.154064in}}{\pgfqpoint{1.721295in}{2.148397in}}{\pgfqpoint{1.735561in}{2.148397in}}%
\pgfpathclose%
\pgfusepath{stroke}%
\end{pgfscope}%
\begin{pgfscope}%
\pgfpathrectangle{\pgfqpoint{0.678743in}{0.883555in}}{\pgfqpoint{4.650000in}{3.020000in}}%
\pgfusepath{clip}%
\pgfsetbuttcap%
\pgfsetroundjoin%
\pgfsetlinewidth{1.003750pt}%
\definecolor{currentstroke}{rgb}{0.000000,0.000000,1.000000}%
\pgfsetstrokecolor{currentstroke}%
\pgfsetdash{}{0pt}%
\pgfpathmoveto{\pgfqpoint{1.841243in}{2.148710in}}%
\pgfpathcurveto{\pgfqpoint{1.855508in}{2.148710in}}{\pgfqpoint{1.869192in}{2.154378in}}{\pgfqpoint{1.879279in}{2.164465in}}%
\pgfpathcurveto{\pgfqpoint{1.889366in}{2.174552in}}{\pgfqpoint{1.895034in}{2.188236in}}{\pgfqpoint{1.895034in}{2.202501in}}%
\pgfpathcurveto{\pgfqpoint{1.895034in}{2.216767in}}{\pgfqpoint{1.889366in}{2.230450in}}{\pgfqpoint{1.879279in}{2.240538in}}%
\pgfpathcurveto{\pgfqpoint{1.869192in}{2.250625in}}{\pgfqpoint{1.855508in}{2.256293in}}{\pgfqpoint{1.841243in}{2.256293in}}%
\pgfpathcurveto{\pgfqpoint{1.826977in}{2.256293in}}{\pgfqpoint{1.813294in}{2.250625in}}{\pgfqpoint{1.803206in}{2.240538in}}%
\pgfpathcurveto{\pgfqpoint{1.793119in}{2.230450in}}{\pgfqpoint{1.787451in}{2.216767in}}{\pgfqpoint{1.787451in}{2.202501in}}%
\pgfpathcurveto{\pgfqpoint{1.787451in}{2.188236in}}{\pgfqpoint{1.793119in}{2.174552in}}{\pgfqpoint{1.803206in}{2.164465in}}%
\pgfpathcurveto{\pgfqpoint{1.813294in}{2.154378in}}{\pgfqpoint{1.826977in}{2.148710in}}{\pgfqpoint{1.841243in}{2.148710in}}%
\pgfpathclose%
\pgfusepath{stroke}%
\end{pgfscope}%
\begin{pgfscope}%
\pgfpathrectangle{\pgfqpoint{0.678743in}{0.883555in}}{\pgfqpoint{4.650000in}{3.020000in}}%
\pgfusepath{clip}%
\pgfsetbuttcap%
\pgfsetroundjoin%
\pgfsetlinewidth{1.003750pt}%
\definecolor{currentstroke}{rgb}{0.000000,0.000000,1.000000}%
\pgfsetstrokecolor{currentstroke}%
\pgfsetdash{}{0pt}%
\pgfpathmoveto{\pgfqpoint{1.946925in}{2.200333in}}%
\pgfpathcurveto{\pgfqpoint{1.961190in}{2.200333in}}{\pgfqpoint{1.974874in}{2.206001in}}{\pgfqpoint{1.984961in}{2.216088in}}%
\pgfpathcurveto{\pgfqpoint{1.995048in}{2.226176in}}{\pgfqpoint{2.000716in}{2.239859in}}{\pgfqpoint{2.000716in}{2.254125in}}%
\pgfpathcurveto{\pgfqpoint{2.000716in}{2.268390in}}{\pgfqpoint{1.995048in}{2.282074in}}{\pgfqpoint{1.984961in}{2.292161in}}%
\pgfpathcurveto{\pgfqpoint{1.974874in}{2.302248in}}{\pgfqpoint{1.961190in}{2.307916in}}{\pgfqpoint{1.946925in}{2.307916in}}%
\pgfpathcurveto{\pgfqpoint{1.932659in}{2.307916in}}{\pgfqpoint{1.918976in}{2.302248in}}{\pgfqpoint{1.908888in}{2.292161in}}%
\pgfpathcurveto{\pgfqpoint{1.898801in}{2.282074in}}{\pgfqpoint{1.893133in}{2.268390in}}{\pgfqpoint{1.893133in}{2.254125in}}%
\pgfpathcurveto{\pgfqpoint{1.893133in}{2.239859in}}{\pgfqpoint{1.898801in}{2.226176in}}{\pgfqpoint{1.908888in}{2.216088in}}%
\pgfpathcurveto{\pgfqpoint{1.918976in}{2.206001in}}{\pgfqpoint{1.932659in}{2.200333in}}{\pgfqpoint{1.946925in}{2.200333in}}%
\pgfpathclose%
\pgfusepath{stroke}%
\end{pgfscope}%
\begin{pgfscope}%
\pgfpathrectangle{\pgfqpoint{0.678743in}{0.883555in}}{\pgfqpoint{4.650000in}{3.020000in}}%
\pgfusepath{clip}%
\pgfsetbuttcap%
\pgfsetroundjoin%
\pgfsetlinewidth{1.003750pt}%
\definecolor{currentstroke}{rgb}{0.000000,0.000000,1.000000}%
\pgfsetstrokecolor{currentstroke}%
\pgfsetdash{}{0pt}%
\pgfpathmoveto{\pgfqpoint{2.052606in}{2.224673in}}%
\pgfpathcurveto{\pgfqpoint{2.066872in}{2.224673in}}{\pgfqpoint{2.080555in}{2.230341in}}{\pgfqpoint{2.090643in}{2.240428in}}%
\pgfpathcurveto{\pgfqpoint{2.100730in}{2.250516in}}{\pgfqpoint{2.106398in}{2.264199in}}{\pgfqpoint{2.106398in}{2.278465in}}%
\pgfpathcurveto{\pgfqpoint{2.106398in}{2.292730in}}{\pgfqpoint{2.100730in}{2.306414in}}{\pgfqpoint{2.090643in}{2.316501in}}%
\pgfpathcurveto{\pgfqpoint{2.080555in}{2.326588in}}{\pgfqpoint{2.066872in}{2.332256in}}{\pgfqpoint{2.052606in}{2.332256in}}%
\pgfpathcurveto{\pgfqpoint{2.038341in}{2.332256in}}{\pgfqpoint{2.024657in}{2.326588in}}{\pgfqpoint{2.014570in}{2.316501in}}%
\pgfpathcurveto{\pgfqpoint{2.004483in}{2.306414in}}{\pgfqpoint{1.998815in}{2.292730in}}{\pgfqpoint{1.998815in}{2.278465in}}%
\pgfpathcurveto{\pgfqpoint{1.998815in}{2.264199in}}{\pgfqpoint{2.004483in}{2.250516in}}{\pgfqpoint{2.014570in}{2.240428in}}%
\pgfpathcurveto{\pgfqpoint{2.024657in}{2.230341in}}{\pgfqpoint{2.038341in}{2.224673in}}{\pgfqpoint{2.052606in}{2.224673in}}%
\pgfpathclose%
\pgfusepath{stroke}%
\end{pgfscope}%
\begin{pgfscope}%
\pgfpathrectangle{\pgfqpoint{0.678743in}{0.883555in}}{\pgfqpoint{4.650000in}{3.020000in}}%
\pgfusepath{clip}%
\pgfsetbuttcap%
\pgfsetroundjoin%
\pgfsetlinewidth{1.003750pt}%
\definecolor{currentstroke}{rgb}{0.000000,0.000000,1.000000}%
\pgfsetstrokecolor{currentstroke}%
\pgfsetdash{}{0pt}%
\pgfpathmoveto{\pgfqpoint{2.158288in}{2.270558in}}%
\pgfpathcurveto{\pgfqpoint{2.172554in}{2.270558in}}{\pgfqpoint{2.186237in}{2.276226in}}{\pgfqpoint{2.196325in}{2.286313in}}%
\pgfpathcurveto{\pgfqpoint{2.206412in}{2.296400in}}{\pgfqpoint{2.212080in}{2.310084in}}{\pgfqpoint{2.212080in}{2.324349in}}%
\pgfpathcurveto{\pgfqpoint{2.212080in}{2.338615in}}{\pgfqpoint{2.206412in}{2.352298in}}{\pgfqpoint{2.196325in}{2.362386in}}%
\pgfpathcurveto{\pgfqpoint{2.186237in}{2.372473in}}{\pgfqpoint{2.172554in}{2.378141in}}{\pgfqpoint{2.158288in}{2.378141in}}%
\pgfpathcurveto{\pgfqpoint{2.144023in}{2.378141in}}{\pgfqpoint{2.130339in}{2.372473in}}{\pgfqpoint{2.120252in}{2.362386in}}%
\pgfpathcurveto{\pgfqpoint{2.110165in}{2.352298in}}{\pgfqpoint{2.104497in}{2.338615in}}{\pgfqpoint{2.104497in}{2.324349in}}%
\pgfpathcurveto{\pgfqpoint{2.104497in}{2.310084in}}{\pgfqpoint{2.110165in}{2.296400in}}{\pgfqpoint{2.120252in}{2.286313in}}%
\pgfpathcurveto{\pgfqpoint{2.130339in}{2.276226in}}{\pgfqpoint{2.144023in}{2.270558in}}{\pgfqpoint{2.158288in}{2.270558in}}%
\pgfpathclose%
\pgfusepath{stroke}%
\end{pgfscope}%
\begin{pgfscope}%
\pgfpathrectangle{\pgfqpoint{0.678743in}{0.883555in}}{\pgfqpoint{4.650000in}{3.020000in}}%
\pgfusepath{clip}%
\pgfsetbuttcap%
\pgfsetroundjoin%
\pgfsetlinewidth{1.003750pt}%
\definecolor{currentstroke}{rgb}{0.000000,0.000000,1.000000}%
\pgfsetstrokecolor{currentstroke}%
\pgfsetdash{}{0pt}%
\pgfpathmoveto{\pgfqpoint{2.263970in}{2.273053in}}%
\pgfpathcurveto{\pgfqpoint{2.278236in}{2.273053in}}{\pgfqpoint{2.291919in}{2.278721in}}{\pgfqpoint{2.302006in}{2.288808in}}%
\pgfpathcurveto{\pgfqpoint{2.312094in}{2.298896in}}{\pgfqpoint{2.317761in}{2.312579in}}{\pgfqpoint{2.317761in}{2.326845in}}%
\pgfpathcurveto{\pgfqpoint{2.317761in}{2.341110in}}{\pgfqpoint{2.312094in}{2.354794in}}{\pgfqpoint{2.302006in}{2.364881in}}%
\pgfpathcurveto{\pgfqpoint{2.291919in}{2.374968in}}{\pgfqpoint{2.278236in}{2.380636in}}{\pgfqpoint{2.263970in}{2.380636in}}%
\pgfpathcurveto{\pgfqpoint{2.249704in}{2.380636in}}{\pgfqpoint{2.236021in}{2.374968in}}{\pgfqpoint{2.225934in}{2.364881in}}%
\pgfpathcurveto{\pgfqpoint{2.215846in}{2.354794in}}{\pgfqpoint{2.210179in}{2.341110in}}{\pgfqpoint{2.210179in}{2.326845in}}%
\pgfpathcurveto{\pgfqpoint{2.210179in}{2.312579in}}{\pgfqpoint{2.215846in}{2.298896in}}{\pgfqpoint{2.225934in}{2.288808in}}%
\pgfpathcurveto{\pgfqpoint{2.236021in}{2.278721in}}{\pgfqpoint{2.249704in}{2.273053in}}{\pgfqpoint{2.263970in}{2.273053in}}%
\pgfpathclose%
\pgfusepath{stroke}%
\end{pgfscope}%
\begin{pgfscope}%
\pgfpathrectangle{\pgfqpoint{0.678743in}{0.883555in}}{\pgfqpoint{4.650000in}{3.020000in}}%
\pgfusepath{clip}%
\pgfsetbuttcap%
\pgfsetroundjoin%
\pgfsetlinewidth{1.003750pt}%
\definecolor{currentstroke}{rgb}{0.000000,0.000000,1.000000}%
\pgfsetstrokecolor{currentstroke}%
\pgfsetdash{}{0pt}%
\pgfpathmoveto{\pgfqpoint{2.369652in}{2.340475in}}%
\pgfpathcurveto{\pgfqpoint{2.383918in}{2.340475in}}{\pgfqpoint{2.397601in}{2.346143in}}{\pgfqpoint{2.407688in}{2.356230in}}%
\pgfpathcurveto{\pgfqpoint{2.417775in}{2.366318in}}{\pgfqpoint{2.423443in}{2.380001in}}{\pgfqpoint{2.423443in}{2.394267in}}%
\pgfpathcurveto{\pgfqpoint{2.423443in}{2.408532in}}{\pgfqpoint{2.417775in}{2.422216in}}{\pgfqpoint{2.407688in}{2.432303in}}%
\pgfpathcurveto{\pgfqpoint{2.397601in}{2.442390in}}{\pgfqpoint{2.383918in}{2.448058in}}{\pgfqpoint{2.369652in}{2.448058in}}%
\pgfpathcurveto{\pgfqpoint{2.355386in}{2.448058in}}{\pgfqpoint{2.341703in}{2.442390in}}{\pgfqpoint{2.331616in}{2.432303in}}%
\pgfpathcurveto{\pgfqpoint{2.321528in}{2.422216in}}{\pgfqpoint{2.315860in}{2.408532in}}{\pgfqpoint{2.315860in}{2.394267in}}%
\pgfpathcurveto{\pgfqpoint{2.315860in}{2.380001in}}{\pgfqpoint{2.321528in}{2.366318in}}{\pgfqpoint{2.331616in}{2.356230in}}%
\pgfpathcurveto{\pgfqpoint{2.341703in}{2.346143in}}{\pgfqpoint{2.355386in}{2.340475in}}{\pgfqpoint{2.369652in}{2.340475in}}%
\pgfpathclose%
\pgfusepath{stroke}%
\end{pgfscope}%
\begin{pgfscope}%
\pgfpathrectangle{\pgfqpoint{0.678743in}{0.883555in}}{\pgfqpoint{4.650000in}{3.020000in}}%
\pgfusepath{clip}%
\pgfsetbuttcap%
\pgfsetroundjoin%
\pgfsetlinewidth{1.003750pt}%
\definecolor{currentstroke}{rgb}{0.000000,0.000000,1.000000}%
\pgfsetstrokecolor{currentstroke}%
\pgfsetdash{}{0pt}%
\pgfpathmoveto{\pgfqpoint{2.475334in}{2.327439in}}%
\pgfpathcurveto{\pgfqpoint{2.489599in}{2.327439in}}{\pgfqpoint{2.503283in}{2.333107in}}{\pgfqpoint{2.513370in}{2.343194in}}%
\pgfpathcurveto{\pgfqpoint{2.523457in}{2.353281in}}{\pgfqpoint{2.529125in}{2.366965in}}{\pgfqpoint{2.529125in}{2.381230in}}%
\pgfpathcurveto{\pgfqpoint{2.529125in}{2.395496in}}{\pgfqpoint{2.523457in}{2.409179in}}{\pgfqpoint{2.513370in}{2.419267in}}%
\pgfpathcurveto{\pgfqpoint{2.503283in}{2.429354in}}{\pgfqpoint{2.489599in}{2.435022in}}{\pgfqpoint{2.475334in}{2.435022in}}%
\pgfpathcurveto{\pgfqpoint{2.461068in}{2.435022in}}{\pgfqpoint{2.447385in}{2.429354in}}{\pgfqpoint{2.437297in}{2.419267in}}%
\pgfpathcurveto{\pgfqpoint{2.427210in}{2.409179in}}{\pgfqpoint{2.421542in}{2.395496in}}{\pgfqpoint{2.421542in}{2.381230in}}%
\pgfpathcurveto{\pgfqpoint{2.421542in}{2.366965in}}{\pgfqpoint{2.427210in}{2.353281in}}{\pgfqpoint{2.437297in}{2.343194in}}%
\pgfpathcurveto{\pgfqpoint{2.447385in}{2.333107in}}{\pgfqpoint{2.461068in}{2.327439in}}{\pgfqpoint{2.475334in}{2.327439in}}%
\pgfpathclose%
\pgfusepath{stroke}%
\end{pgfscope}%
\begin{pgfscope}%
\pgfpathrectangle{\pgfqpoint{0.678743in}{0.883555in}}{\pgfqpoint{4.650000in}{3.020000in}}%
\pgfusepath{clip}%
\pgfsetbuttcap%
\pgfsetroundjoin%
\pgfsetlinewidth{1.003750pt}%
\definecolor{currentstroke}{rgb}{0.000000,0.000000,1.000000}%
\pgfsetstrokecolor{currentstroke}%
\pgfsetdash{}{0pt}%
\pgfpathmoveto{\pgfqpoint{2.581016in}{2.340463in}}%
\pgfpathcurveto{\pgfqpoint{2.595281in}{2.340463in}}{\pgfqpoint{2.608964in}{2.346131in}}{\pgfqpoint{2.619052in}{2.356218in}}%
\pgfpathcurveto{\pgfqpoint{2.629139in}{2.366305in}}{\pgfqpoint{2.634807in}{2.379989in}}{\pgfqpoint{2.634807in}{2.394254in}}%
\pgfpathcurveto{\pgfqpoint{2.634807in}{2.408520in}}{\pgfqpoint{2.629139in}{2.422203in}}{\pgfqpoint{2.619052in}{2.432291in}}%
\pgfpathcurveto{\pgfqpoint{2.608964in}{2.442378in}}{\pgfqpoint{2.595281in}{2.448046in}}{\pgfqpoint{2.581016in}{2.448046in}}%
\pgfpathcurveto{\pgfqpoint{2.566750in}{2.448046in}}{\pgfqpoint{2.553067in}{2.442378in}}{\pgfqpoint{2.542979in}{2.432291in}}%
\pgfpathcurveto{\pgfqpoint{2.532892in}{2.422203in}}{\pgfqpoint{2.527224in}{2.408520in}}{\pgfqpoint{2.527224in}{2.394254in}}%
\pgfpathcurveto{\pgfqpoint{2.527224in}{2.379989in}}{\pgfqpoint{2.532892in}{2.366305in}}{\pgfqpoint{2.542979in}{2.356218in}}%
\pgfpathcurveto{\pgfqpoint{2.553067in}{2.346131in}}{\pgfqpoint{2.566750in}{2.340463in}}{\pgfqpoint{2.581016in}{2.340463in}}%
\pgfpathclose%
\pgfusepath{stroke}%
\end{pgfscope}%
\begin{pgfscope}%
\pgfpathrectangle{\pgfqpoint{0.678743in}{0.883555in}}{\pgfqpoint{4.650000in}{3.020000in}}%
\pgfusepath{clip}%
\pgfsetbuttcap%
\pgfsetroundjoin%
\pgfsetlinewidth{1.003750pt}%
\definecolor{currentstroke}{rgb}{0.000000,0.000000,1.000000}%
\pgfsetstrokecolor{currentstroke}%
\pgfsetdash{}{0pt}%
\pgfpathmoveto{\pgfqpoint{2.686697in}{2.318147in}}%
\pgfpathcurveto{\pgfqpoint{2.700963in}{2.318147in}}{\pgfqpoint{2.714646in}{2.323815in}}{\pgfqpoint{2.724734in}{2.333902in}}%
\pgfpathcurveto{\pgfqpoint{2.734821in}{2.343989in}}{\pgfqpoint{2.740489in}{2.357673in}}{\pgfqpoint{2.740489in}{2.371938in}}%
\pgfpathcurveto{\pgfqpoint{2.740489in}{2.386204in}}{\pgfqpoint{2.734821in}{2.399887in}}{\pgfqpoint{2.724734in}{2.409975in}}%
\pgfpathcurveto{\pgfqpoint{2.714646in}{2.420062in}}{\pgfqpoint{2.700963in}{2.425730in}}{\pgfqpoint{2.686697in}{2.425730in}}%
\pgfpathcurveto{\pgfqpoint{2.672432in}{2.425730in}}{\pgfqpoint{2.658748in}{2.420062in}}{\pgfqpoint{2.648661in}{2.409975in}}%
\pgfpathcurveto{\pgfqpoint{2.638574in}{2.399887in}}{\pgfqpoint{2.632906in}{2.386204in}}{\pgfqpoint{2.632906in}{2.371938in}}%
\pgfpathcurveto{\pgfqpoint{2.632906in}{2.357673in}}{\pgfqpoint{2.638574in}{2.343989in}}{\pgfqpoint{2.648661in}{2.333902in}}%
\pgfpathcurveto{\pgfqpoint{2.658748in}{2.323815in}}{\pgfqpoint{2.672432in}{2.318147in}}{\pgfqpoint{2.686697in}{2.318147in}}%
\pgfpathclose%
\pgfusepath{stroke}%
\end{pgfscope}%
\begin{pgfscope}%
\pgfpathrectangle{\pgfqpoint{0.678743in}{0.883555in}}{\pgfqpoint{4.650000in}{3.020000in}}%
\pgfusepath{clip}%
\pgfsetbuttcap%
\pgfsetroundjoin%
\pgfsetlinewidth{1.003750pt}%
\definecolor{currentstroke}{rgb}{0.000000,0.000000,1.000000}%
\pgfsetstrokecolor{currentstroke}%
\pgfsetdash{}{0pt}%
\pgfpathmoveto{\pgfqpoint{2.792379in}{2.340070in}}%
\pgfpathcurveto{\pgfqpoint{2.806645in}{2.340070in}}{\pgfqpoint{2.820328in}{2.345738in}}{\pgfqpoint{2.830415in}{2.355825in}}%
\pgfpathcurveto{\pgfqpoint{2.840503in}{2.365913in}}{\pgfqpoint{2.846171in}{2.379596in}}{\pgfqpoint{2.846171in}{2.393862in}}%
\pgfpathcurveto{\pgfqpoint{2.846171in}{2.408127in}}{\pgfqpoint{2.840503in}{2.421811in}}{\pgfqpoint{2.830415in}{2.431898in}}%
\pgfpathcurveto{\pgfqpoint{2.820328in}{2.441985in}}{\pgfqpoint{2.806645in}{2.447653in}}{\pgfqpoint{2.792379in}{2.447653in}}%
\pgfpathcurveto{\pgfqpoint{2.778113in}{2.447653in}}{\pgfqpoint{2.764430in}{2.441985in}}{\pgfqpoint{2.754343in}{2.431898in}}%
\pgfpathcurveto{\pgfqpoint{2.744256in}{2.421811in}}{\pgfqpoint{2.738588in}{2.408127in}}{\pgfqpoint{2.738588in}{2.393862in}}%
\pgfpathcurveto{\pgfqpoint{2.738588in}{2.379596in}}{\pgfqpoint{2.744256in}{2.365913in}}{\pgfqpoint{2.754343in}{2.355825in}}%
\pgfpathcurveto{\pgfqpoint{2.764430in}{2.345738in}}{\pgfqpoint{2.778113in}{2.340070in}}{\pgfqpoint{2.792379in}{2.340070in}}%
\pgfpathclose%
\pgfusepath{stroke}%
\end{pgfscope}%
\begin{pgfscope}%
\pgfpathrectangle{\pgfqpoint{0.678743in}{0.883555in}}{\pgfqpoint{4.650000in}{3.020000in}}%
\pgfusepath{clip}%
\pgfsetbuttcap%
\pgfsetroundjoin%
\pgfsetlinewidth{1.003750pt}%
\definecolor{currentstroke}{rgb}{0.000000,0.000000,1.000000}%
\pgfsetstrokecolor{currentstroke}%
\pgfsetdash{}{0pt}%
\pgfpathmoveto{\pgfqpoint{2.898061in}{2.295615in}}%
\pgfpathcurveto{\pgfqpoint{2.912327in}{2.295615in}}{\pgfqpoint{2.926010in}{2.301283in}}{\pgfqpoint{2.936097in}{2.311370in}}%
\pgfpathcurveto{\pgfqpoint{2.946185in}{2.321458in}}{\pgfqpoint{2.951852in}{2.335141in}}{\pgfqpoint{2.951852in}{2.349407in}}%
\pgfpathcurveto{\pgfqpoint{2.951852in}{2.363672in}}{\pgfqpoint{2.946185in}{2.377356in}}{\pgfqpoint{2.936097in}{2.387443in}}%
\pgfpathcurveto{\pgfqpoint{2.926010in}{2.397530in}}{\pgfqpoint{2.912327in}{2.403198in}}{\pgfqpoint{2.898061in}{2.403198in}}%
\pgfpathcurveto{\pgfqpoint{2.883795in}{2.403198in}}{\pgfqpoint{2.870112in}{2.397530in}}{\pgfqpoint{2.860025in}{2.387443in}}%
\pgfpathcurveto{\pgfqpoint{2.849937in}{2.377356in}}{\pgfqpoint{2.844270in}{2.363672in}}{\pgfqpoint{2.844270in}{2.349407in}}%
\pgfpathcurveto{\pgfqpoint{2.844270in}{2.335141in}}{\pgfqpoint{2.849937in}{2.321458in}}{\pgfqpoint{2.860025in}{2.311370in}}%
\pgfpathcurveto{\pgfqpoint{2.870112in}{2.301283in}}{\pgfqpoint{2.883795in}{2.295615in}}{\pgfqpoint{2.898061in}{2.295615in}}%
\pgfpathclose%
\pgfusepath{stroke}%
\end{pgfscope}%
\begin{pgfscope}%
\pgfpathrectangle{\pgfqpoint{0.678743in}{0.883555in}}{\pgfqpoint{4.650000in}{3.020000in}}%
\pgfusepath{clip}%
\pgfsetbuttcap%
\pgfsetroundjoin%
\pgfsetlinewidth{1.003750pt}%
\definecolor{currentstroke}{rgb}{0.000000,0.000000,1.000000}%
\pgfsetstrokecolor{currentstroke}%
\pgfsetdash{}{0pt}%
\pgfpathmoveto{\pgfqpoint{3.003743in}{2.349443in}}%
\pgfpathcurveto{\pgfqpoint{3.018008in}{2.349443in}}{\pgfqpoint{3.031692in}{2.355111in}}{\pgfqpoint{3.041779in}{2.365198in}}%
\pgfpathcurveto{\pgfqpoint{3.051866in}{2.375285in}}{\pgfqpoint{3.057534in}{2.388969in}}{\pgfqpoint{3.057534in}{2.403234in}}%
\pgfpathcurveto{\pgfqpoint{3.057534in}{2.417500in}}{\pgfqpoint{3.051866in}{2.431183in}}{\pgfqpoint{3.041779in}{2.441271in}}%
\pgfpathcurveto{\pgfqpoint{3.031692in}{2.451358in}}{\pgfqpoint{3.018008in}{2.457026in}}{\pgfqpoint{3.003743in}{2.457026in}}%
\pgfpathcurveto{\pgfqpoint{2.989477in}{2.457026in}}{\pgfqpoint{2.975794in}{2.451358in}}{\pgfqpoint{2.965706in}{2.441271in}}%
\pgfpathcurveto{\pgfqpoint{2.955619in}{2.431183in}}{\pgfqpoint{2.949951in}{2.417500in}}{\pgfqpoint{2.949951in}{2.403234in}}%
\pgfpathcurveto{\pgfqpoint{2.949951in}{2.388969in}}{\pgfqpoint{2.955619in}{2.375285in}}{\pgfqpoint{2.965706in}{2.365198in}}%
\pgfpathcurveto{\pgfqpoint{2.975794in}{2.355111in}}{\pgfqpoint{2.989477in}{2.349443in}}{\pgfqpoint{3.003743in}{2.349443in}}%
\pgfpathclose%
\pgfusepath{stroke}%
\end{pgfscope}%
\begin{pgfscope}%
\pgfpathrectangle{\pgfqpoint{0.678743in}{0.883555in}}{\pgfqpoint{4.650000in}{3.020000in}}%
\pgfusepath{clip}%
\pgfsetbuttcap%
\pgfsetroundjoin%
\pgfsetlinewidth{1.003750pt}%
\definecolor{currentstroke}{rgb}{0.000000,0.000000,1.000000}%
\pgfsetstrokecolor{currentstroke}%
\pgfsetdash{}{0pt}%
\pgfpathmoveto{\pgfqpoint{3.109425in}{2.306508in}}%
\pgfpathcurveto{\pgfqpoint{3.123690in}{2.306508in}}{\pgfqpoint{3.137374in}{2.312176in}}{\pgfqpoint{3.147461in}{2.322263in}}%
\pgfpathcurveto{\pgfqpoint{3.157548in}{2.332350in}}{\pgfqpoint{3.163216in}{2.346034in}}{\pgfqpoint{3.163216in}{2.360299in}}%
\pgfpathcurveto{\pgfqpoint{3.163216in}{2.374565in}}{\pgfqpoint{3.157548in}{2.388248in}}{\pgfqpoint{3.147461in}{2.398335in}}%
\pgfpathcurveto{\pgfqpoint{3.137374in}{2.408423in}}{\pgfqpoint{3.123690in}{2.414091in}}{\pgfqpoint{3.109425in}{2.414091in}}%
\pgfpathcurveto{\pgfqpoint{3.095159in}{2.414091in}}{\pgfqpoint{3.081476in}{2.408423in}}{\pgfqpoint{3.071388in}{2.398335in}}%
\pgfpathcurveto{\pgfqpoint{3.061301in}{2.388248in}}{\pgfqpoint{3.055633in}{2.374565in}}{\pgfqpoint{3.055633in}{2.360299in}}%
\pgfpathcurveto{\pgfqpoint{3.055633in}{2.346034in}}{\pgfqpoint{3.061301in}{2.332350in}}{\pgfqpoint{3.071388in}{2.322263in}}%
\pgfpathcurveto{\pgfqpoint{3.081476in}{2.312176in}}{\pgfqpoint{3.095159in}{2.306508in}}{\pgfqpoint{3.109425in}{2.306508in}}%
\pgfpathclose%
\pgfusepath{stroke}%
\end{pgfscope}%
\begin{pgfscope}%
\pgfpathrectangle{\pgfqpoint{0.678743in}{0.883555in}}{\pgfqpoint{4.650000in}{3.020000in}}%
\pgfusepath{clip}%
\pgfsetbuttcap%
\pgfsetroundjoin%
\pgfsetlinewidth{1.003750pt}%
\definecolor{currentstroke}{rgb}{0.000000,0.000000,1.000000}%
\pgfsetstrokecolor{currentstroke}%
\pgfsetdash{}{0pt}%
\pgfpathmoveto{\pgfqpoint{3.215106in}{2.329680in}}%
\pgfpathcurveto{\pgfqpoint{3.229372in}{2.329680in}}{\pgfqpoint{3.243055in}{2.335348in}}{\pgfqpoint{3.253143in}{2.345435in}}%
\pgfpathcurveto{\pgfqpoint{3.263230in}{2.355522in}}{\pgfqpoint{3.268898in}{2.369206in}}{\pgfqpoint{3.268898in}{2.383471in}}%
\pgfpathcurveto{\pgfqpoint{3.268898in}{2.397737in}}{\pgfqpoint{3.263230in}{2.411420in}}{\pgfqpoint{3.253143in}{2.421508in}}%
\pgfpathcurveto{\pgfqpoint{3.243055in}{2.431595in}}{\pgfqpoint{3.229372in}{2.437263in}}{\pgfqpoint{3.215106in}{2.437263in}}%
\pgfpathcurveto{\pgfqpoint{3.200841in}{2.437263in}}{\pgfqpoint{3.187157in}{2.431595in}}{\pgfqpoint{3.177070in}{2.421508in}}%
\pgfpathcurveto{\pgfqpoint{3.166983in}{2.411420in}}{\pgfqpoint{3.161315in}{2.397737in}}{\pgfqpoint{3.161315in}{2.383471in}}%
\pgfpathcurveto{\pgfqpoint{3.161315in}{2.369206in}}{\pgfqpoint{3.166983in}{2.355522in}}{\pgfqpoint{3.177070in}{2.345435in}}%
\pgfpathcurveto{\pgfqpoint{3.187157in}{2.335348in}}{\pgfqpoint{3.200841in}{2.329680in}}{\pgfqpoint{3.215106in}{2.329680in}}%
\pgfpathclose%
\pgfusepath{stroke}%
\end{pgfscope}%
\begin{pgfscope}%
\pgfpathrectangle{\pgfqpoint{0.678743in}{0.883555in}}{\pgfqpoint{4.650000in}{3.020000in}}%
\pgfusepath{clip}%
\pgfsetbuttcap%
\pgfsetroundjoin%
\pgfsetlinewidth{1.003750pt}%
\definecolor{currentstroke}{rgb}{0.000000,0.000000,1.000000}%
\pgfsetstrokecolor{currentstroke}%
\pgfsetdash{}{0pt}%
\pgfpathmoveto{\pgfqpoint{3.320788in}{2.318488in}}%
\pgfpathcurveto{\pgfqpoint{3.335054in}{2.318488in}}{\pgfqpoint{3.348737in}{2.324156in}}{\pgfqpoint{3.358825in}{2.334243in}}%
\pgfpathcurveto{\pgfqpoint{3.368912in}{2.344330in}}{\pgfqpoint{3.374580in}{2.358014in}}{\pgfqpoint{3.374580in}{2.372279in}}%
\pgfpathcurveto{\pgfqpoint{3.374580in}{2.386545in}}{\pgfqpoint{3.368912in}{2.400228in}}{\pgfqpoint{3.358825in}{2.410316in}}%
\pgfpathcurveto{\pgfqpoint{3.348737in}{2.420403in}}{\pgfqpoint{3.335054in}{2.426071in}}{\pgfqpoint{3.320788in}{2.426071in}}%
\pgfpathcurveto{\pgfqpoint{3.306523in}{2.426071in}}{\pgfqpoint{3.292839in}{2.420403in}}{\pgfqpoint{3.282752in}{2.410316in}}%
\pgfpathcurveto{\pgfqpoint{3.272665in}{2.400228in}}{\pgfqpoint{3.266997in}{2.386545in}}{\pgfqpoint{3.266997in}{2.372279in}}%
\pgfpathcurveto{\pgfqpoint{3.266997in}{2.358014in}}{\pgfqpoint{3.272665in}{2.344330in}}{\pgfqpoint{3.282752in}{2.334243in}}%
\pgfpathcurveto{\pgfqpoint{3.292839in}{2.324156in}}{\pgfqpoint{3.306523in}{2.318488in}}{\pgfqpoint{3.320788in}{2.318488in}}%
\pgfpathclose%
\pgfusepath{stroke}%
\end{pgfscope}%
\begin{pgfscope}%
\pgfpathrectangle{\pgfqpoint{0.678743in}{0.883555in}}{\pgfqpoint{4.650000in}{3.020000in}}%
\pgfusepath{clip}%
\pgfsetbuttcap%
\pgfsetroundjoin%
\pgfsetlinewidth{1.003750pt}%
\definecolor{currentstroke}{rgb}{0.000000,0.000000,1.000000}%
\pgfsetstrokecolor{currentstroke}%
\pgfsetdash{}{0pt}%
\pgfpathmoveto{\pgfqpoint{3.426470in}{2.266734in}}%
\pgfpathcurveto{\pgfqpoint{3.440736in}{2.266734in}}{\pgfqpoint{3.454419in}{2.272402in}}{\pgfqpoint{3.464506in}{2.282489in}}%
\pgfpathcurveto{\pgfqpoint{3.474594in}{2.292577in}}{\pgfqpoint{3.480261in}{2.306260in}}{\pgfqpoint{3.480261in}{2.320526in}}%
\pgfpathcurveto{\pgfqpoint{3.480261in}{2.334791in}}{\pgfqpoint{3.474594in}{2.348475in}}{\pgfqpoint{3.464506in}{2.358562in}}%
\pgfpathcurveto{\pgfqpoint{3.454419in}{2.368649in}}{\pgfqpoint{3.440736in}{2.374317in}}{\pgfqpoint{3.426470in}{2.374317in}}%
\pgfpathcurveto{\pgfqpoint{3.412204in}{2.374317in}}{\pgfqpoint{3.398521in}{2.368649in}}{\pgfqpoint{3.388434in}{2.358562in}}%
\pgfpathcurveto{\pgfqpoint{3.378346in}{2.348475in}}{\pgfqpoint{3.372679in}{2.334791in}}{\pgfqpoint{3.372679in}{2.320526in}}%
\pgfpathcurveto{\pgfqpoint{3.372679in}{2.306260in}}{\pgfqpoint{3.378346in}{2.292577in}}{\pgfqpoint{3.388434in}{2.282489in}}%
\pgfpathcurveto{\pgfqpoint{3.398521in}{2.272402in}}{\pgfqpoint{3.412204in}{2.266734in}}{\pgfqpoint{3.426470in}{2.266734in}}%
\pgfpathclose%
\pgfusepath{stroke}%
\end{pgfscope}%
\begin{pgfscope}%
\pgfpathrectangle{\pgfqpoint{0.678743in}{0.883555in}}{\pgfqpoint{4.650000in}{3.020000in}}%
\pgfusepath{clip}%
\pgfsetbuttcap%
\pgfsetroundjoin%
\pgfsetlinewidth{1.003750pt}%
\definecolor{currentstroke}{rgb}{0.000000,0.000000,1.000000}%
\pgfsetstrokecolor{currentstroke}%
\pgfsetdash{}{0pt}%
\pgfpathmoveto{\pgfqpoint{3.532152in}{2.249091in}}%
\pgfpathcurveto{\pgfqpoint{3.546418in}{2.249091in}}{\pgfqpoint{3.560101in}{2.254759in}}{\pgfqpoint{3.570188in}{2.264846in}}%
\pgfpathcurveto{\pgfqpoint{3.580275in}{2.274933in}}{\pgfqpoint{3.585943in}{2.288617in}}{\pgfqpoint{3.585943in}{2.302882in}}%
\pgfpathcurveto{\pgfqpoint{3.585943in}{2.317148in}}{\pgfqpoint{3.580275in}{2.330831in}}{\pgfqpoint{3.570188in}{2.340918in}}%
\pgfpathcurveto{\pgfqpoint{3.560101in}{2.351006in}}{\pgfqpoint{3.546418in}{2.356674in}}{\pgfqpoint{3.532152in}{2.356674in}}%
\pgfpathcurveto{\pgfqpoint{3.517886in}{2.356674in}}{\pgfqpoint{3.504203in}{2.351006in}}{\pgfqpoint{3.494116in}{2.340918in}}%
\pgfpathcurveto{\pgfqpoint{3.484028in}{2.330831in}}{\pgfqpoint{3.478360in}{2.317148in}}{\pgfqpoint{3.478360in}{2.302882in}}%
\pgfpathcurveto{\pgfqpoint{3.478360in}{2.288617in}}{\pgfqpoint{3.484028in}{2.274933in}}{\pgfqpoint{3.494116in}{2.264846in}}%
\pgfpathcurveto{\pgfqpoint{3.504203in}{2.254759in}}{\pgfqpoint{3.517886in}{2.249091in}}{\pgfqpoint{3.532152in}{2.249091in}}%
\pgfpathclose%
\pgfusepath{stroke}%
\end{pgfscope}%
\begin{pgfscope}%
\pgfpathrectangle{\pgfqpoint{0.678743in}{0.883555in}}{\pgfqpoint{4.650000in}{3.020000in}}%
\pgfusepath{clip}%
\pgfsetbuttcap%
\pgfsetroundjoin%
\pgfsetlinewidth{1.003750pt}%
\definecolor{currentstroke}{rgb}{0.000000,0.000000,1.000000}%
\pgfsetstrokecolor{currentstroke}%
\pgfsetdash{}{0pt}%
\pgfpathmoveto{\pgfqpoint{3.637834in}{2.259309in}}%
\pgfpathcurveto{\pgfqpoint{3.652099in}{2.259309in}}{\pgfqpoint{3.665783in}{2.264976in}}{\pgfqpoint{3.675870in}{2.275064in}}%
\pgfpathcurveto{\pgfqpoint{3.685957in}{2.285151in}}{\pgfqpoint{3.691625in}{2.298834in}}{\pgfqpoint{3.691625in}{2.313100in}}%
\pgfpathcurveto{\pgfqpoint{3.691625in}{2.327366in}}{\pgfqpoint{3.685957in}{2.341049in}}{\pgfqpoint{3.675870in}{2.351136in}}%
\pgfpathcurveto{\pgfqpoint{3.665783in}{2.361224in}}{\pgfqpoint{3.652099in}{2.366891in}}{\pgfqpoint{3.637834in}{2.366891in}}%
\pgfpathcurveto{\pgfqpoint{3.623568in}{2.366891in}}{\pgfqpoint{3.609885in}{2.361224in}}{\pgfqpoint{3.599797in}{2.351136in}}%
\pgfpathcurveto{\pgfqpoint{3.589710in}{2.341049in}}{\pgfqpoint{3.584042in}{2.327366in}}{\pgfqpoint{3.584042in}{2.313100in}}%
\pgfpathcurveto{\pgfqpoint{3.584042in}{2.298834in}}{\pgfqpoint{3.589710in}{2.285151in}}{\pgfqpoint{3.599797in}{2.275064in}}%
\pgfpathcurveto{\pgfqpoint{3.609885in}{2.264976in}}{\pgfqpoint{3.623568in}{2.259309in}}{\pgfqpoint{3.637834in}{2.259309in}}%
\pgfpathclose%
\pgfusepath{stroke}%
\end{pgfscope}%
\begin{pgfscope}%
\pgfpathrectangle{\pgfqpoint{0.678743in}{0.883555in}}{\pgfqpoint{4.650000in}{3.020000in}}%
\pgfusepath{clip}%
\pgfsetbuttcap%
\pgfsetroundjoin%
\pgfsetlinewidth{1.003750pt}%
\definecolor{currentstroke}{rgb}{0.000000,0.000000,1.000000}%
\pgfsetstrokecolor{currentstroke}%
\pgfsetdash{}{0pt}%
\pgfpathmoveto{\pgfqpoint{3.743516in}{2.225294in}}%
\pgfpathcurveto{\pgfqpoint{3.757781in}{2.225294in}}{\pgfqpoint{3.771464in}{2.230962in}}{\pgfqpoint{3.781552in}{2.241049in}}%
\pgfpathcurveto{\pgfqpoint{3.791639in}{2.251137in}}{\pgfqpoint{3.797307in}{2.264820in}}{\pgfqpoint{3.797307in}{2.279085in}}%
\pgfpathcurveto{\pgfqpoint{3.797307in}{2.293351in}}{\pgfqpoint{3.791639in}{2.307034in}}{\pgfqpoint{3.781552in}{2.317122in}}%
\pgfpathcurveto{\pgfqpoint{3.771464in}{2.327209in}}{\pgfqpoint{3.757781in}{2.332877in}}{\pgfqpoint{3.743516in}{2.332877in}}%
\pgfpathcurveto{\pgfqpoint{3.729250in}{2.332877in}}{\pgfqpoint{3.715567in}{2.327209in}}{\pgfqpoint{3.705479in}{2.317122in}}%
\pgfpathcurveto{\pgfqpoint{3.695392in}{2.307034in}}{\pgfqpoint{3.689724in}{2.293351in}}{\pgfqpoint{3.689724in}{2.279085in}}%
\pgfpathcurveto{\pgfqpoint{3.689724in}{2.264820in}}{\pgfqpoint{3.695392in}{2.251137in}}{\pgfqpoint{3.705479in}{2.241049in}}%
\pgfpathcurveto{\pgfqpoint{3.715567in}{2.230962in}}{\pgfqpoint{3.729250in}{2.225294in}}{\pgfqpoint{3.743516in}{2.225294in}}%
\pgfpathclose%
\pgfusepath{stroke}%
\end{pgfscope}%
\begin{pgfscope}%
\pgfpathrectangle{\pgfqpoint{0.678743in}{0.883555in}}{\pgfqpoint{4.650000in}{3.020000in}}%
\pgfusepath{clip}%
\pgfsetbuttcap%
\pgfsetroundjoin%
\pgfsetlinewidth{1.003750pt}%
\definecolor{currentstroke}{rgb}{0.000000,0.000000,1.000000}%
\pgfsetstrokecolor{currentstroke}%
\pgfsetdash{}{0pt}%
\pgfpathmoveto{\pgfqpoint{3.849197in}{2.199209in}}%
\pgfpathcurveto{\pgfqpoint{3.863463in}{2.199209in}}{\pgfqpoint{3.877146in}{2.204877in}}{\pgfqpoint{3.887234in}{2.214964in}}%
\pgfpathcurveto{\pgfqpoint{3.897321in}{2.225051in}}{\pgfqpoint{3.902989in}{2.238735in}}{\pgfqpoint{3.902989in}{2.253000in}}%
\pgfpathcurveto{\pgfqpoint{3.902989in}{2.267266in}}{\pgfqpoint{3.897321in}{2.280949in}}{\pgfqpoint{3.887234in}{2.291037in}}%
\pgfpathcurveto{\pgfqpoint{3.877146in}{2.301124in}}{\pgfqpoint{3.863463in}{2.306792in}}{\pgfqpoint{3.849197in}{2.306792in}}%
\pgfpathcurveto{\pgfqpoint{3.834932in}{2.306792in}}{\pgfqpoint{3.821248in}{2.301124in}}{\pgfqpoint{3.811161in}{2.291037in}}%
\pgfpathcurveto{\pgfqpoint{3.801074in}{2.280949in}}{\pgfqpoint{3.795406in}{2.267266in}}{\pgfqpoint{3.795406in}{2.253000in}}%
\pgfpathcurveto{\pgfqpoint{3.795406in}{2.238735in}}{\pgfqpoint{3.801074in}{2.225051in}}{\pgfqpoint{3.811161in}{2.214964in}}%
\pgfpathcurveto{\pgfqpoint{3.821248in}{2.204877in}}{\pgfqpoint{3.834932in}{2.199209in}}{\pgfqpoint{3.849197in}{2.199209in}}%
\pgfpathclose%
\pgfusepath{stroke}%
\end{pgfscope}%
\begin{pgfscope}%
\pgfpathrectangle{\pgfqpoint{0.678743in}{0.883555in}}{\pgfqpoint{4.650000in}{3.020000in}}%
\pgfusepath{clip}%
\pgfsetbuttcap%
\pgfsetroundjoin%
\pgfsetlinewidth{1.003750pt}%
\definecolor{currentstroke}{rgb}{0.000000,0.000000,1.000000}%
\pgfsetstrokecolor{currentstroke}%
\pgfsetdash{}{0pt}%
\pgfpathmoveto{\pgfqpoint{3.954879in}{2.244075in}}%
\pgfpathcurveto{\pgfqpoint{3.969145in}{2.244075in}}{\pgfqpoint{3.982828in}{2.249743in}}{\pgfqpoint{3.992915in}{2.259830in}}%
\pgfpathcurveto{\pgfqpoint{4.003003in}{2.269918in}}{\pgfqpoint{4.008671in}{2.283601in}}{\pgfqpoint{4.008671in}{2.297867in}}%
\pgfpathcurveto{\pgfqpoint{4.008671in}{2.312132in}}{\pgfqpoint{4.003003in}{2.325815in}}{\pgfqpoint{3.992915in}{2.335903in}}%
\pgfpathcurveto{\pgfqpoint{3.982828in}{2.345990in}}{\pgfqpoint{3.969145in}{2.351658in}}{\pgfqpoint{3.954879in}{2.351658in}}%
\pgfpathcurveto{\pgfqpoint{3.940613in}{2.351658in}}{\pgfqpoint{3.926930in}{2.345990in}}{\pgfqpoint{3.916843in}{2.335903in}}%
\pgfpathcurveto{\pgfqpoint{3.906756in}{2.325815in}}{\pgfqpoint{3.901088in}{2.312132in}}{\pgfqpoint{3.901088in}{2.297867in}}%
\pgfpathcurveto{\pgfqpoint{3.901088in}{2.283601in}}{\pgfqpoint{3.906756in}{2.269918in}}{\pgfqpoint{3.916843in}{2.259830in}}%
\pgfpathcurveto{\pgfqpoint{3.926930in}{2.249743in}}{\pgfqpoint{3.940613in}{2.244075in}}{\pgfqpoint{3.954879in}{2.244075in}}%
\pgfpathclose%
\pgfusepath{stroke}%
\end{pgfscope}%
\begin{pgfscope}%
\pgfpathrectangle{\pgfqpoint{0.678743in}{0.883555in}}{\pgfqpoint{4.650000in}{3.020000in}}%
\pgfusepath{clip}%
\pgfsetbuttcap%
\pgfsetroundjoin%
\pgfsetlinewidth{1.003750pt}%
\definecolor{currentstroke}{rgb}{0.000000,0.000000,1.000000}%
\pgfsetstrokecolor{currentstroke}%
\pgfsetdash{}{0pt}%
\pgfpathmoveto{\pgfqpoint{4.060561in}{2.219487in}}%
\pgfpathcurveto{\pgfqpoint{4.074827in}{2.219487in}}{\pgfqpoint{4.088510in}{2.225155in}}{\pgfqpoint{4.098597in}{2.235242in}}%
\pgfpathcurveto{\pgfqpoint{4.108685in}{2.245329in}}{\pgfqpoint{4.114352in}{2.259013in}}{\pgfqpoint{4.114352in}{2.273278in}}%
\pgfpathcurveto{\pgfqpoint{4.114352in}{2.287544in}}{\pgfqpoint{4.108685in}{2.301227in}}{\pgfqpoint{4.098597in}{2.311315in}}%
\pgfpathcurveto{\pgfqpoint{4.088510in}{2.321402in}}{\pgfqpoint{4.074827in}{2.327070in}}{\pgfqpoint{4.060561in}{2.327070in}}%
\pgfpathcurveto{\pgfqpoint{4.046295in}{2.327070in}}{\pgfqpoint{4.032612in}{2.321402in}}{\pgfqpoint{4.022525in}{2.311315in}}%
\pgfpathcurveto{\pgfqpoint{4.012437in}{2.301227in}}{\pgfqpoint{4.006770in}{2.287544in}}{\pgfqpoint{4.006770in}{2.273278in}}%
\pgfpathcurveto{\pgfqpoint{4.006770in}{2.259013in}}{\pgfqpoint{4.012437in}{2.245329in}}{\pgfqpoint{4.022525in}{2.235242in}}%
\pgfpathcurveto{\pgfqpoint{4.032612in}{2.225155in}}{\pgfqpoint{4.046295in}{2.219487in}}{\pgfqpoint{4.060561in}{2.219487in}}%
\pgfpathclose%
\pgfusepath{stroke}%
\end{pgfscope}%
\begin{pgfscope}%
\pgfpathrectangle{\pgfqpoint{0.678743in}{0.883555in}}{\pgfqpoint{4.650000in}{3.020000in}}%
\pgfusepath{clip}%
\pgfsetbuttcap%
\pgfsetroundjoin%
\pgfsetlinewidth{1.003750pt}%
\definecolor{currentstroke}{rgb}{0.000000,0.000000,1.000000}%
\pgfsetstrokecolor{currentstroke}%
\pgfsetdash{}{0pt}%
\pgfpathmoveto{\pgfqpoint{4.166243in}{2.238113in}}%
\pgfpathcurveto{\pgfqpoint{4.180508in}{2.238113in}}{\pgfqpoint{4.194192in}{2.243781in}}{\pgfqpoint{4.204279in}{2.253868in}}%
\pgfpathcurveto{\pgfqpoint{4.214366in}{2.263955in}}{\pgfqpoint{4.220034in}{2.277639in}}{\pgfqpoint{4.220034in}{2.291904in}}%
\pgfpathcurveto{\pgfqpoint{4.220034in}{2.306170in}}{\pgfqpoint{4.214366in}{2.319853in}}{\pgfqpoint{4.204279in}{2.329941in}}%
\pgfpathcurveto{\pgfqpoint{4.194192in}{2.340028in}}{\pgfqpoint{4.180508in}{2.345696in}}{\pgfqpoint{4.166243in}{2.345696in}}%
\pgfpathcurveto{\pgfqpoint{4.151977in}{2.345696in}}{\pgfqpoint{4.138294in}{2.340028in}}{\pgfqpoint{4.128206in}{2.329941in}}%
\pgfpathcurveto{\pgfqpoint{4.118119in}{2.319853in}}{\pgfqpoint{4.112451in}{2.306170in}}{\pgfqpoint{4.112451in}{2.291904in}}%
\pgfpathcurveto{\pgfqpoint{4.112451in}{2.277639in}}{\pgfqpoint{4.118119in}{2.263955in}}{\pgfqpoint{4.128206in}{2.253868in}}%
\pgfpathcurveto{\pgfqpoint{4.138294in}{2.243781in}}{\pgfqpoint{4.151977in}{2.238113in}}{\pgfqpoint{4.166243in}{2.238113in}}%
\pgfpathclose%
\pgfusepath{stroke}%
\end{pgfscope}%
\begin{pgfscope}%
\pgfpathrectangle{\pgfqpoint{0.678743in}{0.883555in}}{\pgfqpoint{4.650000in}{3.020000in}}%
\pgfusepath{clip}%
\pgfsetbuttcap%
\pgfsetroundjoin%
\pgfsetlinewidth{1.003750pt}%
\definecolor{currentstroke}{rgb}{0.000000,0.000000,1.000000}%
\pgfsetstrokecolor{currentstroke}%
\pgfsetdash{}{0pt}%
\pgfpathmoveto{\pgfqpoint{4.271925in}{2.197770in}}%
\pgfpathcurveto{\pgfqpoint{4.286190in}{2.197770in}}{\pgfqpoint{4.299874in}{2.203438in}}{\pgfqpoint{4.309961in}{2.213525in}}%
\pgfpathcurveto{\pgfqpoint{4.320048in}{2.223613in}}{\pgfqpoint{4.325716in}{2.237296in}}{\pgfqpoint{4.325716in}{2.251561in}}%
\pgfpathcurveto{\pgfqpoint{4.325716in}{2.265827in}}{\pgfqpoint{4.320048in}{2.279510in}}{\pgfqpoint{4.309961in}{2.289598in}}%
\pgfpathcurveto{\pgfqpoint{4.299874in}{2.299685in}}{\pgfqpoint{4.286190in}{2.305353in}}{\pgfqpoint{4.271925in}{2.305353in}}%
\pgfpathcurveto{\pgfqpoint{4.257659in}{2.305353in}}{\pgfqpoint{4.243976in}{2.299685in}}{\pgfqpoint{4.233888in}{2.289598in}}%
\pgfpathcurveto{\pgfqpoint{4.223801in}{2.279510in}}{\pgfqpoint{4.218133in}{2.265827in}}{\pgfqpoint{4.218133in}{2.251561in}}%
\pgfpathcurveto{\pgfqpoint{4.218133in}{2.237296in}}{\pgfqpoint{4.223801in}{2.223613in}}{\pgfqpoint{4.233888in}{2.213525in}}%
\pgfpathcurveto{\pgfqpoint{4.243976in}{2.203438in}}{\pgfqpoint{4.257659in}{2.197770in}}{\pgfqpoint{4.271925in}{2.197770in}}%
\pgfpathclose%
\pgfusepath{stroke}%
\end{pgfscope}%
\begin{pgfscope}%
\pgfpathrectangle{\pgfqpoint{0.678743in}{0.883555in}}{\pgfqpoint{4.650000in}{3.020000in}}%
\pgfusepath{clip}%
\pgfsetbuttcap%
\pgfsetroundjoin%
\pgfsetlinewidth{1.003750pt}%
\definecolor{currentstroke}{rgb}{0.000000,0.000000,1.000000}%
\pgfsetstrokecolor{currentstroke}%
\pgfsetdash{}{0pt}%
\pgfpathmoveto{\pgfqpoint{4.377606in}{2.266113in}}%
\pgfpathcurveto{\pgfqpoint{4.391872in}{2.266113in}}{\pgfqpoint{4.405555in}{2.271781in}}{\pgfqpoint{4.415643in}{2.281868in}}%
\pgfpathcurveto{\pgfqpoint{4.425730in}{2.291956in}}{\pgfqpoint{4.431398in}{2.305639in}}{\pgfqpoint{4.431398in}{2.319905in}}%
\pgfpathcurveto{\pgfqpoint{4.431398in}{2.334170in}}{\pgfqpoint{4.425730in}{2.347853in}}{\pgfqpoint{4.415643in}{2.357941in}}%
\pgfpathcurveto{\pgfqpoint{4.405555in}{2.368028in}}{\pgfqpoint{4.391872in}{2.373696in}}{\pgfqpoint{4.377606in}{2.373696in}}%
\pgfpathcurveto{\pgfqpoint{4.363341in}{2.373696in}}{\pgfqpoint{4.349657in}{2.368028in}}{\pgfqpoint{4.339570in}{2.357941in}}%
\pgfpathcurveto{\pgfqpoint{4.329483in}{2.347853in}}{\pgfqpoint{4.323815in}{2.334170in}}{\pgfqpoint{4.323815in}{2.319905in}}%
\pgfpathcurveto{\pgfqpoint{4.323815in}{2.305639in}}{\pgfqpoint{4.329483in}{2.291956in}}{\pgfqpoint{4.339570in}{2.281868in}}%
\pgfpathcurveto{\pgfqpoint{4.349657in}{2.271781in}}{\pgfqpoint{4.363341in}{2.266113in}}{\pgfqpoint{4.377606in}{2.266113in}}%
\pgfpathclose%
\pgfusepath{stroke}%
\end{pgfscope}%
\begin{pgfscope}%
\pgfpathrectangle{\pgfqpoint{0.678743in}{0.883555in}}{\pgfqpoint{4.650000in}{3.020000in}}%
\pgfusepath{clip}%
\pgfsetbuttcap%
\pgfsetroundjoin%
\pgfsetlinewidth{1.003750pt}%
\definecolor{currentstroke}{rgb}{0.000000,0.000000,1.000000}%
\pgfsetstrokecolor{currentstroke}%
\pgfsetdash{}{0pt}%
\pgfpathmoveto{\pgfqpoint{4.483288in}{2.221083in}}%
\pgfpathcurveto{\pgfqpoint{4.497554in}{2.221083in}}{\pgfqpoint{4.511237in}{2.226751in}}{\pgfqpoint{4.521325in}{2.236838in}}%
\pgfpathcurveto{\pgfqpoint{4.531412in}{2.246926in}}{\pgfqpoint{4.537080in}{2.260609in}}{\pgfqpoint{4.537080in}{2.274875in}}%
\pgfpathcurveto{\pgfqpoint{4.537080in}{2.289140in}}{\pgfqpoint{4.531412in}{2.302824in}}{\pgfqpoint{4.521325in}{2.312911in}}%
\pgfpathcurveto{\pgfqpoint{4.511237in}{2.322998in}}{\pgfqpoint{4.497554in}{2.328666in}}{\pgfqpoint{4.483288in}{2.328666in}}%
\pgfpathcurveto{\pgfqpoint{4.469023in}{2.328666in}}{\pgfqpoint{4.455339in}{2.322998in}}{\pgfqpoint{4.445252in}{2.312911in}}%
\pgfpathcurveto{\pgfqpoint{4.435165in}{2.302824in}}{\pgfqpoint{4.429497in}{2.289140in}}{\pgfqpoint{4.429497in}{2.274875in}}%
\pgfpathcurveto{\pgfqpoint{4.429497in}{2.260609in}}{\pgfqpoint{4.435165in}{2.246926in}}{\pgfqpoint{4.445252in}{2.236838in}}%
\pgfpathcurveto{\pgfqpoint{4.455339in}{2.226751in}}{\pgfqpoint{4.469023in}{2.221083in}}{\pgfqpoint{4.483288in}{2.221083in}}%
\pgfpathclose%
\pgfusepath{stroke}%
\end{pgfscope}%
\begin{pgfscope}%
\pgfpathrectangle{\pgfqpoint{0.678743in}{0.883555in}}{\pgfqpoint{4.650000in}{3.020000in}}%
\pgfusepath{clip}%
\pgfsetbuttcap%
\pgfsetroundjoin%
\pgfsetlinewidth{1.003750pt}%
\definecolor{currentstroke}{rgb}{0.000000,0.000000,1.000000}%
\pgfsetstrokecolor{currentstroke}%
\pgfsetdash{}{0pt}%
\pgfpathmoveto{\pgfqpoint{4.588970in}{2.164553in}}%
\pgfpathcurveto{\pgfqpoint{4.603236in}{2.164553in}}{\pgfqpoint{4.616919in}{2.170221in}}{\pgfqpoint{4.627006in}{2.180308in}}%
\pgfpathcurveto{\pgfqpoint{4.637094in}{2.190395in}}{\pgfqpoint{4.642761in}{2.204079in}}{\pgfqpoint{4.642761in}{2.218344in}}%
\pgfpathcurveto{\pgfqpoint{4.642761in}{2.232610in}}{\pgfqpoint{4.637094in}{2.246293in}}{\pgfqpoint{4.627006in}{2.256381in}}%
\pgfpathcurveto{\pgfqpoint{4.616919in}{2.266468in}}{\pgfqpoint{4.603236in}{2.272136in}}{\pgfqpoint{4.588970in}{2.272136in}}%
\pgfpathcurveto{\pgfqpoint{4.574704in}{2.272136in}}{\pgfqpoint{4.561021in}{2.266468in}}{\pgfqpoint{4.550934in}{2.256381in}}%
\pgfpathcurveto{\pgfqpoint{4.540846in}{2.246293in}}{\pgfqpoint{4.535179in}{2.232610in}}{\pgfqpoint{4.535179in}{2.218344in}}%
\pgfpathcurveto{\pgfqpoint{4.535179in}{2.204079in}}{\pgfqpoint{4.540846in}{2.190395in}}{\pgfqpoint{4.550934in}{2.180308in}}%
\pgfpathcurveto{\pgfqpoint{4.561021in}{2.170221in}}{\pgfqpoint{4.574704in}{2.164553in}}{\pgfqpoint{4.588970in}{2.164553in}}%
\pgfpathclose%
\pgfusepath{stroke}%
\end{pgfscope}%
\begin{pgfscope}%
\pgfpathrectangle{\pgfqpoint{0.678743in}{0.883555in}}{\pgfqpoint{4.650000in}{3.020000in}}%
\pgfusepath{clip}%
\pgfsetbuttcap%
\pgfsetroundjoin%
\pgfsetlinewidth{1.003750pt}%
\definecolor{currentstroke}{rgb}{0.000000,0.000000,1.000000}%
\pgfsetstrokecolor{currentstroke}%
\pgfsetdash{}{0pt}%
\pgfpathmoveto{\pgfqpoint{4.694652in}{2.194969in}}%
\pgfpathcurveto{\pgfqpoint{4.708918in}{2.194969in}}{\pgfqpoint{4.722601in}{2.200637in}}{\pgfqpoint{4.732688in}{2.210724in}}%
\pgfpathcurveto{\pgfqpoint{4.742775in}{2.220811in}}{\pgfqpoint{4.748443in}{2.234494in}}{\pgfqpoint{4.748443in}{2.248760in}}%
\pgfpathcurveto{\pgfqpoint{4.748443in}{2.263026in}}{\pgfqpoint{4.742775in}{2.276709in}}{\pgfqpoint{4.732688in}{2.286796in}}%
\pgfpathcurveto{\pgfqpoint{4.722601in}{2.296884in}}{\pgfqpoint{4.708918in}{2.302552in}}{\pgfqpoint{4.694652in}{2.302552in}}%
\pgfpathcurveto{\pgfqpoint{4.680386in}{2.302552in}}{\pgfqpoint{4.666703in}{2.296884in}}{\pgfqpoint{4.656616in}{2.286796in}}%
\pgfpathcurveto{\pgfqpoint{4.646528in}{2.276709in}}{\pgfqpoint{4.640860in}{2.263026in}}{\pgfqpoint{4.640860in}{2.248760in}}%
\pgfpathcurveto{\pgfqpoint{4.640860in}{2.234494in}}{\pgfqpoint{4.646528in}{2.220811in}}{\pgfqpoint{4.656616in}{2.210724in}}%
\pgfpathcurveto{\pgfqpoint{4.666703in}{2.200637in}}{\pgfqpoint{4.680386in}{2.194969in}}{\pgfqpoint{4.694652in}{2.194969in}}%
\pgfpathclose%
\pgfusepath{stroke}%
\end{pgfscope}%
\begin{pgfscope}%
\pgfpathrectangle{\pgfqpoint{0.678743in}{0.883555in}}{\pgfqpoint{4.650000in}{3.020000in}}%
\pgfusepath{clip}%
\pgfsetbuttcap%
\pgfsetroundjoin%
\pgfsetlinewidth{1.003750pt}%
\definecolor{currentstroke}{rgb}{0.000000,0.000000,1.000000}%
\pgfsetstrokecolor{currentstroke}%
\pgfsetdash{}{0pt}%
\pgfpathmoveto{\pgfqpoint{4.800334in}{2.186956in}}%
\pgfpathcurveto{\pgfqpoint{4.814599in}{2.186956in}}{\pgfqpoint{4.828283in}{2.192624in}}{\pgfqpoint{4.838370in}{2.202711in}}%
\pgfpathcurveto{\pgfqpoint{4.848457in}{2.212798in}}{\pgfqpoint{4.854125in}{2.226482in}}{\pgfqpoint{4.854125in}{2.240747in}}%
\pgfpathcurveto{\pgfqpoint{4.854125in}{2.255013in}}{\pgfqpoint{4.848457in}{2.268696in}}{\pgfqpoint{4.838370in}{2.278784in}}%
\pgfpathcurveto{\pgfqpoint{4.828283in}{2.288871in}}{\pgfqpoint{4.814599in}{2.294539in}}{\pgfqpoint{4.800334in}{2.294539in}}%
\pgfpathcurveto{\pgfqpoint{4.786068in}{2.294539in}}{\pgfqpoint{4.772385in}{2.288871in}}{\pgfqpoint{4.762297in}{2.278784in}}%
\pgfpathcurveto{\pgfqpoint{4.752210in}{2.268696in}}{\pgfqpoint{4.746542in}{2.255013in}}{\pgfqpoint{4.746542in}{2.240747in}}%
\pgfpathcurveto{\pgfqpoint{4.746542in}{2.226482in}}{\pgfqpoint{4.752210in}{2.212798in}}{\pgfqpoint{4.762297in}{2.202711in}}%
\pgfpathcurveto{\pgfqpoint{4.772385in}{2.192624in}}{\pgfqpoint{4.786068in}{2.186956in}}{\pgfqpoint{4.800334in}{2.186956in}}%
\pgfpathclose%
\pgfusepath{stroke}%
\end{pgfscope}%
\begin{pgfscope}%
\pgfpathrectangle{\pgfqpoint{0.678743in}{0.883555in}}{\pgfqpoint{4.650000in}{3.020000in}}%
\pgfusepath{clip}%
\pgfsetbuttcap%
\pgfsetroundjoin%
\pgfsetlinewidth{1.003750pt}%
\definecolor{currentstroke}{rgb}{0.000000,0.000000,1.000000}%
\pgfsetstrokecolor{currentstroke}%
\pgfsetdash{}{0pt}%
\pgfpathmoveto{\pgfqpoint{4.906016in}{2.201681in}}%
\pgfpathcurveto{\pgfqpoint{4.920281in}{2.201681in}}{\pgfqpoint{4.933964in}{2.207349in}}{\pgfqpoint{4.944052in}{2.217436in}}%
\pgfpathcurveto{\pgfqpoint{4.954139in}{2.227523in}}{\pgfqpoint{4.959807in}{2.241207in}}{\pgfqpoint{4.959807in}{2.255472in}}%
\pgfpathcurveto{\pgfqpoint{4.959807in}{2.269738in}}{\pgfqpoint{4.954139in}{2.283421in}}{\pgfqpoint{4.944052in}{2.293509in}}%
\pgfpathcurveto{\pgfqpoint{4.933964in}{2.303596in}}{\pgfqpoint{4.920281in}{2.309264in}}{\pgfqpoint{4.906016in}{2.309264in}}%
\pgfpathcurveto{\pgfqpoint{4.891750in}{2.309264in}}{\pgfqpoint{4.878067in}{2.303596in}}{\pgfqpoint{4.867979in}{2.293509in}}%
\pgfpathcurveto{\pgfqpoint{4.857892in}{2.283421in}}{\pgfqpoint{4.852224in}{2.269738in}}{\pgfqpoint{4.852224in}{2.255472in}}%
\pgfpathcurveto{\pgfqpoint{4.852224in}{2.241207in}}{\pgfqpoint{4.857892in}{2.227523in}}{\pgfqpoint{4.867979in}{2.217436in}}%
\pgfpathcurveto{\pgfqpoint{4.878067in}{2.207349in}}{\pgfqpoint{4.891750in}{2.201681in}}{\pgfqpoint{4.906016in}{2.201681in}}%
\pgfpathclose%
\pgfusepath{stroke}%
\end{pgfscope}%
\begin{pgfscope}%
\pgfpathrectangle{\pgfqpoint{0.678743in}{0.883555in}}{\pgfqpoint{4.650000in}{3.020000in}}%
\pgfusepath{clip}%
\pgfsetbuttcap%
\pgfsetroundjoin%
\pgfsetlinewidth{1.003750pt}%
\definecolor{currentstroke}{rgb}{0.000000,0.000000,1.000000}%
\pgfsetstrokecolor{currentstroke}%
\pgfsetdash{}{0pt}%
\pgfpathmoveto{\pgfqpoint{5.011697in}{2.189183in}}%
\pgfpathcurveto{\pgfqpoint{5.025963in}{2.189183in}}{\pgfqpoint{5.039646in}{2.194851in}}{\pgfqpoint{5.049734in}{2.204938in}}%
\pgfpathcurveto{\pgfqpoint{5.059821in}{2.215025in}}{\pgfqpoint{5.065489in}{2.228709in}}{\pgfqpoint{5.065489in}{2.242974in}}%
\pgfpathcurveto{\pgfqpoint{5.065489in}{2.257240in}}{\pgfqpoint{5.059821in}{2.270923in}}{\pgfqpoint{5.049734in}{2.281011in}}%
\pgfpathcurveto{\pgfqpoint{5.039646in}{2.291098in}}{\pgfqpoint{5.025963in}{2.296766in}}{\pgfqpoint{5.011697in}{2.296766in}}%
\pgfpathcurveto{\pgfqpoint{4.997432in}{2.296766in}}{\pgfqpoint{4.983748in}{2.291098in}}{\pgfqpoint{4.973661in}{2.281011in}}%
\pgfpathcurveto{\pgfqpoint{4.963574in}{2.270923in}}{\pgfqpoint{4.957906in}{2.257240in}}{\pgfqpoint{4.957906in}{2.242974in}}%
\pgfpathcurveto{\pgfqpoint{4.957906in}{2.228709in}}{\pgfqpoint{4.963574in}{2.215025in}}{\pgfqpoint{4.973661in}{2.204938in}}%
\pgfpathcurveto{\pgfqpoint{4.983748in}{2.194851in}}{\pgfqpoint{4.997432in}{2.189183in}}{\pgfqpoint{5.011697in}{2.189183in}}%
\pgfpathclose%
\pgfusepath{stroke}%
\end{pgfscope}%
\begin{pgfscope}%
\pgfpathrectangle{\pgfqpoint{0.678743in}{0.883555in}}{\pgfqpoint{4.650000in}{3.020000in}}%
\pgfusepath{clip}%
\pgfsetbuttcap%
\pgfsetroundjoin%
\pgfsetlinewidth{1.003750pt}%
\definecolor{currentstroke}{rgb}{0.000000,0.000000,1.000000}%
\pgfsetstrokecolor{currentstroke}%
\pgfsetdash{}{0pt}%
\pgfpathmoveto{\pgfqpoint{5.117379in}{2.222571in}}%
\pgfpathcurveto{\pgfqpoint{5.131645in}{2.222571in}}{\pgfqpoint{5.145328in}{2.228239in}}{\pgfqpoint{5.155415in}{2.238326in}}%
\pgfpathcurveto{\pgfqpoint{5.165503in}{2.248413in}}{\pgfqpoint{5.171171in}{2.262097in}}{\pgfqpoint{5.171171in}{2.276362in}}%
\pgfpathcurveto{\pgfqpoint{5.171171in}{2.290628in}}{\pgfqpoint{5.165503in}{2.304311in}}{\pgfqpoint{5.155415in}{2.314398in}}%
\pgfpathcurveto{\pgfqpoint{5.145328in}{2.324486in}}{\pgfqpoint{5.131645in}{2.330154in}}{\pgfqpoint{5.117379in}{2.330154in}}%
\pgfpathcurveto{\pgfqpoint{5.103113in}{2.330154in}}{\pgfqpoint{5.089430in}{2.324486in}}{\pgfqpoint{5.079343in}{2.314398in}}%
\pgfpathcurveto{\pgfqpoint{5.069256in}{2.304311in}}{\pgfqpoint{5.063588in}{2.290628in}}{\pgfqpoint{5.063588in}{2.276362in}}%
\pgfpathcurveto{\pgfqpoint{5.063588in}{2.262097in}}{\pgfqpoint{5.069256in}{2.248413in}}{\pgfqpoint{5.079343in}{2.238326in}}%
\pgfpathcurveto{\pgfqpoint{5.089430in}{2.228239in}}{\pgfqpoint{5.103113in}{2.222571in}}{\pgfqpoint{5.117379in}{2.222571in}}%
\pgfpathclose%
\pgfusepath{stroke}%
\end{pgfscope}%
\begin{pgfscope}%
\pgfsetbuttcap%
\pgfsetroundjoin%
\definecolor{currentfill}{rgb}{0.000000,0.000000,0.000000}%
\pgfsetfillcolor{currentfill}%
\pgfsetlinewidth{0.803000pt}%
\definecolor{currentstroke}{rgb}{0.000000,0.000000,0.000000}%
\pgfsetstrokecolor{currentstroke}%
\pgfsetdash{}{0pt}%
\pgfsys@defobject{currentmarker}{\pgfqpoint{0.000000in}{-0.048611in}}{\pgfqpoint{0.000000in}{0.000000in}}{%
\pgfpathmoveto{\pgfqpoint{0.000000in}{0.000000in}}%
\pgfpathlineto{\pgfqpoint{0.000000in}{-0.048611in}}%
\pgfusepath{stroke,fill}%
}%
\begin{pgfscope}%
\pgfsys@transformshift{0.890106in}{0.883555in}%
\pgfsys@useobject{currentmarker}{}%
\end{pgfscope}%
\end{pgfscope}%
\begin{pgfscope}%
\definecolor{textcolor}{rgb}{0.000000,0.000000,0.000000}%
\pgfsetstrokecolor{textcolor}%
\pgfsetfillcolor{textcolor}%
\pgftext[x=0.890106in,y=0.786333in,,top]{\color{textcolor}\rmfamily\fontsize{11.000000}{13.200000}\selectfont \(\displaystyle {20}\)}%
\end{pgfscope}%
\begin{pgfscope}%
\pgfsetbuttcap%
\pgfsetroundjoin%
\definecolor{currentfill}{rgb}{0.000000,0.000000,0.000000}%
\pgfsetfillcolor{currentfill}%
\pgfsetlinewidth{0.803000pt}%
\definecolor{currentstroke}{rgb}{0.000000,0.000000,0.000000}%
\pgfsetstrokecolor{currentstroke}%
\pgfsetdash{}{0pt}%
\pgfsys@defobject{currentmarker}{\pgfqpoint{0.000000in}{-0.048611in}}{\pgfqpoint{0.000000in}{0.000000in}}{%
\pgfpathmoveto{\pgfqpoint{0.000000in}{0.000000in}}%
\pgfpathlineto{\pgfqpoint{0.000000in}{-0.048611in}}%
\pgfusepath{stroke,fill}%
}%
\begin{pgfscope}%
\pgfsys@transformshift{1.418516in}{0.883555in}%
\pgfsys@useobject{currentmarker}{}%
\end{pgfscope}%
\end{pgfscope}%
\begin{pgfscope}%
\definecolor{textcolor}{rgb}{0.000000,0.000000,0.000000}%
\pgfsetstrokecolor{textcolor}%
\pgfsetfillcolor{textcolor}%
\pgftext[x=1.418516in,y=0.786333in,,top]{\color{textcolor}\rmfamily\fontsize{11.000000}{13.200000}\selectfont \(\displaystyle {25}\)}%
\end{pgfscope}%
\begin{pgfscope}%
\pgfsetbuttcap%
\pgfsetroundjoin%
\definecolor{currentfill}{rgb}{0.000000,0.000000,0.000000}%
\pgfsetfillcolor{currentfill}%
\pgfsetlinewidth{0.803000pt}%
\definecolor{currentstroke}{rgb}{0.000000,0.000000,0.000000}%
\pgfsetstrokecolor{currentstroke}%
\pgfsetdash{}{0pt}%
\pgfsys@defobject{currentmarker}{\pgfqpoint{0.000000in}{-0.048611in}}{\pgfqpoint{0.000000in}{0.000000in}}{%
\pgfpathmoveto{\pgfqpoint{0.000000in}{0.000000in}}%
\pgfpathlineto{\pgfqpoint{0.000000in}{-0.048611in}}%
\pgfusepath{stroke,fill}%
}%
\begin{pgfscope}%
\pgfsys@transformshift{1.946925in}{0.883555in}%
\pgfsys@useobject{currentmarker}{}%
\end{pgfscope}%
\end{pgfscope}%
\begin{pgfscope}%
\definecolor{textcolor}{rgb}{0.000000,0.000000,0.000000}%
\pgfsetstrokecolor{textcolor}%
\pgfsetfillcolor{textcolor}%
\pgftext[x=1.946925in,y=0.786333in,,top]{\color{textcolor}\rmfamily\fontsize{11.000000}{13.200000}\selectfont \(\displaystyle {30}\)}%
\end{pgfscope}%
\begin{pgfscope}%
\pgfsetbuttcap%
\pgfsetroundjoin%
\definecolor{currentfill}{rgb}{0.000000,0.000000,0.000000}%
\pgfsetfillcolor{currentfill}%
\pgfsetlinewidth{0.803000pt}%
\definecolor{currentstroke}{rgb}{0.000000,0.000000,0.000000}%
\pgfsetstrokecolor{currentstroke}%
\pgfsetdash{}{0pt}%
\pgfsys@defobject{currentmarker}{\pgfqpoint{0.000000in}{-0.048611in}}{\pgfqpoint{0.000000in}{0.000000in}}{%
\pgfpathmoveto{\pgfqpoint{0.000000in}{0.000000in}}%
\pgfpathlineto{\pgfqpoint{0.000000in}{-0.048611in}}%
\pgfusepath{stroke,fill}%
}%
\begin{pgfscope}%
\pgfsys@transformshift{2.475334in}{0.883555in}%
\pgfsys@useobject{currentmarker}{}%
\end{pgfscope}%
\end{pgfscope}%
\begin{pgfscope}%
\definecolor{textcolor}{rgb}{0.000000,0.000000,0.000000}%
\pgfsetstrokecolor{textcolor}%
\pgfsetfillcolor{textcolor}%
\pgftext[x=2.475334in,y=0.786333in,,top]{\color{textcolor}\rmfamily\fontsize{11.000000}{13.200000}\selectfont \(\displaystyle {35}\)}%
\end{pgfscope}%
\begin{pgfscope}%
\pgfsetbuttcap%
\pgfsetroundjoin%
\definecolor{currentfill}{rgb}{0.000000,0.000000,0.000000}%
\pgfsetfillcolor{currentfill}%
\pgfsetlinewidth{0.803000pt}%
\definecolor{currentstroke}{rgb}{0.000000,0.000000,0.000000}%
\pgfsetstrokecolor{currentstroke}%
\pgfsetdash{}{0pt}%
\pgfsys@defobject{currentmarker}{\pgfqpoint{0.000000in}{-0.048611in}}{\pgfqpoint{0.000000in}{0.000000in}}{%
\pgfpathmoveto{\pgfqpoint{0.000000in}{0.000000in}}%
\pgfpathlineto{\pgfqpoint{0.000000in}{-0.048611in}}%
\pgfusepath{stroke,fill}%
}%
\begin{pgfscope}%
\pgfsys@transformshift{3.003743in}{0.883555in}%
\pgfsys@useobject{currentmarker}{}%
\end{pgfscope}%
\end{pgfscope}%
\begin{pgfscope}%
\definecolor{textcolor}{rgb}{0.000000,0.000000,0.000000}%
\pgfsetstrokecolor{textcolor}%
\pgfsetfillcolor{textcolor}%
\pgftext[x=3.003743in,y=0.786333in,,top]{\color{textcolor}\rmfamily\fontsize{11.000000}{13.200000}\selectfont \(\displaystyle {40}\)}%
\end{pgfscope}%
\begin{pgfscope}%
\pgfsetbuttcap%
\pgfsetroundjoin%
\definecolor{currentfill}{rgb}{0.000000,0.000000,0.000000}%
\pgfsetfillcolor{currentfill}%
\pgfsetlinewidth{0.803000pt}%
\definecolor{currentstroke}{rgb}{0.000000,0.000000,0.000000}%
\pgfsetstrokecolor{currentstroke}%
\pgfsetdash{}{0pt}%
\pgfsys@defobject{currentmarker}{\pgfqpoint{0.000000in}{-0.048611in}}{\pgfqpoint{0.000000in}{0.000000in}}{%
\pgfpathmoveto{\pgfqpoint{0.000000in}{0.000000in}}%
\pgfpathlineto{\pgfqpoint{0.000000in}{-0.048611in}}%
\pgfusepath{stroke,fill}%
}%
\begin{pgfscope}%
\pgfsys@transformshift{3.532152in}{0.883555in}%
\pgfsys@useobject{currentmarker}{}%
\end{pgfscope}%
\end{pgfscope}%
\begin{pgfscope}%
\definecolor{textcolor}{rgb}{0.000000,0.000000,0.000000}%
\pgfsetstrokecolor{textcolor}%
\pgfsetfillcolor{textcolor}%
\pgftext[x=3.532152in,y=0.786333in,,top]{\color{textcolor}\rmfamily\fontsize{11.000000}{13.200000}\selectfont \(\displaystyle {45}\)}%
\end{pgfscope}%
\begin{pgfscope}%
\pgfsetbuttcap%
\pgfsetroundjoin%
\definecolor{currentfill}{rgb}{0.000000,0.000000,0.000000}%
\pgfsetfillcolor{currentfill}%
\pgfsetlinewidth{0.803000pt}%
\definecolor{currentstroke}{rgb}{0.000000,0.000000,0.000000}%
\pgfsetstrokecolor{currentstroke}%
\pgfsetdash{}{0pt}%
\pgfsys@defobject{currentmarker}{\pgfqpoint{0.000000in}{-0.048611in}}{\pgfqpoint{0.000000in}{0.000000in}}{%
\pgfpathmoveto{\pgfqpoint{0.000000in}{0.000000in}}%
\pgfpathlineto{\pgfqpoint{0.000000in}{-0.048611in}}%
\pgfusepath{stroke,fill}%
}%
\begin{pgfscope}%
\pgfsys@transformshift{4.060561in}{0.883555in}%
\pgfsys@useobject{currentmarker}{}%
\end{pgfscope}%
\end{pgfscope}%
\begin{pgfscope}%
\definecolor{textcolor}{rgb}{0.000000,0.000000,0.000000}%
\pgfsetstrokecolor{textcolor}%
\pgfsetfillcolor{textcolor}%
\pgftext[x=4.060561in,y=0.786333in,,top]{\color{textcolor}\rmfamily\fontsize{11.000000}{13.200000}\selectfont \(\displaystyle {50}\)}%
\end{pgfscope}%
\begin{pgfscope}%
\pgfsetbuttcap%
\pgfsetroundjoin%
\definecolor{currentfill}{rgb}{0.000000,0.000000,0.000000}%
\pgfsetfillcolor{currentfill}%
\pgfsetlinewidth{0.803000pt}%
\definecolor{currentstroke}{rgb}{0.000000,0.000000,0.000000}%
\pgfsetstrokecolor{currentstroke}%
\pgfsetdash{}{0pt}%
\pgfsys@defobject{currentmarker}{\pgfqpoint{0.000000in}{-0.048611in}}{\pgfqpoint{0.000000in}{0.000000in}}{%
\pgfpathmoveto{\pgfqpoint{0.000000in}{0.000000in}}%
\pgfpathlineto{\pgfqpoint{0.000000in}{-0.048611in}}%
\pgfusepath{stroke,fill}%
}%
\begin{pgfscope}%
\pgfsys@transformshift{4.588970in}{0.883555in}%
\pgfsys@useobject{currentmarker}{}%
\end{pgfscope}%
\end{pgfscope}%
\begin{pgfscope}%
\definecolor{textcolor}{rgb}{0.000000,0.000000,0.000000}%
\pgfsetstrokecolor{textcolor}%
\pgfsetfillcolor{textcolor}%
\pgftext[x=4.588970in,y=0.786333in,,top]{\color{textcolor}\rmfamily\fontsize{11.000000}{13.200000}\selectfont \(\displaystyle {55}\)}%
\end{pgfscope}%
\begin{pgfscope}%
\pgfsetbuttcap%
\pgfsetroundjoin%
\definecolor{currentfill}{rgb}{0.000000,0.000000,0.000000}%
\pgfsetfillcolor{currentfill}%
\pgfsetlinewidth{0.803000pt}%
\definecolor{currentstroke}{rgb}{0.000000,0.000000,0.000000}%
\pgfsetstrokecolor{currentstroke}%
\pgfsetdash{}{0pt}%
\pgfsys@defobject{currentmarker}{\pgfqpoint{0.000000in}{-0.048611in}}{\pgfqpoint{0.000000in}{0.000000in}}{%
\pgfpathmoveto{\pgfqpoint{0.000000in}{0.000000in}}%
\pgfpathlineto{\pgfqpoint{0.000000in}{-0.048611in}}%
\pgfusepath{stroke,fill}%
}%
\begin{pgfscope}%
\pgfsys@transformshift{5.117379in}{0.883555in}%
\pgfsys@useobject{currentmarker}{}%
\end{pgfscope}%
\end{pgfscope}%
\begin{pgfscope}%
\definecolor{textcolor}{rgb}{0.000000,0.000000,0.000000}%
\pgfsetstrokecolor{textcolor}%
\pgfsetfillcolor{textcolor}%
\pgftext[x=5.117379in,y=0.786333in,,top]{\color{textcolor}\rmfamily\fontsize{11.000000}{13.200000}\selectfont \(\displaystyle {60}\)}%
\end{pgfscope}%
\begin{pgfscope}%
\definecolor{textcolor}{rgb}{0.000000,0.000000,0.000000}%
\pgfsetstrokecolor{textcolor}%
\pgfsetfillcolor{textcolor}%
\pgftext[x=3.003743in,y=0.595592in,,top]{\color{textcolor}\rmfamily\fontsize{16.000000}{19.200000}\selectfont Age}%
\end{pgfscope}%
\begin{pgfscope}%
\pgfsetbuttcap%
\pgfsetroundjoin%
\definecolor{currentfill}{rgb}{0.000000,0.000000,0.000000}%
\pgfsetfillcolor{currentfill}%
\pgfsetlinewidth{0.803000pt}%
\definecolor{currentstroke}{rgb}{0.000000,0.000000,0.000000}%
\pgfsetstrokecolor{currentstroke}%
\pgfsetdash{}{0pt}%
\pgfsys@defobject{currentmarker}{\pgfqpoint{-0.048611in}{0.000000in}}{\pgfqpoint{0.000000in}{0.000000in}}{%
\pgfpathmoveto{\pgfqpoint{0.000000in}{0.000000in}}%
\pgfpathlineto{\pgfqpoint{-0.048611in}{0.000000in}}%
\pgfusepath{stroke,fill}%
}%
\begin{pgfscope}%
\pgfsys@transformshift{0.678743in}{0.883555in}%
\pgfsys@useobject{currentmarker}{}%
\end{pgfscope}%
\end{pgfscope}%
\begin{pgfscope}%
\definecolor{textcolor}{rgb}{0.000000,0.000000,0.000000}%
\pgfsetstrokecolor{textcolor}%
\pgfsetfillcolor{textcolor}%
\pgftext[x=0.268904in, y=0.830748in, left, base]{\color{textcolor}\rmfamily\fontsize{11.000000}{13.200000}\selectfont \(\displaystyle {-1.0}\)}%
\end{pgfscope}%
\begin{pgfscope}%
\pgfsetbuttcap%
\pgfsetroundjoin%
\definecolor{currentfill}{rgb}{0.000000,0.000000,0.000000}%
\pgfsetfillcolor{currentfill}%
\pgfsetlinewidth{0.803000pt}%
\definecolor{currentstroke}{rgb}{0.000000,0.000000,0.000000}%
\pgfsetstrokecolor{currentstroke}%
\pgfsetdash{}{0pt}%
\pgfsys@defobject{currentmarker}{\pgfqpoint{-0.048611in}{0.000000in}}{\pgfqpoint{0.000000in}{0.000000in}}{%
\pgfpathmoveto{\pgfqpoint{0.000000in}{0.000000in}}%
\pgfpathlineto{\pgfqpoint{-0.048611in}{0.000000in}}%
\pgfusepath{stroke,fill}%
}%
\begin{pgfscope}%
\pgfsys@transformshift{0.678743in}{1.286222in}%
\pgfsys@useobject{currentmarker}{}%
\end{pgfscope}%
\end{pgfscope}%
\begin{pgfscope}%
\definecolor{textcolor}{rgb}{0.000000,0.000000,0.000000}%
\pgfsetstrokecolor{textcolor}%
\pgfsetfillcolor{textcolor}%
\pgftext[x=0.268904in, y=1.233415in, left, base]{\color{textcolor}\rmfamily\fontsize{11.000000}{13.200000}\selectfont \(\displaystyle {-0.8}\)}%
\end{pgfscope}%
\begin{pgfscope}%
\pgfsetbuttcap%
\pgfsetroundjoin%
\definecolor{currentfill}{rgb}{0.000000,0.000000,0.000000}%
\pgfsetfillcolor{currentfill}%
\pgfsetlinewidth{0.803000pt}%
\definecolor{currentstroke}{rgb}{0.000000,0.000000,0.000000}%
\pgfsetstrokecolor{currentstroke}%
\pgfsetdash{}{0pt}%
\pgfsys@defobject{currentmarker}{\pgfqpoint{-0.048611in}{0.000000in}}{\pgfqpoint{0.000000in}{0.000000in}}{%
\pgfpathmoveto{\pgfqpoint{0.000000in}{0.000000in}}%
\pgfpathlineto{\pgfqpoint{-0.048611in}{0.000000in}}%
\pgfusepath{stroke,fill}%
}%
\begin{pgfscope}%
\pgfsys@transformshift{0.678743in}{1.688888in}%
\pgfsys@useobject{currentmarker}{}%
\end{pgfscope}%
\end{pgfscope}%
\begin{pgfscope}%
\definecolor{textcolor}{rgb}{0.000000,0.000000,0.000000}%
\pgfsetstrokecolor{textcolor}%
\pgfsetfillcolor{textcolor}%
\pgftext[x=0.268904in, y=1.636082in, left, base]{\color{textcolor}\rmfamily\fontsize{11.000000}{13.200000}\selectfont \(\displaystyle {-0.6}\)}%
\end{pgfscope}%
\begin{pgfscope}%
\pgfsetbuttcap%
\pgfsetroundjoin%
\definecolor{currentfill}{rgb}{0.000000,0.000000,0.000000}%
\pgfsetfillcolor{currentfill}%
\pgfsetlinewidth{0.803000pt}%
\definecolor{currentstroke}{rgb}{0.000000,0.000000,0.000000}%
\pgfsetstrokecolor{currentstroke}%
\pgfsetdash{}{0pt}%
\pgfsys@defobject{currentmarker}{\pgfqpoint{-0.048611in}{0.000000in}}{\pgfqpoint{0.000000in}{0.000000in}}{%
\pgfpathmoveto{\pgfqpoint{0.000000in}{0.000000in}}%
\pgfpathlineto{\pgfqpoint{-0.048611in}{0.000000in}}%
\pgfusepath{stroke,fill}%
}%
\begin{pgfscope}%
\pgfsys@transformshift{0.678743in}{2.091555in}%
\pgfsys@useobject{currentmarker}{}%
\end{pgfscope}%
\end{pgfscope}%
\begin{pgfscope}%
\definecolor{textcolor}{rgb}{0.000000,0.000000,0.000000}%
\pgfsetstrokecolor{textcolor}%
\pgfsetfillcolor{textcolor}%
\pgftext[x=0.268904in, y=2.038748in, left, base]{\color{textcolor}\rmfamily\fontsize{11.000000}{13.200000}\selectfont \(\displaystyle {-0.4}\)}%
\end{pgfscope}%
\begin{pgfscope}%
\pgfsetbuttcap%
\pgfsetroundjoin%
\definecolor{currentfill}{rgb}{0.000000,0.000000,0.000000}%
\pgfsetfillcolor{currentfill}%
\pgfsetlinewidth{0.803000pt}%
\definecolor{currentstroke}{rgb}{0.000000,0.000000,0.000000}%
\pgfsetstrokecolor{currentstroke}%
\pgfsetdash{}{0pt}%
\pgfsys@defobject{currentmarker}{\pgfqpoint{-0.048611in}{0.000000in}}{\pgfqpoint{0.000000in}{0.000000in}}{%
\pgfpathmoveto{\pgfqpoint{0.000000in}{0.000000in}}%
\pgfpathlineto{\pgfqpoint{-0.048611in}{0.000000in}}%
\pgfusepath{stroke,fill}%
}%
\begin{pgfscope}%
\pgfsys@transformshift{0.678743in}{2.494222in}%
\pgfsys@useobject{currentmarker}{}%
\end{pgfscope}%
\end{pgfscope}%
\begin{pgfscope}%
\definecolor{textcolor}{rgb}{0.000000,0.000000,0.000000}%
\pgfsetstrokecolor{textcolor}%
\pgfsetfillcolor{textcolor}%
\pgftext[x=0.268904in, y=2.441415in, left, base]{\color{textcolor}\rmfamily\fontsize{11.000000}{13.200000}\selectfont \(\displaystyle {-0.2}\)}%
\end{pgfscope}%
\begin{pgfscope}%
\pgfsetbuttcap%
\pgfsetroundjoin%
\definecolor{currentfill}{rgb}{0.000000,0.000000,0.000000}%
\pgfsetfillcolor{currentfill}%
\pgfsetlinewidth{0.803000pt}%
\definecolor{currentstroke}{rgb}{0.000000,0.000000,0.000000}%
\pgfsetstrokecolor{currentstroke}%
\pgfsetdash{}{0pt}%
\pgfsys@defobject{currentmarker}{\pgfqpoint{-0.048611in}{0.000000in}}{\pgfqpoint{0.000000in}{0.000000in}}{%
\pgfpathmoveto{\pgfqpoint{0.000000in}{0.000000in}}%
\pgfpathlineto{\pgfqpoint{-0.048611in}{0.000000in}}%
\pgfusepath{stroke,fill}%
}%
\begin{pgfscope}%
\pgfsys@transformshift{0.678743in}{2.896888in}%
\pgfsys@useobject{currentmarker}{}%
\end{pgfscope}%
\end{pgfscope}%
\begin{pgfscope}%
\definecolor{textcolor}{rgb}{0.000000,0.000000,0.000000}%
\pgfsetstrokecolor{textcolor}%
\pgfsetfillcolor{textcolor}%
\pgftext[x=0.387192in, y=2.844082in, left, base]{\color{textcolor}\rmfamily\fontsize{11.000000}{13.200000}\selectfont \(\displaystyle {0.0}\)}%
\end{pgfscope}%
\begin{pgfscope}%
\pgfsetbuttcap%
\pgfsetroundjoin%
\definecolor{currentfill}{rgb}{0.000000,0.000000,0.000000}%
\pgfsetfillcolor{currentfill}%
\pgfsetlinewidth{0.803000pt}%
\definecolor{currentstroke}{rgb}{0.000000,0.000000,0.000000}%
\pgfsetstrokecolor{currentstroke}%
\pgfsetdash{}{0pt}%
\pgfsys@defobject{currentmarker}{\pgfqpoint{-0.048611in}{0.000000in}}{\pgfqpoint{0.000000in}{0.000000in}}{%
\pgfpathmoveto{\pgfqpoint{0.000000in}{0.000000in}}%
\pgfpathlineto{\pgfqpoint{-0.048611in}{0.000000in}}%
\pgfusepath{stroke,fill}%
}%
\begin{pgfscope}%
\pgfsys@transformshift{0.678743in}{3.299555in}%
\pgfsys@useobject{currentmarker}{}%
\end{pgfscope}%
\end{pgfscope}%
\begin{pgfscope}%
\definecolor{textcolor}{rgb}{0.000000,0.000000,0.000000}%
\pgfsetstrokecolor{textcolor}%
\pgfsetfillcolor{textcolor}%
\pgftext[x=0.387192in, y=3.246748in, left, base]{\color{textcolor}\rmfamily\fontsize{11.000000}{13.200000}\selectfont \(\displaystyle {0.2}\)}%
\end{pgfscope}%
\begin{pgfscope}%
\pgfsetbuttcap%
\pgfsetroundjoin%
\definecolor{currentfill}{rgb}{0.000000,0.000000,0.000000}%
\pgfsetfillcolor{currentfill}%
\pgfsetlinewidth{0.803000pt}%
\definecolor{currentstroke}{rgb}{0.000000,0.000000,0.000000}%
\pgfsetstrokecolor{currentstroke}%
\pgfsetdash{}{0pt}%
\pgfsys@defobject{currentmarker}{\pgfqpoint{-0.048611in}{0.000000in}}{\pgfqpoint{0.000000in}{0.000000in}}{%
\pgfpathmoveto{\pgfqpoint{0.000000in}{0.000000in}}%
\pgfpathlineto{\pgfqpoint{-0.048611in}{0.000000in}}%
\pgfusepath{stroke,fill}%
}%
\begin{pgfscope}%
\pgfsys@transformshift{0.678743in}{3.702222in}%
\pgfsys@useobject{currentmarker}{}%
\end{pgfscope}%
\end{pgfscope}%
\begin{pgfscope}%
\definecolor{textcolor}{rgb}{0.000000,0.000000,0.000000}%
\pgfsetstrokecolor{textcolor}%
\pgfsetfillcolor{textcolor}%
\pgftext[x=0.387192in, y=3.649415in, left, base]{\color{textcolor}\rmfamily\fontsize{11.000000}{13.200000}\selectfont \(\displaystyle {0.4}\)}%
\end{pgfscope}%
\begin{pgfscope}%
\definecolor{textcolor}{rgb}{0.000000,0.000000,0.000000}%
\pgfsetstrokecolor{textcolor}%
\pgfsetfillcolor{textcolor}%
\pgftext[x=0.213349in,y=2.393555in,,bottom,rotate=90.000000]{\color{textcolor}\rmfamily\fontsize{16.000000}{19.200000}\selectfont Average log wage}%
\end{pgfscope}%
\begin{pgfscope}%
\pgfpathrectangle{\pgfqpoint{0.678743in}{0.883555in}}{\pgfqpoint{4.650000in}{3.020000in}}%
\pgfusepath{clip}%
\pgfsetrectcap%
\pgfsetroundjoin%
\pgfsetlinewidth{1.505625pt}%
\definecolor{currentstroke}{rgb}{1.000000,0.000000,0.000000}%
\pgfsetstrokecolor{currentstroke}%
\pgfsetdash{}{0pt}%
\pgfpathmoveto{\pgfqpoint{0.890106in}{1.306225in}}%
\pgfpathlineto{\pgfqpoint{0.995788in}{1.461547in}}%
\pgfpathlineto{\pgfqpoint{1.101470in}{1.621368in}}%
\pgfpathlineto{\pgfqpoint{1.207152in}{1.752145in}}%
\pgfpathlineto{\pgfqpoint{1.312834in}{1.875406in}}%
\pgfpathlineto{\pgfqpoint{1.418516in}{2.000256in}}%
\pgfpathlineto{\pgfqpoint{1.524197in}{2.092481in}}%
\pgfpathlineto{\pgfqpoint{1.629879in}{2.179383in}}%
\pgfpathlineto{\pgfqpoint{1.735561in}{2.250350in}}%
\pgfpathlineto{\pgfqpoint{1.841243in}{2.296584in}}%
\pgfpathlineto{\pgfqpoint{1.946925in}{2.345809in}}%
\pgfpathlineto{\pgfqpoint{2.052606in}{2.378037in}}%
\pgfpathlineto{\pgfqpoint{2.158288in}{2.411638in}}%
\pgfpathlineto{\pgfqpoint{2.263970in}{2.436746in}}%
\pgfpathlineto{\pgfqpoint{2.369652in}{2.476907in}}%
\pgfpathlineto{\pgfqpoint{2.475334in}{2.498796in}}%
\pgfpathlineto{\pgfqpoint{2.581016in}{2.498797in}}%
\pgfpathlineto{\pgfqpoint{2.686697in}{2.504093in}}%
\pgfpathlineto{\pgfqpoint{2.792379in}{2.533243in}}%
\pgfpathlineto{\pgfqpoint{2.898061in}{2.556020in}}%
\pgfpathlineto{\pgfqpoint{3.003743in}{2.514448in}}%
\pgfpathlineto{\pgfqpoint{3.109425in}{2.524375in}}%
\pgfpathlineto{\pgfqpoint{3.215106in}{2.516165in}}%
\pgfpathlineto{\pgfqpoint{3.320788in}{2.510009in}}%
\pgfpathlineto{\pgfqpoint{3.426470in}{2.510623in}}%
\pgfpathlineto{\pgfqpoint{3.532152in}{2.478002in}}%
\pgfpathlineto{\pgfqpoint{3.637834in}{2.427259in}}%
\pgfpathlineto{\pgfqpoint{3.743516in}{2.416194in}}%
\pgfpathlineto{\pgfqpoint{3.849197in}{2.360189in}}%
\pgfpathlineto{\pgfqpoint{3.954879in}{2.352945in}}%
\pgfpathlineto{\pgfqpoint{4.060561in}{2.329860in}}%
\pgfpathlineto{\pgfqpoint{4.166243in}{2.304218in}}%
\pgfpathlineto{\pgfqpoint{4.271925in}{2.296382in}}%
\pgfpathlineto{\pgfqpoint{4.377606in}{2.236513in}}%
\pgfpathlineto{\pgfqpoint{4.483288in}{2.186352in}}%
\pgfpathlineto{\pgfqpoint{4.588970in}{2.172680in}}%
\pgfpathlineto{\pgfqpoint{4.694652in}{2.164191in}}%
\pgfpathlineto{\pgfqpoint{4.800334in}{2.131768in}}%
\pgfpathlineto{\pgfqpoint{4.906016in}{2.176958in}}%
\pgfpathlineto{\pgfqpoint{5.011697in}{2.237030in}}%
\pgfpathlineto{\pgfqpoint{5.117379in}{2.240788in}}%
\pgfusepath{stroke}%
\end{pgfscope}%
\begin{pgfscope}%
\pgfsetrectcap%
\pgfsetmiterjoin%
\pgfsetlinewidth{0.803000pt}%
\definecolor{currentstroke}{rgb}{0.000000,0.000000,0.000000}%
\pgfsetstrokecolor{currentstroke}%
\pgfsetdash{}{0pt}%
\pgfpathmoveto{\pgfqpoint{0.678743in}{0.883555in}}%
\pgfpathlineto{\pgfqpoint{0.678743in}{3.903555in}}%
\pgfusepath{stroke}%
\end{pgfscope}%
\begin{pgfscope}%
\pgfsetrectcap%
\pgfsetmiterjoin%
\pgfsetlinewidth{0.803000pt}%
\definecolor{currentstroke}{rgb}{0.000000,0.000000,0.000000}%
\pgfsetstrokecolor{currentstroke}%
\pgfsetdash{}{0pt}%
\pgfpathmoveto{\pgfqpoint{5.328743in}{0.883555in}}%
\pgfpathlineto{\pgfqpoint{5.328743in}{3.903555in}}%
\pgfusepath{stroke}%
\end{pgfscope}%
\begin{pgfscope}%
\pgfsetrectcap%
\pgfsetmiterjoin%
\pgfsetlinewidth{0.803000pt}%
\definecolor{currentstroke}{rgb}{0.000000,0.000000,0.000000}%
\pgfsetstrokecolor{currentstroke}%
\pgfsetdash{}{0pt}%
\pgfpathmoveto{\pgfqpoint{0.678743in}{0.883555in}}%
\pgfpathlineto{\pgfqpoint{5.328743in}{0.883555in}}%
\pgfusepath{stroke}%
\end{pgfscope}%
\begin{pgfscope}%
\pgfsetrectcap%
\pgfsetmiterjoin%
\pgfsetlinewidth{0.803000pt}%
\definecolor{currentstroke}{rgb}{0.000000,0.000000,0.000000}%
\pgfsetstrokecolor{currentstroke}%
\pgfsetdash{}{0pt}%
\pgfpathmoveto{\pgfqpoint{0.678743in}{3.903555in}}%
\pgfpathlineto{\pgfqpoint{5.328743in}{3.903555in}}%
\pgfusepath{stroke}%
\end{pgfscope}%
\begin{pgfscope}%
\pgfsetbuttcap%
\pgfsetmiterjoin%
\definecolor{currentfill}{rgb}{0.300000,0.300000,0.300000}%
\pgfsetfillcolor{currentfill}%
\pgfsetfillopacity{0.500000}%
\pgfsetlinewidth{1.003750pt}%
\definecolor{currentstroke}{rgb}{0.300000,0.300000,0.300000}%
\pgfsetstrokecolor{currentstroke}%
\pgfsetstrokeopacity{0.500000}%
\pgfsetdash{}{0pt}%
\pgfpathmoveto{\pgfqpoint{1.544689in}{-0.027778in}}%
\pgfpathlineto{\pgfqpoint{4.518352in}{-0.027778in}}%
\pgfpathquadraticcurveto{\pgfqpoint{4.557241in}{-0.027778in}}{\pgfqpoint{4.557241in}{0.011111in}}%
\pgfpathlineto{\pgfqpoint{4.557241in}{0.266666in}}%
\pgfpathquadraticcurveto{\pgfqpoint{4.557241in}{0.305555in}}{\pgfqpoint{4.518352in}{0.305555in}}%
\pgfpathlineto{\pgfqpoint{1.544689in}{0.305555in}}%
\pgfpathquadraticcurveto{\pgfqpoint{1.505800in}{0.305555in}}{\pgfqpoint{1.505800in}{0.266666in}}%
\pgfpathlineto{\pgfqpoint{1.505800in}{0.011111in}}%
\pgfpathquadraticcurveto{\pgfqpoint{1.505800in}{-0.027778in}}{\pgfqpoint{1.544689in}{-0.027778in}}%
\pgfpathclose%
\pgfusepath{stroke,fill}%
\end{pgfscope}%
\begin{pgfscope}%
\pgfsetbuttcap%
\pgfsetmiterjoin%
\definecolor{currentfill}{rgb}{1.000000,1.000000,1.000000}%
\pgfsetfillcolor{currentfill}%
\pgfsetlinewidth{1.003750pt}%
\definecolor{currentstroke}{rgb}{0.800000,0.800000,0.800000}%
\pgfsetstrokecolor{currentstroke}%
\pgfsetdash{}{0pt}%
\pgfpathmoveto{\pgfqpoint{1.516911in}{0.000000in}}%
\pgfpathlineto{\pgfqpoint{4.490574in}{0.000000in}}%
\pgfpathquadraticcurveto{\pgfqpoint{4.529463in}{0.000000in}}{\pgfqpoint{4.529463in}{0.038889in}}%
\pgfpathlineto{\pgfqpoint{4.529463in}{0.294444in}}%
\pgfpathquadraticcurveto{\pgfqpoint{4.529463in}{0.333333in}}{\pgfqpoint{4.490574in}{0.333333in}}%
\pgfpathlineto{\pgfqpoint{1.516911in}{0.333333in}}%
\pgfpathquadraticcurveto{\pgfqpoint{1.478022in}{0.333333in}}{\pgfqpoint{1.478022in}{0.294444in}}%
\pgfpathlineto{\pgfqpoint{1.478022in}{0.038889in}}%
\pgfpathquadraticcurveto{\pgfqpoint{1.478022in}{0.000000in}}{\pgfqpoint{1.516911in}{0.000000in}}%
\pgfpathclose%
\pgfusepath{stroke,fill}%
\end{pgfscope}%
\begin{pgfscope}%
\pgfsetrectcap%
\pgfsetroundjoin%
\pgfsetlinewidth{1.505625pt}%
\definecolor{currentstroke}{rgb}{1.000000,0.000000,0.000000}%
\pgfsetstrokecolor{currentstroke}%
\pgfsetdash{}{0pt}%
\pgfpathmoveto{\pgfqpoint{1.555800in}{0.184722in}}%
\pgfpathlineto{\pgfqpoint{1.944689in}{0.184722in}}%
\pgfusepath{stroke}%
\end{pgfscope}%
\begin{pgfscope}%
\definecolor{textcolor}{rgb}{0.000000,0.000000,0.000000}%
\pgfsetstrokecolor{textcolor}%
\pgfsetfillcolor{textcolor}%
\pgftext[x=2.100245in,y=0.116667in,left,base]{\color{textcolor}\rmfamily\fontsize{14.000000}{16.800000}\selectfont Simulations}%
\end{pgfscope}%
\begin{pgfscope}%
\pgfsetbuttcap%
\pgfsetroundjoin%
\pgfsetlinewidth{1.003750pt}%
\definecolor{currentstroke}{rgb}{0.000000,0.000000,1.000000}%
\pgfsetstrokecolor{currentstroke}%
\pgfsetdash{}{0pt}%
\pgfpathmoveto{\pgfqpoint{3.680140in}{0.113917in}}%
\pgfpathcurveto{\pgfqpoint{3.694405in}{0.113917in}}{\pgfqpoint{3.708088in}{0.119585in}}{\pgfqpoint{3.718176in}{0.129672in}}%
\pgfpathcurveto{\pgfqpoint{3.728263in}{0.139759in}}{\pgfqpoint{3.733931in}{0.153443in}}{\pgfqpoint{3.733931in}{0.167708in}}%
\pgfpathcurveto{\pgfqpoint{3.733931in}{0.181974in}}{\pgfqpoint{3.728263in}{0.195657in}}{\pgfqpoint{3.718176in}{0.205744in}}%
\pgfpathcurveto{\pgfqpoint{3.708088in}{0.215832in}}{\pgfqpoint{3.694405in}{0.221500in}}{\pgfqpoint{3.680140in}{0.221500in}}%
\pgfpathcurveto{\pgfqpoint{3.665874in}{0.221500in}}{\pgfqpoint{3.652191in}{0.215832in}}{\pgfqpoint{3.642103in}{0.205744in}}%
\pgfpathcurveto{\pgfqpoint{3.632016in}{0.195657in}}{\pgfqpoint{3.626348in}{0.181974in}}{\pgfqpoint{3.626348in}{0.167708in}}%
\pgfpathcurveto{\pgfqpoint{3.626348in}{0.153443in}}{\pgfqpoint{3.632016in}{0.139759in}}{\pgfqpoint{3.642103in}{0.129672in}}%
\pgfpathcurveto{\pgfqpoint{3.652191in}{0.119585in}}{\pgfqpoint{3.665874in}{0.113917in}}{\pgfqpoint{3.680140in}{0.113917in}}%
\pgfpathclose%
\pgfusepath{stroke}%
\end{pgfscope}%
\begin{pgfscope}%
\definecolor{textcolor}{rgb}{0.000000,0.000000,0.000000}%
\pgfsetstrokecolor{textcolor}%
\pgfsetfillcolor{textcolor}%
\pgftext[x=4.030140in,y=0.116667in,left,base]{\color{textcolor}\rmfamily\fontsize{14.000000}{16.800000}\selectfont Data}%
\end{pgfscope}%
\end{pgfpicture}%
\makeatother%
\endgroup%
 } 
		\end{subfigure}
		\begin{subfigure}{.49\textwidth}
			\centering
			% include second image
			\caption{Men}
			\label{fig:ef}
			\scalebox{0.5}{%% Creator: Matplotlib, PGF backend
%%
%% To include the figure in your LaTeX document, write
%%   \input{<filename>.pgf}
%%
%% Make sure the required packages are loaded in your preamble
%%   \usepackage{pgf}
%%
%% Figures using additional raster images can only be included by \input if
%% they are in the same directory as the main LaTeX file. For loading figures
%% from other directories you can use the `import` package
%%   \usepackage{import}
%% and then include the figures with
%%   \import{<path to file>}{<filename>.pgf}
%%
%% Matplotlib used the following preamble
%%
\begingroup%
\makeatletter%
\begin{pgfpicture}%
\pgfpathrectangle{\pgfpointorigin}{\pgfqpoint{5.328743in}{3.903555in}}%
\pgfusepath{use as bounding box, clip}%
\begin{pgfscope}%
\pgfsetbuttcap%
\pgfsetmiterjoin%
\definecolor{currentfill}{rgb}{1.000000,1.000000,1.000000}%
\pgfsetfillcolor{currentfill}%
\pgfsetlinewidth{0.000000pt}%
\definecolor{currentstroke}{rgb}{1.000000,1.000000,1.000000}%
\pgfsetstrokecolor{currentstroke}%
\pgfsetdash{}{0pt}%
\pgfpathmoveto{\pgfqpoint{0.000000in}{0.000000in}}%
\pgfpathlineto{\pgfqpoint{5.328743in}{0.000000in}}%
\pgfpathlineto{\pgfqpoint{5.328743in}{3.903555in}}%
\pgfpathlineto{\pgfqpoint{0.000000in}{3.903555in}}%
\pgfpathclose%
\pgfusepath{fill}%
\end{pgfscope}%
\begin{pgfscope}%
\pgfsetbuttcap%
\pgfsetmiterjoin%
\definecolor{currentfill}{rgb}{1.000000,1.000000,1.000000}%
\pgfsetfillcolor{currentfill}%
\pgfsetlinewidth{0.000000pt}%
\definecolor{currentstroke}{rgb}{0.000000,0.000000,0.000000}%
\pgfsetstrokecolor{currentstroke}%
\pgfsetstrokeopacity{0.000000}%
\pgfsetdash{}{0pt}%
\pgfpathmoveto{\pgfqpoint{0.678743in}{0.883555in}}%
\pgfpathlineto{\pgfqpoint{5.328743in}{0.883555in}}%
\pgfpathlineto{\pgfqpoint{5.328743in}{3.903555in}}%
\pgfpathlineto{\pgfqpoint{0.678743in}{3.903555in}}%
\pgfpathclose%
\pgfusepath{fill}%
\end{pgfscope}%
\begin{pgfscope}%
\pgfpathrectangle{\pgfqpoint{0.678743in}{0.883555in}}{\pgfqpoint{4.650000in}{3.020000in}}%
\pgfusepath{clip}%
\pgfsetbuttcap%
\pgfsetroundjoin%
\pgfsetlinewidth{1.003750pt}%
\definecolor{currentstroke}{rgb}{0.000000,0.000000,1.000000}%
\pgfsetstrokecolor{currentstroke}%
\pgfsetdash{}{0pt}%
\pgfpathmoveto{\pgfqpoint{0.891328in}{1.913998in}}%
\pgfpathcurveto{\pgfqpoint{0.905594in}{1.913998in}}{\pgfqpoint{0.919277in}{1.919666in}}{\pgfqpoint{0.929365in}{1.929754in}}%
\pgfpathcurveto{\pgfqpoint{0.939452in}{1.939841in}}{\pgfqpoint{0.945120in}{1.953524in}}{\pgfqpoint{0.945120in}{1.967790in}}%
\pgfpathcurveto{\pgfqpoint{0.945120in}{1.982055in}}{\pgfqpoint{0.939452in}{1.995739in}}{\pgfqpoint{0.929365in}{2.005826in}}%
\pgfpathcurveto{\pgfqpoint{0.919277in}{2.015913in}}{\pgfqpoint{0.905594in}{2.021581in}}{\pgfqpoint{0.891328in}{2.021581in}}%
\pgfpathcurveto{\pgfqpoint{0.877063in}{2.021581in}}{\pgfqpoint{0.863379in}{2.015913in}}{\pgfqpoint{0.853292in}{2.005826in}}%
\pgfpathcurveto{\pgfqpoint{0.843205in}{1.995739in}}{\pgfqpoint{0.837537in}{1.982055in}}{\pgfqpoint{0.837537in}{1.967790in}}%
\pgfpathcurveto{\pgfqpoint{0.837537in}{1.953524in}}{\pgfqpoint{0.843205in}{1.939841in}}{\pgfqpoint{0.853292in}{1.929754in}}%
\pgfpathcurveto{\pgfqpoint{0.863379in}{1.919666in}}{\pgfqpoint{0.877063in}{1.913998in}}{\pgfqpoint{0.891328in}{1.913998in}}%
\pgfpathclose%
\pgfusepath{stroke}%
\end{pgfscope}%
\begin{pgfscope}%
\pgfpathrectangle{\pgfqpoint{0.678743in}{0.883555in}}{\pgfqpoint{4.650000in}{3.020000in}}%
\pgfusepath{clip}%
\pgfsetbuttcap%
\pgfsetroundjoin%
\pgfsetlinewidth{1.003750pt}%
\definecolor{currentstroke}{rgb}{0.000000,0.000000,1.000000}%
\pgfsetstrokecolor{currentstroke}%
\pgfsetdash{}{0pt}%
\pgfpathmoveto{\pgfqpoint{0.996949in}{2.035451in}}%
\pgfpathcurveto{\pgfqpoint{1.011215in}{2.035451in}}{\pgfqpoint{1.024898in}{2.041119in}}{\pgfqpoint{1.034985in}{2.051206in}}%
\pgfpathcurveto{\pgfqpoint{1.045073in}{2.061293in}}{\pgfqpoint{1.050740in}{2.074977in}}{\pgfqpoint{1.050740in}{2.089242in}}%
\pgfpathcurveto{\pgfqpoint{1.050740in}{2.103508in}}{\pgfqpoint{1.045073in}{2.117191in}}{\pgfqpoint{1.034985in}{2.127278in}}%
\pgfpathcurveto{\pgfqpoint{1.024898in}{2.137366in}}{\pgfqpoint{1.011215in}{2.143034in}}{\pgfqpoint{0.996949in}{2.143034in}}%
\pgfpathcurveto{\pgfqpoint{0.982683in}{2.143034in}}{\pgfqpoint{0.969000in}{2.137366in}}{\pgfqpoint{0.958913in}{2.127278in}}%
\pgfpathcurveto{\pgfqpoint{0.948825in}{2.117191in}}{\pgfqpoint{0.943158in}{2.103508in}}{\pgfqpoint{0.943158in}{2.089242in}}%
\pgfpathcurveto{\pgfqpoint{0.943158in}{2.074977in}}{\pgfqpoint{0.948825in}{2.061293in}}{\pgfqpoint{0.958913in}{2.051206in}}%
\pgfpathcurveto{\pgfqpoint{0.969000in}{2.041119in}}{\pgfqpoint{0.982683in}{2.035451in}}{\pgfqpoint{0.996949in}{2.035451in}}%
\pgfpathclose%
\pgfusepath{stroke}%
\end{pgfscope}%
\begin{pgfscope}%
\pgfpathrectangle{\pgfqpoint{0.678743in}{0.883555in}}{\pgfqpoint{4.650000in}{3.020000in}}%
\pgfusepath{clip}%
\pgfsetbuttcap%
\pgfsetroundjoin%
\pgfsetlinewidth{1.003750pt}%
\definecolor{currentstroke}{rgb}{0.000000,0.000000,1.000000}%
\pgfsetstrokecolor{currentstroke}%
\pgfsetdash{}{0pt}%
\pgfpathmoveto{\pgfqpoint{1.102570in}{2.158477in}}%
\pgfpathcurveto{\pgfqpoint{1.116835in}{2.158477in}}{\pgfqpoint{1.130519in}{2.164145in}}{\pgfqpoint{1.140606in}{2.174232in}}%
\pgfpathcurveto{\pgfqpoint{1.150693in}{2.184320in}}{\pgfqpoint{1.156361in}{2.198003in}}{\pgfqpoint{1.156361in}{2.212269in}}%
\pgfpathcurveto{\pgfqpoint{1.156361in}{2.226534in}}{\pgfqpoint{1.150693in}{2.240218in}}{\pgfqpoint{1.140606in}{2.250305in}}%
\pgfpathcurveto{\pgfqpoint{1.130519in}{2.260392in}}{\pgfqpoint{1.116835in}{2.266060in}}{\pgfqpoint{1.102570in}{2.266060in}}%
\pgfpathcurveto{\pgfqpoint{1.088304in}{2.266060in}}{\pgfqpoint{1.074621in}{2.260392in}}{\pgfqpoint{1.064533in}{2.250305in}}%
\pgfpathcurveto{\pgfqpoint{1.054446in}{2.240218in}}{\pgfqpoint{1.048778in}{2.226534in}}{\pgfqpoint{1.048778in}{2.212269in}}%
\pgfpathcurveto{\pgfqpoint{1.048778in}{2.198003in}}{\pgfqpoint{1.054446in}{2.184320in}}{\pgfqpoint{1.064533in}{2.174232in}}%
\pgfpathcurveto{\pgfqpoint{1.074621in}{2.164145in}}{\pgfqpoint{1.088304in}{2.158477in}}{\pgfqpoint{1.102570in}{2.158477in}}%
\pgfpathclose%
\pgfusepath{stroke}%
\end{pgfscope}%
\begin{pgfscope}%
\pgfpathrectangle{\pgfqpoint{0.678743in}{0.883555in}}{\pgfqpoint{4.650000in}{3.020000in}}%
\pgfusepath{clip}%
\pgfsetbuttcap%
\pgfsetroundjoin%
\pgfsetlinewidth{1.003750pt}%
\definecolor{currentstroke}{rgb}{0.000000,0.000000,1.000000}%
\pgfsetstrokecolor{currentstroke}%
\pgfsetdash{}{0pt}%
\pgfpathmoveto{\pgfqpoint{1.208190in}{2.269430in}}%
\pgfpathcurveto{\pgfqpoint{1.222456in}{2.269430in}}{\pgfqpoint{1.236139in}{2.275098in}}{\pgfqpoint{1.246227in}{2.285185in}}%
\pgfpathcurveto{\pgfqpoint{1.256314in}{2.295273in}}{\pgfqpoint{1.261982in}{2.308956in}}{\pgfqpoint{1.261982in}{2.323222in}}%
\pgfpathcurveto{\pgfqpoint{1.261982in}{2.337487in}}{\pgfqpoint{1.256314in}{2.351171in}}{\pgfqpoint{1.246227in}{2.361258in}}%
\pgfpathcurveto{\pgfqpoint{1.236139in}{2.371345in}}{\pgfqpoint{1.222456in}{2.377013in}}{\pgfqpoint{1.208190in}{2.377013in}}%
\pgfpathcurveto{\pgfqpoint{1.193925in}{2.377013in}}{\pgfqpoint{1.180241in}{2.371345in}}{\pgfqpoint{1.170154in}{2.361258in}}%
\pgfpathcurveto{\pgfqpoint{1.160067in}{2.351171in}}{\pgfqpoint{1.154399in}{2.337487in}}{\pgfqpoint{1.154399in}{2.323222in}}%
\pgfpathcurveto{\pgfqpoint{1.154399in}{2.308956in}}{\pgfqpoint{1.160067in}{2.295273in}}{\pgfqpoint{1.170154in}{2.285185in}}%
\pgfpathcurveto{\pgfqpoint{1.180241in}{2.275098in}}{\pgfqpoint{1.193925in}{2.269430in}}{\pgfqpoint{1.208190in}{2.269430in}}%
\pgfpathclose%
\pgfusepath{stroke}%
\end{pgfscope}%
\begin{pgfscope}%
\pgfpathrectangle{\pgfqpoint{0.678743in}{0.883555in}}{\pgfqpoint{4.650000in}{3.020000in}}%
\pgfusepath{clip}%
\pgfsetbuttcap%
\pgfsetroundjoin%
\pgfsetlinewidth{1.003750pt}%
\definecolor{currentstroke}{rgb}{0.000000,0.000000,1.000000}%
\pgfsetstrokecolor{currentstroke}%
\pgfsetdash{}{0pt}%
\pgfpathmoveto{\pgfqpoint{1.313811in}{2.404770in}}%
\pgfpathcurveto{\pgfqpoint{1.328077in}{2.404770in}}{\pgfqpoint{1.341760in}{2.410438in}}{\pgfqpoint{1.351847in}{2.420525in}}%
\pgfpathcurveto{\pgfqpoint{1.361935in}{2.430612in}}{\pgfqpoint{1.367603in}{2.444296in}}{\pgfqpoint{1.367603in}{2.458561in}}%
\pgfpathcurveto{\pgfqpoint{1.367603in}{2.472827in}}{\pgfqpoint{1.361935in}{2.486510in}}{\pgfqpoint{1.351847in}{2.496597in}}%
\pgfpathcurveto{\pgfqpoint{1.341760in}{2.506685in}}{\pgfqpoint{1.328077in}{2.512353in}}{\pgfqpoint{1.313811in}{2.512353in}}%
\pgfpathcurveto{\pgfqpoint{1.299545in}{2.512353in}}{\pgfqpoint{1.285862in}{2.506685in}}{\pgfqpoint{1.275775in}{2.496597in}}%
\pgfpathcurveto{\pgfqpoint{1.265688in}{2.486510in}}{\pgfqpoint{1.260020in}{2.472827in}}{\pgfqpoint{1.260020in}{2.458561in}}%
\pgfpathcurveto{\pgfqpoint{1.260020in}{2.444296in}}{\pgfqpoint{1.265688in}{2.430612in}}{\pgfqpoint{1.275775in}{2.420525in}}%
\pgfpathcurveto{\pgfqpoint{1.285862in}{2.410438in}}{\pgfqpoint{1.299545in}{2.404770in}}{\pgfqpoint{1.313811in}{2.404770in}}%
\pgfpathclose%
\pgfusepath{stroke}%
\end{pgfscope}%
\begin{pgfscope}%
\pgfpathrectangle{\pgfqpoint{0.678743in}{0.883555in}}{\pgfqpoint{4.650000in}{3.020000in}}%
\pgfusepath{clip}%
\pgfsetbuttcap%
\pgfsetroundjoin%
\pgfsetlinewidth{1.003750pt}%
\definecolor{currentstroke}{rgb}{0.000000,0.000000,1.000000}%
\pgfsetstrokecolor{currentstroke}%
\pgfsetdash{}{0pt}%
\pgfpathmoveto{\pgfqpoint{1.419432in}{2.499976in}}%
\pgfpathcurveto{\pgfqpoint{1.433698in}{2.499976in}}{\pgfqpoint{1.447381in}{2.505644in}}{\pgfqpoint{1.457468in}{2.515731in}}%
\pgfpathcurveto{\pgfqpoint{1.467556in}{2.525818in}}{\pgfqpoint{1.473223in}{2.539502in}}{\pgfqpoint{1.473223in}{2.553767in}}%
\pgfpathcurveto{\pgfqpoint{1.473223in}{2.568033in}}{\pgfqpoint{1.467556in}{2.581716in}}{\pgfqpoint{1.457468in}{2.591804in}}%
\pgfpathcurveto{\pgfqpoint{1.447381in}{2.601891in}}{\pgfqpoint{1.433698in}{2.607559in}}{\pgfqpoint{1.419432in}{2.607559in}}%
\pgfpathcurveto{\pgfqpoint{1.405166in}{2.607559in}}{\pgfqpoint{1.391483in}{2.601891in}}{\pgfqpoint{1.381396in}{2.591804in}}%
\pgfpathcurveto{\pgfqpoint{1.371308in}{2.581716in}}{\pgfqpoint{1.365640in}{2.568033in}}{\pgfqpoint{1.365640in}{2.553767in}}%
\pgfpathcurveto{\pgfqpoint{1.365640in}{2.539502in}}{\pgfqpoint{1.371308in}{2.525818in}}{\pgfqpoint{1.381396in}{2.515731in}}%
\pgfpathcurveto{\pgfqpoint{1.391483in}{2.505644in}}{\pgfqpoint{1.405166in}{2.499976in}}{\pgfqpoint{1.419432in}{2.499976in}}%
\pgfpathclose%
\pgfusepath{stroke}%
\end{pgfscope}%
\begin{pgfscope}%
\pgfpathrectangle{\pgfqpoint{0.678743in}{0.883555in}}{\pgfqpoint{4.650000in}{3.020000in}}%
\pgfusepath{clip}%
\pgfsetbuttcap%
\pgfsetroundjoin%
\pgfsetlinewidth{1.003750pt}%
\definecolor{currentstroke}{rgb}{0.000000,0.000000,1.000000}%
\pgfsetstrokecolor{currentstroke}%
\pgfsetdash{}{0pt}%
\pgfpathmoveto{\pgfqpoint{1.525053in}{2.560543in}}%
\pgfpathcurveto{\pgfqpoint{1.539318in}{2.560543in}}{\pgfqpoint{1.553002in}{2.566210in}}{\pgfqpoint{1.563089in}{2.576298in}}%
\pgfpathcurveto{\pgfqpoint{1.573176in}{2.586385in}}{\pgfqpoint{1.578844in}{2.600068in}}{\pgfqpoint{1.578844in}{2.614334in}}%
\pgfpathcurveto{\pgfqpoint{1.578844in}{2.628600in}}{\pgfqpoint{1.573176in}{2.642283in}}{\pgfqpoint{1.563089in}{2.652370in}}%
\pgfpathcurveto{\pgfqpoint{1.553002in}{2.662458in}}{\pgfqpoint{1.539318in}{2.668125in}}{\pgfqpoint{1.525053in}{2.668125in}}%
\pgfpathcurveto{\pgfqpoint{1.510787in}{2.668125in}}{\pgfqpoint{1.497104in}{2.662458in}}{\pgfqpoint{1.487016in}{2.652370in}}%
\pgfpathcurveto{\pgfqpoint{1.476929in}{2.642283in}}{\pgfqpoint{1.471261in}{2.628600in}}{\pgfqpoint{1.471261in}{2.614334in}}%
\pgfpathcurveto{\pgfqpoint{1.471261in}{2.600068in}}{\pgfqpoint{1.476929in}{2.586385in}}{\pgfqpoint{1.487016in}{2.576298in}}%
\pgfpathcurveto{\pgfqpoint{1.497104in}{2.566210in}}{\pgfqpoint{1.510787in}{2.560543in}}{\pgfqpoint{1.525053in}{2.560543in}}%
\pgfpathclose%
\pgfusepath{stroke}%
\end{pgfscope}%
\begin{pgfscope}%
\pgfpathrectangle{\pgfqpoint{0.678743in}{0.883555in}}{\pgfqpoint{4.650000in}{3.020000in}}%
\pgfusepath{clip}%
\pgfsetbuttcap%
\pgfsetroundjoin%
\pgfsetlinewidth{1.003750pt}%
\definecolor{currentstroke}{rgb}{0.000000,0.000000,1.000000}%
\pgfsetstrokecolor{currentstroke}%
\pgfsetdash{}{0pt}%
\pgfpathmoveto{\pgfqpoint{1.630673in}{2.643005in}}%
\pgfpathcurveto{\pgfqpoint{1.644939in}{2.643005in}}{\pgfqpoint{1.658622in}{2.648673in}}{\pgfqpoint{1.668710in}{2.658760in}}%
\pgfpathcurveto{\pgfqpoint{1.678797in}{2.668848in}}{\pgfqpoint{1.684465in}{2.682531in}}{\pgfqpoint{1.684465in}{2.696797in}}%
\pgfpathcurveto{\pgfqpoint{1.684465in}{2.711062in}}{\pgfqpoint{1.678797in}{2.724745in}}{\pgfqpoint{1.668710in}{2.734833in}}%
\pgfpathcurveto{\pgfqpoint{1.658622in}{2.744920in}}{\pgfqpoint{1.644939in}{2.750588in}}{\pgfqpoint{1.630673in}{2.750588in}}%
\pgfpathcurveto{\pgfqpoint{1.616408in}{2.750588in}}{\pgfqpoint{1.602724in}{2.744920in}}{\pgfqpoint{1.592637in}{2.734833in}}%
\pgfpathcurveto{\pgfqpoint{1.582550in}{2.724745in}}{\pgfqpoint{1.576882in}{2.711062in}}{\pgfqpoint{1.576882in}{2.696797in}}%
\pgfpathcurveto{\pgfqpoint{1.576882in}{2.682531in}}{\pgfqpoint{1.582550in}{2.668848in}}{\pgfqpoint{1.592637in}{2.658760in}}%
\pgfpathcurveto{\pgfqpoint{1.602724in}{2.648673in}}{\pgfqpoint{1.616408in}{2.643005in}}{\pgfqpoint{1.630673in}{2.643005in}}%
\pgfpathclose%
\pgfusepath{stroke}%
\end{pgfscope}%
\begin{pgfscope}%
\pgfpathrectangle{\pgfqpoint{0.678743in}{0.883555in}}{\pgfqpoint{4.650000in}{3.020000in}}%
\pgfusepath{clip}%
\pgfsetbuttcap%
\pgfsetroundjoin%
\pgfsetlinewidth{1.003750pt}%
\definecolor{currentstroke}{rgb}{0.000000,0.000000,1.000000}%
\pgfsetstrokecolor{currentstroke}%
\pgfsetdash{}{0pt}%
\pgfpathmoveto{\pgfqpoint{1.736294in}{2.701646in}}%
\pgfpathcurveto{\pgfqpoint{1.750560in}{2.701646in}}{\pgfqpoint{1.764243in}{2.707314in}}{\pgfqpoint{1.774330in}{2.717401in}}%
\pgfpathcurveto{\pgfqpoint{1.784418in}{2.727489in}}{\pgfqpoint{1.790085in}{2.741172in}}{\pgfqpoint{1.790085in}{2.755438in}}%
\pgfpathcurveto{\pgfqpoint{1.790085in}{2.769703in}}{\pgfqpoint{1.784418in}{2.783387in}}{\pgfqpoint{1.774330in}{2.793474in}}%
\pgfpathcurveto{\pgfqpoint{1.764243in}{2.803561in}}{\pgfqpoint{1.750560in}{2.809229in}}{\pgfqpoint{1.736294in}{2.809229in}}%
\pgfpathcurveto{\pgfqpoint{1.722028in}{2.809229in}}{\pgfqpoint{1.708345in}{2.803561in}}{\pgfqpoint{1.698258in}{2.793474in}}%
\pgfpathcurveto{\pgfqpoint{1.688170in}{2.783387in}}{\pgfqpoint{1.682503in}{2.769703in}}{\pgfqpoint{1.682503in}{2.755438in}}%
\pgfpathcurveto{\pgfqpoint{1.682503in}{2.741172in}}{\pgfqpoint{1.688170in}{2.727489in}}{\pgfqpoint{1.698258in}{2.717401in}}%
\pgfpathcurveto{\pgfqpoint{1.708345in}{2.707314in}}{\pgfqpoint{1.722028in}{2.701646in}}{\pgfqpoint{1.736294in}{2.701646in}}%
\pgfpathclose%
\pgfusepath{stroke}%
\end{pgfscope}%
\begin{pgfscope}%
\pgfpathrectangle{\pgfqpoint{0.678743in}{0.883555in}}{\pgfqpoint{4.650000in}{3.020000in}}%
\pgfusepath{clip}%
\pgfsetbuttcap%
\pgfsetroundjoin%
\pgfsetlinewidth{1.003750pt}%
\definecolor{currentstroke}{rgb}{0.000000,0.000000,1.000000}%
\pgfsetstrokecolor{currentstroke}%
\pgfsetdash{}{0pt}%
\pgfpathmoveto{\pgfqpoint{1.841915in}{2.771245in}}%
\pgfpathcurveto{\pgfqpoint{1.856180in}{2.771245in}}{\pgfqpoint{1.869864in}{2.776912in}}{\pgfqpoint{1.879951in}{2.787000in}}%
\pgfpathcurveto{\pgfqpoint{1.890038in}{2.797087in}}{\pgfqpoint{1.895706in}{2.810770in}}{\pgfqpoint{1.895706in}{2.825036in}}%
\pgfpathcurveto{\pgfqpoint{1.895706in}{2.839302in}}{\pgfqpoint{1.890038in}{2.852985in}}{\pgfqpoint{1.879951in}{2.863072in}}%
\pgfpathcurveto{\pgfqpoint{1.869864in}{2.873160in}}{\pgfqpoint{1.856180in}{2.878827in}}{\pgfqpoint{1.841915in}{2.878827in}}%
\pgfpathcurveto{\pgfqpoint{1.827649in}{2.878827in}}{\pgfqpoint{1.813966in}{2.873160in}}{\pgfqpoint{1.803878in}{2.863072in}}%
\pgfpathcurveto{\pgfqpoint{1.793791in}{2.852985in}}{\pgfqpoint{1.788123in}{2.839302in}}{\pgfqpoint{1.788123in}{2.825036in}}%
\pgfpathcurveto{\pgfqpoint{1.788123in}{2.810770in}}{\pgfqpoint{1.793791in}{2.797087in}}{\pgfqpoint{1.803878in}{2.787000in}}%
\pgfpathcurveto{\pgfqpoint{1.813966in}{2.776912in}}{\pgfqpoint{1.827649in}{2.771245in}}{\pgfqpoint{1.841915in}{2.771245in}}%
\pgfpathclose%
\pgfusepath{stroke}%
\end{pgfscope}%
\begin{pgfscope}%
\pgfpathrectangle{\pgfqpoint{0.678743in}{0.883555in}}{\pgfqpoint{4.650000in}{3.020000in}}%
\pgfusepath{clip}%
\pgfsetbuttcap%
\pgfsetroundjoin%
\pgfsetlinewidth{1.003750pt}%
\definecolor{currentstroke}{rgb}{0.000000,0.000000,1.000000}%
\pgfsetstrokecolor{currentstroke}%
\pgfsetdash{}{0pt}%
\pgfpathmoveto{\pgfqpoint{1.947536in}{2.843097in}}%
\pgfpathcurveto{\pgfqpoint{1.961801in}{2.843097in}}{\pgfqpoint{1.975484in}{2.848765in}}{\pgfqpoint{1.985572in}{2.858852in}}%
\pgfpathcurveto{\pgfqpoint{1.995659in}{2.868939in}}{\pgfqpoint{2.001327in}{2.882623in}}{\pgfqpoint{2.001327in}{2.896888in}}%
\pgfpathcurveto{\pgfqpoint{2.001327in}{2.911154in}}{\pgfqpoint{1.995659in}{2.924837in}}{\pgfqpoint{1.985572in}{2.934925in}}%
\pgfpathcurveto{\pgfqpoint{1.975484in}{2.945012in}}{\pgfqpoint{1.961801in}{2.950680in}}{\pgfqpoint{1.947536in}{2.950680in}}%
\pgfpathcurveto{\pgfqpoint{1.933270in}{2.950680in}}{\pgfqpoint{1.919587in}{2.945012in}}{\pgfqpoint{1.909499in}{2.934925in}}%
\pgfpathcurveto{\pgfqpoint{1.899412in}{2.924837in}}{\pgfqpoint{1.893744in}{2.911154in}}{\pgfqpoint{1.893744in}{2.896888in}}%
\pgfpathcurveto{\pgfqpoint{1.893744in}{2.882623in}}{\pgfqpoint{1.899412in}{2.868939in}}{\pgfqpoint{1.909499in}{2.858852in}}%
\pgfpathcurveto{\pgfqpoint{1.919587in}{2.848765in}}{\pgfqpoint{1.933270in}{2.843097in}}{\pgfqpoint{1.947536in}{2.843097in}}%
\pgfpathclose%
\pgfusepath{stroke}%
\end{pgfscope}%
\begin{pgfscope}%
\pgfpathrectangle{\pgfqpoint{0.678743in}{0.883555in}}{\pgfqpoint{4.650000in}{3.020000in}}%
\pgfusepath{clip}%
\pgfsetbuttcap%
\pgfsetroundjoin%
\pgfsetlinewidth{1.003750pt}%
\definecolor{currentstroke}{rgb}{0.000000,0.000000,1.000000}%
\pgfsetstrokecolor{currentstroke}%
\pgfsetdash{}{0pt}%
\pgfpathmoveto{\pgfqpoint{2.053156in}{2.878689in}}%
\pgfpathcurveto{\pgfqpoint{2.067422in}{2.878689in}}{\pgfqpoint{2.081105in}{2.884357in}}{\pgfqpoint{2.091193in}{2.894445in}}%
\pgfpathcurveto{\pgfqpoint{2.101280in}{2.904532in}}{\pgfqpoint{2.106948in}{2.918215in}}{\pgfqpoint{2.106948in}{2.932481in}}%
\pgfpathcurveto{\pgfqpoint{2.106948in}{2.946746in}}{\pgfqpoint{2.101280in}{2.960430in}}{\pgfqpoint{2.091193in}{2.970517in}}%
\pgfpathcurveto{\pgfqpoint{2.081105in}{2.980604in}}{\pgfqpoint{2.067422in}{2.986272in}}{\pgfqpoint{2.053156in}{2.986272in}}%
\pgfpathcurveto{\pgfqpoint{2.038891in}{2.986272in}}{\pgfqpoint{2.025207in}{2.980604in}}{\pgfqpoint{2.015120in}{2.970517in}}%
\pgfpathcurveto{\pgfqpoint{2.005033in}{2.960430in}}{\pgfqpoint{1.999365in}{2.946746in}}{\pgfqpoint{1.999365in}{2.932481in}}%
\pgfpathcurveto{\pgfqpoint{1.999365in}{2.918215in}}{\pgfqpoint{2.005033in}{2.904532in}}{\pgfqpoint{2.015120in}{2.894445in}}%
\pgfpathcurveto{\pgfqpoint{2.025207in}{2.884357in}}{\pgfqpoint{2.038891in}{2.878689in}}{\pgfqpoint{2.053156in}{2.878689in}}%
\pgfpathclose%
\pgfusepath{stroke}%
\end{pgfscope}%
\begin{pgfscope}%
\pgfpathrectangle{\pgfqpoint{0.678743in}{0.883555in}}{\pgfqpoint{4.650000in}{3.020000in}}%
\pgfusepath{clip}%
\pgfsetbuttcap%
\pgfsetroundjoin%
\pgfsetlinewidth{1.003750pt}%
\definecolor{currentstroke}{rgb}{0.000000,0.000000,1.000000}%
\pgfsetstrokecolor{currentstroke}%
\pgfsetdash{}{0pt}%
\pgfpathmoveto{\pgfqpoint{2.158777in}{2.924331in}}%
\pgfpathcurveto{\pgfqpoint{2.173043in}{2.924331in}}{\pgfqpoint{2.186726in}{2.929999in}}{\pgfqpoint{2.196813in}{2.940086in}}%
\pgfpathcurveto{\pgfqpoint{2.206901in}{2.950174in}}{\pgfqpoint{2.212568in}{2.963857in}}{\pgfqpoint{2.212568in}{2.978123in}}%
\pgfpathcurveto{\pgfqpoint{2.212568in}{2.992388in}}{\pgfqpoint{2.206901in}{3.006072in}}{\pgfqpoint{2.196813in}{3.016159in}}%
\pgfpathcurveto{\pgfqpoint{2.186726in}{3.026246in}}{\pgfqpoint{2.173043in}{3.031914in}}{\pgfqpoint{2.158777in}{3.031914in}}%
\pgfpathcurveto{\pgfqpoint{2.144511in}{3.031914in}}{\pgfqpoint{2.130828in}{3.026246in}}{\pgfqpoint{2.120741in}{3.016159in}}%
\pgfpathcurveto{\pgfqpoint{2.110653in}{3.006072in}}{\pgfqpoint{2.104986in}{2.992388in}}{\pgfqpoint{2.104986in}{2.978123in}}%
\pgfpathcurveto{\pgfqpoint{2.104986in}{2.963857in}}{\pgfqpoint{2.110653in}{2.950174in}}{\pgfqpoint{2.120741in}{2.940086in}}%
\pgfpathcurveto{\pgfqpoint{2.130828in}{2.929999in}}{\pgfqpoint{2.144511in}{2.924331in}}{\pgfqpoint{2.158777in}{2.924331in}}%
\pgfpathclose%
\pgfusepath{stroke}%
\end{pgfscope}%
\begin{pgfscope}%
\pgfpathrectangle{\pgfqpoint{0.678743in}{0.883555in}}{\pgfqpoint{4.650000in}{3.020000in}}%
\pgfusepath{clip}%
\pgfsetbuttcap%
\pgfsetroundjoin%
\pgfsetlinewidth{1.003750pt}%
\definecolor{currentstroke}{rgb}{0.000000,0.000000,1.000000}%
\pgfsetstrokecolor{currentstroke}%
\pgfsetdash{}{0pt}%
\pgfpathmoveto{\pgfqpoint{2.264398in}{2.947204in}}%
\pgfpathcurveto{\pgfqpoint{2.278663in}{2.947204in}}{\pgfqpoint{2.292347in}{2.952872in}}{\pgfqpoint{2.302434in}{2.962960in}}%
\pgfpathcurveto{\pgfqpoint{2.312521in}{2.973047in}}{\pgfqpoint{2.318189in}{2.986730in}}{\pgfqpoint{2.318189in}{3.000996in}}%
\pgfpathcurveto{\pgfqpoint{2.318189in}{3.015262in}}{\pgfqpoint{2.312521in}{3.028945in}}{\pgfqpoint{2.302434in}{3.039032in}}%
\pgfpathcurveto{\pgfqpoint{2.292347in}{3.049120in}}{\pgfqpoint{2.278663in}{3.054787in}}{\pgfqpoint{2.264398in}{3.054787in}}%
\pgfpathcurveto{\pgfqpoint{2.250132in}{3.054787in}}{\pgfqpoint{2.236449in}{3.049120in}}{\pgfqpoint{2.226361in}{3.039032in}}%
\pgfpathcurveto{\pgfqpoint{2.216274in}{3.028945in}}{\pgfqpoint{2.210606in}{3.015262in}}{\pgfqpoint{2.210606in}{3.000996in}}%
\pgfpathcurveto{\pgfqpoint{2.210606in}{2.986730in}}{\pgfqpoint{2.216274in}{2.973047in}}{\pgfqpoint{2.226361in}{2.962960in}}%
\pgfpathcurveto{\pgfqpoint{2.236449in}{2.952872in}}{\pgfqpoint{2.250132in}{2.947204in}}{\pgfqpoint{2.264398in}{2.947204in}}%
\pgfpathclose%
\pgfusepath{stroke}%
\end{pgfscope}%
\begin{pgfscope}%
\pgfpathrectangle{\pgfqpoint{0.678743in}{0.883555in}}{\pgfqpoint{4.650000in}{3.020000in}}%
\pgfusepath{clip}%
\pgfsetbuttcap%
\pgfsetroundjoin%
\pgfsetlinewidth{1.003750pt}%
\definecolor{currentstroke}{rgb}{0.000000,0.000000,1.000000}%
\pgfsetstrokecolor{currentstroke}%
\pgfsetdash{}{0pt}%
\pgfpathmoveto{\pgfqpoint{2.370018in}{3.020343in}}%
\pgfpathcurveto{\pgfqpoint{2.384284in}{3.020343in}}{\pgfqpoint{2.397967in}{3.026010in}}{\pgfqpoint{2.408055in}{3.036098in}}%
\pgfpathcurveto{\pgfqpoint{2.418142in}{3.046185in}}{\pgfqpoint{2.423810in}{3.059868in}}{\pgfqpoint{2.423810in}{3.074134in}}%
\pgfpathcurveto{\pgfqpoint{2.423810in}{3.088400in}}{\pgfqpoint{2.418142in}{3.102083in}}{\pgfqpoint{2.408055in}{3.112170in}}%
\pgfpathcurveto{\pgfqpoint{2.397967in}{3.122258in}}{\pgfqpoint{2.384284in}{3.127925in}}{\pgfqpoint{2.370018in}{3.127925in}}%
\pgfpathcurveto{\pgfqpoint{2.355753in}{3.127925in}}{\pgfqpoint{2.342069in}{3.122258in}}{\pgfqpoint{2.331982in}{3.112170in}}%
\pgfpathcurveto{\pgfqpoint{2.321895in}{3.102083in}}{\pgfqpoint{2.316227in}{3.088400in}}{\pgfqpoint{2.316227in}{3.074134in}}%
\pgfpathcurveto{\pgfqpoint{2.316227in}{3.059868in}}{\pgfqpoint{2.321895in}{3.046185in}}{\pgfqpoint{2.331982in}{3.036098in}}%
\pgfpathcurveto{\pgfqpoint{2.342069in}{3.026010in}}{\pgfqpoint{2.355753in}{3.020343in}}{\pgfqpoint{2.370018in}{3.020343in}}%
\pgfpathclose%
\pgfusepath{stroke}%
\end{pgfscope}%
\begin{pgfscope}%
\pgfpathrectangle{\pgfqpoint{0.678743in}{0.883555in}}{\pgfqpoint{4.650000in}{3.020000in}}%
\pgfusepath{clip}%
\pgfsetbuttcap%
\pgfsetroundjoin%
\pgfsetlinewidth{1.003750pt}%
\definecolor{currentstroke}{rgb}{0.000000,0.000000,1.000000}%
\pgfsetstrokecolor{currentstroke}%
\pgfsetdash{}{0pt}%
\pgfpathmoveto{\pgfqpoint{2.475639in}{3.030804in}}%
\pgfpathcurveto{\pgfqpoint{2.489905in}{3.030804in}}{\pgfqpoint{2.503588in}{3.036472in}}{\pgfqpoint{2.513675in}{3.046559in}}%
\pgfpathcurveto{\pgfqpoint{2.523763in}{3.056646in}}{\pgfqpoint{2.529431in}{3.070330in}}{\pgfqpoint{2.529431in}{3.084595in}}%
\pgfpathcurveto{\pgfqpoint{2.529431in}{3.098861in}}{\pgfqpoint{2.523763in}{3.112544in}}{\pgfqpoint{2.513675in}{3.122632in}}%
\pgfpathcurveto{\pgfqpoint{2.503588in}{3.132719in}}{\pgfqpoint{2.489905in}{3.138387in}}{\pgfqpoint{2.475639in}{3.138387in}}%
\pgfpathcurveto{\pgfqpoint{2.461373in}{3.138387in}}{\pgfqpoint{2.447690in}{3.132719in}}{\pgfqpoint{2.437603in}{3.122632in}}%
\pgfpathcurveto{\pgfqpoint{2.427516in}{3.112544in}}{\pgfqpoint{2.421848in}{3.098861in}}{\pgfqpoint{2.421848in}{3.084595in}}%
\pgfpathcurveto{\pgfqpoint{2.421848in}{3.070330in}}{\pgfqpoint{2.427516in}{3.056646in}}{\pgfqpoint{2.437603in}{3.046559in}}%
\pgfpathcurveto{\pgfqpoint{2.447690in}{3.036472in}}{\pgfqpoint{2.461373in}{3.030804in}}{\pgfqpoint{2.475639in}{3.030804in}}%
\pgfpathclose%
\pgfusepath{stroke}%
\end{pgfscope}%
\begin{pgfscope}%
\pgfpathrectangle{\pgfqpoint{0.678743in}{0.883555in}}{\pgfqpoint{4.650000in}{3.020000in}}%
\pgfusepath{clip}%
\pgfsetbuttcap%
\pgfsetroundjoin%
\pgfsetlinewidth{1.003750pt}%
\definecolor{currentstroke}{rgb}{0.000000,0.000000,1.000000}%
\pgfsetstrokecolor{currentstroke}%
\pgfsetdash{}{0pt}%
\pgfpathmoveto{\pgfqpoint{2.581260in}{3.069174in}}%
\pgfpathcurveto{\pgfqpoint{2.595526in}{3.069174in}}{\pgfqpoint{2.609209in}{3.074841in}}{\pgfqpoint{2.619296in}{3.084929in}}%
\pgfpathcurveto{\pgfqpoint{2.629384in}{3.095016in}}{\pgfqpoint{2.635051in}{3.108699in}}{\pgfqpoint{2.635051in}{3.122965in}}%
\pgfpathcurveto{\pgfqpoint{2.635051in}{3.137231in}}{\pgfqpoint{2.629384in}{3.150914in}}{\pgfqpoint{2.619296in}{3.161001in}}%
\pgfpathcurveto{\pgfqpoint{2.609209in}{3.171089in}}{\pgfqpoint{2.595526in}{3.176757in}}{\pgfqpoint{2.581260in}{3.176757in}}%
\pgfpathcurveto{\pgfqpoint{2.566994in}{3.176757in}}{\pgfqpoint{2.553311in}{3.171089in}}{\pgfqpoint{2.543224in}{3.161001in}}%
\pgfpathcurveto{\pgfqpoint{2.533136in}{3.150914in}}{\pgfqpoint{2.527468in}{3.137231in}}{\pgfqpoint{2.527468in}{3.122965in}}%
\pgfpathcurveto{\pgfqpoint{2.527468in}{3.108699in}}{\pgfqpoint{2.533136in}{3.095016in}}{\pgfqpoint{2.543224in}{3.084929in}}%
\pgfpathcurveto{\pgfqpoint{2.553311in}{3.074841in}}{\pgfqpoint{2.566994in}{3.069174in}}{\pgfqpoint{2.581260in}{3.069174in}}%
\pgfpathclose%
\pgfusepath{stroke}%
\end{pgfscope}%
\begin{pgfscope}%
\pgfpathrectangle{\pgfqpoint{0.678743in}{0.883555in}}{\pgfqpoint{4.650000in}{3.020000in}}%
\pgfusepath{clip}%
\pgfsetbuttcap%
\pgfsetroundjoin%
\pgfsetlinewidth{1.003750pt}%
\definecolor{currentstroke}{rgb}{0.000000,0.000000,1.000000}%
\pgfsetstrokecolor{currentstroke}%
\pgfsetdash{}{0pt}%
\pgfpathmoveto{\pgfqpoint{2.686881in}{3.068721in}}%
\pgfpathcurveto{\pgfqpoint{2.701146in}{3.068721in}}{\pgfqpoint{2.714830in}{3.074389in}}{\pgfqpoint{2.724917in}{3.084476in}}%
\pgfpathcurveto{\pgfqpoint{2.735004in}{3.094563in}}{\pgfqpoint{2.740672in}{3.108247in}}{\pgfqpoint{2.740672in}{3.122512in}}%
\pgfpathcurveto{\pgfqpoint{2.740672in}{3.136778in}}{\pgfqpoint{2.735004in}{3.150461in}}{\pgfqpoint{2.724917in}{3.160549in}}%
\pgfpathcurveto{\pgfqpoint{2.714830in}{3.170636in}}{\pgfqpoint{2.701146in}{3.176304in}}{\pgfqpoint{2.686881in}{3.176304in}}%
\pgfpathcurveto{\pgfqpoint{2.672615in}{3.176304in}}{\pgfqpoint{2.658932in}{3.170636in}}{\pgfqpoint{2.648844in}{3.160549in}}%
\pgfpathcurveto{\pgfqpoint{2.638757in}{3.150461in}}{\pgfqpoint{2.633089in}{3.136778in}}{\pgfqpoint{2.633089in}{3.122512in}}%
\pgfpathcurveto{\pgfqpoint{2.633089in}{3.108247in}}{\pgfqpoint{2.638757in}{3.094563in}}{\pgfqpoint{2.648844in}{3.084476in}}%
\pgfpathcurveto{\pgfqpoint{2.658932in}{3.074389in}}{\pgfqpoint{2.672615in}{3.068721in}}{\pgfqpoint{2.686881in}{3.068721in}}%
\pgfpathclose%
\pgfusepath{stroke}%
\end{pgfscope}%
\begin{pgfscope}%
\pgfpathrectangle{\pgfqpoint{0.678743in}{0.883555in}}{\pgfqpoint{4.650000in}{3.020000in}}%
\pgfusepath{clip}%
\pgfsetbuttcap%
\pgfsetroundjoin%
\pgfsetlinewidth{1.003750pt}%
\definecolor{currentstroke}{rgb}{0.000000,0.000000,1.000000}%
\pgfsetstrokecolor{currentstroke}%
\pgfsetdash{}{0pt}%
\pgfpathmoveto{\pgfqpoint{2.792501in}{3.119482in}}%
\pgfpathcurveto{\pgfqpoint{2.806767in}{3.119482in}}{\pgfqpoint{2.820450in}{3.125150in}}{\pgfqpoint{2.830538in}{3.135237in}}%
\pgfpathcurveto{\pgfqpoint{2.840625in}{3.145325in}}{\pgfqpoint{2.846293in}{3.159008in}}{\pgfqpoint{2.846293in}{3.173273in}}%
\pgfpathcurveto{\pgfqpoint{2.846293in}{3.187539in}}{\pgfqpoint{2.840625in}{3.201222in}}{\pgfqpoint{2.830538in}{3.211310in}}%
\pgfpathcurveto{\pgfqpoint{2.820450in}{3.221397in}}{\pgfqpoint{2.806767in}{3.227065in}}{\pgfqpoint{2.792501in}{3.227065in}}%
\pgfpathcurveto{\pgfqpoint{2.778236in}{3.227065in}}{\pgfqpoint{2.764552in}{3.221397in}}{\pgfqpoint{2.754465in}{3.211310in}}%
\pgfpathcurveto{\pgfqpoint{2.744378in}{3.201222in}}{\pgfqpoint{2.738710in}{3.187539in}}{\pgfqpoint{2.738710in}{3.173273in}}%
\pgfpathcurveto{\pgfqpoint{2.738710in}{3.159008in}}{\pgfqpoint{2.744378in}{3.145325in}}{\pgfqpoint{2.754465in}{3.135237in}}%
\pgfpathcurveto{\pgfqpoint{2.764552in}{3.125150in}}{\pgfqpoint{2.778236in}{3.119482in}}{\pgfqpoint{2.792501in}{3.119482in}}%
\pgfpathclose%
\pgfusepath{stroke}%
\end{pgfscope}%
\begin{pgfscope}%
\pgfpathrectangle{\pgfqpoint{0.678743in}{0.883555in}}{\pgfqpoint{4.650000in}{3.020000in}}%
\pgfusepath{clip}%
\pgfsetbuttcap%
\pgfsetroundjoin%
\pgfsetlinewidth{1.003750pt}%
\definecolor{currentstroke}{rgb}{0.000000,0.000000,1.000000}%
\pgfsetstrokecolor{currentstroke}%
\pgfsetdash{}{0pt}%
\pgfpathmoveto{\pgfqpoint{2.898122in}{3.118383in}}%
\pgfpathcurveto{\pgfqpoint{2.912388in}{3.118383in}}{\pgfqpoint{2.926071in}{3.124051in}}{\pgfqpoint{2.936158in}{3.134138in}}%
\pgfpathcurveto{\pgfqpoint{2.946246in}{3.144226in}}{\pgfqpoint{2.951913in}{3.157909in}}{\pgfqpoint{2.951913in}{3.172175in}}%
\pgfpathcurveto{\pgfqpoint{2.951913in}{3.186440in}}{\pgfqpoint{2.946246in}{3.200124in}}{\pgfqpoint{2.936158in}{3.210211in}}%
\pgfpathcurveto{\pgfqpoint{2.926071in}{3.220298in}}{\pgfqpoint{2.912388in}{3.225966in}}{\pgfqpoint{2.898122in}{3.225966in}}%
\pgfpathcurveto{\pgfqpoint{2.883856in}{3.225966in}}{\pgfqpoint{2.870173in}{3.220298in}}{\pgfqpoint{2.860086in}{3.210211in}}%
\pgfpathcurveto{\pgfqpoint{2.849998in}{3.200124in}}{\pgfqpoint{2.844331in}{3.186440in}}{\pgfqpoint{2.844331in}{3.172175in}}%
\pgfpathcurveto{\pgfqpoint{2.844331in}{3.157909in}}{\pgfqpoint{2.849998in}{3.144226in}}{\pgfqpoint{2.860086in}{3.134138in}}%
\pgfpathcurveto{\pgfqpoint{2.870173in}{3.124051in}}{\pgfqpoint{2.883856in}{3.118383in}}{\pgfqpoint{2.898122in}{3.118383in}}%
\pgfpathclose%
\pgfusepath{stroke}%
\end{pgfscope}%
\begin{pgfscope}%
\pgfpathrectangle{\pgfqpoint{0.678743in}{0.883555in}}{\pgfqpoint{4.650000in}{3.020000in}}%
\pgfusepath{clip}%
\pgfsetbuttcap%
\pgfsetroundjoin%
\pgfsetlinewidth{1.003750pt}%
\definecolor{currentstroke}{rgb}{0.000000,0.000000,1.000000}%
\pgfsetstrokecolor{currentstroke}%
\pgfsetdash{}{0pt}%
\pgfpathmoveto{\pgfqpoint{3.003743in}{3.174238in}}%
\pgfpathcurveto{\pgfqpoint{3.018008in}{3.174238in}}{\pgfqpoint{3.031692in}{3.179906in}}{\pgfqpoint{3.041779in}{3.189993in}}%
\pgfpathcurveto{\pgfqpoint{3.051866in}{3.200080in}}{\pgfqpoint{3.057534in}{3.213764in}}{\pgfqpoint{3.057534in}{3.228029in}}%
\pgfpathcurveto{\pgfqpoint{3.057534in}{3.242295in}}{\pgfqpoint{3.051866in}{3.255978in}}{\pgfqpoint{3.041779in}{3.266066in}}%
\pgfpathcurveto{\pgfqpoint{3.031692in}{3.276153in}}{\pgfqpoint{3.018008in}{3.281821in}}{\pgfqpoint{3.003743in}{3.281821in}}%
\pgfpathcurveto{\pgfqpoint{2.989477in}{3.281821in}}{\pgfqpoint{2.975794in}{3.276153in}}{\pgfqpoint{2.965706in}{3.266066in}}%
\pgfpathcurveto{\pgfqpoint{2.955619in}{3.255978in}}{\pgfqpoint{2.949951in}{3.242295in}}{\pgfqpoint{2.949951in}{3.228029in}}%
\pgfpathcurveto{\pgfqpoint{2.949951in}{3.213764in}}{\pgfqpoint{2.955619in}{3.200080in}}{\pgfqpoint{2.965706in}{3.189993in}}%
\pgfpathcurveto{\pgfqpoint{2.975794in}{3.179906in}}{\pgfqpoint{2.989477in}{3.174238in}}{\pgfqpoint{3.003743in}{3.174238in}}%
\pgfpathclose%
\pgfusepath{stroke}%
\end{pgfscope}%
\begin{pgfscope}%
\pgfpathrectangle{\pgfqpoint{0.678743in}{0.883555in}}{\pgfqpoint{4.650000in}{3.020000in}}%
\pgfusepath{clip}%
\pgfsetbuttcap%
\pgfsetroundjoin%
\pgfsetlinewidth{1.003750pt}%
\definecolor{currentstroke}{rgb}{0.000000,0.000000,1.000000}%
\pgfsetstrokecolor{currentstroke}%
\pgfsetdash{}{0pt}%
\pgfpathmoveto{\pgfqpoint{3.109364in}{3.185792in}}%
\pgfpathcurveto{\pgfqpoint{3.123629in}{3.185792in}}{\pgfqpoint{3.137312in}{3.191460in}}{\pgfqpoint{3.147400in}{3.201547in}}%
\pgfpathcurveto{\pgfqpoint{3.157487in}{3.211635in}}{\pgfqpoint{3.163155in}{3.225318in}}{\pgfqpoint{3.163155in}{3.239584in}}%
\pgfpathcurveto{\pgfqpoint{3.163155in}{3.253849in}}{\pgfqpoint{3.157487in}{3.267533in}}{\pgfqpoint{3.147400in}{3.277620in}}%
\pgfpathcurveto{\pgfqpoint{3.137312in}{3.287707in}}{\pgfqpoint{3.123629in}{3.293375in}}{\pgfqpoint{3.109364in}{3.293375in}}%
\pgfpathcurveto{\pgfqpoint{3.095098in}{3.293375in}}{\pgfqpoint{3.081415in}{3.287707in}}{\pgfqpoint{3.071327in}{3.277620in}}%
\pgfpathcurveto{\pgfqpoint{3.061240in}{3.267533in}}{\pgfqpoint{3.055572in}{3.253849in}}{\pgfqpoint{3.055572in}{3.239584in}}%
\pgfpathcurveto{\pgfqpoint{3.055572in}{3.225318in}}{\pgfqpoint{3.061240in}{3.211635in}}{\pgfqpoint{3.071327in}{3.201547in}}%
\pgfpathcurveto{\pgfqpoint{3.081415in}{3.191460in}}{\pgfqpoint{3.095098in}{3.185792in}}{\pgfqpoint{3.109364in}{3.185792in}}%
\pgfpathclose%
\pgfusepath{stroke}%
\end{pgfscope}%
\begin{pgfscope}%
\pgfpathrectangle{\pgfqpoint{0.678743in}{0.883555in}}{\pgfqpoint{4.650000in}{3.020000in}}%
\pgfusepath{clip}%
\pgfsetbuttcap%
\pgfsetroundjoin%
\pgfsetlinewidth{1.003750pt}%
\definecolor{currentstroke}{rgb}{0.000000,0.000000,1.000000}%
\pgfsetstrokecolor{currentstroke}%
\pgfsetdash{}{0pt}%
\pgfpathmoveto{\pgfqpoint{3.214984in}{3.164687in}}%
\pgfpathcurveto{\pgfqpoint{3.229250in}{3.164687in}}{\pgfqpoint{3.242933in}{3.170354in}}{\pgfqpoint{3.253021in}{3.180442in}}%
\pgfpathcurveto{\pgfqpoint{3.263108in}{3.190529in}}{\pgfqpoint{3.268776in}{3.204212in}}{\pgfqpoint{3.268776in}{3.218478in}}%
\pgfpathcurveto{\pgfqpoint{3.268776in}{3.232744in}}{\pgfqpoint{3.263108in}{3.246427in}}{\pgfqpoint{3.253021in}{3.256514in}}%
\pgfpathcurveto{\pgfqpoint{3.242933in}{3.266602in}}{\pgfqpoint{3.229250in}{3.272269in}}{\pgfqpoint{3.214984in}{3.272269in}}%
\pgfpathcurveto{\pgfqpoint{3.200719in}{3.272269in}}{\pgfqpoint{3.187035in}{3.266602in}}{\pgfqpoint{3.176948in}{3.256514in}}%
\pgfpathcurveto{\pgfqpoint{3.166861in}{3.246427in}}{\pgfqpoint{3.161193in}{3.232744in}}{\pgfqpoint{3.161193in}{3.218478in}}%
\pgfpathcurveto{\pgfqpoint{3.161193in}{3.204212in}}{\pgfqpoint{3.166861in}{3.190529in}}{\pgfqpoint{3.176948in}{3.180442in}}%
\pgfpathcurveto{\pgfqpoint{3.187035in}{3.170354in}}{\pgfqpoint{3.200719in}{3.164687in}}{\pgfqpoint{3.214984in}{3.164687in}}%
\pgfpathclose%
\pgfusepath{stroke}%
\end{pgfscope}%
\begin{pgfscope}%
\pgfpathrectangle{\pgfqpoint{0.678743in}{0.883555in}}{\pgfqpoint{4.650000in}{3.020000in}}%
\pgfusepath{clip}%
\pgfsetbuttcap%
\pgfsetroundjoin%
\pgfsetlinewidth{1.003750pt}%
\definecolor{currentstroke}{rgb}{0.000000,0.000000,1.000000}%
\pgfsetstrokecolor{currentstroke}%
\pgfsetdash{}{0pt}%
\pgfpathmoveto{\pgfqpoint{3.320605in}{3.215900in}}%
\pgfpathcurveto{\pgfqpoint{3.334871in}{3.215900in}}{\pgfqpoint{3.348554in}{3.221568in}}{\pgfqpoint{3.358641in}{3.231655in}}%
\pgfpathcurveto{\pgfqpoint{3.368729in}{3.241743in}}{\pgfqpoint{3.374396in}{3.255426in}}{\pgfqpoint{3.374396in}{3.269692in}}%
\pgfpathcurveto{\pgfqpoint{3.374396in}{3.283957in}}{\pgfqpoint{3.368729in}{3.297641in}}{\pgfqpoint{3.358641in}{3.307728in}}%
\pgfpathcurveto{\pgfqpoint{3.348554in}{3.317815in}}{\pgfqpoint{3.334871in}{3.323483in}}{\pgfqpoint{3.320605in}{3.323483in}}%
\pgfpathcurveto{\pgfqpoint{3.306339in}{3.323483in}}{\pgfqpoint{3.292656in}{3.317815in}}{\pgfqpoint{3.282569in}{3.307728in}}%
\pgfpathcurveto{\pgfqpoint{3.272481in}{3.297641in}}{\pgfqpoint{3.266814in}{3.283957in}}{\pgfqpoint{3.266814in}{3.269692in}}%
\pgfpathcurveto{\pgfqpoint{3.266814in}{3.255426in}}{\pgfqpoint{3.272481in}{3.241743in}}{\pgfqpoint{3.282569in}{3.231655in}}%
\pgfpathcurveto{\pgfqpoint{3.292656in}{3.221568in}}{\pgfqpoint{3.306339in}{3.215900in}}{\pgfqpoint{3.320605in}{3.215900in}}%
\pgfpathclose%
\pgfusepath{stroke}%
\end{pgfscope}%
\begin{pgfscope}%
\pgfpathrectangle{\pgfqpoint{0.678743in}{0.883555in}}{\pgfqpoint{4.650000in}{3.020000in}}%
\pgfusepath{clip}%
\pgfsetbuttcap%
\pgfsetroundjoin%
\pgfsetlinewidth{1.003750pt}%
\definecolor{currentstroke}{rgb}{0.000000,0.000000,1.000000}%
\pgfsetstrokecolor{currentstroke}%
\pgfsetdash{}{0pt}%
\pgfpathmoveto{\pgfqpoint{3.426226in}{3.203267in}}%
\pgfpathcurveto{\pgfqpoint{3.440491in}{3.203267in}}{\pgfqpoint{3.454175in}{3.208934in}}{\pgfqpoint{3.464262in}{3.219022in}}%
\pgfpathcurveto{\pgfqpoint{3.474349in}{3.229109in}}{\pgfqpoint{3.480017in}{3.242792in}}{\pgfqpoint{3.480017in}{3.257058in}}%
\pgfpathcurveto{\pgfqpoint{3.480017in}{3.271324in}}{\pgfqpoint{3.474349in}{3.285007in}}{\pgfqpoint{3.464262in}{3.295094in}}%
\pgfpathcurveto{\pgfqpoint{3.454175in}{3.305182in}}{\pgfqpoint{3.440491in}{3.310849in}}{\pgfqpoint{3.426226in}{3.310849in}}%
\pgfpathcurveto{\pgfqpoint{3.411960in}{3.310849in}}{\pgfqpoint{3.398277in}{3.305182in}}{\pgfqpoint{3.388189in}{3.295094in}}%
\pgfpathcurveto{\pgfqpoint{3.378102in}{3.285007in}}{\pgfqpoint{3.372434in}{3.271324in}}{\pgfqpoint{3.372434in}{3.257058in}}%
\pgfpathcurveto{\pgfqpoint{3.372434in}{3.242792in}}{\pgfqpoint{3.378102in}{3.229109in}}{\pgfqpoint{3.388189in}{3.219022in}}%
\pgfpathcurveto{\pgfqpoint{3.398277in}{3.208934in}}{\pgfqpoint{3.411960in}{3.203267in}}{\pgfqpoint{3.426226in}{3.203267in}}%
\pgfpathclose%
\pgfusepath{stroke}%
\end{pgfscope}%
\begin{pgfscope}%
\pgfpathrectangle{\pgfqpoint{0.678743in}{0.883555in}}{\pgfqpoint{4.650000in}{3.020000in}}%
\pgfusepath{clip}%
\pgfsetbuttcap%
\pgfsetroundjoin%
\pgfsetlinewidth{1.003750pt}%
\definecolor{currentstroke}{rgb}{0.000000,0.000000,1.000000}%
\pgfsetstrokecolor{currentstroke}%
\pgfsetdash{}{0pt}%
\pgfpathmoveto{\pgfqpoint{3.531846in}{3.177216in}}%
\pgfpathcurveto{\pgfqpoint{3.546112in}{3.177216in}}{\pgfqpoint{3.559795in}{3.182884in}}{\pgfqpoint{3.569883in}{3.192971in}}%
\pgfpathcurveto{\pgfqpoint{3.579970in}{3.203058in}}{\pgfqpoint{3.585638in}{3.216742in}}{\pgfqpoint{3.585638in}{3.231007in}}%
\pgfpathcurveto{\pgfqpoint{3.585638in}{3.245273in}}{\pgfqpoint{3.579970in}{3.258956in}}{\pgfqpoint{3.569883in}{3.269044in}}%
\pgfpathcurveto{\pgfqpoint{3.559795in}{3.279131in}}{\pgfqpoint{3.546112in}{3.284799in}}{\pgfqpoint{3.531846in}{3.284799in}}%
\pgfpathcurveto{\pgfqpoint{3.517581in}{3.284799in}}{\pgfqpoint{3.503897in}{3.279131in}}{\pgfqpoint{3.493810in}{3.269044in}}%
\pgfpathcurveto{\pgfqpoint{3.483723in}{3.258956in}}{\pgfqpoint{3.478055in}{3.245273in}}{\pgfqpoint{3.478055in}{3.231007in}}%
\pgfpathcurveto{\pgfqpoint{3.478055in}{3.216742in}}{\pgfqpoint{3.483723in}{3.203058in}}{\pgfqpoint{3.493810in}{3.192971in}}%
\pgfpathcurveto{\pgfqpoint{3.503897in}{3.182884in}}{\pgfqpoint{3.517581in}{3.177216in}}{\pgfqpoint{3.531846in}{3.177216in}}%
\pgfpathclose%
\pgfusepath{stroke}%
\end{pgfscope}%
\begin{pgfscope}%
\pgfpathrectangle{\pgfqpoint{0.678743in}{0.883555in}}{\pgfqpoint{4.650000in}{3.020000in}}%
\pgfusepath{clip}%
\pgfsetbuttcap%
\pgfsetroundjoin%
\pgfsetlinewidth{1.003750pt}%
\definecolor{currentstroke}{rgb}{0.000000,0.000000,1.000000}%
\pgfsetstrokecolor{currentstroke}%
\pgfsetdash{}{0pt}%
\pgfpathmoveto{\pgfqpoint{3.637467in}{3.114371in}}%
\pgfpathcurveto{\pgfqpoint{3.651733in}{3.114371in}}{\pgfqpoint{3.665416in}{3.120039in}}{\pgfqpoint{3.675503in}{3.130126in}}%
\pgfpathcurveto{\pgfqpoint{3.685591in}{3.140214in}}{\pgfqpoint{3.691259in}{3.153897in}}{\pgfqpoint{3.691259in}{3.168163in}}%
\pgfpathcurveto{\pgfqpoint{3.691259in}{3.182428in}}{\pgfqpoint{3.685591in}{3.196112in}}{\pgfqpoint{3.675503in}{3.206199in}}%
\pgfpathcurveto{\pgfqpoint{3.665416in}{3.216286in}}{\pgfqpoint{3.651733in}{3.221954in}}{\pgfqpoint{3.637467in}{3.221954in}}%
\pgfpathcurveto{\pgfqpoint{3.623201in}{3.221954in}}{\pgfqpoint{3.609518in}{3.216286in}}{\pgfqpoint{3.599431in}{3.206199in}}%
\pgfpathcurveto{\pgfqpoint{3.589344in}{3.196112in}}{\pgfqpoint{3.583676in}{3.182428in}}{\pgfqpoint{3.583676in}{3.168163in}}%
\pgfpathcurveto{\pgfqpoint{3.583676in}{3.153897in}}{\pgfqpoint{3.589344in}{3.140214in}}{\pgfqpoint{3.599431in}{3.130126in}}%
\pgfpathcurveto{\pgfqpoint{3.609518in}{3.120039in}}{\pgfqpoint{3.623201in}{3.114371in}}{\pgfqpoint{3.637467in}{3.114371in}}%
\pgfpathclose%
\pgfusepath{stroke}%
\end{pgfscope}%
\begin{pgfscope}%
\pgfpathrectangle{\pgfqpoint{0.678743in}{0.883555in}}{\pgfqpoint{4.650000in}{3.020000in}}%
\pgfusepath{clip}%
\pgfsetbuttcap%
\pgfsetroundjoin%
\pgfsetlinewidth{1.003750pt}%
\definecolor{currentstroke}{rgb}{0.000000,0.000000,1.000000}%
\pgfsetstrokecolor{currentstroke}%
\pgfsetdash{}{0pt}%
\pgfpathmoveto{\pgfqpoint{3.743088in}{3.154175in}}%
\pgfpathcurveto{\pgfqpoint{3.757354in}{3.154175in}}{\pgfqpoint{3.771037in}{3.159843in}}{\pgfqpoint{3.781124in}{3.169930in}}%
\pgfpathcurveto{\pgfqpoint{3.791211in}{3.180018in}}{\pgfqpoint{3.796879in}{3.193701in}}{\pgfqpoint{3.796879in}{3.207966in}}%
\pgfpathcurveto{\pgfqpoint{3.796879in}{3.222232in}}{\pgfqpoint{3.791211in}{3.235915in}}{\pgfqpoint{3.781124in}{3.246003in}}%
\pgfpathcurveto{\pgfqpoint{3.771037in}{3.256090in}}{\pgfqpoint{3.757354in}{3.261758in}}{\pgfqpoint{3.743088in}{3.261758in}}%
\pgfpathcurveto{\pgfqpoint{3.728822in}{3.261758in}}{\pgfqpoint{3.715139in}{3.256090in}}{\pgfqpoint{3.705052in}{3.246003in}}%
\pgfpathcurveto{\pgfqpoint{3.694964in}{3.235915in}}{\pgfqpoint{3.689296in}{3.222232in}}{\pgfqpoint{3.689296in}{3.207966in}}%
\pgfpathcurveto{\pgfqpoint{3.689296in}{3.193701in}}{\pgfqpoint{3.694964in}{3.180018in}}{\pgfqpoint{3.705052in}{3.169930in}}%
\pgfpathcurveto{\pgfqpoint{3.715139in}{3.159843in}}{\pgfqpoint{3.728822in}{3.154175in}}{\pgfqpoint{3.743088in}{3.154175in}}%
\pgfpathclose%
\pgfusepath{stroke}%
\end{pgfscope}%
\begin{pgfscope}%
\pgfpathrectangle{\pgfqpoint{0.678743in}{0.883555in}}{\pgfqpoint{4.650000in}{3.020000in}}%
\pgfusepath{clip}%
\pgfsetbuttcap%
\pgfsetroundjoin%
\pgfsetlinewidth{1.003750pt}%
\definecolor{currentstroke}{rgb}{0.000000,0.000000,1.000000}%
\pgfsetstrokecolor{currentstroke}%
\pgfsetdash{}{0pt}%
\pgfpathmoveto{\pgfqpoint{3.848709in}{3.171347in}}%
\pgfpathcurveto{\pgfqpoint{3.862974in}{3.171347in}}{\pgfqpoint{3.876658in}{3.177015in}}{\pgfqpoint{3.886745in}{3.187102in}}%
\pgfpathcurveto{\pgfqpoint{3.896832in}{3.197190in}}{\pgfqpoint{3.902500in}{3.210873in}}{\pgfqpoint{3.902500in}{3.225139in}}%
\pgfpathcurveto{\pgfqpoint{3.902500in}{3.239404in}}{\pgfqpoint{3.896832in}{3.253088in}}{\pgfqpoint{3.886745in}{3.263175in}}%
\pgfpathcurveto{\pgfqpoint{3.876658in}{3.273262in}}{\pgfqpoint{3.862974in}{3.278930in}}{\pgfqpoint{3.848709in}{3.278930in}}%
\pgfpathcurveto{\pgfqpoint{3.834443in}{3.278930in}}{\pgfqpoint{3.820760in}{3.273262in}}{\pgfqpoint{3.810672in}{3.263175in}}%
\pgfpathcurveto{\pgfqpoint{3.800585in}{3.253088in}}{\pgfqpoint{3.794917in}{3.239404in}}{\pgfqpoint{3.794917in}{3.225139in}}%
\pgfpathcurveto{\pgfqpoint{3.794917in}{3.210873in}}{\pgfqpoint{3.800585in}{3.197190in}}{\pgfqpoint{3.810672in}{3.187102in}}%
\pgfpathcurveto{\pgfqpoint{3.820760in}{3.177015in}}{\pgfqpoint{3.834443in}{3.171347in}}{\pgfqpoint{3.848709in}{3.171347in}}%
\pgfpathclose%
\pgfusepath{stroke}%
\end{pgfscope}%
\begin{pgfscope}%
\pgfpathrectangle{\pgfqpoint{0.678743in}{0.883555in}}{\pgfqpoint{4.650000in}{3.020000in}}%
\pgfusepath{clip}%
\pgfsetbuttcap%
\pgfsetroundjoin%
\pgfsetlinewidth{1.003750pt}%
\definecolor{currentstroke}{rgb}{0.000000,0.000000,1.000000}%
\pgfsetstrokecolor{currentstroke}%
\pgfsetdash{}{0pt}%
\pgfpathmoveto{\pgfqpoint{3.954329in}{3.159254in}}%
\pgfpathcurveto{\pgfqpoint{3.968595in}{3.159254in}}{\pgfqpoint{3.982278in}{3.164921in}}{\pgfqpoint{3.992366in}{3.175009in}}%
\pgfpathcurveto{\pgfqpoint{4.002453in}{3.185096in}}{\pgfqpoint{4.008121in}{3.198779in}}{\pgfqpoint{4.008121in}{3.213045in}}%
\pgfpathcurveto{\pgfqpoint{4.008121in}{3.227311in}}{\pgfqpoint{4.002453in}{3.240994in}}{\pgfqpoint{3.992366in}{3.251081in}}%
\pgfpathcurveto{\pgfqpoint{3.982278in}{3.261169in}}{\pgfqpoint{3.968595in}{3.266836in}}{\pgfqpoint{3.954329in}{3.266836in}}%
\pgfpathcurveto{\pgfqpoint{3.940064in}{3.266836in}}{\pgfqpoint{3.926380in}{3.261169in}}{\pgfqpoint{3.916293in}{3.251081in}}%
\pgfpathcurveto{\pgfqpoint{3.906206in}{3.240994in}}{\pgfqpoint{3.900538in}{3.227311in}}{\pgfqpoint{3.900538in}{3.213045in}}%
\pgfpathcurveto{\pgfqpoint{3.900538in}{3.198779in}}{\pgfqpoint{3.906206in}{3.185096in}}{\pgfqpoint{3.916293in}{3.175009in}}%
\pgfpathcurveto{\pgfqpoint{3.926380in}{3.164921in}}{\pgfqpoint{3.940064in}{3.159254in}}{\pgfqpoint{3.954329in}{3.159254in}}%
\pgfpathclose%
\pgfusepath{stroke}%
\end{pgfscope}%
\begin{pgfscope}%
\pgfpathrectangle{\pgfqpoint{0.678743in}{0.883555in}}{\pgfqpoint{4.650000in}{3.020000in}}%
\pgfusepath{clip}%
\pgfsetbuttcap%
\pgfsetroundjoin%
\pgfsetlinewidth{1.003750pt}%
\definecolor{currentstroke}{rgb}{0.000000,0.000000,1.000000}%
\pgfsetstrokecolor{currentstroke}%
\pgfsetdash{}{0pt}%
\pgfpathmoveto{\pgfqpoint{4.059950in}{3.189700in}}%
\pgfpathcurveto{\pgfqpoint{4.074216in}{3.189700in}}{\pgfqpoint{4.087899in}{3.195368in}}{\pgfqpoint{4.097986in}{3.205455in}}%
\pgfpathcurveto{\pgfqpoint{4.108074in}{3.215542in}}{\pgfqpoint{4.113741in}{3.229226in}}{\pgfqpoint{4.113741in}{3.243491in}}%
\pgfpathcurveto{\pgfqpoint{4.113741in}{3.257757in}}{\pgfqpoint{4.108074in}{3.271440in}}{\pgfqpoint{4.097986in}{3.281528in}}%
\pgfpathcurveto{\pgfqpoint{4.087899in}{3.291615in}}{\pgfqpoint{4.074216in}{3.297283in}}{\pgfqpoint{4.059950in}{3.297283in}}%
\pgfpathcurveto{\pgfqpoint{4.045684in}{3.297283in}}{\pgfqpoint{4.032001in}{3.291615in}}{\pgfqpoint{4.021914in}{3.281528in}}%
\pgfpathcurveto{\pgfqpoint{4.011826in}{3.271440in}}{\pgfqpoint{4.006159in}{3.257757in}}{\pgfqpoint{4.006159in}{3.243491in}}%
\pgfpathcurveto{\pgfqpoint{4.006159in}{3.229226in}}{\pgfqpoint{4.011826in}{3.215542in}}{\pgfqpoint{4.021914in}{3.205455in}}%
\pgfpathcurveto{\pgfqpoint{4.032001in}{3.195368in}}{\pgfqpoint{4.045684in}{3.189700in}}{\pgfqpoint{4.059950in}{3.189700in}}%
\pgfpathclose%
\pgfusepath{stroke}%
\end{pgfscope}%
\begin{pgfscope}%
\pgfpathrectangle{\pgfqpoint{0.678743in}{0.883555in}}{\pgfqpoint{4.650000in}{3.020000in}}%
\pgfusepath{clip}%
\pgfsetbuttcap%
\pgfsetroundjoin%
\pgfsetlinewidth{1.003750pt}%
\definecolor{currentstroke}{rgb}{0.000000,0.000000,1.000000}%
\pgfsetstrokecolor{currentstroke}%
\pgfsetdash{}{0pt}%
\pgfpathmoveto{\pgfqpoint{4.165571in}{3.124797in}}%
\pgfpathcurveto{\pgfqpoint{4.179836in}{3.124797in}}{\pgfqpoint{4.193520in}{3.130465in}}{\pgfqpoint{4.203607in}{3.140553in}}%
\pgfpathcurveto{\pgfqpoint{4.213694in}{3.150640in}}{\pgfqpoint{4.219362in}{3.164323in}}{\pgfqpoint{4.219362in}{3.178589in}}%
\pgfpathcurveto{\pgfqpoint{4.219362in}{3.192854in}}{\pgfqpoint{4.213694in}{3.206538in}}{\pgfqpoint{4.203607in}{3.216625in}}%
\pgfpathcurveto{\pgfqpoint{4.193520in}{3.226712in}}{\pgfqpoint{4.179836in}{3.232380in}}{\pgfqpoint{4.165571in}{3.232380in}}%
\pgfpathcurveto{\pgfqpoint{4.151305in}{3.232380in}}{\pgfqpoint{4.137622in}{3.226712in}}{\pgfqpoint{4.127534in}{3.216625in}}%
\pgfpathcurveto{\pgfqpoint{4.117447in}{3.206538in}}{\pgfqpoint{4.111779in}{3.192854in}}{\pgfqpoint{4.111779in}{3.178589in}}%
\pgfpathcurveto{\pgfqpoint{4.111779in}{3.164323in}}{\pgfqpoint{4.117447in}{3.150640in}}{\pgfqpoint{4.127534in}{3.140553in}}%
\pgfpathcurveto{\pgfqpoint{4.137622in}{3.130465in}}{\pgfqpoint{4.151305in}{3.124797in}}{\pgfqpoint{4.165571in}{3.124797in}}%
\pgfpathclose%
\pgfusepath{stroke}%
\end{pgfscope}%
\begin{pgfscope}%
\pgfpathrectangle{\pgfqpoint{0.678743in}{0.883555in}}{\pgfqpoint{4.650000in}{3.020000in}}%
\pgfusepath{clip}%
\pgfsetbuttcap%
\pgfsetroundjoin%
\pgfsetlinewidth{1.003750pt}%
\definecolor{currentstroke}{rgb}{0.000000,0.000000,1.000000}%
\pgfsetstrokecolor{currentstroke}%
\pgfsetdash{}{0pt}%
\pgfpathmoveto{\pgfqpoint{4.271192in}{3.127881in}}%
\pgfpathcurveto{\pgfqpoint{4.285457in}{3.127881in}}{\pgfqpoint{4.299140in}{3.133549in}}{\pgfqpoint{4.309228in}{3.143636in}}%
\pgfpathcurveto{\pgfqpoint{4.319315in}{3.153724in}}{\pgfqpoint{4.324983in}{3.167407in}}{\pgfqpoint{4.324983in}{3.181673in}}%
\pgfpathcurveto{\pgfqpoint{4.324983in}{3.195938in}}{\pgfqpoint{4.319315in}{3.209622in}}{\pgfqpoint{4.309228in}{3.219709in}}%
\pgfpathcurveto{\pgfqpoint{4.299140in}{3.229796in}}{\pgfqpoint{4.285457in}{3.235464in}}{\pgfqpoint{4.271192in}{3.235464in}}%
\pgfpathcurveto{\pgfqpoint{4.256926in}{3.235464in}}{\pgfqpoint{4.243243in}{3.229796in}}{\pgfqpoint{4.233155in}{3.219709in}}%
\pgfpathcurveto{\pgfqpoint{4.223068in}{3.209622in}}{\pgfqpoint{4.217400in}{3.195938in}}{\pgfqpoint{4.217400in}{3.181673in}}%
\pgfpathcurveto{\pgfqpoint{4.217400in}{3.167407in}}{\pgfqpoint{4.223068in}{3.153724in}}{\pgfqpoint{4.233155in}{3.143636in}}%
\pgfpathcurveto{\pgfqpoint{4.243243in}{3.133549in}}{\pgfqpoint{4.256926in}{3.127881in}}{\pgfqpoint{4.271192in}{3.127881in}}%
\pgfpathclose%
\pgfusepath{stroke}%
\end{pgfscope}%
\begin{pgfscope}%
\pgfpathrectangle{\pgfqpoint{0.678743in}{0.883555in}}{\pgfqpoint{4.650000in}{3.020000in}}%
\pgfusepath{clip}%
\pgfsetbuttcap%
\pgfsetroundjoin%
\pgfsetlinewidth{1.003750pt}%
\definecolor{currentstroke}{rgb}{0.000000,0.000000,1.000000}%
\pgfsetstrokecolor{currentstroke}%
\pgfsetdash{}{0pt}%
\pgfpathmoveto{\pgfqpoint{4.376812in}{3.092647in}}%
\pgfpathcurveto{\pgfqpoint{4.391078in}{3.092647in}}{\pgfqpoint{4.404761in}{3.098314in}}{\pgfqpoint{4.414849in}{3.108402in}}%
\pgfpathcurveto{\pgfqpoint{4.424936in}{3.118489in}}{\pgfqpoint{4.430604in}{3.132172in}}{\pgfqpoint{4.430604in}{3.146438in}}%
\pgfpathcurveto{\pgfqpoint{4.430604in}{3.160704in}}{\pgfqpoint{4.424936in}{3.174387in}}{\pgfqpoint{4.414849in}{3.184474in}}%
\pgfpathcurveto{\pgfqpoint{4.404761in}{3.194562in}}{\pgfqpoint{4.391078in}{3.200229in}}{\pgfqpoint{4.376812in}{3.200229in}}%
\pgfpathcurveto{\pgfqpoint{4.362547in}{3.200229in}}{\pgfqpoint{4.348863in}{3.194562in}}{\pgfqpoint{4.338776in}{3.184474in}}%
\pgfpathcurveto{\pgfqpoint{4.328689in}{3.174387in}}{\pgfqpoint{4.323021in}{3.160704in}}{\pgfqpoint{4.323021in}{3.146438in}}%
\pgfpathcurveto{\pgfqpoint{4.323021in}{3.132172in}}{\pgfqpoint{4.328689in}{3.118489in}}{\pgfqpoint{4.338776in}{3.108402in}}%
\pgfpathcurveto{\pgfqpoint{4.348863in}{3.098314in}}{\pgfqpoint{4.362547in}{3.092647in}}{\pgfqpoint{4.376812in}{3.092647in}}%
\pgfpathclose%
\pgfusepath{stroke}%
\end{pgfscope}%
\begin{pgfscope}%
\pgfpathrectangle{\pgfqpoint{0.678743in}{0.883555in}}{\pgfqpoint{4.650000in}{3.020000in}}%
\pgfusepath{clip}%
\pgfsetbuttcap%
\pgfsetroundjoin%
\pgfsetlinewidth{1.003750pt}%
\definecolor{currentstroke}{rgb}{0.000000,0.000000,1.000000}%
\pgfsetstrokecolor{currentstroke}%
\pgfsetdash{}{0pt}%
\pgfpathmoveto{\pgfqpoint{4.482433in}{3.104491in}}%
\pgfpathcurveto{\pgfqpoint{4.496699in}{3.104491in}}{\pgfqpoint{4.510382in}{3.110159in}}{\pgfqpoint{4.520469in}{3.120246in}}%
\pgfpathcurveto{\pgfqpoint{4.530557in}{3.130334in}}{\pgfqpoint{4.536224in}{3.144017in}}{\pgfqpoint{4.536224in}{3.158283in}}%
\pgfpathcurveto{\pgfqpoint{4.536224in}{3.172548in}}{\pgfqpoint{4.530557in}{3.186232in}}{\pgfqpoint{4.520469in}{3.196319in}}%
\pgfpathcurveto{\pgfqpoint{4.510382in}{3.206406in}}{\pgfqpoint{4.496699in}{3.212074in}}{\pgfqpoint{4.482433in}{3.212074in}}%
\pgfpathcurveto{\pgfqpoint{4.468167in}{3.212074in}}{\pgfqpoint{4.454484in}{3.206406in}}{\pgfqpoint{4.444397in}{3.196319in}}%
\pgfpathcurveto{\pgfqpoint{4.434309in}{3.186232in}}{\pgfqpoint{4.428642in}{3.172548in}}{\pgfqpoint{4.428642in}{3.158283in}}%
\pgfpathcurveto{\pgfqpoint{4.428642in}{3.144017in}}{\pgfqpoint{4.434309in}{3.130334in}}{\pgfqpoint{4.444397in}{3.120246in}}%
\pgfpathcurveto{\pgfqpoint{4.454484in}{3.110159in}}{\pgfqpoint{4.468167in}{3.104491in}}{\pgfqpoint{4.482433in}{3.104491in}}%
\pgfpathclose%
\pgfusepath{stroke}%
\end{pgfscope}%
\begin{pgfscope}%
\pgfpathrectangle{\pgfqpoint{0.678743in}{0.883555in}}{\pgfqpoint{4.650000in}{3.020000in}}%
\pgfusepath{clip}%
\pgfsetbuttcap%
\pgfsetroundjoin%
\pgfsetlinewidth{1.003750pt}%
\definecolor{currentstroke}{rgb}{0.000000,0.000000,1.000000}%
\pgfsetstrokecolor{currentstroke}%
\pgfsetdash{}{0pt}%
\pgfpathmoveto{\pgfqpoint{4.588054in}{3.108037in}}%
\pgfpathcurveto{\pgfqpoint{4.602319in}{3.108037in}}{\pgfqpoint{4.616003in}{3.113705in}}{\pgfqpoint{4.626090in}{3.123792in}}%
\pgfpathcurveto{\pgfqpoint{4.636177in}{3.133880in}}{\pgfqpoint{4.641845in}{3.147563in}}{\pgfqpoint{4.641845in}{3.161828in}}%
\pgfpathcurveto{\pgfqpoint{4.641845in}{3.176094in}}{\pgfqpoint{4.636177in}{3.189777in}}{\pgfqpoint{4.626090in}{3.199865in}}%
\pgfpathcurveto{\pgfqpoint{4.616003in}{3.209952in}}{\pgfqpoint{4.602319in}{3.215620in}}{\pgfqpoint{4.588054in}{3.215620in}}%
\pgfpathcurveto{\pgfqpoint{4.573788in}{3.215620in}}{\pgfqpoint{4.560105in}{3.209952in}}{\pgfqpoint{4.550017in}{3.199865in}}%
\pgfpathcurveto{\pgfqpoint{4.539930in}{3.189777in}}{\pgfqpoint{4.534262in}{3.176094in}}{\pgfqpoint{4.534262in}{3.161828in}}%
\pgfpathcurveto{\pgfqpoint{4.534262in}{3.147563in}}{\pgfqpoint{4.539930in}{3.133880in}}{\pgfqpoint{4.550017in}{3.123792in}}%
\pgfpathcurveto{\pgfqpoint{4.560105in}{3.113705in}}{\pgfqpoint{4.573788in}{3.108037in}}{\pgfqpoint{4.588054in}{3.108037in}}%
\pgfpathclose%
\pgfusepath{stroke}%
\end{pgfscope}%
\begin{pgfscope}%
\pgfpathrectangle{\pgfqpoint{0.678743in}{0.883555in}}{\pgfqpoint{4.650000in}{3.020000in}}%
\pgfusepath{clip}%
\pgfsetbuttcap%
\pgfsetroundjoin%
\pgfsetlinewidth{1.003750pt}%
\definecolor{currentstroke}{rgb}{0.000000,0.000000,1.000000}%
\pgfsetstrokecolor{currentstroke}%
\pgfsetdash{}{0pt}%
\pgfpathmoveto{\pgfqpoint{4.693674in}{3.056524in}}%
\pgfpathcurveto{\pgfqpoint{4.707940in}{3.056524in}}{\pgfqpoint{4.721623in}{3.062192in}}{\pgfqpoint{4.731711in}{3.072279in}}%
\pgfpathcurveto{\pgfqpoint{4.741798in}{3.082366in}}{\pgfqpoint{4.747466in}{3.096050in}}{\pgfqpoint{4.747466in}{3.110315in}}%
\pgfpathcurveto{\pgfqpoint{4.747466in}{3.124581in}}{\pgfqpoint{4.741798in}{3.138264in}}{\pgfqpoint{4.731711in}{3.148352in}}%
\pgfpathcurveto{\pgfqpoint{4.721623in}{3.158439in}}{\pgfqpoint{4.707940in}{3.164107in}}{\pgfqpoint{4.693674in}{3.164107in}}%
\pgfpathcurveto{\pgfqpoint{4.679409in}{3.164107in}}{\pgfqpoint{4.665725in}{3.158439in}}{\pgfqpoint{4.655638in}{3.148352in}}%
\pgfpathcurveto{\pgfqpoint{4.645551in}{3.138264in}}{\pgfqpoint{4.639883in}{3.124581in}}{\pgfqpoint{4.639883in}{3.110315in}}%
\pgfpathcurveto{\pgfqpoint{4.639883in}{3.096050in}}{\pgfqpoint{4.645551in}{3.082366in}}{\pgfqpoint{4.655638in}{3.072279in}}%
\pgfpathcurveto{\pgfqpoint{4.665725in}{3.062192in}}{\pgfqpoint{4.679409in}{3.056524in}}{\pgfqpoint{4.693674in}{3.056524in}}%
\pgfpathclose%
\pgfusepath{stroke}%
\end{pgfscope}%
\begin{pgfscope}%
\pgfpathrectangle{\pgfqpoint{0.678743in}{0.883555in}}{\pgfqpoint{4.650000in}{3.020000in}}%
\pgfusepath{clip}%
\pgfsetbuttcap%
\pgfsetroundjoin%
\pgfsetlinewidth{1.003750pt}%
\definecolor{currentstroke}{rgb}{0.000000,0.000000,1.000000}%
\pgfsetstrokecolor{currentstroke}%
\pgfsetdash{}{0pt}%
\pgfpathmoveto{\pgfqpoint{4.799295in}{3.017521in}}%
\pgfpathcurveto{\pgfqpoint{4.813561in}{3.017521in}}{\pgfqpoint{4.827244in}{3.023188in}}{\pgfqpoint{4.837331in}{3.033276in}}%
\pgfpathcurveto{\pgfqpoint{4.847419in}{3.043363in}}{\pgfqpoint{4.853087in}{3.057046in}}{\pgfqpoint{4.853087in}{3.071312in}}%
\pgfpathcurveto{\pgfqpoint{4.853087in}{3.085578in}}{\pgfqpoint{4.847419in}{3.099261in}}{\pgfqpoint{4.837331in}{3.109348in}}%
\pgfpathcurveto{\pgfqpoint{4.827244in}{3.119436in}}{\pgfqpoint{4.813561in}{3.125104in}}{\pgfqpoint{4.799295in}{3.125104in}}%
\pgfpathcurveto{\pgfqpoint{4.785029in}{3.125104in}}{\pgfqpoint{4.771346in}{3.119436in}}{\pgfqpoint{4.761259in}{3.109348in}}%
\pgfpathcurveto{\pgfqpoint{4.751172in}{3.099261in}}{\pgfqpoint{4.745504in}{3.085578in}}{\pgfqpoint{4.745504in}{3.071312in}}%
\pgfpathcurveto{\pgfqpoint{4.745504in}{3.057046in}}{\pgfqpoint{4.751172in}{3.043363in}}{\pgfqpoint{4.761259in}{3.033276in}}%
\pgfpathcurveto{\pgfqpoint{4.771346in}{3.023188in}}{\pgfqpoint{4.785029in}{3.017521in}}{\pgfqpoint{4.799295in}{3.017521in}}%
\pgfpathclose%
\pgfusepath{stroke}%
\end{pgfscope}%
\begin{pgfscope}%
\pgfpathrectangle{\pgfqpoint{0.678743in}{0.883555in}}{\pgfqpoint{4.650000in}{3.020000in}}%
\pgfusepath{clip}%
\pgfsetbuttcap%
\pgfsetroundjoin%
\pgfsetlinewidth{1.003750pt}%
\definecolor{currentstroke}{rgb}{0.000000,0.000000,1.000000}%
\pgfsetstrokecolor{currentstroke}%
\pgfsetdash{}{0pt}%
\pgfpathmoveto{\pgfqpoint{4.904916in}{3.053750in}}%
\pgfpathcurveto{\pgfqpoint{4.919182in}{3.053750in}}{\pgfqpoint{4.932865in}{3.059417in}}{\pgfqpoint{4.942952in}{3.069505in}}%
\pgfpathcurveto{\pgfqpoint{4.953039in}{3.079592in}}{\pgfqpoint{4.958707in}{3.093275in}}{\pgfqpoint{4.958707in}{3.107541in}}%
\pgfpathcurveto{\pgfqpoint{4.958707in}{3.121807in}}{\pgfqpoint{4.953039in}{3.135490in}}{\pgfqpoint{4.942952in}{3.145577in}}%
\pgfpathcurveto{\pgfqpoint{4.932865in}{3.155665in}}{\pgfqpoint{4.919182in}{3.161332in}}{\pgfqpoint{4.904916in}{3.161332in}}%
\pgfpathcurveto{\pgfqpoint{4.890650in}{3.161332in}}{\pgfqpoint{4.876967in}{3.155665in}}{\pgfqpoint{4.866880in}{3.145577in}}%
\pgfpathcurveto{\pgfqpoint{4.856792in}{3.135490in}}{\pgfqpoint{4.851124in}{3.121807in}}{\pgfqpoint{4.851124in}{3.107541in}}%
\pgfpathcurveto{\pgfqpoint{4.851124in}{3.093275in}}{\pgfqpoint{4.856792in}{3.079592in}}{\pgfqpoint{4.866880in}{3.069505in}}%
\pgfpathcurveto{\pgfqpoint{4.876967in}{3.059417in}}{\pgfqpoint{4.890650in}{3.053750in}}{\pgfqpoint{4.904916in}{3.053750in}}%
\pgfpathclose%
\pgfusepath{stroke}%
\end{pgfscope}%
\begin{pgfscope}%
\pgfpathrectangle{\pgfqpoint{0.678743in}{0.883555in}}{\pgfqpoint{4.650000in}{3.020000in}}%
\pgfusepath{clip}%
\pgfsetbuttcap%
\pgfsetroundjoin%
\pgfsetlinewidth{1.003750pt}%
\definecolor{currentstroke}{rgb}{0.000000,0.000000,1.000000}%
\pgfsetstrokecolor{currentstroke}%
\pgfsetdash{}{0pt}%
\pgfpathmoveto{\pgfqpoint{5.010537in}{3.012508in}}%
\pgfpathcurveto{\pgfqpoint{5.024802in}{3.012508in}}{\pgfqpoint{5.038486in}{3.018176in}}{\pgfqpoint{5.048573in}{3.028263in}}%
\pgfpathcurveto{\pgfqpoint{5.058660in}{3.038350in}}{\pgfqpoint{5.064328in}{3.052034in}}{\pgfqpoint{5.064328in}{3.066299in}}%
\pgfpathcurveto{\pgfqpoint{5.064328in}{3.080565in}}{\pgfqpoint{5.058660in}{3.094248in}}{\pgfqpoint{5.048573in}{3.104336in}}%
\pgfpathcurveto{\pgfqpoint{5.038486in}{3.114423in}}{\pgfqpoint{5.024802in}{3.120091in}}{\pgfqpoint{5.010537in}{3.120091in}}%
\pgfpathcurveto{\pgfqpoint{4.996271in}{3.120091in}}{\pgfqpoint{4.982588in}{3.114423in}}{\pgfqpoint{4.972500in}{3.104336in}}%
\pgfpathcurveto{\pgfqpoint{4.962413in}{3.094248in}}{\pgfqpoint{4.956745in}{3.080565in}}{\pgfqpoint{4.956745in}{3.066299in}}%
\pgfpathcurveto{\pgfqpoint{4.956745in}{3.052034in}}{\pgfqpoint{4.962413in}{3.038350in}}{\pgfqpoint{4.972500in}{3.028263in}}%
\pgfpathcurveto{\pgfqpoint{4.982588in}{3.018176in}}{\pgfqpoint{4.996271in}{3.012508in}}{\pgfqpoint{5.010537in}{3.012508in}}%
\pgfpathclose%
\pgfusepath{stroke}%
\end{pgfscope}%
\begin{pgfscope}%
\pgfpathrectangle{\pgfqpoint{0.678743in}{0.883555in}}{\pgfqpoint{4.650000in}{3.020000in}}%
\pgfusepath{clip}%
\pgfsetbuttcap%
\pgfsetroundjoin%
\pgfsetlinewidth{1.003750pt}%
\definecolor{currentstroke}{rgb}{0.000000,0.000000,1.000000}%
\pgfsetstrokecolor{currentstroke}%
\pgfsetdash{}{0pt}%
\pgfpathmoveto{\pgfqpoint{5.116157in}{2.959582in}}%
\pgfpathcurveto{\pgfqpoint{5.130423in}{2.959582in}}{\pgfqpoint{5.144106in}{2.965250in}}{\pgfqpoint{5.154194in}{2.975338in}}%
\pgfpathcurveto{\pgfqpoint{5.164281in}{2.985425in}}{\pgfqpoint{5.169949in}{2.999108in}}{\pgfqpoint{5.169949in}{3.013374in}}%
\pgfpathcurveto{\pgfqpoint{5.169949in}{3.027640in}}{\pgfqpoint{5.164281in}{3.041323in}}{\pgfqpoint{5.154194in}{3.051410in}}%
\pgfpathcurveto{\pgfqpoint{5.144106in}{3.061498in}}{\pgfqpoint{5.130423in}{3.067165in}}{\pgfqpoint{5.116157in}{3.067165in}}%
\pgfpathcurveto{\pgfqpoint{5.101892in}{3.067165in}}{\pgfqpoint{5.088208in}{3.061498in}}{\pgfqpoint{5.078121in}{3.051410in}}%
\pgfpathcurveto{\pgfqpoint{5.068034in}{3.041323in}}{\pgfqpoint{5.062366in}{3.027640in}}{\pgfqpoint{5.062366in}{3.013374in}}%
\pgfpathcurveto{\pgfqpoint{5.062366in}{2.999108in}}{\pgfqpoint{5.068034in}{2.985425in}}{\pgfqpoint{5.078121in}{2.975338in}}%
\pgfpathcurveto{\pgfqpoint{5.088208in}{2.965250in}}{\pgfqpoint{5.101892in}{2.959582in}}{\pgfqpoint{5.116157in}{2.959582in}}%
\pgfpathclose%
\pgfusepath{stroke}%
\end{pgfscope}%
\begin{pgfscope}%
\pgfsetbuttcap%
\pgfsetroundjoin%
\definecolor{currentfill}{rgb}{0.000000,0.000000,0.000000}%
\pgfsetfillcolor{currentfill}%
\pgfsetlinewidth{0.803000pt}%
\definecolor{currentstroke}{rgb}{0.000000,0.000000,0.000000}%
\pgfsetstrokecolor{currentstroke}%
\pgfsetdash{}{0pt}%
\pgfsys@defobject{currentmarker}{\pgfqpoint{0.000000in}{-0.048611in}}{\pgfqpoint{0.000000in}{0.000000in}}{%
\pgfpathmoveto{\pgfqpoint{0.000000in}{0.000000in}}%
\pgfpathlineto{\pgfqpoint{0.000000in}{-0.048611in}}%
\pgfusepath{stroke,fill}%
}%
\begin{pgfscope}%
\pgfsys@transformshift{0.891328in}{0.883555in}%
\pgfsys@useobject{currentmarker}{}%
\end{pgfscope}%
\end{pgfscope}%
\begin{pgfscope}%
\definecolor{textcolor}{rgb}{0.000000,0.000000,0.000000}%
\pgfsetstrokecolor{textcolor}%
\pgfsetfillcolor{textcolor}%
\pgftext[x=0.891328in,y=0.786333in,,top]{\color{textcolor}\rmfamily\fontsize{11.000000}{13.200000}\selectfont \(\displaystyle 20\)}%
\end{pgfscope}%
\begin{pgfscope}%
\pgfsetbuttcap%
\pgfsetroundjoin%
\definecolor{currentfill}{rgb}{0.000000,0.000000,0.000000}%
\pgfsetfillcolor{currentfill}%
\pgfsetlinewidth{0.803000pt}%
\definecolor{currentstroke}{rgb}{0.000000,0.000000,0.000000}%
\pgfsetstrokecolor{currentstroke}%
\pgfsetdash{}{0pt}%
\pgfsys@defobject{currentmarker}{\pgfqpoint{0.000000in}{-0.048611in}}{\pgfqpoint{0.000000in}{0.000000in}}{%
\pgfpathmoveto{\pgfqpoint{0.000000in}{0.000000in}}%
\pgfpathlineto{\pgfqpoint{0.000000in}{-0.048611in}}%
\pgfusepath{stroke,fill}%
}%
\begin{pgfscope}%
\pgfsys@transformshift{1.419432in}{0.883555in}%
\pgfsys@useobject{currentmarker}{}%
\end{pgfscope}%
\end{pgfscope}%
\begin{pgfscope}%
\definecolor{textcolor}{rgb}{0.000000,0.000000,0.000000}%
\pgfsetstrokecolor{textcolor}%
\pgfsetfillcolor{textcolor}%
\pgftext[x=1.419432in,y=0.786333in,,top]{\color{textcolor}\rmfamily\fontsize{11.000000}{13.200000}\selectfont \(\displaystyle 25\)}%
\end{pgfscope}%
\begin{pgfscope}%
\pgfsetbuttcap%
\pgfsetroundjoin%
\definecolor{currentfill}{rgb}{0.000000,0.000000,0.000000}%
\pgfsetfillcolor{currentfill}%
\pgfsetlinewidth{0.803000pt}%
\definecolor{currentstroke}{rgb}{0.000000,0.000000,0.000000}%
\pgfsetstrokecolor{currentstroke}%
\pgfsetdash{}{0pt}%
\pgfsys@defobject{currentmarker}{\pgfqpoint{0.000000in}{-0.048611in}}{\pgfqpoint{0.000000in}{0.000000in}}{%
\pgfpathmoveto{\pgfqpoint{0.000000in}{0.000000in}}%
\pgfpathlineto{\pgfqpoint{0.000000in}{-0.048611in}}%
\pgfusepath{stroke,fill}%
}%
\begin{pgfscope}%
\pgfsys@transformshift{1.947536in}{0.883555in}%
\pgfsys@useobject{currentmarker}{}%
\end{pgfscope}%
\end{pgfscope}%
\begin{pgfscope}%
\definecolor{textcolor}{rgb}{0.000000,0.000000,0.000000}%
\pgfsetstrokecolor{textcolor}%
\pgfsetfillcolor{textcolor}%
\pgftext[x=1.947536in,y=0.786333in,,top]{\color{textcolor}\rmfamily\fontsize{11.000000}{13.200000}\selectfont \(\displaystyle 30\)}%
\end{pgfscope}%
\begin{pgfscope}%
\pgfsetbuttcap%
\pgfsetroundjoin%
\definecolor{currentfill}{rgb}{0.000000,0.000000,0.000000}%
\pgfsetfillcolor{currentfill}%
\pgfsetlinewidth{0.803000pt}%
\definecolor{currentstroke}{rgb}{0.000000,0.000000,0.000000}%
\pgfsetstrokecolor{currentstroke}%
\pgfsetdash{}{0pt}%
\pgfsys@defobject{currentmarker}{\pgfqpoint{0.000000in}{-0.048611in}}{\pgfqpoint{0.000000in}{0.000000in}}{%
\pgfpathmoveto{\pgfqpoint{0.000000in}{0.000000in}}%
\pgfpathlineto{\pgfqpoint{0.000000in}{-0.048611in}}%
\pgfusepath{stroke,fill}%
}%
\begin{pgfscope}%
\pgfsys@transformshift{2.475639in}{0.883555in}%
\pgfsys@useobject{currentmarker}{}%
\end{pgfscope}%
\end{pgfscope}%
\begin{pgfscope}%
\definecolor{textcolor}{rgb}{0.000000,0.000000,0.000000}%
\pgfsetstrokecolor{textcolor}%
\pgfsetfillcolor{textcolor}%
\pgftext[x=2.475639in,y=0.786333in,,top]{\color{textcolor}\rmfamily\fontsize{11.000000}{13.200000}\selectfont \(\displaystyle 35\)}%
\end{pgfscope}%
\begin{pgfscope}%
\pgfsetbuttcap%
\pgfsetroundjoin%
\definecolor{currentfill}{rgb}{0.000000,0.000000,0.000000}%
\pgfsetfillcolor{currentfill}%
\pgfsetlinewidth{0.803000pt}%
\definecolor{currentstroke}{rgb}{0.000000,0.000000,0.000000}%
\pgfsetstrokecolor{currentstroke}%
\pgfsetdash{}{0pt}%
\pgfsys@defobject{currentmarker}{\pgfqpoint{0.000000in}{-0.048611in}}{\pgfqpoint{0.000000in}{0.000000in}}{%
\pgfpathmoveto{\pgfqpoint{0.000000in}{0.000000in}}%
\pgfpathlineto{\pgfqpoint{0.000000in}{-0.048611in}}%
\pgfusepath{stroke,fill}%
}%
\begin{pgfscope}%
\pgfsys@transformshift{3.003743in}{0.883555in}%
\pgfsys@useobject{currentmarker}{}%
\end{pgfscope}%
\end{pgfscope}%
\begin{pgfscope}%
\definecolor{textcolor}{rgb}{0.000000,0.000000,0.000000}%
\pgfsetstrokecolor{textcolor}%
\pgfsetfillcolor{textcolor}%
\pgftext[x=3.003743in,y=0.786333in,,top]{\color{textcolor}\rmfamily\fontsize{11.000000}{13.200000}\selectfont \(\displaystyle 40\)}%
\end{pgfscope}%
\begin{pgfscope}%
\pgfsetbuttcap%
\pgfsetroundjoin%
\definecolor{currentfill}{rgb}{0.000000,0.000000,0.000000}%
\pgfsetfillcolor{currentfill}%
\pgfsetlinewidth{0.803000pt}%
\definecolor{currentstroke}{rgb}{0.000000,0.000000,0.000000}%
\pgfsetstrokecolor{currentstroke}%
\pgfsetdash{}{0pt}%
\pgfsys@defobject{currentmarker}{\pgfqpoint{0.000000in}{-0.048611in}}{\pgfqpoint{0.000000in}{0.000000in}}{%
\pgfpathmoveto{\pgfqpoint{0.000000in}{0.000000in}}%
\pgfpathlineto{\pgfqpoint{0.000000in}{-0.048611in}}%
\pgfusepath{stroke,fill}%
}%
\begin{pgfscope}%
\pgfsys@transformshift{3.531846in}{0.883555in}%
\pgfsys@useobject{currentmarker}{}%
\end{pgfscope}%
\end{pgfscope}%
\begin{pgfscope}%
\definecolor{textcolor}{rgb}{0.000000,0.000000,0.000000}%
\pgfsetstrokecolor{textcolor}%
\pgfsetfillcolor{textcolor}%
\pgftext[x=3.531846in,y=0.786333in,,top]{\color{textcolor}\rmfamily\fontsize{11.000000}{13.200000}\selectfont \(\displaystyle 45\)}%
\end{pgfscope}%
\begin{pgfscope}%
\pgfsetbuttcap%
\pgfsetroundjoin%
\definecolor{currentfill}{rgb}{0.000000,0.000000,0.000000}%
\pgfsetfillcolor{currentfill}%
\pgfsetlinewidth{0.803000pt}%
\definecolor{currentstroke}{rgb}{0.000000,0.000000,0.000000}%
\pgfsetstrokecolor{currentstroke}%
\pgfsetdash{}{0pt}%
\pgfsys@defobject{currentmarker}{\pgfqpoint{0.000000in}{-0.048611in}}{\pgfqpoint{0.000000in}{0.000000in}}{%
\pgfpathmoveto{\pgfqpoint{0.000000in}{0.000000in}}%
\pgfpathlineto{\pgfqpoint{0.000000in}{-0.048611in}}%
\pgfusepath{stroke,fill}%
}%
\begin{pgfscope}%
\pgfsys@transformshift{4.059950in}{0.883555in}%
\pgfsys@useobject{currentmarker}{}%
\end{pgfscope}%
\end{pgfscope}%
\begin{pgfscope}%
\definecolor{textcolor}{rgb}{0.000000,0.000000,0.000000}%
\pgfsetstrokecolor{textcolor}%
\pgfsetfillcolor{textcolor}%
\pgftext[x=4.059950in,y=0.786333in,,top]{\color{textcolor}\rmfamily\fontsize{11.000000}{13.200000}\selectfont \(\displaystyle 50\)}%
\end{pgfscope}%
\begin{pgfscope}%
\pgfsetbuttcap%
\pgfsetroundjoin%
\definecolor{currentfill}{rgb}{0.000000,0.000000,0.000000}%
\pgfsetfillcolor{currentfill}%
\pgfsetlinewidth{0.803000pt}%
\definecolor{currentstroke}{rgb}{0.000000,0.000000,0.000000}%
\pgfsetstrokecolor{currentstroke}%
\pgfsetdash{}{0pt}%
\pgfsys@defobject{currentmarker}{\pgfqpoint{0.000000in}{-0.048611in}}{\pgfqpoint{0.000000in}{0.000000in}}{%
\pgfpathmoveto{\pgfqpoint{0.000000in}{0.000000in}}%
\pgfpathlineto{\pgfqpoint{0.000000in}{-0.048611in}}%
\pgfusepath{stroke,fill}%
}%
\begin{pgfscope}%
\pgfsys@transformshift{4.588054in}{0.883555in}%
\pgfsys@useobject{currentmarker}{}%
\end{pgfscope}%
\end{pgfscope}%
\begin{pgfscope}%
\definecolor{textcolor}{rgb}{0.000000,0.000000,0.000000}%
\pgfsetstrokecolor{textcolor}%
\pgfsetfillcolor{textcolor}%
\pgftext[x=4.588054in,y=0.786333in,,top]{\color{textcolor}\rmfamily\fontsize{11.000000}{13.200000}\selectfont \(\displaystyle 55\)}%
\end{pgfscope}%
\begin{pgfscope}%
\pgfsetbuttcap%
\pgfsetroundjoin%
\definecolor{currentfill}{rgb}{0.000000,0.000000,0.000000}%
\pgfsetfillcolor{currentfill}%
\pgfsetlinewidth{0.803000pt}%
\definecolor{currentstroke}{rgb}{0.000000,0.000000,0.000000}%
\pgfsetstrokecolor{currentstroke}%
\pgfsetdash{}{0pt}%
\pgfsys@defobject{currentmarker}{\pgfqpoint{0.000000in}{-0.048611in}}{\pgfqpoint{0.000000in}{0.000000in}}{%
\pgfpathmoveto{\pgfqpoint{0.000000in}{0.000000in}}%
\pgfpathlineto{\pgfqpoint{0.000000in}{-0.048611in}}%
\pgfusepath{stroke,fill}%
}%
\begin{pgfscope}%
\pgfsys@transformshift{5.116157in}{0.883555in}%
\pgfsys@useobject{currentmarker}{}%
\end{pgfscope}%
\end{pgfscope}%
\begin{pgfscope}%
\definecolor{textcolor}{rgb}{0.000000,0.000000,0.000000}%
\pgfsetstrokecolor{textcolor}%
\pgfsetfillcolor{textcolor}%
\pgftext[x=5.116157in,y=0.786333in,,top]{\color{textcolor}\rmfamily\fontsize{11.000000}{13.200000}\selectfont \(\displaystyle 60\)}%
\end{pgfscope}%
\begin{pgfscope}%
\definecolor{textcolor}{rgb}{0.000000,0.000000,0.000000}%
\pgfsetstrokecolor{textcolor}%
\pgfsetfillcolor{textcolor}%
\pgftext[x=3.003743in,y=0.595592in,,top]{\color{textcolor}\rmfamily\fontsize{16.000000}{19.200000}\selectfont Age}%
\end{pgfscope}%
\begin{pgfscope}%
\pgfsetbuttcap%
\pgfsetroundjoin%
\definecolor{currentfill}{rgb}{0.000000,0.000000,0.000000}%
\pgfsetfillcolor{currentfill}%
\pgfsetlinewidth{0.803000pt}%
\definecolor{currentstroke}{rgb}{0.000000,0.000000,0.000000}%
\pgfsetstrokecolor{currentstroke}%
\pgfsetdash{}{0pt}%
\pgfsys@defobject{currentmarker}{\pgfqpoint{-0.048611in}{0.000000in}}{\pgfqpoint{0.000000in}{0.000000in}}{%
\pgfpathmoveto{\pgfqpoint{0.000000in}{0.000000in}}%
\pgfpathlineto{\pgfqpoint{-0.048611in}{0.000000in}}%
\pgfusepath{stroke,fill}%
}%
\begin{pgfscope}%
\pgfsys@transformshift{0.678743in}{0.883555in}%
\pgfsys@useobject{currentmarker}{}%
\end{pgfscope}%
\end{pgfscope}%
\begin{pgfscope}%
\definecolor{textcolor}{rgb}{0.000000,0.000000,0.000000}%
\pgfsetstrokecolor{textcolor}%
\pgfsetfillcolor{textcolor}%
\pgftext[x=0.268904in,y=0.830748in,left,base]{\color{textcolor}\rmfamily\fontsize{11.000000}{13.200000}\selectfont \(\displaystyle -1.0\)}%
\end{pgfscope}%
\begin{pgfscope}%
\pgfsetbuttcap%
\pgfsetroundjoin%
\definecolor{currentfill}{rgb}{0.000000,0.000000,0.000000}%
\pgfsetfillcolor{currentfill}%
\pgfsetlinewidth{0.803000pt}%
\definecolor{currentstroke}{rgb}{0.000000,0.000000,0.000000}%
\pgfsetstrokecolor{currentstroke}%
\pgfsetdash{}{0pt}%
\pgfsys@defobject{currentmarker}{\pgfqpoint{-0.048611in}{0.000000in}}{\pgfqpoint{0.000000in}{0.000000in}}{%
\pgfpathmoveto{\pgfqpoint{0.000000in}{0.000000in}}%
\pgfpathlineto{\pgfqpoint{-0.048611in}{0.000000in}}%
\pgfusepath{stroke,fill}%
}%
\begin{pgfscope}%
\pgfsys@transformshift{0.678743in}{1.286222in}%
\pgfsys@useobject{currentmarker}{}%
\end{pgfscope}%
\end{pgfscope}%
\begin{pgfscope}%
\definecolor{textcolor}{rgb}{0.000000,0.000000,0.000000}%
\pgfsetstrokecolor{textcolor}%
\pgfsetfillcolor{textcolor}%
\pgftext[x=0.268904in,y=1.233415in,left,base]{\color{textcolor}\rmfamily\fontsize{11.000000}{13.200000}\selectfont \(\displaystyle -0.8\)}%
\end{pgfscope}%
\begin{pgfscope}%
\pgfsetbuttcap%
\pgfsetroundjoin%
\definecolor{currentfill}{rgb}{0.000000,0.000000,0.000000}%
\pgfsetfillcolor{currentfill}%
\pgfsetlinewidth{0.803000pt}%
\definecolor{currentstroke}{rgb}{0.000000,0.000000,0.000000}%
\pgfsetstrokecolor{currentstroke}%
\pgfsetdash{}{0pt}%
\pgfsys@defobject{currentmarker}{\pgfqpoint{-0.048611in}{0.000000in}}{\pgfqpoint{0.000000in}{0.000000in}}{%
\pgfpathmoveto{\pgfqpoint{0.000000in}{0.000000in}}%
\pgfpathlineto{\pgfqpoint{-0.048611in}{0.000000in}}%
\pgfusepath{stroke,fill}%
}%
\begin{pgfscope}%
\pgfsys@transformshift{0.678743in}{1.688888in}%
\pgfsys@useobject{currentmarker}{}%
\end{pgfscope}%
\end{pgfscope}%
\begin{pgfscope}%
\definecolor{textcolor}{rgb}{0.000000,0.000000,0.000000}%
\pgfsetstrokecolor{textcolor}%
\pgfsetfillcolor{textcolor}%
\pgftext[x=0.268904in,y=1.636082in,left,base]{\color{textcolor}\rmfamily\fontsize{11.000000}{13.200000}\selectfont \(\displaystyle -0.6\)}%
\end{pgfscope}%
\begin{pgfscope}%
\pgfsetbuttcap%
\pgfsetroundjoin%
\definecolor{currentfill}{rgb}{0.000000,0.000000,0.000000}%
\pgfsetfillcolor{currentfill}%
\pgfsetlinewidth{0.803000pt}%
\definecolor{currentstroke}{rgb}{0.000000,0.000000,0.000000}%
\pgfsetstrokecolor{currentstroke}%
\pgfsetdash{}{0pt}%
\pgfsys@defobject{currentmarker}{\pgfqpoint{-0.048611in}{0.000000in}}{\pgfqpoint{0.000000in}{0.000000in}}{%
\pgfpathmoveto{\pgfqpoint{0.000000in}{0.000000in}}%
\pgfpathlineto{\pgfqpoint{-0.048611in}{0.000000in}}%
\pgfusepath{stroke,fill}%
}%
\begin{pgfscope}%
\pgfsys@transformshift{0.678743in}{2.091555in}%
\pgfsys@useobject{currentmarker}{}%
\end{pgfscope}%
\end{pgfscope}%
\begin{pgfscope}%
\definecolor{textcolor}{rgb}{0.000000,0.000000,0.000000}%
\pgfsetstrokecolor{textcolor}%
\pgfsetfillcolor{textcolor}%
\pgftext[x=0.268904in,y=2.038748in,left,base]{\color{textcolor}\rmfamily\fontsize{11.000000}{13.200000}\selectfont \(\displaystyle -0.4\)}%
\end{pgfscope}%
\begin{pgfscope}%
\pgfsetbuttcap%
\pgfsetroundjoin%
\definecolor{currentfill}{rgb}{0.000000,0.000000,0.000000}%
\pgfsetfillcolor{currentfill}%
\pgfsetlinewidth{0.803000pt}%
\definecolor{currentstroke}{rgb}{0.000000,0.000000,0.000000}%
\pgfsetstrokecolor{currentstroke}%
\pgfsetdash{}{0pt}%
\pgfsys@defobject{currentmarker}{\pgfqpoint{-0.048611in}{0.000000in}}{\pgfqpoint{0.000000in}{0.000000in}}{%
\pgfpathmoveto{\pgfqpoint{0.000000in}{0.000000in}}%
\pgfpathlineto{\pgfqpoint{-0.048611in}{0.000000in}}%
\pgfusepath{stroke,fill}%
}%
\begin{pgfscope}%
\pgfsys@transformshift{0.678743in}{2.494222in}%
\pgfsys@useobject{currentmarker}{}%
\end{pgfscope}%
\end{pgfscope}%
\begin{pgfscope}%
\definecolor{textcolor}{rgb}{0.000000,0.000000,0.000000}%
\pgfsetstrokecolor{textcolor}%
\pgfsetfillcolor{textcolor}%
\pgftext[x=0.268904in,y=2.441415in,left,base]{\color{textcolor}\rmfamily\fontsize{11.000000}{13.200000}\selectfont \(\displaystyle -0.2\)}%
\end{pgfscope}%
\begin{pgfscope}%
\pgfsetbuttcap%
\pgfsetroundjoin%
\definecolor{currentfill}{rgb}{0.000000,0.000000,0.000000}%
\pgfsetfillcolor{currentfill}%
\pgfsetlinewidth{0.803000pt}%
\definecolor{currentstroke}{rgb}{0.000000,0.000000,0.000000}%
\pgfsetstrokecolor{currentstroke}%
\pgfsetdash{}{0pt}%
\pgfsys@defobject{currentmarker}{\pgfqpoint{-0.048611in}{0.000000in}}{\pgfqpoint{0.000000in}{0.000000in}}{%
\pgfpathmoveto{\pgfqpoint{0.000000in}{0.000000in}}%
\pgfpathlineto{\pgfqpoint{-0.048611in}{0.000000in}}%
\pgfusepath{stroke,fill}%
}%
\begin{pgfscope}%
\pgfsys@transformshift{0.678743in}{2.896888in}%
\pgfsys@useobject{currentmarker}{}%
\end{pgfscope}%
\end{pgfscope}%
\begin{pgfscope}%
\definecolor{textcolor}{rgb}{0.000000,0.000000,0.000000}%
\pgfsetstrokecolor{textcolor}%
\pgfsetfillcolor{textcolor}%
\pgftext[x=0.387192in,y=2.844082in,left,base]{\color{textcolor}\rmfamily\fontsize{11.000000}{13.200000}\selectfont \(\displaystyle 0.0\)}%
\end{pgfscope}%
\begin{pgfscope}%
\pgfsetbuttcap%
\pgfsetroundjoin%
\definecolor{currentfill}{rgb}{0.000000,0.000000,0.000000}%
\pgfsetfillcolor{currentfill}%
\pgfsetlinewidth{0.803000pt}%
\definecolor{currentstroke}{rgb}{0.000000,0.000000,0.000000}%
\pgfsetstrokecolor{currentstroke}%
\pgfsetdash{}{0pt}%
\pgfsys@defobject{currentmarker}{\pgfqpoint{-0.048611in}{0.000000in}}{\pgfqpoint{0.000000in}{0.000000in}}{%
\pgfpathmoveto{\pgfqpoint{0.000000in}{0.000000in}}%
\pgfpathlineto{\pgfqpoint{-0.048611in}{0.000000in}}%
\pgfusepath{stroke,fill}%
}%
\begin{pgfscope}%
\pgfsys@transformshift{0.678743in}{3.299555in}%
\pgfsys@useobject{currentmarker}{}%
\end{pgfscope}%
\end{pgfscope}%
\begin{pgfscope}%
\definecolor{textcolor}{rgb}{0.000000,0.000000,0.000000}%
\pgfsetstrokecolor{textcolor}%
\pgfsetfillcolor{textcolor}%
\pgftext[x=0.387192in,y=3.246748in,left,base]{\color{textcolor}\rmfamily\fontsize{11.000000}{13.200000}\selectfont \(\displaystyle 0.2\)}%
\end{pgfscope}%
\begin{pgfscope}%
\pgfsetbuttcap%
\pgfsetroundjoin%
\definecolor{currentfill}{rgb}{0.000000,0.000000,0.000000}%
\pgfsetfillcolor{currentfill}%
\pgfsetlinewidth{0.803000pt}%
\definecolor{currentstroke}{rgb}{0.000000,0.000000,0.000000}%
\pgfsetstrokecolor{currentstroke}%
\pgfsetdash{}{0pt}%
\pgfsys@defobject{currentmarker}{\pgfqpoint{-0.048611in}{0.000000in}}{\pgfqpoint{0.000000in}{0.000000in}}{%
\pgfpathmoveto{\pgfqpoint{0.000000in}{0.000000in}}%
\pgfpathlineto{\pgfqpoint{-0.048611in}{0.000000in}}%
\pgfusepath{stroke,fill}%
}%
\begin{pgfscope}%
\pgfsys@transformshift{0.678743in}{3.702222in}%
\pgfsys@useobject{currentmarker}{}%
\end{pgfscope}%
\end{pgfscope}%
\begin{pgfscope}%
\definecolor{textcolor}{rgb}{0.000000,0.000000,0.000000}%
\pgfsetstrokecolor{textcolor}%
\pgfsetfillcolor{textcolor}%
\pgftext[x=0.387192in,y=3.649415in,left,base]{\color{textcolor}\rmfamily\fontsize{11.000000}{13.200000}\selectfont \(\displaystyle 0.4\)}%
\end{pgfscope}%
\begin{pgfscope}%
\definecolor{textcolor}{rgb}{0.000000,0.000000,0.000000}%
\pgfsetstrokecolor{textcolor}%
\pgfsetfillcolor{textcolor}%
\pgftext[x=0.213349in,y=2.393555in,,bottom,rotate=90.000000]{\color{textcolor}\rmfamily\fontsize{16.000000}{19.200000}\selectfont Average log wage}%
\end{pgfscope}%
\begin{pgfscope}%
\pgfpathrectangle{\pgfqpoint{0.678743in}{0.883555in}}{\pgfqpoint{4.650000in}{3.020000in}}%
\pgfusepath{clip}%
\pgfsetrectcap%
\pgfsetroundjoin%
\pgfsetlinewidth{1.505625pt}%
\definecolor{currentstroke}{rgb}{1.000000,0.000000,0.000000}%
\pgfsetstrokecolor{currentstroke}%
\pgfsetdash{}{0pt}%
\pgfpathmoveto{\pgfqpoint{0.891328in}{2.022682in}}%
\pgfpathlineto{\pgfqpoint{0.996949in}{2.142794in}}%
\pgfpathlineto{\pgfqpoint{1.102570in}{2.253859in}}%
\pgfpathlineto{\pgfqpoint{1.208190in}{2.355726in}}%
\pgfpathlineto{\pgfqpoint{1.313811in}{2.453325in}}%
\pgfpathlineto{\pgfqpoint{1.419432in}{2.542649in}}%
\pgfpathlineto{\pgfqpoint{1.525053in}{2.623968in}}%
\pgfpathlineto{\pgfqpoint{1.630673in}{2.703185in}}%
\pgfpathlineto{\pgfqpoint{1.736294in}{2.771257in}}%
\pgfpathlineto{\pgfqpoint{1.841915in}{2.836793in}}%
\pgfpathlineto{\pgfqpoint{1.947536in}{2.896075in}}%
\pgfpathlineto{\pgfqpoint{2.053156in}{2.954784in}}%
\pgfpathlineto{\pgfqpoint{2.158777in}{3.002890in}}%
\pgfpathlineto{\pgfqpoint{2.264398in}{3.045247in}}%
\pgfpathlineto{\pgfqpoint{2.370018in}{3.081291in}}%
\pgfpathlineto{\pgfqpoint{2.475639in}{3.116768in}}%
\pgfpathlineto{\pgfqpoint{2.581260in}{3.145267in}}%
\pgfpathlineto{\pgfqpoint{2.686881in}{3.168626in}}%
\pgfpathlineto{\pgfqpoint{2.792501in}{3.191007in}}%
\pgfpathlineto{\pgfqpoint{2.898122in}{3.207115in}}%
\pgfpathlineto{\pgfqpoint{3.003743in}{3.219923in}}%
\pgfpathlineto{\pgfqpoint{3.109364in}{3.230160in}}%
\pgfpathlineto{\pgfqpoint{3.214984in}{3.237289in}}%
\pgfpathlineto{\pgfqpoint{3.320605in}{3.240030in}}%
\pgfpathlineto{\pgfqpoint{3.426226in}{3.239269in}}%
\pgfpathlineto{\pgfqpoint{3.531846in}{3.236918in}}%
\pgfpathlineto{\pgfqpoint{3.637467in}{3.230289in}}%
\pgfpathlineto{\pgfqpoint{3.743088in}{3.224726in}}%
\pgfpathlineto{\pgfqpoint{3.848709in}{3.214301in}}%
\pgfpathlineto{\pgfqpoint{3.954329in}{3.201114in}}%
\pgfpathlineto{\pgfqpoint{4.059950in}{3.190422in}}%
\pgfpathlineto{\pgfqpoint{4.165571in}{3.174876in}}%
\pgfpathlineto{\pgfqpoint{4.271192in}{3.156882in}}%
\pgfpathlineto{\pgfqpoint{4.376812in}{3.139273in}}%
\pgfpathlineto{\pgfqpoint{4.482433in}{3.122545in}}%
\pgfpathlineto{\pgfqpoint{4.588054in}{3.103097in}}%
\pgfpathlineto{\pgfqpoint{4.693674in}{3.084771in}}%
\pgfpathlineto{\pgfqpoint{4.799295in}{3.063739in}}%
\pgfpathlineto{\pgfqpoint{4.904916in}{3.043274in}}%
\pgfpathlineto{\pgfqpoint{5.010537in}{3.020076in}}%
\pgfpathlineto{\pgfqpoint{5.116157in}{2.998205in}}%
\pgfusepath{stroke}%
\end{pgfscope}%
\begin{pgfscope}%
\pgfsetrectcap%
\pgfsetmiterjoin%
\pgfsetlinewidth{0.803000pt}%
\definecolor{currentstroke}{rgb}{0.000000,0.000000,0.000000}%
\pgfsetstrokecolor{currentstroke}%
\pgfsetdash{}{0pt}%
\pgfpathmoveto{\pgfqpoint{0.678743in}{0.883555in}}%
\pgfpathlineto{\pgfqpoint{0.678743in}{3.903555in}}%
\pgfusepath{stroke}%
\end{pgfscope}%
\begin{pgfscope}%
\pgfsetrectcap%
\pgfsetmiterjoin%
\pgfsetlinewidth{0.803000pt}%
\definecolor{currentstroke}{rgb}{0.000000,0.000000,0.000000}%
\pgfsetstrokecolor{currentstroke}%
\pgfsetdash{}{0pt}%
\pgfpathmoveto{\pgfqpoint{5.328743in}{0.883555in}}%
\pgfpathlineto{\pgfqpoint{5.328743in}{3.903555in}}%
\pgfusepath{stroke}%
\end{pgfscope}%
\begin{pgfscope}%
\pgfsetrectcap%
\pgfsetmiterjoin%
\pgfsetlinewidth{0.803000pt}%
\definecolor{currentstroke}{rgb}{0.000000,0.000000,0.000000}%
\pgfsetstrokecolor{currentstroke}%
\pgfsetdash{}{0pt}%
\pgfpathmoveto{\pgfqpoint{0.678743in}{0.883555in}}%
\pgfpathlineto{\pgfqpoint{5.328743in}{0.883555in}}%
\pgfusepath{stroke}%
\end{pgfscope}%
\begin{pgfscope}%
\pgfsetrectcap%
\pgfsetmiterjoin%
\pgfsetlinewidth{0.803000pt}%
\definecolor{currentstroke}{rgb}{0.000000,0.000000,0.000000}%
\pgfsetstrokecolor{currentstroke}%
\pgfsetdash{}{0pt}%
\pgfpathmoveto{\pgfqpoint{0.678743in}{3.903555in}}%
\pgfpathlineto{\pgfqpoint{5.328743in}{3.903555in}}%
\pgfusepath{stroke}%
\end{pgfscope}%
\begin{pgfscope}%
\pgfsetbuttcap%
\pgfsetmiterjoin%
\definecolor{currentfill}{rgb}{0.300000,0.300000,0.300000}%
\pgfsetfillcolor{currentfill}%
\pgfsetfillopacity{0.500000}%
\pgfsetlinewidth{1.003750pt}%
\definecolor{currentstroke}{rgb}{0.300000,0.300000,0.300000}%
\pgfsetstrokecolor{currentstroke}%
\pgfsetstrokeopacity{0.500000}%
\pgfsetdash{}{0pt}%
\pgfpathmoveto{\pgfqpoint{1.544689in}{-0.027778in}}%
\pgfpathlineto{\pgfqpoint{4.518352in}{-0.027778in}}%
\pgfpathquadraticcurveto{\pgfqpoint{4.557241in}{-0.027778in}}{\pgfqpoint{4.557241in}{0.011111in}}%
\pgfpathlineto{\pgfqpoint{4.557241in}{0.266666in}}%
\pgfpathquadraticcurveto{\pgfqpoint{4.557241in}{0.305555in}}{\pgfqpoint{4.518352in}{0.305555in}}%
\pgfpathlineto{\pgfqpoint{1.544689in}{0.305555in}}%
\pgfpathquadraticcurveto{\pgfqpoint{1.505800in}{0.305555in}}{\pgfqpoint{1.505800in}{0.266666in}}%
\pgfpathlineto{\pgfqpoint{1.505800in}{0.011111in}}%
\pgfpathquadraticcurveto{\pgfqpoint{1.505800in}{-0.027778in}}{\pgfqpoint{1.544689in}{-0.027778in}}%
\pgfpathclose%
\pgfusepath{stroke,fill}%
\end{pgfscope}%
\begin{pgfscope}%
\pgfsetbuttcap%
\pgfsetmiterjoin%
\definecolor{currentfill}{rgb}{1.000000,1.000000,1.000000}%
\pgfsetfillcolor{currentfill}%
\pgfsetlinewidth{1.003750pt}%
\definecolor{currentstroke}{rgb}{0.800000,0.800000,0.800000}%
\pgfsetstrokecolor{currentstroke}%
\pgfsetdash{}{0pt}%
\pgfpathmoveto{\pgfqpoint{1.516911in}{0.000000in}}%
\pgfpathlineto{\pgfqpoint{4.490574in}{0.000000in}}%
\pgfpathquadraticcurveto{\pgfqpoint{4.529463in}{0.000000in}}{\pgfqpoint{4.529463in}{0.038889in}}%
\pgfpathlineto{\pgfqpoint{4.529463in}{0.294444in}}%
\pgfpathquadraticcurveto{\pgfqpoint{4.529463in}{0.333333in}}{\pgfqpoint{4.490574in}{0.333333in}}%
\pgfpathlineto{\pgfqpoint{1.516911in}{0.333333in}}%
\pgfpathquadraticcurveto{\pgfqpoint{1.478022in}{0.333333in}}{\pgfqpoint{1.478022in}{0.294444in}}%
\pgfpathlineto{\pgfqpoint{1.478022in}{0.038889in}}%
\pgfpathquadraticcurveto{\pgfqpoint{1.478022in}{0.000000in}}{\pgfqpoint{1.516911in}{0.000000in}}%
\pgfpathclose%
\pgfusepath{stroke,fill}%
\end{pgfscope}%
\begin{pgfscope}%
\pgfsetrectcap%
\pgfsetroundjoin%
\pgfsetlinewidth{1.505625pt}%
\definecolor{currentstroke}{rgb}{1.000000,0.000000,0.000000}%
\pgfsetstrokecolor{currentstroke}%
\pgfsetdash{}{0pt}%
\pgfpathmoveto{\pgfqpoint{1.555800in}{0.184722in}}%
\pgfpathlineto{\pgfqpoint{1.944689in}{0.184722in}}%
\pgfusepath{stroke}%
\end{pgfscope}%
\begin{pgfscope}%
\definecolor{textcolor}{rgb}{0.000000,0.000000,0.000000}%
\pgfsetstrokecolor{textcolor}%
\pgfsetfillcolor{textcolor}%
\pgftext[x=2.100245in,y=0.116667in,left,base]{\color{textcolor}\rmfamily\fontsize{14.000000}{16.800000}\selectfont Simulations}%
\end{pgfscope}%
\begin{pgfscope}%
\pgfsetbuttcap%
\pgfsetroundjoin%
\pgfsetlinewidth{1.003750pt}%
\definecolor{currentstroke}{rgb}{0.000000,0.000000,1.000000}%
\pgfsetstrokecolor{currentstroke}%
\pgfsetdash{}{0pt}%
\pgfpathmoveto{\pgfqpoint{3.680140in}{0.113917in}}%
\pgfpathcurveto{\pgfqpoint{3.694405in}{0.113917in}}{\pgfqpoint{3.708088in}{0.119585in}}{\pgfqpoint{3.718176in}{0.129672in}}%
\pgfpathcurveto{\pgfqpoint{3.728263in}{0.139759in}}{\pgfqpoint{3.733931in}{0.153443in}}{\pgfqpoint{3.733931in}{0.167708in}}%
\pgfpathcurveto{\pgfqpoint{3.733931in}{0.181974in}}{\pgfqpoint{3.728263in}{0.195657in}}{\pgfqpoint{3.718176in}{0.205744in}}%
\pgfpathcurveto{\pgfqpoint{3.708088in}{0.215832in}}{\pgfqpoint{3.694405in}{0.221500in}}{\pgfqpoint{3.680140in}{0.221500in}}%
\pgfpathcurveto{\pgfqpoint{3.665874in}{0.221500in}}{\pgfqpoint{3.652191in}{0.215832in}}{\pgfqpoint{3.642103in}{0.205744in}}%
\pgfpathcurveto{\pgfqpoint{3.632016in}{0.195657in}}{\pgfqpoint{3.626348in}{0.181974in}}{\pgfqpoint{3.626348in}{0.167708in}}%
\pgfpathcurveto{\pgfqpoint{3.626348in}{0.153443in}}{\pgfqpoint{3.632016in}{0.139759in}}{\pgfqpoint{3.642103in}{0.129672in}}%
\pgfpathcurveto{\pgfqpoint{3.652191in}{0.119585in}}{\pgfqpoint{3.665874in}{0.113917in}}{\pgfqpoint{3.680140in}{0.113917in}}%
\pgfpathclose%
\pgfusepath{stroke}%
\end{pgfscope}%
\begin{pgfscope}%
\definecolor{textcolor}{rgb}{0.000000,0.000000,0.000000}%
\pgfsetstrokecolor{textcolor}%
\pgfsetfillcolor{textcolor}%
\pgftext[x=4.030140in,y=0.116667in,left,base]{\color{textcolor}\rmfamily\fontsize{14.000000}{16.800000}\selectfont Data}%
\end{pgfscope}%
\end{pgfpicture}%
\makeatother%
\endgroup%
 } 
		\end{subfigure}
	\end{center}	
	\begin{minipage}{0.99\textwidth} % choose width suitably
		{\footnotesize \textsc{Notes.} This figure depicts simulated and empirical low wages over the life cycle. Data on wages are constructed dividing the annual labor income by the total number of hours. \par}
	\end{minipage}
\end{figure}
\FloatBarrier
%%%%%%%%%%%%%%%%%%%%
%MECHANISM
%%%%%%%%%%%%%%%%%%%%%%%%%%%%%%%%%%
\subsection{Mechanisms}
\begin{figure}[h!]
\centering
\caption{\\Cumulative distribution of love shock $\psi$ at meeting}
\hspace*{-1.3cm} 
\label{fig:psidist}
\resizebox{0.8\textwidth}{!}{%% Creator: Matplotlib, PGF backend
%%
%% To include the figure in your LaTeX document, write
%%   \input{<filename>.pgf}
%%
%% Make sure the required packages are loaded in your preamble
%%   \usepackage{pgf}
%%
%% and, on pdftex
%%   \usepackage[utf8]{inputenc}\DeclareUnicodeCharacter{2212}{-}
%%
%% or, on luatex and xetex
%%   \usepackage{unicode-math}
%%
%% Figures using additional raster images can only be included by \input if
%% they are in the same directory as the main LaTeX file. For loading figures
%% from other directories you can use the `import` package
%%   \usepackage{import}
%%
%% and then include the figures with
%%   \import{<path to file>}{<filename>.pgf}
%%
%% Matplotlib used the following preamble
%%
\begingroup%
\makeatletter%
\begin{pgfpicture}%
\pgfpathrectangle{\pgfpointorigin}{\pgfqpoint{5.132292in}{2.021147in}}%
\pgfusepath{use as bounding box, clip}%
\begin{pgfscope}%
\pgfsetbuttcap%
\pgfsetmiterjoin%
\definecolor{currentfill}{rgb}{1.000000,1.000000,1.000000}%
\pgfsetfillcolor{currentfill}%
\pgfsetlinewidth{0.000000pt}%
\definecolor{currentstroke}{rgb}{1.000000,1.000000,1.000000}%
\pgfsetstrokecolor{currentstroke}%
\pgfsetdash{}{0pt}%
\pgfpathmoveto{\pgfqpoint{0.000000in}{0.000000in}}%
\pgfpathlineto{\pgfqpoint{5.132292in}{0.000000in}}%
\pgfpathlineto{\pgfqpoint{5.132292in}{2.021147in}}%
\pgfpathlineto{\pgfqpoint{0.000000in}{2.021147in}}%
\pgfpathclose%
\pgfusepath{fill}%
\end{pgfscope}%
\begin{pgfscope}%
\pgfsetbuttcap%
\pgfsetmiterjoin%
\definecolor{currentfill}{rgb}{1.000000,1.000000,1.000000}%
\pgfsetfillcolor{currentfill}%
\pgfsetlinewidth{0.000000pt}%
\definecolor{currentstroke}{rgb}{0.000000,0.000000,0.000000}%
\pgfsetstrokecolor{currentstroke}%
\pgfsetstrokeopacity{0.000000}%
\pgfsetdash{}{0pt}%
\pgfpathmoveto{\pgfqpoint{0.482292in}{0.648420in}}%
\pgfpathlineto{\pgfqpoint{5.132292in}{0.648420in}}%
\pgfpathlineto{\pgfqpoint{5.132292in}{2.021147in}}%
\pgfpathlineto{\pgfqpoint{0.482292in}{2.021147in}}%
\pgfpathclose%
\pgfusepath{fill}%
\end{pgfscope}%
\begin{pgfscope}%
\pgfsetbuttcap%
\pgfsetroundjoin%
\definecolor{currentfill}{rgb}{0.000000,0.000000,0.000000}%
\pgfsetfillcolor{currentfill}%
\pgfsetlinewidth{0.803000pt}%
\definecolor{currentstroke}{rgb}{0.000000,0.000000,0.000000}%
\pgfsetstrokecolor{currentstroke}%
\pgfsetdash{}{0pt}%
\pgfsys@defobject{currentmarker}{\pgfqpoint{0.000000in}{-0.048611in}}{\pgfqpoint{0.000000in}{0.000000in}}{%
\pgfpathmoveto{\pgfqpoint{0.000000in}{0.000000in}}%
\pgfpathlineto{\pgfqpoint{0.000000in}{-0.048611in}}%
\pgfusepath{stroke,fill}%
}%
\begin{pgfscope}%
\pgfsys@transformshift{1.161376in}{0.648420in}%
\pgfsys@useobject{currentmarker}{}%
\end{pgfscope}%
\end{pgfscope}%
\begin{pgfscope}%
\definecolor{textcolor}{rgb}{0.000000,0.000000,0.000000}%
\pgfsetstrokecolor{textcolor}%
\pgfsetfillcolor{textcolor}%
\pgftext[x=1.161376in,y=0.551198in,,top]{\color{textcolor}\rmfamily\fontsize{11.000000}{13.200000}\selectfont \(\displaystyle {-4}\)}%
\end{pgfscope}%
\begin{pgfscope}%
\pgfsetbuttcap%
\pgfsetroundjoin%
\definecolor{currentfill}{rgb}{0.000000,0.000000,0.000000}%
\pgfsetfillcolor{currentfill}%
\pgfsetlinewidth{0.803000pt}%
\definecolor{currentstroke}{rgb}{0.000000,0.000000,0.000000}%
\pgfsetstrokecolor{currentstroke}%
\pgfsetdash{}{0pt}%
\pgfsys@defobject{currentmarker}{\pgfqpoint{0.000000in}{-0.048611in}}{\pgfqpoint{0.000000in}{0.000000in}}{%
\pgfpathmoveto{\pgfqpoint{0.000000in}{0.000000in}}%
\pgfpathlineto{\pgfqpoint{0.000000in}{-0.048611in}}%
\pgfusepath{stroke,fill}%
}%
\begin{pgfscope}%
\pgfsys@transformshift{1.906501in}{0.648420in}%
\pgfsys@useobject{currentmarker}{}%
\end{pgfscope}%
\end{pgfscope}%
\begin{pgfscope}%
\definecolor{textcolor}{rgb}{0.000000,0.000000,0.000000}%
\pgfsetstrokecolor{textcolor}%
\pgfsetfillcolor{textcolor}%
\pgftext[x=1.906501in,y=0.551198in,,top]{\color{textcolor}\rmfamily\fontsize{11.000000}{13.200000}\selectfont \(\displaystyle {-2}\)}%
\end{pgfscope}%
\begin{pgfscope}%
\pgfsetbuttcap%
\pgfsetroundjoin%
\definecolor{currentfill}{rgb}{0.000000,0.000000,0.000000}%
\pgfsetfillcolor{currentfill}%
\pgfsetlinewidth{0.803000pt}%
\definecolor{currentstroke}{rgb}{0.000000,0.000000,0.000000}%
\pgfsetstrokecolor{currentstroke}%
\pgfsetdash{}{0pt}%
\pgfsys@defobject{currentmarker}{\pgfqpoint{0.000000in}{-0.048611in}}{\pgfqpoint{0.000000in}{0.000000in}}{%
\pgfpathmoveto{\pgfqpoint{0.000000in}{0.000000in}}%
\pgfpathlineto{\pgfqpoint{0.000000in}{-0.048611in}}%
\pgfusepath{stroke,fill}%
}%
\begin{pgfscope}%
\pgfsys@transformshift{2.651626in}{0.648420in}%
\pgfsys@useobject{currentmarker}{}%
\end{pgfscope}%
\end{pgfscope}%
\begin{pgfscope}%
\definecolor{textcolor}{rgb}{0.000000,0.000000,0.000000}%
\pgfsetstrokecolor{textcolor}%
\pgfsetfillcolor{textcolor}%
\pgftext[x=2.651626in,y=0.551198in,,top]{\color{textcolor}\rmfamily\fontsize{11.000000}{13.200000}\selectfont \(\displaystyle {0}\)}%
\end{pgfscope}%
\begin{pgfscope}%
\pgfsetbuttcap%
\pgfsetroundjoin%
\definecolor{currentfill}{rgb}{0.000000,0.000000,0.000000}%
\pgfsetfillcolor{currentfill}%
\pgfsetlinewidth{0.803000pt}%
\definecolor{currentstroke}{rgb}{0.000000,0.000000,0.000000}%
\pgfsetstrokecolor{currentstroke}%
\pgfsetdash{}{0pt}%
\pgfsys@defobject{currentmarker}{\pgfqpoint{0.000000in}{-0.048611in}}{\pgfqpoint{0.000000in}{0.000000in}}{%
\pgfpathmoveto{\pgfqpoint{0.000000in}{0.000000in}}%
\pgfpathlineto{\pgfqpoint{0.000000in}{-0.048611in}}%
\pgfusepath{stroke,fill}%
}%
\begin{pgfscope}%
\pgfsys@transformshift{3.396752in}{0.648420in}%
\pgfsys@useobject{currentmarker}{}%
\end{pgfscope}%
\end{pgfscope}%
\begin{pgfscope}%
\definecolor{textcolor}{rgb}{0.000000,0.000000,0.000000}%
\pgfsetstrokecolor{textcolor}%
\pgfsetfillcolor{textcolor}%
\pgftext[x=3.396752in,y=0.551198in,,top]{\color{textcolor}\rmfamily\fontsize{11.000000}{13.200000}\selectfont \(\displaystyle {2}\)}%
\end{pgfscope}%
\begin{pgfscope}%
\pgfsetbuttcap%
\pgfsetroundjoin%
\definecolor{currentfill}{rgb}{0.000000,0.000000,0.000000}%
\pgfsetfillcolor{currentfill}%
\pgfsetlinewidth{0.803000pt}%
\definecolor{currentstroke}{rgb}{0.000000,0.000000,0.000000}%
\pgfsetstrokecolor{currentstroke}%
\pgfsetdash{}{0pt}%
\pgfsys@defobject{currentmarker}{\pgfqpoint{0.000000in}{-0.048611in}}{\pgfqpoint{0.000000in}{0.000000in}}{%
\pgfpathmoveto{\pgfqpoint{0.000000in}{0.000000in}}%
\pgfpathlineto{\pgfqpoint{0.000000in}{-0.048611in}}%
\pgfusepath{stroke,fill}%
}%
\begin{pgfscope}%
\pgfsys@transformshift{4.141877in}{0.648420in}%
\pgfsys@useobject{currentmarker}{}%
\end{pgfscope}%
\end{pgfscope}%
\begin{pgfscope}%
\definecolor{textcolor}{rgb}{0.000000,0.000000,0.000000}%
\pgfsetstrokecolor{textcolor}%
\pgfsetfillcolor{textcolor}%
\pgftext[x=4.141877in,y=0.551198in,,top]{\color{textcolor}\rmfamily\fontsize{11.000000}{13.200000}\selectfont \(\displaystyle {4}\)}%
\end{pgfscope}%
\begin{pgfscope}%
\pgfsetbuttcap%
\pgfsetroundjoin%
\definecolor{currentfill}{rgb}{0.000000,0.000000,0.000000}%
\pgfsetfillcolor{currentfill}%
\pgfsetlinewidth{0.803000pt}%
\definecolor{currentstroke}{rgb}{0.000000,0.000000,0.000000}%
\pgfsetstrokecolor{currentstroke}%
\pgfsetdash{}{0pt}%
\pgfsys@defobject{currentmarker}{\pgfqpoint{0.000000in}{-0.048611in}}{\pgfqpoint{0.000000in}{0.000000in}}{%
\pgfpathmoveto{\pgfqpoint{0.000000in}{0.000000in}}%
\pgfpathlineto{\pgfqpoint{0.000000in}{-0.048611in}}%
\pgfusepath{stroke,fill}%
}%
\begin{pgfscope}%
\pgfsys@transformshift{4.887002in}{0.648420in}%
\pgfsys@useobject{currentmarker}{}%
\end{pgfscope}%
\end{pgfscope}%
\begin{pgfscope}%
\definecolor{textcolor}{rgb}{0.000000,0.000000,0.000000}%
\pgfsetstrokecolor{textcolor}%
\pgfsetfillcolor{textcolor}%
\pgftext[x=4.887002in,y=0.551198in,,top]{\color{textcolor}\rmfamily\fontsize{11.000000}{13.200000}\selectfont \(\displaystyle {6}\)}%
\end{pgfscope}%
\begin{pgfscope}%
\definecolor{textcolor}{rgb}{0.000000,0.000000,0.000000}%
\pgfsetstrokecolor{textcolor}%
\pgfsetfillcolor{textcolor}%
\pgftext[x=2.807292in,y=0.360457in,,top]{\color{textcolor}\rmfamily\fontsize{11.000000}{13.200000}\selectfont Love Shock \(\displaystyle \psi\)}%
\end{pgfscope}%
\begin{pgfscope}%
\pgfsetbuttcap%
\pgfsetroundjoin%
\definecolor{currentfill}{rgb}{0.000000,0.000000,0.000000}%
\pgfsetfillcolor{currentfill}%
\pgfsetlinewidth{0.803000pt}%
\definecolor{currentstroke}{rgb}{0.000000,0.000000,0.000000}%
\pgfsetstrokecolor{currentstroke}%
\pgfsetdash{}{0pt}%
\pgfsys@defobject{currentmarker}{\pgfqpoint{-0.048611in}{0.000000in}}{\pgfqpoint{0.000000in}{0.000000in}}{%
\pgfpathmoveto{\pgfqpoint{0.000000in}{0.000000in}}%
\pgfpathlineto{\pgfqpoint{-0.048611in}{0.000000in}}%
\pgfusepath{stroke,fill}%
}%
\begin{pgfscope}%
\pgfsys@transformshift{0.482292in}{0.710786in}%
\pgfsys@useobject{currentmarker}{}%
\end{pgfscope}%
\end{pgfscope}%
\begin{pgfscope}%
\definecolor{textcolor}{rgb}{0.000000,0.000000,0.000000}%
\pgfsetstrokecolor{textcolor}%
\pgfsetfillcolor{textcolor}%
\pgftext[x=0.190741in, y=0.657979in, left, base]{\color{textcolor}\rmfamily\fontsize{11.000000}{13.200000}\selectfont \(\displaystyle {0.0}\)}%
\end{pgfscope}%
\begin{pgfscope}%
\pgfsetbuttcap%
\pgfsetroundjoin%
\definecolor{currentfill}{rgb}{0.000000,0.000000,0.000000}%
\pgfsetfillcolor{currentfill}%
\pgfsetlinewidth{0.803000pt}%
\definecolor{currentstroke}{rgb}{0.000000,0.000000,0.000000}%
\pgfsetstrokecolor{currentstroke}%
\pgfsetdash{}{0pt}%
\pgfsys@defobject{currentmarker}{\pgfqpoint{-0.048611in}{0.000000in}}{\pgfqpoint{0.000000in}{0.000000in}}{%
\pgfpathmoveto{\pgfqpoint{0.000000in}{0.000000in}}%
\pgfpathlineto{\pgfqpoint{-0.048611in}{0.000000in}}%
\pgfusepath{stroke,fill}%
}%
\begin{pgfscope}%
\pgfsys@transformshift{0.482292in}{1.334768in}%
\pgfsys@useobject{currentmarker}{}%
\end{pgfscope}%
\end{pgfscope}%
\begin{pgfscope}%
\definecolor{textcolor}{rgb}{0.000000,0.000000,0.000000}%
\pgfsetstrokecolor{textcolor}%
\pgfsetfillcolor{textcolor}%
\pgftext[x=0.190741in, y=1.281961in, left, base]{\color{textcolor}\rmfamily\fontsize{11.000000}{13.200000}\selectfont \(\displaystyle {0.5}\)}%
\end{pgfscope}%
\begin{pgfscope}%
\pgfsetbuttcap%
\pgfsetroundjoin%
\definecolor{currentfill}{rgb}{0.000000,0.000000,0.000000}%
\pgfsetfillcolor{currentfill}%
\pgfsetlinewidth{0.803000pt}%
\definecolor{currentstroke}{rgb}{0.000000,0.000000,0.000000}%
\pgfsetstrokecolor{currentstroke}%
\pgfsetdash{}{0pt}%
\pgfsys@defobject{currentmarker}{\pgfqpoint{-0.048611in}{0.000000in}}{\pgfqpoint{0.000000in}{0.000000in}}{%
\pgfpathmoveto{\pgfqpoint{0.000000in}{0.000000in}}%
\pgfpathlineto{\pgfqpoint{-0.048611in}{0.000000in}}%
\pgfusepath{stroke,fill}%
}%
\begin{pgfscope}%
\pgfsys@transformshift{0.482292in}{1.958750in}%
\pgfsys@useobject{currentmarker}{}%
\end{pgfscope}%
\end{pgfscope}%
\begin{pgfscope}%
\definecolor{textcolor}{rgb}{0.000000,0.000000,0.000000}%
\pgfsetstrokecolor{textcolor}%
\pgfsetfillcolor{textcolor}%
\pgftext[x=0.190741in, y=1.905944in, left, base]{\color{textcolor}\rmfamily\fontsize{11.000000}{13.200000}\selectfont \(\displaystyle {1.0}\)}%
\end{pgfscope}%
\begin{pgfscope}%
\definecolor{textcolor}{rgb}{0.000000,0.000000,0.000000}%
\pgfsetstrokecolor{textcolor}%
\pgfsetfillcolor{textcolor}%
\pgftext[x=0.135185in,y=1.334783in,,bottom,rotate=90.000000]{\color{textcolor}\rmfamily\fontsize{11.000000}{13.200000}\selectfont Probability}%
\end{pgfscope}%
\begin{pgfscope}%
\pgfpathrectangle{\pgfqpoint{0.482292in}{0.648420in}}{\pgfqpoint{4.650000in}{1.372727in}}%
\pgfusepath{clip}%
\pgfsetrectcap%
\pgfsetroundjoin%
\pgfsetlinewidth{1.505625pt}%
\definecolor{currentstroke}{rgb}{1.000000,0.000000,0.000000}%
\pgfsetstrokecolor{currentstroke}%
\pgfsetdash{}{0pt}%
\pgfpathmoveto{\pgfqpoint{0.693655in}{0.710864in}}%
\pgfpathlineto{\pgfqpoint{1.232132in}{0.712188in}}%
\pgfpathlineto{\pgfqpoint{1.264767in}{0.712616in}}%
\pgfpathlineto{\pgfqpoint{1.411625in}{0.713901in}}%
\pgfpathlineto{\pgfqpoint{1.432022in}{0.714251in}}%
\pgfpathlineto{\pgfqpoint{1.538085in}{0.715497in}}%
\pgfpathlineto{\pgfqpoint{1.558482in}{0.715809in}}%
\pgfpathlineto{\pgfqpoint{1.648228in}{0.717055in}}%
\pgfpathlineto{\pgfqpoint{1.664546in}{0.717444in}}%
\pgfpathlineto{\pgfqpoint{1.831800in}{0.718885in}}%
\pgfpathlineto{\pgfqpoint{1.888911in}{0.725426in}}%
\pgfpathlineto{\pgfqpoint{1.892991in}{0.727101in}}%
\pgfpathlineto{\pgfqpoint{1.897070in}{0.727101in}}%
\pgfpathlineto{\pgfqpoint{1.901149in}{0.728775in}}%
\pgfpathlineto{\pgfqpoint{1.909308in}{0.728775in}}%
\pgfpathlineto{\pgfqpoint{1.913387in}{0.730605in}}%
\pgfpathlineto{\pgfqpoint{1.917467in}{0.730605in}}%
\pgfpathlineto{\pgfqpoint{1.921546in}{0.733253in}}%
\pgfpathlineto{\pgfqpoint{1.929705in}{0.733565in}}%
\pgfpathlineto{\pgfqpoint{1.933784in}{0.735317in}}%
\pgfpathlineto{\pgfqpoint{1.937864in}{0.735317in}}%
\pgfpathlineto{\pgfqpoint{1.941943in}{0.737770in}}%
\pgfpathlineto{\pgfqpoint{1.950102in}{0.737770in}}%
\pgfpathlineto{\pgfqpoint{1.954181in}{0.740495in}}%
\pgfpathlineto{\pgfqpoint{1.958261in}{0.740495in}}%
\pgfpathlineto{\pgfqpoint{1.962340in}{0.743338in}}%
\pgfpathlineto{\pgfqpoint{1.970499in}{0.743533in}}%
\pgfpathlineto{\pgfqpoint{1.974578in}{0.746142in}}%
\pgfpathlineto{\pgfqpoint{1.982737in}{0.746142in}}%
\pgfpathlineto{\pgfqpoint{1.986816in}{0.748400in}}%
\pgfpathlineto{\pgfqpoint{1.990896in}{0.748400in}}%
\pgfpathlineto{\pgfqpoint{1.994975in}{0.750697in}}%
\pgfpathlineto{\pgfqpoint{2.003134in}{0.750697in}}%
\pgfpathlineto{\pgfqpoint{2.007213in}{0.752449in}}%
\pgfpathlineto{\pgfqpoint{2.015372in}{0.752722in}}%
\pgfpathlineto{\pgfqpoint{2.019451in}{0.755409in}}%
\pgfpathlineto{\pgfqpoint{2.027610in}{0.755409in}}%
\pgfpathlineto{\pgfqpoint{2.031689in}{0.757550in}}%
\pgfpathlineto{\pgfqpoint{2.039848in}{0.757550in}}%
\pgfpathlineto{\pgfqpoint{2.043927in}{0.759653in}}%
\pgfpathlineto{\pgfqpoint{2.052086in}{0.759653in}}%
\pgfpathlineto{\pgfqpoint{2.056165in}{0.761834in}}%
\pgfpathlineto{\pgfqpoint{2.064324in}{0.761872in}}%
\pgfpathlineto{\pgfqpoint{2.068404in}{0.764326in}}%
\pgfpathlineto{\pgfqpoint{2.076562in}{0.764326in}}%
\pgfpathlineto{\pgfqpoint{2.080642in}{0.766311in}}%
\pgfpathlineto{\pgfqpoint{2.092880in}{0.766311in}}%
\pgfpathlineto{\pgfqpoint{2.096959in}{0.768414in}}%
\pgfpathlineto{\pgfqpoint{2.105118in}{0.768414in}}%
\pgfpathlineto{\pgfqpoint{2.109197in}{0.771140in}}%
\pgfpathlineto{\pgfqpoint{2.117356in}{0.771140in}}%
\pgfpathlineto{\pgfqpoint{2.121435in}{0.773087in}}%
\pgfpathlineto{\pgfqpoint{2.133674in}{0.773087in}}%
\pgfpathlineto{\pgfqpoint{2.137753in}{0.775423in}}%
\pgfpathlineto{\pgfqpoint{2.145912in}{0.775423in}}%
\pgfpathlineto{\pgfqpoint{2.149991in}{0.777370in}}%
\pgfpathlineto{\pgfqpoint{2.178547in}{0.778888in}}%
\pgfpathlineto{\pgfqpoint{2.182626in}{0.780602in}}%
\pgfpathlineto{\pgfqpoint{2.194864in}{0.780602in}}%
\pgfpathlineto{\pgfqpoint{2.198943in}{0.782432in}}%
\pgfpathlineto{\pgfqpoint{2.227499in}{0.783249in}}%
\pgfpathlineto{\pgfqpoint{2.235658in}{0.784457in}}%
\pgfpathlineto{\pgfqpoint{2.264213in}{0.785742in}}%
\pgfpathlineto{\pgfqpoint{2.272372in}{0.786910in}}%
\pgfpathlineto{\pgfqpoint{2.305007in}{0.788272in}}%
\pgfpathlineto{\pgfqpoint{2.309086in}{0.790180in}}%
\pgfpathlineto{\pgfqpoint{2.329483in}{0.790180in}}%
\pgfpathlineto{\pgfqpoint{2.333563in}{0.792127in}}%
\pgfpathlineto{\pgfqpoint{2.353960in}{0.792127in}}%
\pgfpathlineto{\pgfqpoint{2.358039in}{0.794425in}}%
\pgfpathlineto{\pgfqpoint{2.643595in}{0.794425in}}%
\pgfpathlineto{\pgfqpoint{2.647674in}{1.549200in}}%
\pgfpathlineto{\pgfqpoint{2.937310in}{1.549200in}}%
\pgfpathlineto{\pgfqpoint{2.941389in}{1.555742in}}%
\pgfpathlineto{\pgfqpoint{2.961786in}{1.555742in}}%
\pgfpathlineto{\pgfqpoint{2.965865in}{1.562789in}}%
\pgfpathlineto{\pgfqpoint{2.986262in}{1.562789in}}%
\pgfpathlineto{\pgfqpoint{2.990341in}{1.570071in}}%
\pgfpathlineto{\pgfqpoint{3.006659in}{1.570071in}}%
\pgfpathlineto{\pgfqpoint{3.010738in}{1.576690in}}%
\pgfpathlineto{\pgfqpoint{3.027056in}{1.576690in}}%
\pgfpathlineto{\pgfqpoint{3.031135in}{1.585179in}}%
\pgfpathlineto{\pgfqpoint{3.043373in}{1.585179in}}%
\pgfpathlineto{\pgfqpoint{3.047453in}{1.594485in}}%
\pgfpathlineto{\pgfqpoint{3.063770in}{1.594485in}}%
\pgfpathlineto{\pgfqpoint{3.067850in}{1.603129in}}%
\pgfpathlineto{\pgfqpoint{3.080088in}{1.603129in}}%
\pgfpathlineto{\pgfqpoint{3.084167in}{1.611734in}}%
\pgfpathlineto{\pgfqpoint{3.096405in}{1.611734in}}%
\pgfpathlineto{\pgfqpoint{3.100484in}{1.623727in}}%
\pgfpathlineto{\pgfqpoint{3.112723in}{1.623727in}}%
\pgfpathlineto{\pgfqpoint{3.116802in}{1.633812in}}%
\pgfpathlineto{\pgfqpoint{3.129040in}{1.633812in}}%
\pgfpathlineto{\pgfqpoint{3.133119in}{1.646740in}}%
\pgfpathlineto{\pgfqpoint{3.145358in}{1.646740in}}%
\pgfpathlineto{\pgfqpoint{3.149437in}{1.659901in}}%
\pgfpathlineto{\pgfqpoint{3.157596in}{1.659901in}}%
\pgfpathlineto{\pgfqpoint{3.161675in}{1.671660in}}%
\pgfpathlineto{\pgfqpoint{3.173913in}{1.671660in}}%
\pgfpathlineto{\pgfqpoint{3.177993in}{1.681901in}}%
\pgfpathlineto{\pgfqpoint{3.186151in}{1.681901in}}%
\pgfpathlineto{\pgfqpoint{3.190231in}{1.694400in}}%
\pgfpathlineto{\pgfqpoint{3.198389in}{1.694400in}}%
\pgfpathlineto{\pgfqpoint{3.202469in}{1.708184in}}%
\pgfpathlineto{\pgfqpoint{3.214707in}{1.708184in}}%
\pgfpathlineto{\pgfqpoint{3.218786in}{1.723837in}}%
\pgfpathlineto{\pgfqpoint{3.226945in}{1.723837in}}%
\pgfpathlineto{\pgfqpoint{3.235104in}{1.741865in}}%
\pgfpathlineto{\pgfqpoint{3.239183in}{1.741865in}}%
\pgfpathlineto{\pgfqpoint{3.243262in}{1.755727in}}%
\pgfpathlineto{\pgfqpoint{3.251421in}{1.755727in}}%
\pgfpathlineto{\pgfqpoint{3.255501in}{1.769784in}}%
\pgfpathlineto{\pgfqpoint{3.263659in}{1.769784in}}%
\pgfpathlineto{\pgfqpoint{3.267739in}{1.782984in}}%
\pgfpathlineto{\pgfqpoint{3.275897in}{1.782984in}}%
\pgfpathlineto{\pgfqpoint{3.284056in}{1.801246in}}%
\pgfpathlineto{\pgfqpoint{3.288136in}{1.801246in}}%
\pgfpathlineto{\pgfqpoint{3.292215in}{1.812227in}}%
\pgfpathlineto{\pgfqpoint{3.296294in}{1.812227in}}%
\pgfpathlineto{\pgfqpoint{3.300374in}{1.823051in}}%
\pgfpathlineto{\pgfqpoint{3.308532in}{1.823051in}}%
\pgfpathlineto{\pgfqpoint{3.312612in}{1.832786in}}%
\pgfpathlineto{\pgfqpoint{3.320771in}{1.832786in}}%
\pgfpathlineto{\pgfqpoint{3.324850in}{1.842287in}}%
\pgfpathlineto{\pgfqpoint{3.328929in}{1.846881in}}%
\pgfpathlineto{\pgfqpoint{3.333009in}{1.846881in}}%
\pgfpathlineto{\pgfqpoint{3.337088in}{1.855565in}}%
\pgfpathlineto{\pgfqpoint{3.341167in}{1.855565in}}%
\pgfpathlineto{\pgfqpoint{3.345247in}{1.863664in}}%
\pgfpathlineto{\pgfqpoint{3.353406in}{1.863664in}}%
\pgfpathlineto{\pgfqpoint{3.357485in}{1.871646in}}%
\pgfpathlineto{\pgfqpoint{3.361564in}{1.871646in}}%
\pgfpathlineto{\pgfqpoint{3.365644in}{1.878888in}}%
\pgfpathlineto{\pgfqpoint{3.369723in}{1.882354in}}%
\pgfpathlineto{\pgfqpoint{3.373802in}{1.882354in}}%
\pgfpathlineto{\pgfqpoint{3.377882in}{1.888779in}}%
\pgfpathlineto{\pgfqpoint{3.381961in}{1.888779in}}%
\pgfpathlineto{\pgfqpoint{3.386040in}{1.894697in}}%
\pgfpathlineto{\pgfqpoint{3.394199in}{1.894697in}}%
\pgfpathlineto{\pgfqpoint{3.398279in}{1.898825in}}%
\pgfpathlineto{\pgfqpoint{3.402358in}{1.898825in}}%
\pgfpathlineto{\pgfqpoint{3.410517in}{1.905756in}}%
\pgfpathlineto{\pgfqpoint{3.414596in}{1.905756in}}%
\pgfpathlineto{\pgfqpoint{3.418675in}{1.908637in}}%
\pgfpathlineto{\pgfqpoint{3.422755in}{1.908637in}}%
\pgfpathlineto{\pgfqpoint{3.426834in}{1.910857in}}%
\pgfpathlineto{\pgfqpoint{3.443152in}{1.912336in}}%
\pgfpathlineto{\pgfqpoint{3.447231in}{1.915802in}}%
\pgfpathlineto{\pgfqpoint{3.479866in}{1.915802in}}%
\pgfpathlineto{\pgfqpoint{3.483945in}{1.918995in}}%
\pgfpathlineto{\pgfqpoint{3.516580in}{1.918995in}}%
\pgfpathlineto{\pgfqpoint{3.524739in}{1.921876in}}%
\pgfpathlineto{\pgfqpoint{3.549215in}{1.921876in}}%
\pgfpathlineto{\pgfqpoint{3.553295in}{1.924641in}}%
\pgfpathlineto{\pgfqpoint{3.581850in}{1.924641in}}%
\pgfpathlineto{\pgfqpoint{3.585930in}{1.926977in}}%
\pgfpathlineto{\pgfqpoint{3.610406in}{1.928301in}}%
\pgfpathlineto{\pgfqpoint{3.614485in}{1.931766in}}%
\pgfpathlineto{\pgfqpoint{3.643041in}{1.931766in}}%
\pgfpathlineto{\pgfqpoint{3.647120in}{1.934297in}}%
\pgfpathlineto{\pgfqpoint{3.671596in}{1.934959in}}%
\pgfpathlineto{\pgfqpoint{3.675676in}{1.937568in}}%
\pgfpathlineto{\pgfqpoint{3.700152in}{1.937568in}}%
\pgfpathlineto{\pgfqpoint{3.708311in}{1.939126in}}%
\pgfpathlineto{\pgfqpoint{3.728708in}{1.939904in}}%
\pgfpathlineto{\pgfqpoint{3.732787in}{1.942474in}}%
\pgfpathlineto{\pgfqpoint{3.753184in}{1.942474in}}%
\pgfpathlineto{\pgfqpoint{3.757263in}{1.944733in}}%
\pgfpathlineto{\pgfqpoint{3.785819in}{1.946134in}}%
\pgfpathlineto{\pgfqpoint{3.798057in}{1.946758in}}%
\pgfpathlineto{\pgfqpoint{3.830692in}{1.948198in}}%
\pgfpathlineto{\pgfqpoint{3.838851in}{1.949327in}}%
\pgfpathlineto{\pgfqpoint{3.879644in}{1.950690in}}%
\pgfpathlineto{\pgfqpoint{3.887803in}{1.951858in}}%
\pgfpathlineto{\pgfqpoint{3.928597in}{1.952832in}}%
\pgfpathlineto{\pgfqpoint{3.940835in}{1.953416in}}%
\pgfpathlineto{\pgfqpoint{3.993867in}{1.954740in}}%
\pgfpathlineto{\pgfqpoint{4.006105in}{1.955441in}}%
\pgfpathlineto{\pgfqpoint{4.099930in}{1.956959in}}%
\pgfpathlineto{\pgfqpoint{4.116248in}{1.957349in}}%
\pgfpathlineto{\pgfqpoint{4.768947in}{1.958750in}}%
\pgfpathlineto{\pgfqpoint{4.768947in}{1.958750in}}%
\pgfusepath{stroke}%
\end{pgfscope}%
\begin{pgfscope}%
\pgfpathrectangle{\pgfqpoint{0.482292in}{0.648420in}}{\pgfqpoint{4.650000in}{1.372727in}}%
\pgfusepath{clip}%
\pgfsetrectcap%
\pgfsetroundjoin%
\pgfsetlinewidth{1.505625pt}%
\definecolor{currentstroke}{rgb}{0.000000,0.000000,1.000000}%
\pgfsetstrokecolor{currentstroke}%
\pgfsetdash{}{0pt}%
\pgfpathmoveto{\pgfqpoint{1.116792in}{0.710817in}}%
\pgfpathlineto{\pgfqpoint{1.870765in}{0.712294in}}%
\pgfpathlineto{\pgfqpoint{1.946924in}{0.713956in}}%
\pgfpathlineto{\pgfqpoint{2.053546in}{0.715372in}}%
\pgfpathlineto{\pgfqpoint{2.087818in}{0.715864in}}%
\pgfpathlineto{\pgfqpoint{2.209672in}{0.717342in}}%
\pgfpathlineto{\pgfqpoint{2.251559in}{0.717527in}}%
\pgfpathlineto{\pgfqpoint{2.647585in}{0.717619in}}%
\pgfpathlineto{\pgfqpoint{2.651393in}{1.049548in}}%
\pgfpathlineto{\pgfqpoint{2.936989in}{1.049548in}}%
\pgfpathlineto{\pgfqpoint{2.940797in}{1.102704in}}%
\pgfpathlineto{\pgfqpoint{2.963645in}{1.102704in}}%
\pgfpathlineto{\pgfqpoint{2.967453in}{1.147889in}}%
\pgfpathlineto{\pgfqpoint{2.986492in}{1.147889in}}%
\pgfpathlineto{\pgfqpoint{2.990300in}{1.181747in}}%
\pgfpathlineto{\pgfqpoint{3.005532in}{1.181747in}}%
\pgfpathlineto{\pgfqpoint{3.009340in}{1.212988in}}%
\pgfpathlineto{\pgfqpoint{3.024572in}{1.212988in}}%
\pgfpathlineto{\pgfqpoint{3.028380in}{1.238320in}}%
\pgfpathlineto{\pgfqpoint{3.043612in}{1.238320in}}%
\pgfpathlineto{\pgfqpoint{3.047420in}{1.260050in}}%
\pgfpathlineto{\pgfqpoint{3.062651in}{1.260050in}}%
\pgfpathlineto{\pgfqpoint{3.066459in}{1.280273in}}%
\pgfpathlineto{\pgfqpoint{3.081691in}{1.280273in}}%
\pgfpathlineto{\pgfqpoint{3.085499in}{1.298648in}}%
\pgfpathlineto{\pgfqpoint{3.096923in}{1.298648in}}%
\pgfpathlineto{\pgfqpoint{3.100731in}{1.313145in}}%
\pgfpathlineto{\pgfqpoint{3.112155in}{1.313145in}}%
\pgfpathlineto{\pgfqpoint{3.115963in}{1.326719in}}%
\pgfpathlineto{\pgfqpoint{3.127386in}{1.326719in}}%
\pgfpathlineto{\pgfqpoint{3.131194in}{1.337615in}}%
\pgfpathlineto{\pgfqpoint{3.142618in}{1.337615in}}%
\pgfpathlineto{\pgfqpoint{3.146426in}{1.346972in}}%
\pgfpathlineto{\pgfqpoint{3.157850in}{1.346972in}}%
\pgfpathlineto{\pgfqpoint{3.161658in}{1.357191in}}%
\pgfpathlineto{\pgfqpoint{3.173082in}{1.357191in}}%
\pgfpathlineto{\pgfqpoint{3.176890in}{1.367810in}}%
\pgfpathlineto{\pgfqpoint{3.188313in}{1.367810in}}%
\pgfpathlineto{\pgfqpoint{3.192121in}{1.377106in}}%
\pgfpathlineto{\pgfqpoint{3.199737in}{1.377106in}}%
\pgfpathlineto{\pgfqpoint{3.203545in}{1.384647in}}%
\pgfpathlineto{\pgfqpoint{3.214969in}{1.384647in}}%
\pgfpathlineto{\pgfqpoint{3.218777in}{1.391603in}}%
\pgfpathlineto{\pgfqpoint{3.226393in}{1.391603in}}%
\pgfpathlineto{\pgfqpoint{3.230201in}{1.400160in}}%
\pgfpathlineto{\pgfqpoint{3.234009in}{1.435495in}}%
\pgfpathlineto{\pgfqpoint{3.237817in}{1.435495in}}%
\pgfpathlineto{\pgfqpoint{3.241625in}{1.442112in}}%
\pgfpathlineto{\pgfqpoint{3.253049in}{1.442112in}}%
\pgfpathlineto{\pgfqpoint{3.256856in}{1.447006in}}%
\pgfpathlineto{\pgfqpoint{3.264472in}{1.447006in}}%
\pgfpathlineto{\pgfqpoint{3.268280in}{1.452393in}}%
\pgfpathlineto{\pgfqpoint{3.275896in}{1.452393in}}%
\pgfpathlineto{\pgfqpoint{3.279704in}{1.458580in}}%
\pgfpathlineto{\pgfqpoint{3.283512in}{1.485512in}}%
\pgfpathlineto{\pgfqpoint{3.287320in}{1.485512in}}%
\pgfpathlineto{\pgfqpoint{3.291128in}{1.492314in}}%
\pgfpathlineto{\pgfqpoint{3.298744in}{1.492314in}}%
\pgfpathlineto{\pgfqpoint{3.302552in}{1.498963in}}%
\pgfpathlineto{\pgfqpoint{3.310168in}{1.498963in}}%
\pgfpathlineto{\pgfqpoint{3.313976in}{1.505549in}}%
\pgfpathlineto{\pgfqpoint{3.321592in}{1.505549in}}%
\pgfpathlineto{\pgfqpoint{3.325399in}{1.511613in}}%
\pgfpathlineto{\pgfqpoint{3.329207in}{1.534667in}}%
\pgfpathlineto{\pgfqpoint{3.333015in}{1.534667in}}%
\pgfpathlineto{\pgfqpoint{3.336823in}{1.541593in}}%
\pgfpathlineto{\pgfqpoint{3.340631in}{1.541593in}}%
\pgfpathlineto{\pgfqpoint{3.344439in}{1.548395in}}%
\pgfpathlineto{\pgfqpoint{3.352055in}{1.548395in}}%
\pgfpathlineto{\pgfqpoint{3.355863in}{1.555505in}}%
\pgfpathlineto{\pgfqpoint{3.363479in}{1.555505in}}%
\pgfpathlineto{\pgfqpoint{3.367287in}{1.562554in}}%
\pgfpathlineto{\pgfqpoint{3.371095in}{1.581976in}}%
\pgfpathlineto{\pgfqpoint{3.374903in}{1.581976in}}%
\pgfpathlineto{\pgfqpoint{3.378711in}{1.589240in}}%
\pgfpathlineto{\pgfqpoint{3.382519in}{1.589240in}}%
\pgfpathlineto{\pgfqpoint{3.386327in}{1.597119in}}%
\pgfpathlineto{\pgfqpoint{3.393942in}{1.597119in}}%
\pgfpathlineto{\pgfqpoint{3.397750in}{1.605153in}}%
\pgfpathlineto{\pgfqpoint{3.405366in}{1.605153in}}%
\pgfpathlineto{\pgfqpoint{3.409174in}{1.630269in}}%
\pgfpathlineto{\pgfqpoint{3.412982in}{1.630269in}}%
\pgfpathlineto{\pgfqpoint{3.416790in}{1.638703in}}%
\pgfpathlineto{\pgfqpoint{3.424406in}{1.638703in}}%
\pgfpathlineto{\pgfqpoint{3.428214in}{1.647475in}}%
\pgfpathlineto{\pgfqpoint{3.432022in}{1.647475in}}%
\pgfpathlineto{\pgfqpoint{3.435830in}{1.655478in}}%
\pgfpathlineto{\pgfqpoint{3.443446in}{1.655478in}}%
\pgfpathlineto{\pgfqpoint{3.447254in}{1.676562in}}%
\pgfpathlineto{\pgfqpoint{3.451062in}{1.676562in}}%
\pgfpathlineto{\pgfqpoint{3.454870in}{1.684072in}}%
\pgfpathlineto{\pgfqpoint{3.458678in}{1.684072in}}%
\pgfpathlineto{\pgfqpoint{3.462485in}{1.690659in}}%
\pgfpathlineto{\pgfqpoint{3.481525in}{1.690659in}}%
\pgfpathlineto{\pgfqpoint{3.485333in}{1.702417in}}%
\pgfpathlineto{\pgfqpoint{3.515797in}{1.702417in}}%
\pgfpathlineto{\pgfqpoint{3.519605in}{1.713128in}}%
\pgfpathlineto{\pgfqpoint{3.523413in}{1.713128in}}%
\pgfpathlineto{\pgfqpoint{3.527221in}{1.736244in}}%
\pgfpathlineto{\pgfqpoint{3.550068in}{1.736244in}}%
\pgfpathlineto{\pgfqpoint{3.553876in}{1.744339in}}%
\pgfpathlineto{\pgfqpoint{3.580532in}{1.744339in}}%
\pgfpathlineto{\pgfqpoint{3.584340in}{1.753296in}}%
\pgfpathlineto{\pgfqpoint{3.595764in}{1.753296in}}%
\pgfpathlineto{\pgfqpoint{3.599571in}{1.768624in}}%
\pgfpathlineto{\pgfqpoint{3.610995in}{1.768624in}}%
\pgfpathlineto{\pgfqpoint{3.614803in}{1.775118in}}%
\pgfpathlineto{\pgfqpoint{3.641459in}{1.775118in}}%
\pgfpathlineto{\pgfqpoint{3.645267in}{1.780967in}}%
\pgfpathlineto{\pgfqpoint{3.664307in}{1.780967in}}%
\pgfpathlineto{\pgfqpoint{3.668114in}{1.794294in}}%
\pgfpathlineto{\pgfqpoint{3.671922in}{1.794294in}}%
\pgfpathlineto{\pgfqpoint{3.675730in}{1.800512in}}%
\pgfpathlineto{\pgfqpoint{3.698578in}{1.800512in}}%
\pgfpathlineto{\pgfqpoint{3.702386in}{1.806698in}}%
\pgfpathlineto{\pgfqpoint{3.725234in}{1.806698in}}%
\pgfpathlineto{\pgfqpoint{3.729042in}{1.815255in}}%
\pgfpathlineto{\pgfqpoint{3.732850in}{1.819934in}}%
\pgfpathlineto{\pgfqpoint{3.755697in}{1.819934in}}%
\pgfpathlineto{\pgfqpoint{3.759505in}{1.824274in}}%
\pgfpathlineto{\pgfqpoint{3.782353in}{1.824274in}}%
\pgfpathlineto{\pgfqpoint{3.789969in}{1.835878in}}%
\pgfpathlineto{\pgfqpoint{3.805200in}{1.835878in}}%
\pgfpathlineto{\pgfqpoint{3.809008in}{1.839971in}}%
\pgfpathlineto{\pgfqpoint{3.812816in}{1.839971in}}%
\pgfpathlineto{\pgfqpoint{3.816624in}{1.850313in}}%
\pgfpathlineto{\pgfqpoint{3.831856in}{1.850313in}}%
\pgfpathlineto{\pgfqpoint{3.835664in}{1.853699in}}%
\pgfpathlineto{\pgfqpoint{3.843280in}{1.853699in}}%
\pgfpathlineto{\pgfqpoint{3.847088in}{1.859424in}}%
\pgfpathlineto{\pgfqpoint{3.858512in}{1.859424in}}%
\pgfpathlineto{\pgfqpoint{3.862320in}{1.862718in}}%
\pgfpathlineto{\pgfqpoint{3.881359in}{1.862718in}}%
\pgfpathlineto{\pgfqpoint{3.885167in}{1.865765in}}%
\pgfpathlineto{\pgfqpoint{3.896591in}{1.865765in}}%
\pgfpathlineto{\pgfqpoint{3.900399in}{1.870843in}}%
\pgfpathlineto{\pgfqpoint{3.904207in}{1.870843in}}%
\pgfpathlineto{\pgfqpoint{3.908015in}{1.873675in}}%
\pgfpathlineto{\pgfqpoint{3.911823in}{1.873675in}}%
\pgfpathlineto{\pgfqpoint{3.915631in}{1.881278in}}%
\pgfpathlineto{\pgfqpoint{3.927055in}{1.881278in}}%
\pgfpathlineto{\pgfqpoint{3.930863in}{1.884140in}}%
\pgfpathlineto{\pgfqpoint{3.949902in}{1.884140in}}%
\pgfpathlineto{\pgfqpoint{3.953710in}{1.890204in}}%
\pgfpathlineto{\pgfqpoint{3.972750in}{1.890204in}}%
\pgfpathlineto{\pgfqpoint{3.976558in}{1.892759in}}%
\pgfpathlineto{\pgfqpoint{3.995598in}{1.892759in}}%
\pgfpathlineto{\pgfqpoint{4.003214in}{1.898268in}}%
\pgfpathlineto{\pgfqpoint{4.007022in}{1.902854in}}%
\pgfpathlineto{\pgfqpoint{4.018445in}{1.902854in}}%
\pgfpathlineto{\pgfqpoint{4.022253in}{1.904763in}}%
\pgfpathlineto{\pgfqpoint{4.037485in}{1.904763in}}%
\pgfpathlineto{\pgfqpoint{4.041293in}{1.907133in}}%
\pgfpathlineto{\pgfqpoint{4.048909in}{1.907133in}}%
\pgfpathlineto{\pgfqpoint{4.052717in}{1.909380in}}%
\pgfpathlineto{\pgfqpoint{4.060333in}{1.909380in}}%
\pgfpathlineto{\pgfqpoint{4.064141in}{1.911227in}}%
\pgfpathlineto{\pgfqpoint{4.079372in}{1.911227in}}%
\pgfpathlineto{\pgfqpoint{4.083180in}{1.913258in}}%
\pgfpathlineto{\pgfqpoint{4.086988in}{1.913258in}}%
\pgfpathlineto{\pgfqpoint{4.090796in}{1.915936in}}%
\pgfpathlineto{\pgfqpoint{4.094604in}{1.915936in}}%
\pgfpathlineto{\pgfqpoint{4.098412in}{1.917721in}}%
\pgfpathlineto{\pgfqpoint{4.102220in}{1.917721in}}%
\pgfpathlineto{\pgfqpoint{4.106028in}{1.919445in}}%
\pgfpathlineto{\pgfqpoint{4.109836in}{1.923969in}}%
\pgfpathlineto{\pgfqpoint{4.140300in}{1.925478in}}%
\pgfpathlineto{\pgfqpoint{4.144108in}{1.928371in}}%
\pgfpathlineto{\pgfqpoint{4.166955in}{1.929540in}}%
\pgfpathlineto{\pgfqpoint{4.170763in}{1.931726in}}%
\pgfpathlineto{\pgfqpoint{4.185995in}{1.933049in}}%
\pgfpathlineto{\pgfqpoint{4.193611in}{1.934404in}}%
\pgfpathlineto{\pgfqpoint{4.220266in}{1.935235in}}%
\pgfpathlineto{\pgfqpoint{4.250730in}{1.942160in}}%
\pgfpathlineto{\pgfqpoint{4.269770in}{1.943084in}}%
\pgfpathlineto{\pgfqpoint{4.281194in}{1.944653in}}%
\pgfpathlineto{\pgfqpoint{4.315465in}{1.945484in}}%
\pgfpathlineto{\pgfqpoint{4.323081in}{1.946469in}}%
\pgfpathlineto{\pgfqpoint{4.349737in}{1.947916in}}%
\pgfpathlineto{\pgfqpoint{4.361160in}{1.948439in}}%
\pgfpathlineto{\pgfqpoint{4.399240in}{1.949670in}}%
\pgfpathlineto{\pgfqpoint{4.406856in}{1.951179in}}%
\pgfpathlineto{\pgfqpoint{4.463975in}{1.952564in}}%
\pgfpathlineto{\pgfqpoint{4.475399in}{1.953087in}}%
\pgfpathlineto{\pgfqpoint{4.543942in}{1.954103in}}%
\pgfpathlineto{\pgfqpoint{4.559173in}{1.954995in}}%
\pgfpathlineto{\pgfqpoint{4.681028in}{1.956380in}}%
\pgfpathlineto{\pgfqpoint{4.711491in}{1.957735in}}%
\pgfpathlineto{\pgfqpoint{4.920928in}{1.958750in}}%
\pgfpathlineto{\pgfqpoint{4.920928in}{1.958750in}}%
\pgfusepath{stroke}%
\end{pgfscope}%
\begin{pgfscope}%
\pgfsetrectcap%
\pgfsetmiterjoin%
\pgfsetlinewidth{0.803000pt}%
\definecolor{currentstroke}{rgb}{0.000000,0.000000,0.000000}%
\pgfsetstrokecolor{currentstroke}%
\pgfsetdash{}{0pt}%
\pgfpathmoveto{\pgfqpoint{0.482292in}{0.648420in}}%
\pgfpathlineto{\pgfqpoint{0.482292in}{2.021147in}}%
\pgfusepath{stroke}%
\end{pgfscope}%
\begin{pgfscope}%
\pgfsetrectcap%
\pgfsetmiterjoin%
\pgfsetlinewidth{0.803000pt}%
\definecolor{currentstroke}{rgb}{0.000000,0.000000,0.000000}%
\pgfsetstrokecolor{currentstroke}%
\pgfsetdash{}{0pt}%
\pgfpathmoveto{\pgfqpoint{5.132292in}{0.648420in}}%
\pgfpathlineto{\pgfqpoint{5.132292in}{2.021147in}}%
\pgfusepath{stroke}%
\end{pgfscope}%
\begin{pgfscope}%
\pgfsetrectcap%
\pgfsetmiterjoin%
\pgfsetlinewidth{0.803000pt}%
\definecolor{currentstroke}{rgb}{0.000000,0.000000,0.000000}%
\pgfsetstrokecolor{currentstroke}%
\pgfsetdash{}{0pt}%
\pgfpathmoveto{\pgfqpoint{0.482292in}{0.648420in}}%
\pgfpathlineto{\pgfqpoint{5.132292in}{0.648420in}}%
\pgfusepath{stroke}%
\end{pgfscope}%
\begin{pgfscope}%
\pgfsetrectcap%
\pgfsetmiterjoin%
\pgfsetlinewidth{0.803000pt}%
\definecolor{currentstroke}{rgb}{0.000000,0.000000,0.000000}%
\pgfsetstrokecolor{currentstroke}%
\pgfsetdash{}{0pt}%
\pgfpathmoveto{\pgfqpoint{0.482292in}{2.021147in}}%
\pgfpathlineto{\pgfqpoint{5.132292in}{2.021147in}}%
\pgfusepath{stroke}%
\end{pgfscope}%
\begin{pgfscope}%
\pgfsetbuttcap%
\pgfsetmiterjoin%
\definecolor{currentfill}{rgb}{0.300000,0.300000,0.300000}%
\pgfsetfillcolor{currentfill}%
\pgfsetfillopacity{0.500000}%
\pgfsetlinewidth{1.003750pt}%
\definecolor{currentstroke}{rgb}{0.300000,0.300000,0.300000}%
\pgfsetstrokecolor{currentstroke}%
\pgfsetstrokeopacity{0.500000}%
\pgfsetdash{}{0pt}%
\pgfpathmoveto{\pgfqpoint{1.825553in}{-0.027778in}}%
\pgfpathlineto{\pgfqpoint{3.844586in}{-0.027778in}}%
\pgfpathquadraticcurveto{\pgfqpoint{3.865792in}{-0.027778in}}{\pgfqpoint{3.865792in}{-0.006572in}}%
\pgfpathlineto{\pgfqpoint{3.865792in}{0.134604in}}%
\pgfpathquadraticcurveto{\pgfqpoint{3.865792in}{0.155810in}}{\pgfqpoint{3.844586in}{0.155810in}}%
\pgfpathlineto{\pgfqpoint{1.825553in}{0.155810in}}%
\pgfpathquadraticcurveto{\pgfqpoint{1.804347in}{0.155810in}}{\pgfqpoint{1.804347in}{0.134604in}}%
\pgfpathlineto{\pgfqpoint{1.804347in}{-0.006572in}}%
\pgfpathquadraticcurveto{\pgfqpoint{1.804347in}{-0.027778in}}{\pgfqpoint{1.825553in}{-0.027778in}}%
\pgfpathclose%
\pgfusepath{stroke,fill}%
\end{pgfscope}%
\begin{pgfscope}%
\pgfsetbuttcap%
\pgfsetmiterjoin%
\definecolor{currentfill}{rgb}{1.000000,1.000000,1.000000}%
\pgfsetfillcolor{currentfill}%
\pgfsetlinewidth{1.003750pt}%
\definecolor{currentstroke}{rgb}{0.800000,0.800000,0.800000}%
\pgfsetstrokecolor{currentstroke}%
\pgfsetdash{}{0pt}%
\pgfpathmoveto{\pgfqpoint{1.797775in}{0.000000in}}%
\pgfpathlineto{\pgfqpoint{3.816809in}{0.000000in}}%
\pgfpathquadraticcurveto{\pgfqpoint{3.838014in}{0.000000in}}{\pgfqpoint{3.838014in}{0.021206in}}%
\pgfpathlineto{\pgfqpoint{3.838014in}{0.162382in}}%
\pgfpathquadraticcurveto{\pgfqpoint{3.838014in}{0.183588in}}{\pgfqpoint{3.816809in}{0.183588in}}%
\pgfpathlineto{\pgfqpoint{1.797775in}{0.183588in}}%
\pgfpathquadraticcurveto{\pgfqpoint{1.776569in}{0.183588in}}{\pgfqpoint{1.776569in}{0.162382in}}%
\pgfpathlineto{\pgfqpoint{1.776569in}{0.021206in}}%
\pgfpathquadraticcurveto{\pgfqpoint{1.776569in}{0.000000in}}{\pgfqpoint{1.797775in}{0.000000in}}%
\pgfpathclose%
\pgfusepath{stroke,fill}%
\end{pgfscope}%
\begin{pgfscope}%
\pgfsetrectcap%
\pgfsetroundjoin%
\pgfsetlinewidth{1.505625pt}%
\definecolor{currentstroke}{rgb}{1.000000,0.000000,0.000000}%
\pgfsetstrokecolor{currentstroke}%
\pgfsetdash{}{0pt}%
\pgfpathmoveto{\pgfqpoint{1.818980in}{0.101126in}}%
\pgfpathlineto{\pgfqpoint{2.031036in}{0.101126in}}%
\pgfusepath{stroke}%
\end{pgfscope}%
\begin{pgfscope}%
\definecolor{textcolor}{rgb}{0.000000,0.000000,0.000000}%
\pgfsetstrokecolor{textcolor}%
\pgfsetfillcolor{textcolor}%
\pgftext[x=2.115858in,y=0.064016in,left,base]{\color{textcolor}\rmfamily\fontsize{7.634000}{9.160800}\selectfont Cohabitaition}%
\end{pgfscope}%
\begin{pgfscope}%
\pgfsetrectcap%
\pgfsetroundjoin%
\pgfsetlinewidth{1.505625pt}%
\definecolor{currentstroke}{rgb}{0.000000,0.000000,1.000000}%
\pgfsetstrokecolor{currentstroke}%
\pgfsetdash{}{0pt}%
\pgfpathmoveto{\pgfqpoint{3.036257in}{0.101126in}}%
\pgfpathlineto{\pgfqpoint{3.248313in}{0.101126in}}%
\pgfusepath{stroke}%
\end{pgfscope}%
\begin{pgfscope}%
\definecolor{textcolor}{rgb}{0.000000,0.000000,0.000000}%
\pgfsetstrokecolor{textcolor}%
\pgfsetfillcolor{textcolor}%
\pgftext[x=3.333135in,y=0.064016in,left,base]{\color{textcolor}\rmfamily\fontsize{7.634000}{9.160800}\selectfont Marriage}%
\end{pgfscope}%
\end{pgfpicture}%
\makeatother%
\endgroup%
}

\end{figure}

\begin{figure}[h!]
\centering
\caption{\\Distribution of women Pareto weight $\theta$---marriage and cohabitation}
\label{fig:thetadist}
\hspace*{-1.1cm} 
\resizebox{0.8\textwidth}{!}{%% Creator: Matplotlib, PGF backend
%%
%% To include the figure in your LaTeX document, write
%%   \input{<filename>.pgf}
%%
%% Make sure the required packages are loaded in your preamble
%%   \usepackage{pgf}
%%
%% Figures using additional raster images can only be included by \input if
%% they are in the same directory as the main LaTeX file. For loading figures
%% from other directories you can use the `import` package
%%   \usepackage{import}
%% and then include the figures with
%%   \import{<path to file>}{<filename>.pgf}
%%
%% Matplotlib used the following preamble
%%
\begingroup%
\makeatletter%
\begin{pgfpicture}%
\pgfpathrectangle{\pgfpointorigin}{\pgfqpoint{5.014005in}{2.172927in}}%
\pgfusepath{use as bounding box, clip}%
\begin{pgfscope}%
\pgfsetbuttcap%
\pgfsetmiterjoin%
\definecolor{currentfill}{rgb}{1.000000,1.000000,1.000000}%
\pgfsetfillcolor{currentfill}%
\pgfsetlinewidth{0.000000pt}%
\definecolor{currentstroke}{rgb}{1.000000,1.000000,1.000000}%
\pgfsetstrokecolor{currentstroke}%
\pgfsetdash{}{0pt}%
\pgfpathmoveto{\pgfqpoint{0.000000in}{0.000000in}}%
\pgfpathlineto{\pgfqpoint{5.014005in}{0.000000in}}%
\pgfpathlineto{\pgfqpoint{5.014005in}{2.172927in}}%
\pgfpathlineto{\pgfqpoint{0.000000in}{2.172927in}}%
\pgfpathclose%
\pgfusepath{fill}%
\end{pgfscope}%
\begin{pgfscope}%
\pgfsetbuttcap%
\pgfsetmiterjoin%
\definecolor{currentfill}{rgb}{1.000000,1.000000,1.000000}%
\pgfsetfillcolor{currentfill}%
\pgfsetlinewidth{0.000000pt}%
\definecolor{currentstroke}{rgb}{0.000000,0.000000,0.000000}%
\pgfsetstrokecolor{currentstroke}%
\pgfsetstrokeopacity{0.000000}%
\pgfsetdash{}{0pt}%
\pgfpathmoveto{\pgfqpoint{0.364005in}{0.800199in}}%
\pgfpathlineto{\pgfqpoint{5.014005in}{0.800199in}}%
\pgfpathlineto{\pgfqpoint{5.014005in}{2.172927in}}%
\pgfpathlineto{\pgfqpoint{0.364005in}{2.172927in}}%
\pgfpathclose%
\pgfusepath{fill}%
\end{pgfscope}%
\begin{pgfscope}%
\pgfpathrectangle{\pgfqpoint{0.364005in}{0.800199in}}{\pgfqpoint{4.650000in}{1.372727in}}%
\pgfusepath{clip}%
\pgfsetbuttcap%
\pgfsetroundjoin%
\definecolor{currentfill}{rgb}{1.000000,0.000000,0.000000}%
\pgfsetfillcolor{currentfill}%
\pgfsetfillopacity{0.250000}%
\pgfsetlinewidth{0.000000pt}%
\definecolor{currentstroke}{rgb}{0.000000,0.000000,0.000000}%
\pgfsetstrokecolor{currentstroke}%
\pgfsetdash{}{0pt}%
\pgfpathmoveto{\pgfqpoint{0.575368in}{0.803363in}}%
\pgfpathlineto{\pgfqpoint{0.575368in}{0.800199in}}%
\pgfpathlineto{\pgfqpoint{0.608654in}{0.800199in}}%
\pgfpathlineto{\pgfqpoint{0.641939in}{0.800199in}}%
\pgfpathlineto{\pgfqpoint{0.675225in}{0.800199in}}%
\pgfpathlineto{\pgfqpoint{0.708511in}{0.800199in}}%
\pgfpathlineto{\pgfqpoint{0.741796in}{0.800199in}}%
\pgfpathlineto{\pgfqpoint{0.775082in}{0.800199in}}%
\pgfpathlineto{\pgfqpoint{0.808368in}{0.800199in}}%
\pgfpathlineto{\pgfqpoint{0.841653in}{0.800199in}}%
\pgfpathlineto{\pgfqpoint{0.874939in}{0.800199in}}%
\pgfpathlineto{\pgfqpoint{0.908224in}{0.800199in}}%
\pgfpathlineto{\pgfqpoint{0.941510in}{0.800199in}}%
\pgfpathlineto{\pgfqpoint{0.974796in}{0.800199in}}%
\pgfpathlineto{\pgfqpoint{1.008081in}{0.800199in}}%
\pgfpathlineto{\pgfqpoint{1.041367in}{0.800199in}}%
\pgfpathlineto{\pgfqpoint{1.074652in}{0.800199in}}%
\pgfpathlineto{\pgfqpoint{1.107938in}{0.800199in}}%
\pgfpathlineto{\pgfqpoint{1.141224in}{0.800199in}}%
\pgfpathlineto{\pgfqpoint{1.174509in}{0.800199in}}%
\pgfpathlineto{\pgfqpoint{1.207795in}{0.800199in}}%
\pgfpathlineto{\pgfqpoint{1.241080in}{0.800199in}}%
\pgfpathlineto{\pgfqpoint{1.274366in}{0.800199in}}%
\pgfpathlineto{\pgfqpoint{1.307652in}{0.800199in}}%
\pgfpathlineto{\pgfqpoint{1.340937in}{0.800199in}}%
\pgfpathlineto{\pgfqpoint{1.374223in}{0.800199in}}%
\pgfpathlineto{\pgfqpoint{1.407509in}{0.800199in}}%
\pgfpathlineto{\pgfqpoint{1.440794in}{0.800199in}}%
\pgfpathlineto{\pgfqpoint{1.474080in}{0.800199in}}%
\pgfpathlineto{\pgfqpoint{1.507365in}{0.800199in}}%
\pgfpathlineto{\pgfqpoint{1.540651in}{0.800199in}}%
\pgfpathlineto{\pgfqpoint{1.573937in}{0.800199in}}%
\pgfpathlineto{\pgfqpoint{1.607222in}{0.800199in}}%
\pgfpathlineto{\pgfqpoint{1.640508in}{0.800199in}}%
\pgfpathlineto{\pgfqpoint{1.673793in}{0.800199in}}%
\pgfpathlineto{\pgfqpoint{1.707079in}{0.800199in}}%
\pgfpathlineto{\pgfqpoint{1.740365in}{0.800199in}}%
\pgfpathlineto{\pgfqpoint{1.773650in}{0.800199in}}%
\pgfpathlineto{\pgfqpoint{1.806936in}{0.800199in}}%
\pgfpathlineto{\pgfqpoint{1.840221in}{0.800199in}}%
\pgfpathlineto{\pgfqpoint{1.873507in}{0.800199in}}%
\pgfpathlineto{\pgfqpoint{1.906793in}{0.800199in}}%
\pgfpathlineto{\pgfqpoint{1.940078in}{0.800199in}}%
\pgfpathlineto{\pgfqpoint{1.973364in}{0.800199in}}%
\pgfpathlineto{\pgfqpoint{2.006650in}{0.800199in}}%
\pgfpathlineto{\pgfqpoint{2.039935in}{0.800199in}}%
\pgfpathlineto{\pgfqpoint{2.073221in}{0.800199in}}%
\pgfpathlineto{\pgfqpoint{2.106506in}{0.800199in}}%
\pgfpathlineto{\pgfqpoint{2.139792in}{0.800199in}}%
\pgfpathlineto{\pgfqpoint{2.173078in}{0.800199in}}%
\pgfpathlineto{\pgfqpoint{2.206363in}{0.800199in}}%
\pgfpathlineto{\pgfqpoint{2.239649in}{0.800199in}}%
\pgfpathlineto{\pgfqpoint{2.272934in}{0.800199in}}%
\pgfpathlineto{\pgfqpoint{2.306220in}{0.800199in}}%
\pgfpathlineto{\pgfqpoint{2.339506in}{0.800199in}}%
\pgfpathlineto{\pgfqpoint{2.372791in}{0.800199in}}%
\pgfpathlineto{\pgfqpoint{2.406077in}{0.800199in}}%
\pgfpathlineto{\pgfqpoint{2.439362in}{0.800199in}}%
\pgfpathlineto{\pgfqpoint{2.472648in}{0.800199in}}%
\pgfpathlineto{\pgfqpoint{2.505934in}{0.800199in}}%
\pgfpathlineto{\pgfqpoint{2.539219in}{0.800199in}}%
\pgfpathlineto{\pgfqpoint{2.572505in}{0.800199in}}%
\pgfpathlineto{\pgfqpoint{2.605791in}{0.800199in}}%
\pgfpathlineto{\pgfqpoint{2.639076in}{0.800199in}}%
\pgfpathlineto{\pgfqpoint{2.672362in}{0.800199in}}%
\pgfpathlineto{\pgfqpoint{2.705647in}{0.800199in}}%
\pgfpathlineto{\pgfqpoint{2.738933in}{0.800199in}}%
\pgfpathlineto{\pgfqpoint{2.772219in}{0.800199in}}%
\pgfpathlineto{\pgfqpoint{2.805504in}{0.800199in}}%
\pgfpathlineto{\pgfqpoint{2.838790in}{0.800199in}}%
\pgfpathlineto{\pgfqpoint{2.872075in}{0.800199in}}%
\pgfpathlineto{\pgfqpoint{2.905361in}{0.800199in}}%
\pgfpathlineto{\pgfqpoint{2.938647in}{0.800199in}}%
\pgfpathlineto{\pgfqpoint{2.971932in}{0.800199in}}%
\pgfpathlineto{\pgfqpoint{3.005218in}{0.800199in}}%
\pgfpathlineto{\pgfqpoint{3.038504in}{0.800199in}}%
\pgfpathlineto{\pgfqpoint{3.071789in}{0.800199in}}%
\pgfpathlineto{\pgfqpoint{3.105075in}{0.800199in}}%
\pgfpathlineto{\pgfqpoint{3.138360in}{0.800199in}}%
\pgfpathlineto{\pgfqpoint{3.171646in}{0.800199in}}%
\pgfpathlineto{\pgfqpoint{3.204932in}{0.800199in}}%
\pgfpathlineto{\pgfqpoint{3.238217in}{0.800199in}}%
\pgfpathlineto{\pgfqpoint{3.271503in}{0.800199in}}%
\pgfpathlineto{\pgfqpoint{3.304788in}{0.800199in}}%
\pgfpathlineto{\pgfqpoint{3.338074in}{0.800199in}}%
\pgfpathlineto{\pgfqpoint{3.371360in}{0.800199in}}%
\pgfpathlineto{\pgfqpoint{3.404645in}{0.800199in}}%
\pgfpathlineto{\pgfqpoint{3.437931in}{0.800199in}}%
\pgfpathlineto{\pgfqpoint{3.471216in}{0.800199in}}%
\pgfpathlineto{\pgfqpoint{3.504502in}{0.800199in}}%
\pgfpathlineto{\pgfqpoint{3.537788in}{0.800199in}}%
\pgfpathlineto{\pgfqpoint{3.571073in}{0.800199in}}%
\pgfpathlineto{\pgfqpoint{3.604359in}{0.800199in}}%
\pgfpathlineto{\pgfqpoint{3.637645in}{0.800199in}}%
\pgfpathlineto{\pgfqpoint{3.670930in}{0.800199in}}%
\pgfpathlineto{\pgfqpoint{3.704216in}{0.800199in}}%
\pgfpathlineto{\pgfqpoint{3.737501in}{0.800199in}}%
\pgfpathlineto{\pgfqpoint{3.770787in}{0.800199in}}%
\pgfpathlineto{\pgfqpoint{3.804073in}{0.800199in}}%
\pgfpathlineto{\pgfqpoint{3.837358in}{0.800199in}}%
\pgfpathlineto{\pgfqpoint{3.870644in}{0.800199in}}%
\pgfpathlineto{\pgfqpoint{3.903929in}{0.800199in}}%
\pgfpathlineto{\pgfqpoint{3.937215in}{0.800199in}}%
\pgfpathlineto{\pgfqpoint{3.970501in}{0.800199in}}%
\pgfpathlineto{\pgfqpoint{4.003786in}{0.800199in}}%
\pgfpathlineto{\pgfqpoint{4.037072in}{0.800199in}}%
\pgfpathlineto{\pgfqpoint{4.070357in}{0.800199in}}%
\pgfpathlineto{\pgfqpoint{4.103643in}{0.800199in}}%
\pgfpathlineto{\pgfqpoint{4.136929in}{0.800199in}}%
\pgfpathlineto{\pgfqpoint{4.170214in}{0.800199in}}%
\pgfpathlineto{\pgfqpoint{4.203500in}{0.800199in}}%
\pgfpathlineto{\pgfqpoint{4.236786in}{0.800199in}}%
\pgfpathlineto{\pgfqpoint{4.270071in}{0.800199in}}%
\pgfpathlineto{\pgfqpoint{4.303357in}{0.800199in}}%
\pgfpathlineto{\pgfqpoint{4.336642in}{0.800199in}}%
\pgfpathlineto{\pgfqpoint{4.369928in}{0.800199in}}%
\pgfpathlineto{\pgfqpoint{4.403214in}{0.800199in}}%
\pgfpathlineto{\pgfqpoint{4.436499in}{0.800199in}}%
\pgfpathlineto{\pgfqpoint{4.469785in}{0.800199in}}%
\pgfpathlineto{\pgfqpoint{4.503070in}{0.800199in}}%
\pgfpathlineto{\pgfqpoint{4.536356in}{0.800199in}}%
\pgfpathlineto{\pgfqpoint{4.569642in}{0.800199in}}%
\pgfpathlineto{\pgfqpoint{4.602927in}{0.800199in}}%
\pgfpathlineto{\pgfqpoint{4.636213in}{0.800199in}}%
\pgfpathlineto{\pgfqpoint{4.669498in}{0.800199in}}%
\pgfpathlineto{\pgfqpoint{4.702784in}{0.800199in}}%
\pgfpathlineto{\pgfqpoint{4.736070in}{0.800199in}}%
\pgfpathlineto{\pgfqpoint{4.769355in}{0.800199in}}%
\pgfpathlineto{\pgfqpoint{4.802641in}{0.800199in}}%
\pgfpathlineto{\pgfqpoint{4.802641in}{0.802068in}}%
\pgfpathlineto{\pgfqpoint{4.802641in}{0.802068in}}%
\pgfpathlineto{\pgfqpoint{4.769355in}{0.801281in}}%
\pgfpathlineto{\pgfqpoint{4.736070in}{0.800822in}}%
\pgfpathlineto{\pgfqpoint{4.702784in}{0.800572in}}%
\pgfpathlineto{\pgfqpoint{4.669498in}{0.800451in}}%
\pgfpathlineto{\pgfqpoint{4.636213in}{0.800415in}}%
\pgfpathlineto{\pgfqpoint{4.602927in}{0.800437in}}%
\pgfpathlineto{\pgfqpoint{4.569642in}{0.800506in}}%
\pgfpathlineto{\pgfqpoint{4.536356in}{0.800620in}}%
\pgfpathlineto{\pgfqpoint{4.503070in}{0.800786in}}%
\pgfpathlineto{\pgfqpoint{4.469785in}{0.801012in}}%
\pgfpathlineto{\pgfqpoint{4.436499in}{0.801315in}}%
\pgfpathlineto{\pgfqpoint{4.403214in}{0.801713in}}%
\pgfpathlineto{\pgfqpoint{4.369928in}{0.802231in}}%
\pgfpathlineto{\pgfqpoint{4.336642in}{0.802897in}}%
\pgfpathlineto{\pgfqpoint{4.303357in}{0.803746in}}%
\pgfpathlineto{\pgfqpoint{4.270071in}{0.804812in}}%
\pgfpathlineto{\pgfqpoint{4.236786in}{0.806132in}}%
\pgfpathlineto{\pgfqpoint{4.203500in}{0.807739in}}%
\pgfpathlineto{\pgfqpoint{4.170214in}{0.809658in}}%
\pgfpathlineto{\pgfqpoint{4.136929in}{0.811904in}}%
\pgfpathlineto{\pgfqpoint{4.103643in}{0.814474in}}%
\pgfpathlineto{\pgfqpoint{4.070357in}{0.817350in}}%
\pgfpathlineto{\pgfqpoint{4.037072in}{0.820496in}}%
\pgfpathlineto{\pgfqpoint{4.003786in}{0.823861in}}%
\pgfpathlineto{\pgfqpoint{3.970501in}{0.827391in}}%
\pgfpathlineto{\pgfqpoint{3.937215in}{0.831032in}}%
\pgfpathlineto{\pgfqpoint{3.903929in}{0.834745in}}%
\pgfpathlineto{\pgfqpoint{3.870644in}{0.838513in}}%
\pgfpathlineto{\pgfqpoint{3.837358in}{0.842356in}}%
\pgfpathlineto{\pgfqpoint{3.804073in}{0.846335in}}%
\pgfpathlineto{\pgfqpoint{3.770787in}{0.850556in}}%
\pgfpathlineto{\pgfqpoint{3.737501in}{0.855171in}}%
\pgfpathlineto{\pgfqpoint{3.704216in}{0.860374in}}%
\pgfpathlineto{\pgfqpoint{3.670930in}{0.866397in}}%
\pgfpathlineto{\pgfqpoint{3.637645in}{0.873493in}}%
\pgfpathlineto{\pgfqpoint{3.604359in}{0.881921in}}%
\pgfpathlineto{\pgfqpoint{3.571073in}{0.891924in}}%
\pgfpathlineto{\pgfqpoint{3.537788in}{0.903697in}}%
\pgfpathlineto{\pgfqpoint{3.504502in}{0.917352in}}%
\pgfpathlineto{\pgfqpoint{3.471216in}{0.932880in}}%
\pgfpathlineto{\pgfqpoint{3.437931in}{0.950116in}}%
\pgfpathlineto{\pgfqpoint{3.404645in}{0.968718in}}%
\pgfpathlineto{\pgfqpoint{3.371360in}{0.988154in}}%
\pgfpathlineto{\pgfqpoint{3.338074in}{1.007724in}}%
\pgfpathlineto{\pgfqpoint{3.304788in}{1.026596in}}%
\pgfpathlineto{\pgfqpoint{3.271503in}{1.043874in}}%
\pgfpathlineto{\pgfqpoint{3.238217in}{1.058670in}}%
\pgfpathlineto{\pgfqpoint{3.204932in}{1.070201in}}%
\pgfpathlineto{\pgfqpoint{3.171646in}{1.077855in}}%
\pgfpathlineto{\pgfqpoint{3.138360in}{1.081263in}}%
\pgfpathlineto{\pgfqpoint{3.105075in}{1.080337in}}%
\pgfpathlineto{\pgfqpoint{3.071789in}{1.075283in}}%
\pgfpathlineto{\pgfqpoint{3.038504in}{1.066594in}}%
\pgfpathlineto{\pgfqpoint{3.005218in}{1.055026in}}%
\pgfpathlineto{\pgfqpoint{2.971932in}{1.041559in}}%
\pgfpathlineto{\pgfqpoint{2.938647in}{1.027354in}}%
\pgfpathlineto{\pgfqpoint{2.905361in}{1.013702in}}%
\pgfpathlineto{\pgfqpoint{2.872075in}{1.001978in}}%
\pgfpathlineto{\pgfqpoint{2.838790in}{0.993579in}}%
\pgfpathlineto{\pgfqpoint{2.805504in}{0.989865in}}%
\pgfpathlineto{\pgfqpoint{2.772219in}{0.992091in}}%
\pgfpathlineto{\pgfqpoint{2.738933in}{1.001326in}}%
\pgfpathlineto{\pgfqpoint{2.705647in}{1.018374in}}%
\pgfpathlineto{\pgfqpoint{2.672362in}{1.043688in}}%
\pgfpathlineto{\pgfqpoint{2.639076in}{1.077289in}}%
\pgfpathlineto{\pgfqpoint{2.605791in}{1.118703in}}%
\pgfpathlineto{\pgfqpoint{2.572505in}{1.166915in}}%
\pgfpathlineto{\pgfqpoint{2.539219in}{1.220367in}}%
\pgfpathlineto{\pgfqpoint{2.505934in}{1.276985in}}%
\pgfpathlineto{\pgfqpoint{2.472648in}{1.334271in}}%
\pgfpathlineto{\pgfqpoint{2.439362in}{1.389432in}}%
\pgfpathlineto{\pgfqpoint{2.406077in}{1.439561in}}%
\pgfpathlineto{\pgfqpoint{2.372791in}{1.481859in}}%
\pgfpathlineto{\pgfqpoint{2.339506in}{1.513861in}}%
\pgfpathlineto{\pgfqpoint{2.306220in}{1.533675in}}%
\pgfpathlineto{\pgfqpoint{2.272934in}{1.540174in}}%
\pgfpathlineto{\pgfqpoint{2.239649in}{1.533141in}}%
\pgfpathlineto{\pgfqpoint{2.206363in}{1.513329in}}%
\pgfpathlineto{\pgfqpoint{2.173078in}{1.482418in}}%
\pgfpathlineto{\pgfqpoint{2.139792in}{1.442891in}}%
\pgfpathlineto{\pgfqpoint{2.106506in}{1.397808in}}%
\pgfpathlineto{\pgfqpoint{2.073221in}{1.350531in}}%
\pgfpathlineto{\pgfqpoint{2.039935in}{1.304416in}}%
\pgfpathlineto{\pgfqpoint{2.006650in}{1.262521in}}%
\pgfpathlineto{\pgfqpoint{1.973364in}{1.227360in}}%
\pgfpathlineto{\pgfqpoint{1.940078in}{1.200717in}}%
\pgfpathlineto{\pgfqpoint{1.906793in}{1.183559in}}%
\pgfpathlineto{\pgfqpoint{1.873507in}{1.176029in}}%
\pgfpathlineto{\pgfqpoint{1.840221in}{1.177515in}}%
\pgfpathlineto{\pgfqpoint{1.806936in}{1.186789in}}%
\pgfpathlineto{\pgfqpoint{1.773650in}{1.202187in}}%
\pgfpathlineto{\pgfqpoint{1.740365in}{1.221810in}}%
\pgfpathlineto{\pgfqpoint{1.707079in}{1.243720in}}%
\pgfpathlineto{\pgfqpoint{1.673793in}{1.266123in}}%
\pgfpathlineto{\pgfqpoint{1.640508in}{1.287496in}}%
\pgfpathlineto{\pgfqpoint{1.607222in}{1.306677in}}%
\pgfpathlineto{\pgfqpoint{1.573937in}{1.322888in}}%
\pgfpathlineto{\pgfqpoint{1.540651in}{1.335709in}}%
\pgfpathlineto{\pgfqpoint{1.507365in}{1.345008in}}%
\pgfpathlineto{\pgfqpoint{1.474080in}{1.350840in}}%
\pgfpathlineto{\pgfqpoint{1.440794in}{1.353336in}}%
\pgfpathlineto{\pgfqpoint{1.407509in}{1.352603in}}%
\pgfpathlineto{\pgfqpoint{1.374223in}{1.348642in}}%
\pgfpathlineto{\pgfqpoint{1.340937in}{1.341310in}}%
\pgfpathlineto{\pgfqpoint{1.307652in}{1.330319in}}%
\pgfpathlineto{\pgfqpoint{1.274366in}{1.315290in}}%
\pgfpathlineto{\pgfqpoint{1.241080in}{1.295838in}}%
\pgfpathlineto{\pgfqpoint{1.207795in}{1.271692in}}%
\pgfpathlineto{\pgfqpoint{1.174509in}{1.242798in}}%
\pgfpathlineto{\pgfqpoint{1.141224in}{1.209425in}}%
\pgfpathlineto{\pgfqpoint{1.107938in}{1.172198in}}%
\pgfpathlineto{\pgfqpoint{1.074652in}{1.132100in}}%
\pgfpathlineto{\pgfqpoint{1.041367in}{1.090393in}}%
\pgfpathlineto{\pgfqpoint{1.008081in}{1.048501in}}%
\pgfpathlineto{\pgfqpoint{0.974796in}{1.007859in}}%
\pgfpathlineto{\pgfqpoint{0.941510in}{0.969771in}}%
\pgfpathlineto{\pgfqpoint{0.908224in}{0.935280in}}%
\pgfpathlineto{\pgfqpoint{0.874939in}{0.905091in}}%
\pgfpathlineto{\pgfqpoint{0.841653in}{0.879544in}}%
\pgfpathlineto{\pgfqpoint{0.808368in}{0.858637in}}%
\pgfpathlineto{\pgfqpoint{0.775082in}{0.842085in}}%
\pgfpathlineto{\pgfqpoint{0.741796in}{0.829406in}}%
\pgfpathlineto{\pgfqpoint{0.708511in}{0.820006in}}%
\pgfpathlineto{\pgfqpoint{0.675225in}{0.813260in}}%
\pgfpathlineto{\pgfqpoint{0.641939in}{0.808572in}}%
\pgfpathlineto{\pgfqpoint{0.608654in}{0.805417in}}%
\pgfpathlineto{\pgfqpoint{0.575368in}{0.803363in}}%
\pgfpathclose%
\pgfusepath{fill}%
\end{pgfscope}%
\begin{pgfscope}%
\pgfpathrectangle{\pgfqpoint{0.364005in}{0.800199in}}{\pgfqpoint{4.650000in}{1.372727in}}%
\pgfusepath{clip}%
\pgfsetbuttcap%
\pgfsetroundjoin%
\definecolor{currentfill}{rgb}{0.000000,0.000000,1.000000}%
\pgfsetfillcolor{currentfill}%
\pgfsetfillopacity{0.250000}%
\pgfsetlinewidth{0.000000pt}%
\definecolor{currentstroke}{rgb}{0.000000,0.000000,0.000000}%
\pgfsetstrokecolor{currentstroke}%
\pgfsetdash{}{0pt}%
\pgfpathmoveto{\pgfqpoint{0.575368in}{0.800528in}}%
\pgfpathlineto{\pgfqpoint{0.575368in}{0.800199in}}%
\pgfpathlineto{\pgfqpoint{0.608654in}{0.800199in}}%
\pgfpathlineto{\pgfqpoint{0.641939in}{0.800199in}}%
\pgfpathlineto{\pgfqpoint{0.675225in}{0.800199in}}%
\pgfpathlineto{\pgfqpoint{0.708511in}{0.800199in}}%
\pgfpathlineto{\pgfqpoint{0.741796in}{0.800199in}}%
\pgfpathlineto{\pgfqpoint{0.775082in}{0.800199in}}%
\pgfpathlineto{\pgfqpoint{0.808368in}{0.800199in}}%
\pgfpathlineto{\pgfqpoint{0.841653in}{0.800199in}}%
\pgfpathlineto{\pgfqpoint{0.874939in}{0.800199in}}%
\pgfpathlineto{\pgfqpoint{0.908224in}{0.800199in}}%
\pgfpathlineto{\pgfqpoint{0.941510in}{0.800199in}}%
\pgfpathlineto{\pgfqpoint{0.974796in}{0.800199in}}%
\pgfpathlineto{\pgfqpoint{1.008081in}{0.800199in}}%
\pgfpathlineto{\pgfqpoint{1.041367in}{0.800199in}}%
\pgfpathlineto{\pgfqpoint{1.074652in}{0.800199in}}%
\pgfpathlineto{\pgfqpoint{1.107938in}{0.800199in}}%
\pgfpathlineto{\pgfqpoint{1.141224in}{0.800199in}}%
\pgfpathlineto{\pgfqpoint{1.174509in}{0.800199in}}%
\pgfpathlineto{\pgfqpoint{1.207795in}{0.800199in}}%
\pgfpathlineto{\pgfqpoint{1.241080in}{0.800199in}}%
\pgfpathlineto{\pgfqpoint{1.274366in}{0.800199in}}%
\pgfpathlineto{\pgfqpoint{1.307652in}{0.800199in}}%
\pgfpathlineto{\pgfqpoint{1.340937in}{0.800199in}}%
\pgfpathlineto{\pgfqpoint{1.374223in}{0.800199in}}%
\pgfpathlineto{\pgfqpoint{1.407509in}{0.800199in}}%
\pgfpathlineto{\pgfqpoint{1.440794in}{0.800199in}}%
\pgfpathlineto{\pgfqpoint{1.474080in}{0.800199in}}%
\pgfpathlineto{\pgfqpoint{1.507365in}{0.800199in}}%
\pgfpathlineto{\pgfqpoint{1.540651in}{0.800199in}}%
\pgfpathlineto{\pgfqpoint{1.573937in}{0.800199in}}%
\pgfpathlineto{\pgfqpoint{1.607222in}{0.800199in}}%
\pgfpathlineto{\pgfqpoint{1.640508in}{0.800199in}}%
\pgfpathlineto{\pgfqpoint{1.673793in}{0.800199in}}%
\pgfpathlineto{\pgfqpoint{1.707079in}{0.800199in}}%
\pgfpathlineto{\pgfqpoint{1.740365in}{0.800199in}}%
\pgfpathlineto{\pgfqpoint{1.773650in}{0.800199in}}%
\pgfpathlineto{\pgfqpoint{1.806936in}{0.800199in}}%
\pgfpathlineto{\pgfqpoint{1.840221in}{0.800199in}}%
\pgfpathlineto{\pgfqpoint{1.873507in}{0.800199in}}%
\pgfpathlineto{\pgfqpoint{1.906793in}{0.800199in}}%
\pgfpathlineto{\pgfqpoint{1.940078in}{0.800199in}}%
\pgfpathlineto{\pgfqpoint{1.973364in}{0.800199in}}%
\pgfpathlineto{\pgfqpoint{2.006650in}{0.800199in}}%
\pgfpathlineto{\pgfqpoint{2.039935in}{0.800199in}}%
\pgfpathlineto{\pgfqpoint{2.073221in}{0.800199in}}%
\pgfpathlineto{\pgfqpoint{2.106506in}{0.800199in}}%
\pgfpathlineto{\pgfqpoint{2.139792in}{0.800199in}}%
\pgfpathlineto{\pgfqpoint{2.173078in}{0.800199in}}%
\pgfpathlineto{\pgfqpoint{2.206363in}{0.800199in}}%
\pgfpathlineto{\pgfqpoint{2.239649in}{0.800199in}}%
\pgfpathlineto{\pgfqpoint{2.272934in}{0.800199in}}%
\pgfpathlineto{\pgfqpoint{2.306220in}{0.800199in}}%
\pgfpathlineto{\pgfqpoint{2.339506in}{0.800199in}}%
\pgfpathlineto{\pgfqpoint{2.372791in}{0.800199in}}%
\pgfpathlineto{\pgfqpoint{2.406077in}{0.800199in}}%
\pgfpathlineto{\pgfqpoint{2.439362in}{0.800199in}}%
\pgfpathlineto{\pgfqpoint{2.472648in}{0.800199in}}%
\pgfpathlineto{\pgfqpoint{2.505934in}{0.800199in}}%
\pgfpathlineto{\pgfqpoint{2.539219in}{0.800199in}}%
\pgfpathlineto{\pgfqpoint{2.572505in}{0.800199in}}%
\pgfpathlineto{\pgfqpoint{2.605791in}{0.800199in}}%
\pgfpathlineto{\pgfqpoint{2.639076in}{0.800199in}}%
\pgfpathlineto{\pgfqpoint{2.672362in}{0.800199in}}%
\pgfpathlineto{\pgfqpoint{2.705647in}{0.800199in}}%
\pgfpathlineto{\pgfqpoint{2.738933in}{0.800199in}}%
\pgfpathlineto{\pgfqpoint{2.772219in}{0.800199in}}%
\pgfpathlineto{\pgfqpoint{2.805504in}{0.800199in}}%
\pgfpathlineto{\pgfqpoint{2.838790in}{0.800199in}}%
\pgfpathlineto{\pgfqpoint{2.872075in}{0.800199in}}%
\pgfpathlineto{\pgfqpoint{2.905361in}{0.800199in}}%
\pgfpathlineto{\pgfqpoint{2.938647in}{0.800199in}}%
\pgfpathlineto{\pgfqpoint{2.971932in}{0.800199in}}%
\pgfpathlineto{\pgfqpoint{3.005218in}{0.800199in}}%
\pgfpathlineto{\pgfqpoint{3.038504in}{0.800199in}}%
\pgfpathlineto{\pgfqpoint{3.071789in}{0.800199in}}%
\pgfpathlineto{\pgfqpoint{3.105075in}{0.800199in}}%
\pgfpathlineto{\pgfqpoint{3.138360in}{0.800199in}}%
\pgfpathlineto{\pgfqpoint{3.171646in}{0.800199in}}%
\pgfpathlineto{\pgfqpoint{3.204932in}{0.800199in}}%
\pgfpathlineto{\pgfqpoint{3.238217in}{0.800199in}}%
\pgfpathlineto{\pgfqpoint{3.271503in}{0.800199in}}%
\pgfpathlineto{\pgfqpoint{3.304788in}{0.800199in}}%
\pgfpathlineto{\pgfqpoint{3.338074in}{0.800199in}}%
\pgfpathlineto{\pgfqpoint{3.371360in}{0.800199in}}%
\pgfpathlineto{\pgfqpoint{3.404645in}{0.800199in}}%
\pgfpathlineto{\pgfqpoint{3.437931in}{0.800199in}}%
\pgfpathlineto{\pgfqpoint{3.471216in}{0.800199in}}%
\pgfpathlineto{\pgfqpoint{3.504502in}{0.800199in}}%
\pgfpathlineto{\pgfqpoint{3.537788in}{0.800199in}}%
\pgfpathlineto{\pgfqpoint{3.571073in}{0.800199in}}%
\pgfpathlineto{\pgfqpoint{3.604359in}{0.800199in}}%
\pgfpathlineto{\pgfqpoint{3.637645in}{0.800199in}}%
\pgfpathlineto{\pgfqpoint{3.670930in}{0.800199in}}%
\pgfpathlineto{\pgfqpoint{3.704216in}{0.800199in}}%
\pgfpathlineto{\pgfqpoint{3.737501in}{0.800199in}}%
\pgfpathlineto{\pgfqpoint{3.770787in}{0.800199in}}%
\pgfpathlineto{\pgfqpoint{3.804073in}{0.800199in}}%
\pgfpathlineto{\pgfqpoint{3.837358in}{0.800199in}}%
\pgfpathlineto{\pgfqpoint{3.870644in}{0.800199in}}%
\pgfpathlineto{\pgfqpoint{3.903929in}{0.800199in}}%
\pgfpathlineto{\pgfqpoint{3.937215in}{0.800199in}}%
\pgfpathlineto{\pgfqpoint{3.970501in}{0.800199in}}%
\pgfpathlineto{\pgfqpoint{4.003786in}{0.800199in}}%
\pgfpathlineto{\pgfqpoint{4.037072in}{0.800199in}}%
\pgfpathlineto{\pgfqpoint{4.070357in}{0.800199in}}%
\pgfpathlineto{\pgfqpoint{4.103643in}{0.800199in}}%
\pgfpathlineto{\pgfqpoint{4.136929in}{0.800199in}}%
\pgfpathlineto{\pgfqpoint{4.170214in}{0.800199in}}%
\pgfpathlineto{\pgfqpoint{4.203500in}{0.800199in}}%
\pgfpathlineto{\pgfqpoint{4.236786in}{0.800199in}}%
\pgfpathlineto{\pgfqpoint{4.270071in}{0.800199in}}%
\pgfpathlineto{\pgfqpoint{4.303357in}{0.800199in}}%
\pgfpathlineto{\pgfqpoint{4.336642in}{0.800199in}}%
\pgfpathlineto{\pgfqpoint{4.369928in}{0.800199in}}%
\pgfpathlineto{\pgfqpoint{4.403214in}{0.800199in}}%
\pgfpathlineto{\pgfqpoint{4.436499in}{0.800199in}}%
\pgfpathlineto{\pgfqpoint{4.469785in}{0.800199in}}%
\pgfpathlineto{\pgfqpoint{4.503070in}{0.800199in}}%
\pgfpathlineto{\pgfqpoint{4.536356in}{0.800199in}}%
\pgfpathlineto{\pgfqpoint{4.569642in}{0.800199in}}%
\pgfpathlineto{\pgfqpoint{4.602927in}{0.800199in}}%
\pgfpathlineto{\pgfqpoint{4.636213in}{0.800199in}}%
\pgfpathlineto{\pgfqpoint{4.669498in}{0.800199in}}%
\pgfpathlineto{\pgfqpoint{4.702784in}{0.800199in}}%
\pgfpathlineto{\pgfqpoint{4.736070in}{0.800199in}}%
\pgfpathlineto{\pgfqpoint{4.769355in}{0.800199in}}%
\pgfpathlineto{\pgfqpoint{4.802641in}{0.800199in}}%
\pgfpathlineto{\pgfqpoint{4.802641in}{0.800392in}}%
\pgfpathlineto{\pgfqpoint{4.802641in}{0.800392in}}%
\pgfpathlineto{\pgfqpoint{4.769355in}{0.800312in}}%
\pgfpathlineto{\pgfqpoint{4.736070in}{0.800268in}}%
\pgfpathlineto{\pgfqpoint{4.702784in}{0.800246in}}%
\pgfpathlineto{\pgfqpoint{4.669498in}{0.800240in}}%
\pgfpathlineto{\pgfqpoint{4.636213in}{0.800244in}}%
\pgfpathlineto{\pgfqpoint{4.602927in}{0.800257in}}%
\pgfpathlineto{\pgfqpoint{4.569642in}{0.800279in}}%
\pgfpathlineto{\pgfqpoint{4.536356in}{0.800310in}}%
\pgfpathlineto{\pgfqpoint{4.503070in}{0.800353in}}%
\pgfpathlineto{\pgfqpoint{4.469785in}{0.800409in}}%
\pgfpathlineto{\pgfqpoint{4.436499in}{0.800481in}}%
\pgfpathlineto{\pgfqpoint{4.403214in}{0.800574in}}%
\pgfpathlineto{\pgfqpoint{4.369928in}{0.800691in}}%
\pgfpathlineto{\pgfqpoint{4.336642in}{0.800840in}}%
\pgfpathlineto{\pgfqpoint{4.303357in}{0.801026in}}%
\pgfpathlineto{\pgfqpoint{4.270071in}{0.801257in}}%
\pgfpathlineto{\pgfqpoint{4.236786in}{0.801542in}}%
\pgfpathlineto{\pgfqpoint{4.203500in}{0.801892in}}%
\pgfpathlineto{\pgfqpoint{4.170214in}{0.802315in}}%
\pgfpathlineto{\pgfqpoint{4.136929in}{0.802823in}}%
\pgfpathlineto{\pgfqpoint{4.103643in}{0.803427in}}%
\pgfpathlineto{\pgfqpoint{4.070357in}{0.804141in}}%
\pgfpathlineto{\pgfqpoint{4.037072in}{0.804978in}}%
\pgfpathlineto{\pgfqpoint{4.003786in}{0.805959in}}%
\pgfpathlineto{\pgfqpoint{3.970501in}{0.807111in}}%
\pgfpathlineto{\pgfqpoint{3.937215in}{0.808473in}}%
\pgfpathlineto{\pgfqpoint{3.903929in}{0.810098in}}%
\pgfpathlineto{\pgfqpoint{3.870644in}{0.812060in}}%
\pgfpathlineto{\pgfqpoint{3.837358in}{0.814457in}}%
\pgfpathlineto{\pgfqpoint{3.804073in}{0.817414in}}%
\pgfpathlineto{\pgfqpoint{3.770787in}{0.821084in}}%
\pgfpathlineto{\pgfqpoint{3.737501in}{0.825647in}}%
\pgfpathlineto{\pgfqpoint{3.704216in}{0.831307in}}%
\pgfpathlineto{\pgfqpoint{3.670930in}{0.838279in}}%
\pgfpathlineto{\pgfqpoint{3.637645in}{0.846780in}}%
\pgfpathlineto{\pgfqpoint{3.604359in}{0.857006in}}%
\pgfpathlineto{\pgfqpoint{3.571073in}{0.869108in}}%
\pgfpathlineto{\pgfqpoint{3.537788in}{0.883164in}}%
\pgfpathlineto{\pgfqpoint{3.504502in}{0.899143in}}%
\pgfpathlineto{\pgfqpoint{3.471216in}{0.916879in}}%
\pgfpathlineto{\pgfqpoint{3.437931in}{0.936044in}}%
\pgfpathlineto{\pgfqpoint{3.404645in}{0.956144in}}%
\pgfpathlineto{\pgfqpoint{3.371360in}{0.976521in}}%
\pgfpathlineto{\pgfqpoint{3.338074in}{0.996392in}}%
\pgfpathlineto{\pgfqpoint{3.304788in}{1.014899in}}%
\pgfpathlineto{\pgfqpoint{3.271503in}{1.031186in}}%
\pgfpathlineto{\pgfqpoint{3.238217in}{1.044481in}}%
\pgfpathlineto{\pgfqpoint{3.204932in}{1.054179in}}%
\pgfpathlineto{\pgfqpoint{3.171646in}{1.059913in}}%
\pgfpathlineto{\pgfqpoint{3.138360in}{1.061610in}}%
\pgfpathlineto{\pgfqpoint{3.105075in}{1.059516in}}%
\pgfpathlineto{\pgfqpoint{3.071789in}{1.054203in}}%
\pgfpathlineto{\pgfqpoint{3.038504in}{1.046558in}}%
\pgfpathlineto{\pgfqpoint{3.005218in}{1.037757in}}%
\pgfpathlineto{\pgfqpoint{2.971932in}{1.029238in}}%
\pgfpathlineto{\pgfqpoint{2.938647in}{1.022666in}}%
\pgfpathlineto{\pgfqpoint{2.905361in}{1.019892in}}%
\pgfpathlineto{\pgfqpoint{2.872075in}{1.022890in}}%
\pgfpathlineto{\pgfqpoint{2.838790in}{1.033678in}}%
\pgfpathlineto{\pgfqpoint{2.805504in}{1.054187in}}%
\pgfpathlineto{\pgfqpoint{2.772219in}{1.086099in}}%
\pgfpathlineto{\pgfqpoint{2.738933in}{1.130643in}}%
\pgfpathlineto{\pgfqpoint{2.705647in}{1.188367in}}%
\pgfpathlineto{\pgfqpoint{2.672362in}{1.258927in}}%
\pgfpathlineto{\pgfqpoint{2.639076in}{1.340924in}}%
\pgfpathlineto{\pgfqpoint{2.605791in}{1.431828in}}%
\pgfpathlineto{\pgfqpoint{2.572505in}{1.528031in}}%
\pgfpathlineto{\pgfqpoint{2.539219in}{1.625043in}}%
\pgfpathlineto{\pgfqpoint{2.505934in}{1.717821in}}%
\pgfpathlineto{\pgfqpoint{2.472648in}{1.801196in}}%
\pgfpathlineto{\pgfqpoint{2.439362in}{1.870358in}}%
\pgfpathlineto{\pgfqpoint{2.406077in}{1.921305in}}%
\pgfpathlineto{\pgfqpoint{2.372791in}{1.951224in}}%
\pgfpathlineto{\pgfqpoint{2.339506in}{1.958728in}}%
\pgfpathlineto{\pgfqpoint{2.306220in}{1.943946in}}%
\pgfpathlineto{\pgfqpoint{2.272934in}{1.908444in}}%
\pgfpathlineto{\pgfqpoint{2.239649in}{1.855024in}}%
\pgfpathlineto{\pgfqpoint{2.206363in}{1.787420in}}%
\pgfpathlineto{\pgfqpoint{2.173078in}{1.709941in}}%
\pgfpathlineto{\pgfqpoint{2.139792in}{1.627108in}}%
\pgfpathlineto{\pgfqpoint{2.106506in}{1.543303in}}%
\pgfpathlineto{\pgfqpoint{2.073221in}{1.462484in}}%
\pgfpathlineto{\pgfqpoint{2.039935in}{1.387946in}}%
\pgfpathlineto{\pgfqpoint{2.006650in}{1.322175in}}%
\pgfpathlineto{\pgfqpoint{1.973364in}{1.266769in}}%
\pgfpathlineto{\pgfqpoint{1.940078in}{1.222443in}}%
\pgfpathlineto{\pgfqpoint{1.906793in}{1.189091in}}%
\pgfpathlineto{\pgfqpoint{1.873507in}{1.165915in}}%
\pgfpathlineto{\pgfqpoint{1.840221in}{1.151574in}}%
\pgfpathlineto{\pgfqpoint{1.806936in}{1.144362in}}%
\pgfpathlineto{\pgfqpoint{1.773650in}{1.142386in}}%
\pgfpathlineto{\pgfqpoint{1.740365in}{1.143726in}}%
\pgfpathlineto{\pgfqpoint{1.707079in}{1.146575in}}%
\pgfpathlineto{\pgfqpoint{1.673793in}{1.149343in}}%
\pgfpathlineto{\pgfqpoint{1.640508in}{1.150724in}}%
\pgfpathlineto{\pgfqpoint{1.607222in}{1.149736in}}%
\pgfpathlineto{\pgfqpoint{1.573937in}{1.145722in}}%
\pgfpathlineto{\pgfqpoint{1.540651in}{1.138332in}}%
\pgfpathlineto{\pgfqpoint{1.507365in}{1.127488in}}%
\pgfpathlineto{\pgfqpoint{1.474080in}{1.113343in}}%
\pgfpathlineto{\pgfqpoint{1.440794in}{1.096231in}}%
\pgfpathlineto{\pgfqpoint{1.407509in}{1.076627in}}%
\pgfpathlineto{\pgfqpoint{1.374223in}{1.055096in}}%
\pgfpathlineto{\pgfqpoint{1.340937in}{1.032259in}}%
\pgfpathlineto{\pgfqpoint{1.307652in}{1.008750in}}%
\pgfpathlineto{\pgfqpoint{1.274366in}{0.985184in}}%
\pgfpathlineto{\pgfqpoint{1.241080in}{0.962121in}}%
\pgfpathlineto{\pgfqpoint{1.207795in}{0.940046in}}%
\pgfpathlineto{\pgfqpoint{1.174509in}{0.919349in}}%
\pgfpathlineto{\pgfqpoint{1.141224in}{0.900320in}}%
\pgfpathlineto{\pgfqpoint{1.107938in}{0.883146in}}%
\pgfpathlineto{\pgfqpoint{1.074652in}{0.867921in}}%
\pgfpathlineto{\pgfqpoint{1.041367in}{0.854660in}}%
\pgfpathlineto{\pgfqpoint{1.008081in}{0.843307in}}%
\pgfpathlineto{\pgfqpoint{0.974796in}{0.833759in}}%
\pgfpathlineto{\pgfqpoint{0.941510in}{0.825872in}}%
\pgfpathlineto{\pgfqpoint{0.908224in}{0.819481in}}%
\pgfpathlineto{\pgfqpoint{0.874939in}{0.814402in}}%
\pgfpathlineto{\pgfqpoint{0.841653in}{0.810450in}}%
\pgfpathlineto{\pgfqpoint{0.808368in}{0.807441in}}%
\pgfpathlineto{\pgfqpoint{0.775082in}{0.805203in}}%
\pgfpathlineto{\pgfqpoint{0.741796in}{0.803578in}}%
\pgfpathlineto{\pgfqpoint{0.708511in}{0.802427in}}%
\pgfpathlineto{\pgfqpoint{0.675225in}{0.801632in}}%
\pgfpathlineto{\pgfqpoint{0.641939in}{0.801098in}}%
\pgfpathlineto{\pgfqpoint{0.608654in}{0.800750in}}%
\pgfpathlineto{\pgfqpoint{0.575368in}{0.800528in}}%
\pgfpathclose%
\pgfusepath{fill}%
\end{pgfscope}%
\begin{pgfscope}%
\pgfsetbuttcap%
\pgfsetroundjoin%
\definecolor{currentfill}{rgb}{0.000000,0.000000,0.000000}%
\pgfsetfillcolor{currentfill}%
\pgfsetlinewidth{0.803000pt}%
\definecolor{currentstroke}{rgb}{0.000000,0.000000,0.000000}%
\pgfsetstrokecolor{currentstroke}%
\pgfsetdash{}{0pt}%
\pgfsys@defobject{currentmarker}{\pgfqpoint{0.000000in}{-0.048611in}}{\pgfqpoint{0.000000in}{0.000000in}}{%
\pgfpathmoveto{\pgfqpoint{0.000000in}{0.000000in}}%
\pgfpathlineto{\pgfqpoint{0.000000in}{-0.048611in}}%
\pgfusepath{stroke,fill}%
}%
\begin{pgfscope}%
\pgfsys@transformshift{0.767517in}{0.800199in}%
\pgfsys@useobject{currentmarker}{}%
\end{pgfscope}%
\end{pgfscope}%
\begin{pgfscope}%
\definecolor{textcolor}{rgb}{0.000000,0.000000,0.000000}%
\pgfsetstrokecolor{textcolor}%
\pgfsetfillcolor{textcolor}%
\pgftext[x=0.767517in,y=0.702977in,,top]{\color{textcolor}\rmfamily\fontsize{11.000000}{13.200000}\selectfont \(\displaystyle 0.0\)}%
\end{pgfscope}%
\begin{pgfscope}%
\pgfsetbuttcap%
\pgfsetroundjoin%
\definecolor{currentfill}{rgb}{0.000000,0.000000,0.000000}%
\pgfsetfillcolor{currentfill}%
\pgfsetlinewidth{0.803000pt}%
\definecolor{currentstroke}{rgb}{0.000000,0.000000,0.000000}%
\pgfsetstrokecolor{currentstroke}%
\pgfsetdash{}{0pt}%
\pgfsys@defobject{currentmarker}{\pgfqpoint{0.000000in}{-0.048611in}}{\pgfqpoint{0.000000in}{0.000000in}}{%
\pgfpathmoveto{\pgfqpoint{0.000000in}{0.000000in}}%
\pgfpathlineto{\pgfqpoint{0.000000in}{-0.048611in}}%
\pgfusepath{stroke,fill}%
}%
\begin{pgfscope}%
\pgfsys@transformshift{1.536112in}{0.800199in}%
\pgfsys@useobject{currentmarker}{}%
\end{pgfscope}%
\end{pgfscope}%
\begin{pgfscope}%
\definecolor{textcolor}{rgb}{0.000000,0.000000,0.000000}%
\pgfsetstrokecolor{textcolor}%
\pgfsetfillcolor{textcolor}%
\pgftext[x=1.536112in,y=0.702977in,,top]{\color{textcolor}\rmfamily\fontsize{11.000000}{13.200000}\selectfont \(\displaystyle 0.2\)}%
\end{pgfscope}%
\begin{pgfscope}%
\pgfsetbuttcap%
\pgfsetroundjoin%
\definecolor{currentfill}{rgb}{0.000000,0.000000,0.000000}%
\pgfsetfillcolor{currentfill}%
\pgfsetlinewidth{0.803000pt}%
\definecolor{currentstroke}{rgb}{0.000000,0.000000,0.000000}%
\pgfsetstrokecolor{currentstroke}%
\pgfsetdash{}{0pt}%
\pgfsys@defobject{currentmarker}{\pgfqpoint{0.000000in}{-0.048611in}}{\pgfqpoint{0.000000in}{0.000000in}}{%
\pgfpathmoveto{\pgfqpoint{0.000000in}{0.000000in}}%
\pgfpathlineto{\pgfqpoint{0.000000in}{-0.048611in}}%
\pgfusepath{stroke,fill}%
}%
\begin{pgfscope}%
\pgfsys@transformshift{2.304707in}{0.800199in}%
\pgfsys@useobject{currentmarker}{}%
\end{pgfscope}%
\end{pgfscope}%
\begin{pgfscope}%
\definecolor{textcolor}{rgb}{0.000000,0.000000,0.000000}%
\pgfsetstrokecolor{textcolor}%
\pgfsetfillcolor{textcolor}%
\pgftext[x=2.304707in,y=0.702977in,,top]{\color{textcolor}\rmfamily\fontsize{11.000000}{13.200000}\selectfont \(\displaystyle 0.4\)}%
\end{pgfscope}%
\begin{pgfscope}%
\pgfsetbuttcap%
\pgfsetroundjoin%
\definecolor{currentfill}{rgb}{0.000000,0.000000,0.000000}%
\pgfsetfillcolor{currentfill}%
\pgfsetlinewidth{0.803000pt}%
\definecolor{currentstroke}{rgb}{0.000000,0.000000,0.000000}%
\pgfsetstrokecolor{currentstroke}%
\pgfsetdash{}{0pt}%
\pgfsys@defobject{currentmarker}{\pgfqpoint{0.000000in}{-0.048611in}}{\pgfqpoint{0.000000in}{0.000000in}}{%
\pgfpathmoveto{\pgfqpoint{0.000000in}{0.000000in}}%
\pgfpathlineto{\pgfqpoint{0.000000in}{-0.048611in}}%
\pgfusepath{stroke,fill}%
}%
\begin{pgfscope}%
\pgfsys@transformshift{3.073302in}{0.800199in}%
\pgfsys@useobject{currentmarker}{}%
\end{pgfscope}%
\end{pgfscope}%
\begin{pgfscope}%
\definecolor{textcolor}{rgb}{0.000000,0.000000,0.000000}%
\pgfsetstrokecolor{textcolor}%
\pgfsetfillcolor{textcolor}%
\pgftext[x=3.073302in,y=0.702977in,,top]{\color{textcolor}\rmfamily\fontsize{11.000000}{13.200000}\selectfont \(\displaystyle 0.6\)}%
\end{pgfscope}%
\begin{pgfscope}%
\pgfsetbuttcap%
\pgfsetroundjoin%
\definecolor{currentfill}{rgb}{0.000000,0.000000,0.000000}%
\pgfsetfillcolor{currentfill}%
\pgfsetlinewidth{0.803000pt}%
\definecolor{currentstroke}{rgb}{0.000000,0.000000,0.000000}%
\pgfsetstrokecolor{currentstroke}%
\pgfsetdash{}{0pt}%
\pgfsys@defobject{currentmarker}{\pgfqpoint{0.000000in}{-0.048611in}}{\pgfqpoint{0.000000in}{0.000000in}}{%
\pgfpathmoveto{\pgfqpoint{0.000000in}{0.000000in}}%
\pgfpathlineto{\pgfqpoint{0.000000in}{-0.048611in}}%
\pgfusepath{stroke,fill}%
}%
\begin{pgfscope}%
\pgfsys@transformshift{3.841897in}{0.800199in}%
\pgfsys@useobject{currentmarker}{}%
\end{pgfscope}%
\end{pgfscope}%
\begin{pgfscope}%
\definecolor{textcolor}{rgb}{0.000000,0.000000,0.000000}%
\pgfsetstrokecolor{textcolor}%
\pgfsetfillcolor{textcolor}%
\pgftext[x=3.841897in,y=0.702977in,,top]{\color{textcolor}\rmfamily\fontsize{11.000000}{13.200000}\selectfont \(\displaystyle 0.8\)}%
\end{pgfscope}%
\begin{pgfscope}%
\pgfsetbuttcap%
\pgfsetroundjoin%
\definecolor{currentfill}{rgb}{0.000000,0.000000,0.000000}%
\pgfsetfillcolor{currentfill}%
\pgfsetlinewidth{0.803000pt}%
\definecolor{currentstroke}{rgb}{0.000000,0.000000,0.000000}%
\pgfsetstrokecolor{currentstroke}%
\pgfsetdash{}{0pt}%
\pgfsys@defobject{currentmarker}{\pgfqpoint{0.000000in}{-0.048611in}}{\pgfqpoint{0.000000in}{0.000000in}}{%
\pgfpathmoveto{\pgfqpoint{0.000000in}{0.000000in}}%
\pgfpathlineto{\pgfqpoint{0.000000in}{-0.048611in}}%
\pgfusepath{stroke,fill}%
}%
\begin{pgfscope}%
\pgfsys@transformshift{4.610492in}{0.800199in}%
\pgfsys@useobject{currentmarker}{}%
\end{pgfscope}%
\end{pgfscope}%
\begin{pgfscope}%
\definecolor{textcolor}{rgb}{0.000000,0.000000,0.000000}%
\pgfsetstrokecolor{textcolor}%
\pgfsetfillcolor{textcolor}%
\pgftext[x=4.610492in,y=0.702977in,,top]{\color{textcolor}\rmfamily\fontsize{11.000000}{13.200000}\selectfont \(\displaystyle 1.0\)}%
\end{pgfscope}%
\begin{pgfscope}%
\definecolor{textcolor}{rgb}{0.000000,0.000000,0.000000}%
\pgfsetstrokecolor{textcolor}%
\pgfsetfillcolor{textcolor}%
\pgftext[x=2.689005in,y=0.512236in,,top]{\color{textcolor}\rmfamily\fontsize{11.000000}{13.200000}\selectfont Pareto weight of Women}%
\end{pgfscope}%
\begin{pgfscope}%
\pgfsetbuttcap%
\pgfsetroundjoin%
\definecolor{currentfill}{rgb}{0.000000,0.000000,0.000000}%
\pgfsetfillcolor{currentfill}%
\pgfsetlinewidth{0.803000pt}%
\definecolor{currentstroke}{rgb}{0.000000,0.000000,0.000000}%
\pgfsetstrokecolor{currentstroke}%
\pgfsetdash{}{0pt}%
\pgfsys@defobject{currentmarker}{\pgfqpoint{-0.048611in}{0.000000in}}{\pgfqpoint{0.000000in}{0.000000in}}{%
\pgfpathmoveto{\pgfqpoint{0.000000in}{0.000000in}}%
\pgfpathlineto{\pgfqpoint{-0.048611in}{0.000000in}}%
\pgfusepath{stroke,fill}%
}%
\begin{pgfscope}%
\pgfsys@transformshift{0.364005in}{0.800199in}%
\pgfsys@useobject{currentmarker}{}%
\end{pgfscope}%
\end{pgfscope}%
\begin{pgfscope}%
\definecolor{textcolor}{rgb}{0.000000,0.000000,0.000000}%
\pgfsetstrokecolor{textcolor}%
\pgfsetfillcolor{textcolor}%
\pgftext[x=0.190741in,y=0.747393in,left,base]{\color{textcolor}\rmfamily\fontsize{11.000000}{13.200000}\selectfont \(\displaystyle 0\)}%
\end{pgfscope}%
\begin{pgfscope}%
\pgfsetbuttcap%
\pgfsetroundjoin%
\definecolor{currentfill}{rgb}{0.000000,0.000000,0.000000}%
\pgfsetfillcolor{currentfill}%
\pgfsetlinewidth{0.803000pt}%
\definecolor{currentstroke}{rgb}{0.000000,0.000000,0.000000}%
\pgfsetstrokecolor{currentstroke}%
\pgfsetdash{}{0pt}%
\pgfsys@defobject{currentmarker}{\pgfqpoint{-0.048611in}{0.000000in}}{\pgfqpoint{0.000000in}{0.000000in}}{%
\pgfpathmoveto{\pgfqpoint{0.000000in}{0.000000in}}%
\pgfpathlineto{\pgfqpoint{-0.048611in}{0.000000in}}%
\pgfusepath{stroke,fill}%
}%
\begin{pgfscope}%
\pgfsys@transformshift{0.364005in}{1.362103in}%
\pgfsys@useobject{currentmarker}{}%
\end{pgfscope}%
\end{pgfscope}%
\begin{pgfscope}%
\definecolor{textcolor}{rgb}{0.000000,0.000000,0.000000}%
\pgfsetstrokecolor{textcolor}%
\pgfsetfillcolor{textcolor}%
\pgftext[x=0.190741in,y=1.309296in,left,base]{\color{textcolor}\rmfamily\fontsize{11.000000}{13.200000}\selectfont \(\displaystyle 2\)}%
\end{pgfscope}%
\begin{pgfscope}%
\pgfsetbuttcap%
\pgfsetroundjoin%
\definecolor{currentfill}{rgb}{0.000000,0.000000,0.000000}%
\pgfsetfillcolor{currentfill}%
\pgfsetlinewidth{0.803000pt}%
\definecolor{currentstroke}{rgb}{0.000000,0.000000,0.000000}%
\pgfsetstrokecolor{currentstroke}%
\pgfsetdash{}{0pt}%
\pgfsys@defobject{currentmarker}{\pgfqpoint{-0.048611in}{0.000000in}}{\pgfqpoint{0.000000in}{0.000000in}}{%
\pgfpathmoveto{\pgfqpoint{0.000000in}{0.000000in}}%
\pgfpathlineto{\pgfqpoint{-0.048611in}{0.000000in}}%
\pgfusepath{stroke,fill}%
}%
\begin{pgfscope}%
\pgfsys@transformshift{0.364005in}{1.924006in}%
\pgfsys@useobject{currentmarker}{}%
\end{pgfscope}%
\end{pgfscope}%
\begin{pgfscope}%
\definecolor{textcolor}{rgb}{0.000000,0.000000,0.000000}%
\pgfsetstrokecolor{textcolor}%
\pgfsetfillcolor{textcolor}%
\pgftext[x=0.190741in,y=1.871200in,left,base]{\color{textcolor}\rmfamily\fontsize{11.000000}{13.200000}\selectfont \(\displaystyle 4\)}%
\end{pgfscope}%
\begin{pgfscope}%
\definecolor{textcolor}{rgb}{0.000000,0.000000,0.000000}%
\pgfsetstrokecolor{textcolor}%
\pgfsetfillcolor{textcolor}%
\pgftext[x=0.135185in,y=1.486563in,,bottom,rotate=90.000000]{\color{textcolor}\rmfamily\fontsize{11.000000}{13.200000}\selectfont Denisty}%
\end{pgfscope}%
\begin{pgfscope}%
\pgfpathrectangle{\pgfqpoint{0.364005in}{0.800199in}}{\pgfqpoint{4.650000in}{1.372727in}}%
\pgfusepath{clip}%
\pgfsetrectcap%
\pgfsetroundjoin%
\pgfsetlinewidth{0.010037pt}%
\definecolor{currentstroke}{rgb}{1.000000,0.000000,0.000000}%
\pgfsetstrokecolor{currentstroke}%
\pgfsetdash{}{0pt}%
\pgfpathmoveto{\pgfqpoint{0.575368in}{0.803363in}}%
\pgfpathlineto{\pgfqpoint{0.641939in}{0.808572in}}%
\pgfpathlineto{\pgfqpoint{0.675225in}{0.813260in}}%
\pgfpathlineto{\pgfqpoint{0.708511in}{0.820006in}}%
\pgfpathlineto{\pgfqpoint{0.741796in}{0.829406in}}%
\pgfpathlineto{\pgfqpoint{0.775082in}{0.842085in}}%
\pgfpathlineto{\pgfqpoint{0.808368in}{0.858637in}}%
\pgfpathlineto{\pgfqpoint{0.841653in}{0.879544in}}%
\pgfpathlineto{\pgfqpoint{0.874939in}{0.905091in}}%
\pgfpathlineto{\pgfqpoint{0.908224in}{0.935280in}}%
\pgfpathlineto{\pgfqpoint{0.941510in}{0.969771in}}%
\pgfpathlineto{\pgfqpoint{0.974796in}{1.007859in}}%
\pgfpathlineto{\pgfqpoint{1.141224in}{1.209425in}}%
\pgfpathlineto{\pgfqpoint{1.174509in}{1.242798in}}%
\pgfpathlineto{\pgfqpoint{1.207795in}{1.271692in}}%
\pgfpathlineto{\pgfqpoint{1.241080in}{1.295838in}}%
\pgfpathlineto{\pgfqpoint{1.274366in}{1.315290in}}%
\pgfpathlineto{\pgfqpoint{1.307652in}{1.330319in}}%
\pgfpathlineto{\pgfqpoint{1.340937in}{1.341310in}}%
\pgfpathlineto{\pgfqpoint{1.374223in}{1.348642in}}%
\pgfpathlineto{\pgfqpoint{1.407509in}{1.352603in}}%
\pgfpathlineto{\pgfqpoint{1.440794in}{1.353336in}}%
\pgfpathlineto{\pgfqpoint{1.474080in}{1.350840in}}%
\pgfpathlineto{\pgfqpoint{1.507365in}{1.345008in}}%
\pgfpathlineto{\pgfqpoint{1.540651in}{1.335709in}}%
\pgfpathlineto{\pgfqpoint{1.573937in}{1.322888in}}%
\pgfpathlineto{\pgfqpoint{1.607222in}{1.306677in}}%
\pgfpathlineto{\pgfqpoint{1.640508in}{1.287496in}}%
\pgfpathlineto{\pgfqpoint{1.773650in}{1.202187in}}%
\pgfpathlineto{\pgfqpoint{1.806936in}{1.186789in}}%
\pgfpathlineto{\pgfqpoint{1.840221in}{1.177515in}}%
\pgfpathlineto{\pgfqpoint{1.873507in}{1.176029in}}%
\pgfpathlineto{\pgfqpoint{1.906793in}{1.183559in}}%
\pgfpathlineto{\pgfqpoint{1.940078in}{1.200717in}}%
\pgfpathlineto{\pgfqpoint{1.973364in}{1.227360in}}%
\pgfpathlineto{\pgfqpoint{2.006650in}{1.262521in}}%
\pgfpathlineto{\pgfqpoint{2.039935in}{1.304416in}}%
\pgfpathlineto{\pgfqpoint{2.139792in}{1.442891in}}%
\pgfpathlineto{\pgfqpoint{2.173078in}{1.482418in}}%
\pgfpathlineto{\pgfqpoint{2.206363in}{1.513329in}}%
\pgfpathlineto{\pgfqpoint{2.239649in}{1.533141in}}%
\pgfpathlineto{\pgfqpoint{2.272934in}{1.540174in}}%
\pgfpathlineto{\pgfqpoint{2.306220in}{1.533675in}}%
\pgfpathlineto{\pgfqpoint{2.339506in}{1.513861in}}%
\pgfpathlineto{\pgfqpoint{2.372791in}{1.481859in}}%
\pgfpathlineto{\pgfqpoint{2.406077in}{1.439561in}}%
\pgfpathlineto{\pgfqpoint{2.439362in}{1.389432in}}%
\pgfpathlineto{\pgfqpoint{2.505934in}{1.276985in}}%
\pgfpathlineto{\pgfqpoint{2.539219in}{1.220367in}}%
\pgfpathlineto{\pgfqpoint{2.572505in}{1.166915in}}%
\pgfpathlineto{\pgfqpoint{2.605791in}{1.118703in}}%
\pgfpathlineto{\pgfqpoint{2.639076in}{1.077289in}}%
\pgfpathlineto{\pgfqpoint{2.672362in}{1.043688in}}%
\pgfpathlineto{\pgfqpoint{2.705647in}{1.018374in}}%
\pgfpathlineto{\pgfqpoint{2.738933in}{1.001326in}}%
\pgfpathlineto{\pgfqpoint{2.772219in}{0.992091in}}%
\pgfpathlineto{\pgfqpoint{2.805504in}{0.989865in}}%
\pgfpathlineto{\pgfqpoint{2.838790in}{0.993579in}}%
\pgfpathlineto{\pgfqpoint{2.872075in}{1.001978in}}%
\pgfpathlineto{\pgfqpoint{2.905361in}{1.013702in}}%
\pgfpathlineto{\pgfqpoint{3.005218in}{1.055026in}}%
\pgfpathlineto{\pgfqpoint{3.038504in}{1.066594in}}%
\pgfpathlineto{\pgfqpoint{3.071789in}{1.075283in}}%
\pgfpathlineto{\pgfqpoint{3.105075in}{1.080337in}}%
\pgfpathlineto{\pgfqpoint{3.138360in}{1.081263in}}%
\pgfpathlineto{\pgfqpoint{3.171646in}{1.077855in}}%
\pgfpathlineto{\pgfqpoint{3.204932in}{1.070201in}}%
\pgfpathlineto{\pgfqpoint{3.238217in}{1.058670in}}%
\pgfpathlineto{\pgfqpoint{3.271503in}{1.043874in}}%
\pgfpathlineto{\pgfqpoint{3.338074in}{1.007724in}}%
\pgfpathlineto{\pgfqpoint{3.437931in}{0.950116in}}%
\pgfpathlineto{\pgfqpoint{3.504502in}{0.917352in}}%
\pgfpathlineto{\pgfqpoint{3.537788in}{0.903697in}}%
\pgfpathlineto{\pgfqpoint{3.571073in}{0.891924in}}%
\pgfpathlineto{\pgfqpoint{3.637645in}{0.873493in}}%
\pgfpathlineto{\pgfqpoint{3.704216in}{0.860374in}}%
\pgfpathlineto{\pgfqpoint{3.770787in}{0.850556in}}%
\pgfpathlineto{\pgfqpoint{3.903929in}{0.834745in}}%
\pgfpathlineto{\pgfqpoint{4.070357in}{0.817350in}}%
\pgfpathlineto{\pgfqpoint{4.170214in}{0.809658in}}%
\pgfpathlineto{\pgfqpoint{4.270071in}{0.804812in}}%
\pgfpathlineto{\pgfqpoint{4.403214in}{0.801713in}}%
\pgfpathlineto{\pgfqpoint{4.636213in}{0.800415in}}%
\pgfpathlineto{\pgfqpoint{4.802641in}{0.802068in}}%
\pgfpathlineto{\pgfqpoint{4.802641in}{0.802068in}}%
\pgfusepath{stroke}%
\end{pgfscope}%
\begin{pgfscope}%
\pgfpathrectangle{\pgfqpoint{0.364005in}{0.800199in}}{\pgfqpoint{4.650000in}{1.372727in}}%
\pgfusepath{clip}%
\pgfsetrectcap%
\pgfsetroundjoin%
\pgfsetlinewidth{0.010037pt}%
\definecolor{currentstroke}{rgb}{0.000000,0.000000,1.000000}%
\pgfsetstrokecolor{currentstroke}%
\pgfsetdash{}{0pt}%
\pgfpathmoveto{\pgfqpoint{0.575368in}{0.800528in}}%
\pgfpathlineto{\pgfqpoint{0.708511in}{0.802427in}}%
\pgfpathlineto{\pgfqpoint{0.775082in}{0.805203in}}%
\pgfpathlineto{\pgfqpoint{0.841653in}{0.810450in}}%
\pgfpathlineto{\pgfqpoint{0.908224in}{0.819481in}}%
\pgfpathlineto{\pgfqpoint{0.974796in}{0.833759in}}%
\pgfpathlineto{\pgfqpoint{1.008081in}{0.843307in}}%
\pgfpathlineto{\pgfqpoint{1.041367in}{0.854660in}}%
\pgfpathlineto{\pgfqpoint{1.074652in}{0.867921in}}%
\pgfpathlineto{\pgfqpoint{1.107938in}{0.883146in}}%
\pgfpathlineto{\pgfqpoint{1.141224in}{0.900320in}}%
\pgfpathlineto{\pgfqpoint{1.207795in}{0.940046in}}%
\pgfpathlineto{\pgfqpoint{1.274366in}{0.985184in}}%
\pgfpathlineto{\pgfqpoint{1.374223in}{1.055096in}}%
\pgfpathlineto{\pgfqpoint{1.407509in}{1.076627in}}%
\pgfpathlineto{\pgfqpoint{1.440794in}{1.096231in}}%
\pgfpathlineto{\pgfqpoint{1.474080in}{1.113343in}}%
\pgfpathlineto{\pgfqpoint{1.507365in}{1.127488in}}%
\pgfpathlineto{\pgfqpoint{1.540651in}{1.138332in}}%
\pgfpathlineto{\pgfqpoint{1.573937in}{1.145722in}}%
\pgfpathlineto{\pgfqpoint{1.607222in}{1.149736in}}%
\pgfpathlineto{\pgfqpoint{1.640508in}{1.150724in}}%
\pgfpathlineto{\pgfqpoint{1.707079in}{1.146575in}}%
\pgfpathlineto{\pgfqpoint{1.773650in}{1.142386in}}%
\pgfpathlineto{\pgfqpoint{1.806936in}{1.144362in}}%
\pgfpathlineto{\pgfqpoint{1.840221in}{1.151574in}}%
\pgfpathlineto{\pgfqpoint{1.873507in}{1.165915in}}%
\pgfpathlineto{\pgfqpoint{1.906793in}{1.189091in}}%
\pgfpathlineto{\pgfqpoint{1.940078in}{1.222443in}}%
\pgfpathlineto{\pgfqpoint{1.973364in}{1.266769in}}%
\pgfpathlineto{\pgfqpoint{2.006650in}{1.322175in}}%
\pgfpathlineto{\pgfqpoint{2.039935in}{1.387946in}}%
\pgfpathlineto{\pgfqpoint{2.073221in}{1.462484in}}%
\pgfpathlineto{\pgfqpoint{2.139792in}{1.627108in}}%
\pgfpathlineto{\pgfqpoint{2.173078in}{1.709941in}}%
\pgfpathlineto{\pgfqpoint{2.206363in}{1.787420in}}%
\pgfpathlineto{\pgfqpoint{2.239649in}{1.855024in}}%
\pgfpathlineto{\pgfqpoint{2.272934in}{1.908444in}}%
\pgfpathlineto{\pgfqpoint{2.306220in}{1.943946in}}%
\pgfpathlineto{\pgfqpoint{2.339506in}{1.958728in}}%
\pgfpathlineto{\pgfqpoint{2.372791in}{1.951224in}}%
\pgfpathlineto{\pgfqpoint{2.406077in}{1.921305in}}%
\pgfpathlineto{\pgfqpoint{2.439362in}{1.870358in}}%
\pgfpathlineto{\pgfqpoint{2.472648in}{1.801196in}}%
\pgfpathlineto{\pgfqpoint{2.505934in}{1.717821in}}%
\pgfpathlineto{\pgfqpoint{2.572505in}{1.528031in}}%
\pgfpathlineto{\pgfqpoint{2.605791in}{1.431828in}}%
\pgfpathlineto{\pgfqpoint{2.639076in}{1.340924in}}%
\pgfpathlineto{\pgfqpoint{2.672362in}{1.258927in}}%
\pgfpathlineto{\pgfqpoint{2.705647in}{1.188367in}}%
\pgfpathlineto{\pgfqpoint{2.738933in}{1.130643in}}%
\pgfpathlineto{\pgfqpoint{2.772219in}{1.086099in}}%
\pgfpathlineto{\pgfqpoint{2.805504in}{1.054187in}}%
\pgfpathlineto{\pgfqpoint{2.838790in}{1.033678in}}%
\pgfpathlineto{\pgfqpoint{2.872075in}{1.022890in}}%
\pgfpathlineto{\pgfqpoint{2.905361in}{1.019892in}}%
\pgfpathlineto{\pgfqpoint{2.938647in}{1.022666in}}%
\pgfpathlineto{\pgfqpoint{2.971932in}{1.029238in}}%
\pgfpathlineto{\pgfqpoint{3.071789in}{1.054203in}}%
\pgfpathlineto{\pgfqpoint{3.105075in}{1.059516in}}%
\pgfpathlineto{\pgfqpoint{3.138360in}{1.061610in}}%
\pgfpathlineto{\pgfqpoint{3.171646in}{1.059913in}}%
\pgfpathlineto{\pgfqpoint{3.204932in}{1.054179in}}%
\pgfpathlineto{\pgfqpoint{3.238217in}{1.044481in}}%
\pgfpathlineto{\pgfqpoint{3.271503in}{1.031186in}}%
\pgfpathlineto{\pgfqpoint{3.304788in}{1.014899in}}%
\pgfpathlineto{\pgfqpoint{3.371360in}{0.976521in}}%
\pgfpathlineto{\pgfqpoint{3.471216in}{0.916879in}}%
\pgfpathlineto{\pgfqpoint{3.504502in}{0.899143in}}%
\pgfpathlineto{\pgfqpoint{3.537788in}{0.883164in}}%
\pgfpathlineto{\pgfqpoint{3.571073in}{0.869108in}}%
\pgfpathlineto{\pgfqpoint{3.604359in}{0.857006in}}%
\pgfpathlineto{\pgfqpoint{3.637645in}{0.846780in}}%
\pgfpathlineto{\pgfqpoint{3.704216in}{0.831307in}}%
\pgfpathlineto{\pgfqpoint{3.770787in}{0.821084in}}%
\pgfpathlineto{\pgfqpoint{3.837358in}{0.814457in}}%
\pgfpathlineto{\pgfqpoint{3.937215in}{0.808473in}}%
\pgfpathlineto{\pgfqpoint{4.070357in}{0.804141in}}%
\pgfpathlineto{\pgfqpoint{4.270071in}{0.801257in}}%
\pgfpathlineto{\pgfqpoint{4.636213in}{0.800244in}}%
\pgfpathlineto{\pgfqpoint{4.802641in}{0.800392in}}%
\pgfpathlineto{\pgfqpoint{4.802641in}{0.800392in}}%
\pgfusepath{stroke}%
\end{pgfscope}%
\begin{pgfscope}%
\pgfpathrectangle{\pgfqpoint{0.364005in}{0.800199in}}{\pgfqpoint{4.650000in}{1.372727in}}%
\pgfusepath{clip}%
\pgfsetrectcap%
\pgfsetroundjoin%
\pgfsetlinewidth{1.505625pt}%
\definecolor{currentstroke}{rgb}{1.000000,0.000000,0.000000}%
\pgfsetstrokecolor{currentstroke}%
\pgfsetdash{}{0pt}%
\pgfpathmoveto{\pgfqpoint{0.575368in}{0.808577in}}%
\pgfpathlineto{\pgfqpoint{0.608654in}{0.813961in}}%
\pgfpathlineto{\pgfqpoint{0.641939in}{0.822171in}}%
\pgfpathlineto{\pgfqpoint{0.675225in}{0.834278in}}%
\pgfpathlineto{\pgfqpoint{0.708511in}{0.851539in}}%
\pgfpathlineto{\pgfqpoint{0.741796in}{0.875322in}}%
\pgfpathlineto{\pgfqpoint{0.775082in}{0.906971in}}%
\pgfpathlineto{\pgfqpoint{0.808368in}{0.947616in}}%
\pgfpathlineto{\pgfqpoint{0.841653in}{0.997935in}}%
\pgfpathlineto{\pgfqpoint{0.874939in}{1.057911in}}%
\pgfpathlineto{\pgfqpoint{0.908224in}{1.126611in}}%
\pgfpathlineto{\pgfqpoint{0.974796in}{1.281238in}}%
\pgfpathlineto{\pgfqpoint{1.008081in}{1.360253in}}%
\pgfpathlineto{\pgfqpoint{1.041367in}{1.434673in}}%
\pgfpathlineto{\pgfqpoint{1.074652in}{1.499998in}}%
\pgfpathlineto{\pgfqpoint{1.107938in}{1.552197in}}%
\pgfpathlineto{\pgfqpoint{1.141224in}{1.588214in}}%
\pgfpathlineto{\pgfqpoint{1.174509in}{1.606360in}}%
\pgfpathlineto{\pgfqpoint{1.207795in}{1.606510in}}%
\pgfpathlineto{\pgfqpoint{1.241080in}{1.590062in}}%
\pgfpathlineto{\pgfqpoint{1.274366in}{1.559685in}}%
\pgfpathlineto{\pgfqpoint{1.307652in}{1.518886in}}%
\pgfpathlineto{\pgfqpoint{1.340937in}{1.471506in}}%
\pgfpathlineto{\pgfqpoint{1.407509in}{1.371143in}}%
\pgfpathlineto{\pgfqpoint{1.440794in}{1.323570in}}%
\pgfpathlineto{\pgfqpoint{1.474080in}{1.279902in}}%
\pgfpathlineto{\pgfqpoint{1.507365in}{1.240719in}}%
\pgfpathlineto{\pgfqpoint{1.540651in}{1.205962in}}%
\pgfpathlineto{\pgfqpoint{1.573937in}{1.175186in}}%
\pgfpathlineto{\pgfqpoint{1.607222in}{1.147813in}}%
\pgfpathlineto{\pgfqpoint{1.640508in}{1.123354in}}%
\pgfpathlineto{\pgfqpoint{1.673793in}{1.101572in}}%
\pgfpathlineto{\pgfqpoint{1.707079in}{1.082568in}}%
\pgfpathlineto{\pgfqpoint{1.740365in}{1.066790in}}%
\pgfpathlineto{\pgfqpoint{1.773650in}{1.054979in}}%
\pgfpathlineto{\pgfqpoint{1.806936in}{1.048065in}}%
\pgfpathlineto{\pgfqpoint{1.840221in}{1.047028in}}%
\pgfpathlineto{\pgfqpoint{1.873507in}{1.052736in}}%
\pgfpathlineto{\pgfqpoint{1.906793in}{1.065795in}}%
\pgfpathlineto{\pgfqpoint{1.940078in}{1.086399in}}%
\pgfpathlineto{\pgfqpoint{1.973364in}{1.114226in}}%
\pgfpathlineto{\pgfqpoint{2.006650in}{1.148371in}}%
\pgfpathlineto{\pgfqpoint{2.039935in}{1.187345in}}%
\pgfpathlineto{\pgfqpoint{2.139792in}{1.311269in}}%
\pgfpathlineto{\pgfqpoint{2.173078in}{1.346370in}}%
\pgfpathlineto{\pgfqpoint{2.206363in}{1.374226in}}%
\pgfpathlineto{\pgfqpoint{2.239649in}{1.392892in}}%
\pgfpathlineto{\pgfqpoint{2.272934in}{1.401040in}}%
\pgfpathlineto{\pgfqpoint{2.306220in}{1.398057in}}%
\pgfpathlineto{\pgfqpoint{2.339506in}{1.384083in}}%
\pgfpathlineto{\pgfqpoint{2.372791in}{1.359964in}}%
\pgfpathlineto{\pgfqpoint{2.406077in}{1.327147in}}%
\pgfpathlineto{\pgfqpoint{2.439362in}{1.287533in}}%
\pgfpathlineto{\pgfqpoint{2.505934in}{1.196710in}}%
\pgfpathlineto{\pgfqpoint{2.572505in}{1.105067in}}%
\pgfpathlineto{\pgfqpoint{2.605791in}{1.063668in}}%
\pgfpathlineto{\pgfqpoint{2.639076in}{1.027071in}}%
\pgfpathlineto{\pgfqpoint{2.672362in}{0.996167in}}%
\pgfpathlineto{\pgfqpoint{2.705647in}{0.971459in}}%
\pgfpathlineto{\pgfqpoint{2.738933in}{0.953097in}}%
\pgfpathlineto{\pgfqpoint{2.772219in}{0.940926in}}%
\pgfpathlineto{\pgfqpoint{2.805504in}{0.934550in}}%
\pgfpathlineto{\pgfqpoint{2.838790in}{0.933386in}}%
\pgfpathlineto{\pgfqpoint{2.872075in}{0.936712in}}%
\pgfpathlineto{\pgfqpoint{2.905361in}{0.943708in}}%
\pgfpathlineto{\pgfqpoint{2.938647in}{0.953485in}}%
\pgfpathlineto{\pgfqpoint{3.005218in}{0.977638in}}%
\pgfpathlineto{\pgfqpoint{3.071789in}{1.001674in}}%
\pgfpathlineto{\pgfqpoint{3.105075in}{1.011487in}}%
\pgfpathlineto{\pgfqpoint{3.138360in}{1.018898in}}%
\pgfpathlineto{\pgfqpoint{3.171646in}{1.023426in}}%
\pgfpathlineto{\pgfqpoint{3.204932in}{1.024814in}}%
\pgfpathlineto{\pgfqpoint{3.238217in}{1.023042in}}%
\pgfpathlineto{\pgfqpoint{3.271503in}{1.018324in}}%
\pgfpathlineto{\pgfqpoint{3.304788in}{1.011074in}}%
\pgfpathlineto{\pgfqpoint{3.371360in}{0.991289in}}%
\pgfpathlineto{\pgfqpoint{3.504502in}{0.947477in}}%
\pgfpathlineto{\pgfqpoint{3.571073in}{0.929897in}}%
\pgfpathlineto{\pgfqpoint{3.637645in}{0.916126in}}%
\pgfpathlineto{\pgfqpoint{3.737501in}{0.899786in}}%
\pgfpathlineto{\pgfqpoint{3.870644in}{0.877579in}}%
\pgfpathlineto{\pgfqpoint{4.003786in}{0.850686in}}%
\pgfpathlineto{\pgfqpoint{4.103643in}{0.831483in}}%
\pgfpathlineto{\pgfqpoint{4.170214in}{0.821266in}}%
\pgfpathlineto{\pgfqpoint{4.236786in}{0.813628in}}%
\pgfpathlineto{\pgfqpoint{4.336642in}{0.806454in}}%
\pgfpathlineto{\pgfqpoint{4.436499in}{0.802837in}}%
\pgfpathlineto{\pgfqpoint{4.602927in}{0.800781in}}%
\pgfpathlineto{\pgfqpoint{4.736070in}{0.801857in}}%
\pgfpathlineto{\pgfqpoint{4.802641in}{0.805165in}}%
\pgfpathlineto{\pgfqpoint{4.802641in}{0.805165in}}%
\pgfusepath{stroke}%
\end{pgfscope}%
\begin{pgfscope}%
\pgfpathrectangle{\pgfqpoint{0.364005in}{0.800199in}}{\pgfqpoint{4.650000in}{1.372727in}}%
\pgfusepath{clip}%
\pgfsetrectcap%
\pgfsetroundjoin%
\pgfsetlinewidth{1.505625pt}%
\definecolor{currentstroke}{rgb}{0.000000,0.000000,1.000000}%
\pgfsetstrokecolor{currentstroke}%
\pgfsetdash{}{0pt}%
\pgfpathmoveto{\pgfqpoint{0.575368in}{0.800431in}}%
\pgfpathlineto{\pgfqpoint{0.741796in}{0.802520in}}%
\pgfpathlineto{\pgfqpoint{0.841653in}{0.807241in}}%
\pgfpathlineto{\pgfqpoint{0.908224in}{0.813506in}}%
\pgfpathlineto{\pgfqpoint{0.974796in}{0.823575in}}%
\pgfpathlineto{\pgfqpoint{1.041367in}{0.838660in}}%
\pgfpathlineto{\pgfqpoint{1.107938in}{0.859773in}}%
\pgfpathlineto{\pgfqpoint{1.174509in}{0.887268in}}%
\pgfpathlineto{\pgfqpoint{1.241080in}{0.920247in}}%
\pgfpathlineto{\pgfqpoint{1.374223in}{0.991010in}}%
\pgfpathlineto{\pgfqpoint{1.407509in}{1.006598in}}%
\pgfpathlineto{\pgfqpoint{1.440794in}{1.020252in}}%
\pgfpathlineto{\pgfqpoint{1.474080in}{1.031552in}}%
\pgfpathlineto{\pgfqpoint{1.507365in}{1.040227in}}%
\pgfpathlineto{\pgfqpoint{1.540651in}{1.046185in}}%
\pgfpathlineto{\pgfqpoint{1.573937in}{1.049548in}}%
\pgfpathlineto{\pgfqpoint{1.607222in}{1.050686in}}%
\pgfpathlineto{\pgfqpoint{1.740365in}{1.050024in}}%
\pgfpathlineto{\pgfqpoint{1.773650in}{1.055315in}}%
\pgfpathlineto{\pgfqpoint{1.806936in}{1.066342in}}%
\pgfpathlineto{\pgfqpoint{1.840221in}{1.085098in}}%
\pgfpathlineto{\pgfqpoint{1.873507in}{1.113488in}}%
\pgfpathlineto{\pgfqpoint{1.906793in}{1.153141in}}%
\pgfpathlineto{\pgfqpoint{1.940078in}{1.205206in}}%
\pgfpathlineto{\pgfqpoint{1.973364in}{1.270144in}}%
\pgfpathlineto{\pgfqpoint{2.006650in}{1.347554in}}%
\pgfpathlineto{\pgfqpoint{2.039935in}{1.436048in}}%
\pgfpathlineto{\pgfqpoint{2.073221in}{1.533206in}}%
\pgfpathlineto{\pgfqpoint{2.173078in}{1.838569in}}%
\pgfpathlineto{\pgfqpoint{2.206363in}{1.929065in}}%
\pgfpathlineto{\pgfqpoint{2.239649in}{2.005405in}}%
\pgfpathlineto{\pgfqpoint{2.272934in}{2.062935in}}%
\pgfpathlineto{\pgfqpoint{2.306220in}{2.097853in}}%
\pgfpathlineto{\pgfqpoint{2.339506in}{2.107559in}}%
\pgfpathlineto{\pgfqpoint{2.372791in}{2.090933in}}%
\pgfpathlineto{\pgfqpoint{2.406077in}{2.048497in}}%
\pgfpathlineto{\pgfqpoint{2.439362in}{1.982428in}}%
\pgfpathlineto{\pgfqpoint{2.472648in}{1.896406in}}%
\pgfpathlineto{\pgfqpoint{2.505934in}{1.795310in}}%
\pgfpathlineto{\pgfqpoint{2.639076in}{1.353786in}}%
\pgfpathlineto{\pgfqpoint{2.672362in}{1.259742in}}%
\pgfpathlineto{\pgfqpoint{2.705647in}{1.179265in}}%
\pgfpathlineto{\pgfqpoint{2.738933in}{1.113760in}}%
\pgfpathlineto{\pgfqpoint{2.772219in}{1.063492in}}%
\pgfpathlineto{\pgfqpoint{2.805504in}{1.027774in}}%
\pgfpathlineto{\pgfqpoint{2.838790in}{1.005216in}}%
\pgfpathlineto{\pgfqpoint{2.872075in}{0.993964in}}%
\pgfpathlineto{\pgfqpoint{2.905361in}{0.991914in}}%
\pgfpathlineto{\pgfqpoint{2.938647in}{0.996875in}}%
\pgfpathlineto{\pgfqpoint{2.971932in}{1.006685in}}%
\pgfpathlineto{\pgfqpoint{3.105075in}{1.055878in}}%
\pgfpathlineto{\pgfqpoint{3.138360in}{1.062861in}}%
\pgfpathlineto{\pgfqpoint{3.171646in}{1.065579in}}%
\pgfpathlineto{\pgfqpoint{3.204932in}{1.063572in}}%
\pgfpathlineto{\pgfqpoint{3.238217in}{1.056764in}}%
\pgfpathlineto{\pgfqpoint{3.271503in}{1.045442in}}%
\pgfpathlineto{\pgfqpoint{3.304788in}{1.030208in}}%
\pgfpathlineto{\pgfqpoint{3.338074in}{1.011898in}}%
\pgfpathlineto{\pgfqpoint{3.404645in}{0.969977in}}%
\pgfpathlineto{\pgfqpoint{3.471216in}{0.927341in}}%
\pgfpathlineto{\pgfqpoint{3.537788in}{0.889778in}}%
\pgfpathlineto{\pgfqpoint{3.571073in}{0.873908in}}%
\pgfpathlineto{\pgfqpoint{3.604359in}{0.860164in}}%
\pgfpathlineto{\pgfqpoint{3.637645in}{0.848512in}}%
\pgfpathlineto{\pgfqpoint{3.670930in}{0.838820in}}%
\pgfpathlineto{\pgfqpoint{3.737501in}{0.824507in}}%
\pgfpathlineto{\pgfqpoint{3.804073in}{0.815424in}}%
\pgfpathlineto{\pgfqpoint{3.870644in}{0.809855in}}%
\pgfpathlineto{\pgfqpoint{3.970501in}{0.805327in}}%
\pgfpathlineto{\pgfqpoint{4.136929in}{0.802329in}}%
\pgfpathlineto{\pgfqpoint{4.503070in}{0.800498in}}%
\pgfpathlineto{\pgfqpoint{4.802641in}{0.800340in}}%
\pgfpathlineto{\pgfqpoint{4.802641in}{0.800340in}}%
\pgfusepath{stroke}%
\end{pgfscope}%
\begin{pgfscope}%
\pgfsetrectcap%
\pgfsetmiterjoin%
\pgfsetlinewidth{0.803000pt}%
\definecolor{currentstroke}{rgb}{0.000000,0.000000,0.000000}%
\pgfsetstrokecolor{currentstroke}%
\pgfsetdash{}{0pt}%
\pgfpathmoveto{\pgfqpoint{0.364005in}{0.800199in}}%
\pgfpathlineto{\pgfqpoint{0.364005in}{2.172927in}}%
\pgfusepath{stroke}%
\end{pgfscope}%
\begin{pgfscope}%
\pgfsetrectcap%
\pgfsetmiterjoin%
\pgfsetlinewidth{0.803000pt}%
\definecolor{currentstroke}{rgb}{0.000000,0.000000,0.000000}%
\pgfsetstrokecolor{currentstroke}%
\pgfsetdash{}{0pt}%
\pgfpathmoveto{\pgfqpoint{5.014005in}{0.800199in}}%
\pgfpathlineto{\pgfqpoint{5.014005in}{2.172927in}}%
\pgfusepath{stroke}%
\end{pgfscope}%
\begin{pgfscope}%
\pgfsetrectcap%
\pgfsetmiterjoin%
\pgfsetlinewidth{0.803000pt}%
\definecolor{currentstroke}{rgb}{0.000000,0.000000,0.000000}%
\pgfsetstrokecolor{currentstroke}%
\pgfsetdash{}{0pt}%
\pgfpathmoveto{\pgfqpoint{0.364005in}{0.800199in}}%
\pgfpathlineto{\pgfqpoint{5.014005in}{0.800199in}}%
\pgfusepath{stroke}%
\end{pgfscope}%
\begin{pgfscope}%
\pgfsetrectcap%
\pgfsetmiterjoin%
\pgfsetlinewidth{0.803000pt}%
\definecolor{currentstroke}{rgb}{0.000000,0.000000,0.000000}%
\pgfsetstrokecolor{currentstroke}%
\pgfsetdash{}{0pt}%
\pgfpathmoveto{\pgfqpoint{0.364005in}{2.172927in}}%
\pgfpathlineto{\pgfqpoint{5.014005in}{2.172927in}}%
\pgfusepath{stroke}%
\end{pgfscope}%
\begin{pgfscope}%
\pgfsetbuttcap%
\pgfsetmiterjoin%
\definecolor{currentfill}{rgb}{0.300000,0.300000,0.300000}%
\pgfsetfillcolor{currentfill}%
\pgfsetfillopacity{0.500000}%
\pgfsetlinewidth{1.003750pt}%
\definecolor{currentstroke}{rgb}{0.300000,0.300000,0.300000}%
\pgfsetstrokecolor{currentstroke}%
\pgfsetstrokeopacity{0.500000}%
\pgfsetdash{}{0pt}%
\pgfpathmoveto{\pgfqpoint{1.289185in}{-0.027778in}}%
\pgfpathlineto{\pgfqpoint{4.144380in}{-0.027778in}}%
\pgfpathquadraticcurveto{\pgfqpoint{4.165586in}{-0.027778in}}{\pgfqpoint{4.165586in}{-0.006572in}}%
\pgfpathlineto{\pgfqpoint{4.165586in}{0.286384in}}%
\pgfpathquadraticcurveto{\pgfqpoint{4.165586in}{0.307589in}}{\pgfqpoint{4.144380in}{0.307589in}}%
\pgfpathlineto{\pgfqpoint{1.289185in}{0.307589in}}%
\pgfpathquadraticcurveto{\pgfqpoint{1.267979in}{0.307589in}}{\pgfqpoint{1.267979in}{0.286384in}}%
\pgfpathlineto{\pgfqpoint{1.267979in}{-0.006572in}}%
\pgfpathquadraticcurveto{\pgfqpoint{1.267979in}{-0.027778in}}{\pgfqpoint{1.289185in}{-0.027778in}}%
\pgfpathclose%
\pgfusepath{stroke,fill}%
\end{pgfscope}%
\begin{pgfscope}%
\pgfsetbuttcap%
\pgfsetmiterjoin%
\definecolor{currentfill}{rgb}{1.000000,1.000000,1.000000}%
\pgfsetfillcolor{currentfill}%
\pgfsetlinewidth{1.003750pt}%
\definecolor{currentstroke}{rgb}{0.800000,0.800000,0.800000}%
\pgfsetstrokecolor{currentstroke}%
\pgfsetdash{}{0pt}%
\pgfpathmoveto{\pgfqpoint{1.261407in}{0.000000in}}%
\pgfpathlineto{\pgfqpoint{4.116602in}{0.000000in}}%
\pgfpathquadraticcurveto{\pgfqpoint{4.137808in}{0.000000in}}{\pgfqpoint{4.137808in}{0.021206in}}%
\pgfpathlineto{\pgfqpoint{4.137808in}{0.314162in}}%
\pgfpathquadraticcurveto{\pgfqpoint{4.137808in}{0.335367in}}{\pgfqpoint{4.116602in}{0.335367in}}%
\pgfpathlineto{\pgfqpoint{1.261407in}{0.335367in}}%
\pgfpathquadraticcurveto{\pgfqpoint{1.240201in}{0.335367in}}{\pgfqpoint{1.240201in}{0.314162in}}%
\pgfpathlineto{\pgfqpoint{1.240201in}{0.021206in}}%
\pgfpathquadraticcurveto{\pgfqpoint{1.240201in}{0.000000in}}{\pgfqpoint{1.261407in}{0.000000in}}%
\pgfpathclose%
\pgfusepath{stroke,fill}%
\end{pgfscope}%
\begin{pgfscope}%
\pgfsetrectcap%
\pgfsetroundjoin%
\pgfsetlinewidth{0.010037pt}%
\definecolor{currentstroke}{rgb}{1.000000,0.000000,0.000000}%
\pgfsetstrokecolor{currentstroke}%
\pgfsetdash{}{0pt}%
\pgfpathmoveto{\pgfqpoint{1.282612in}{0.252905in}}%
\pgfpathlineto{\pgfqpoint{1.494668in}{0.252905in}}%
\pgfusepath{stroke}%
\end{pgfscope}%
\begin{pgfscope}%
\definecolor{textcolor}{rgb}{0.000000,0.000000,0.000000}%
\pgfsetstrokecolor{textcolor}%
\pgfsetfillcolor{textcolor}%
\pgftext[x=1.579490in,y=0.215796in,left,base]{\color{textcolor}\rmfamily\fontsize{7.634000}{9.160800}\selectfont Cohabitaition}%
\end{pgfscope}%
\begin{pgfscope}%
\pgfsetrectcap%
\pgfsetroundjoin%
\pgfsetlinewidth{0.010037pt}%
\definecolor{currentstroke}{rgb}{0.000000,0.000000,1.000000}%
\pgfsetstrokecolor{currentstroke}%
\pgfsetdash{}{0pt}%
\pgfpathmoveto{\pgfqpoint{1.282612in}{0.101126in}}%
\pgfpathlineto{\pgfqpoint{1.494668in}{0.101126in}}%
\pgfusepath{stroke}%
\end{pgfscope}%
\begin{pgfscope}%
\definecolor{textcolor}{rgb}{0.000000,0.000000,0.000000}%
\pgfsetstrokecolor{textcolor}%
\pgfsetfillcolor{textcolor}%
\pgftext[x=1.579490in,y=0.064016in,left,base]{\color{textcolor}\rmfamily\fontsize{7.634000}{9.160800}\selectfont Marriage}%
\end{pgfscope}%
\begin{pgfscope}%
\pgfsetrectcap%
\pgfsetroundjoin%
\pgfsetlinewidth{1.505625pt}%
\definecolor{currentstroke}{rgb}{1.000000,0.000000,0.000000}%
\pgfsetstrokecolor{currentstroke}%
\pgfsetdash{}{0pt}%
\pgfpathmoveto{\pgfqpoint{2.499889in}{0.252905in}}%
\pgfpathlineto{\pgfqpoint{2.711945in}{0.252905in}}%
\pgfusepath{stroke}%
\end{pgfscope}%
\begin{pgfscope}%
\definecolor{textcolor}{rgb}{0.000000,0.000000,0.000000}%
\pgfsetstrokecolor{textcolor}%
\pgfsetfillcolor{textcolor}%
\pgftext[x=2.796767in,y=0.215796in,left,base]{\color{textcolor}\rmfamily\fontsize{7.634000}{9.160800}\selectfont Cohabitaition at meeting}%
\end{pgfscope}%
\begin{pgfscope}%
\pgfsetrectcap%
\pgfsetroundjoin%
\pgfsetlinewidth{1.505625pt}%
\definecolor{currentstroke}{rgb}{0.000000,0.000000,1.000000}%
\pgfsetstrokecolor{currentstroke}%
\pgfsetdash{}{0pt}%
\pgfpathmoveto{\pgfqpoint{2.499889in}{0.101126in}}%
\pgfpathlineto{\pgfqpoint{2.711945in}{0.101126in}}%
\pgfusepath{stroke}%
\end{pgfscope}%
\begin{pgfscope}%
\definecolor{textcolor}{rgb}{0.000000,0.000000,0.000000}%
\pgfsetstrokecolor{textcolor}%
\pgfsetfillcolor{textcolor}%
\pgftext[x=2.796767in,y=0.064016in,left,base]{\color{textcolor}\rmfamily\fontsize{7.634000}{9.160800}\selectfont Marriage at meeting}%
\end{pgfscope}%
\end{pgfpicture}%
\makeatother%
\endgroup%
}
\begin{minipage}{0.99\textwidth} % choose width suitably

\hspace{30em}

{\footnotesize \textsc{Notes.} This figure depicts the distribution women Pareto weight for married and cohabiting couples. The thick solid lines represent the distribution of this variable in the period the couple decides to get together, while the lighter areas depict the distribution of $\theta$ considering every relationship duration.\par}
\end{minipage}
\end{figure}
\FloatBarrier
%%%%%%%%%%%%%%%%%%%%%%%%%%%%%%%%%%%%%%
%Symmetry in income and assets
%%%%%%%%%%%%%%%%%%%%%%%%%%%%%%%%%%%%%%
\subsection{Simulations-Symmetry}
\begin{figure}[ht]
\begin{center}
\caption{\\ Log Income and assets mean and variances by age---simulated data}
\label{fig:symmetry}

\begin{subfigure}{.49\textwidth}
\centering
% include first image
\caption{Average productivity}
\label{fig:sub-firs1t}
\scalebox{0.5}{%% Creator: Matplotlib, PGF backend
%%
%% To include the figure in your LaTeX document, write
%%   \input{<filename>.pgf}
%%
%% Make sure the required packages are loaded in your preamble
%%   \usepackage{pgf}
%%
%% and, on pdftex
%%   \usepackage[utf8]{inputenc}\DeclareUnicodeCharacter{2212}{-}
%%
%% or, on luatex and xetex
%%   \usepackage{unicode-math}
%%
%% Figures using additional raster images can only be included by \input if
%% they are in the same directory as the main LaTeX file. For loading figures
%% from other directories you can use the `import` package
%%   \usepackage{import}
%%
%% and then include the figures with
%%   \import{<path to file>}{<filename>.pgf}
%%
%% Matplotlib used the following preamble
%%
\begingroup%
\makeatletter%
\begin{pgfpicture}%
\pgfpathrectangle{\pgfpointorigin}{\pgfqpoint{5.757863in}{4.222999in}}%
\pgfusepath{use as bounding box, clip}%
\begin{pgfscope}%
\pgfsetbuttcap%
\pgfsetmiterjoin%
\definecolor{currentfill}{rgb}{1.000000,1.000000,1.000000}%
\pgfsetfillcolor{currentfill}%
\pgfsetlinewidth{0.000000pt}%
\definecolor{currentstroke}{rgb}{1.000000,1.000000,1.000000}%
\pgfsetstrokecolor{currentstroke}%
\pgfsetdash{}{0pt}%
\pgfpathmoveto{\pgfqpoint{0.000000in}{0.000000in}}%
\pgfpathlineto{\pgfqpoint{5.757863in}{0.000000in}}%
\pgfpathlineto{\pgfqpoint{5.757863in}{4.222999in}}%
\pgfpathlineto{\pgfqpoint{0.000000in}{4.222999in}}%
\pgfpathclose%
\pgfusepath{fill}%
\end{pgfscope}%
\begin{pgfscope}%
\pgfsetbuttcap%
\pgfsetmiterjoin%
\definecolor{currentfill}{rgb}{1.000000,1.000000,1.000000}%
\pgfsetfillcolor{currentfill}%
\pgfsetlinewidth{0.000000pt}%
\definecolor{currentstroke}{rgb}{0.000000,0.000000,0.000000}%
\pgfsetstrokecolor{currentstroke}%
\pgfsetstrokeopacity{0.000000}%
\pgfsetdash{}{0pt}%
\pgfpathmoveto{\pgfqpoint{0.754784in}{1.202999in}}%
\pgfpathlineto{\pgfqpoint{5.404784in}{1.202999in}}%
\pgfpathlineto{\pgfqpoint{5.404784in}{4.222999in}}%
\pgfpathlineto{\pgfqpoint{0.754784in}{4.222999in}}%
\pgfpathclose%
\pgfusepath{fill}%
\end{pgfscope}%
\begin{pgfscope}%
\pgfsetbuttcap%
\pgfsetroundjoin%
\definecolor{currentfill}{rgb}{0.000000,0.000000,0.000000}%
\pgfsetfillcolor{currentfill}%
\pgfsetlinewidth{0.803000pt}%
\definecolor{currentstroke}{rgb}{0.000000,0.000000,0.000000}%
\pgfsetstrokecolor{currentstroke}%
\pgfsetdash{}{0pt}%
\pgfsys@defobject{currentmarker}{\pgfqpoint{0.000000in}{-0.048611in}}{\pgfqpoint{0.000000in}{0.000000in}}{%
\pgfpathmoveto{\pgfqpoint{0.000000in}{0.000000in}}%
\pgfpathlineto{\pgfqpoint{0.000000in}{-0.048611in}}%
\pgfusepath{stroke,fill}%
}%
\begin{pgfscope}%
\pgfsys@transformshift{0.966148in}{1.202999in}%
\pgfsys@useobject{currentmarker}{}%
\end{pgfscope}%
\end{pgfscope}%
\begin{pgfscope}%
\definecolor{textcolor}{rgb}{0.000000,0.000000,0.000000}%
\pgfsetstrokecolor{textcolor}%
\pgfsetfillcolor{textcolor}%
\pgftext[x=0.966148in,y=1.105777in,,top]{\color{textcolor}\rmfamily\fontsize{11.000000}{13.200000}\selectfont \(\displaystyle {25}\)}%
\end{pgfscope}%
\begin{pgfscope}%
\pgfsetbuttcap%
\pgfsetroundjoin%
\definecolor{currentfill}{rgb}{0.000000,0.000000,0.000000}%
\pgfsetfillcolor{currentfill}%
\pgfsetlinewidth{0.803000pt}%
\definecolor{currentstroke}{rgb}{0.000000,0.000000,0.000000}%
\pgfsetstrokecolor{currentstroke}%
\pgfsetdash{}{0pt}%
\pgfsys@defobject{currentmarker}{\pgfqpoint{0.000000in}{-0.048611in}}{\pgfqpoint{0.000000in}{0.000000in}}{%
\pgfpathmoveto{\pgfqpoint{0.000000in}{0.000000in}}%
\pgfpathlineto{\pgfqpoint{0.000000in}{-0.048611in}}%
\pgfusepath{stroke,fill}%
}%
\begin{pgfscope}%
\pgfsys@transformshift{1.587806in}{1.202999in}%
\pgfsys@useobject{currentmarker}{}%
\end{pgfscope}%
\end{pgfscope}%
\begin{pgfscope}%
\definecolor{textcolor}{rgb}{0.000000,0.000000,0.000000}%
\pgfsetstrokecolor{textcolor}%
\pgfsetfillcolor{textcolor}%
\pgftext[x=1.587806in,y=1.105777in,,top]{\color{textcolor}\rmfamily\fontsize{11.000000}{13.200000}\selectfont \(\displaystyle {30}\)}%
\end{pgfscope}%
\begin{pgfscope}%
\pgfsetbuttcap%
\pgfsetroundjoin%
\definecolor{currentfill}{rgb}{0.000000,0.000000,0.000000}%
\pgfsetfillcolor{currentfill}%
\pgfsetlinewidth{0.803000pt}%
\definecolor{currentstroke}{rgb}{0.000000,0.000000,0.000000}%
\pgfsetstrokecolor{currentstroke}%
\pgfsetdash{}{0pt}%
\pgfsys@defobject{currentmarker}{\pgfqpoint{0.000000in}{-0.048611in}}{\pgfqpoint{0.000000in}{0.000000in}}{%
\pgfpathmoveto{\pgfqpoint{0.000000in}{0.000000in}}%
\pgfpathlineto{\pgfqpoint{0.000000in}{-0.048611in}}%
\pgfusepath{stroke,fill}%
}%
\begin{pgfscope}%
\pgfsys@transformshift{2.209464in}{1.202999in}%
\pgfsys@useobject{currentmarker}{}%
\end{pgfscope}%
\end{pgfscope}%
\begin{pgfscope}%
\definecolor{textcolor}{rgb}{0.000000,0.000000,0.000000}%
\pgfsetstrokecolor{textcolor}%
\pgfsetfillcolor{textcolor}%
\pgftext[x=2.209464in,y=1.105777in,,top]{\color{textcolor}\rmfamily\fontsize{11.000000}{13.200000}\selectfont \(\displaystyle {35}\)}%
\end{pgfscope}%
\begin{pgfscope}%
\pgfsetbuttcap%
\pgfsetroundjoin%
\definecolor{currentfill}{rgb}{0.000000,0.000000,0.000000}%
\pgfsetfillcolor{currentfill}%
\pgfsetlinewidth{0.803000pt}%
\definecolor{currentstroke}{rgb}{0.000000,0.000000,0.000000}%
\pgfsetstrokecolor{currentstroke}%
\pgfsetdash{}{0pt}%
\pgfsys@defobject{currentmarker}{\pgfqpoint{0.000000in}{-0.048611in}}{\pgfqpoint{0.000000in}{0.000000in}}{%
\pgfpathmoveto{\pgfqpoint{0.000000in}{0.000000in}}%
\pgfpathlineto{\pgfqpoint{0.000000in}{-0.048611in}}%
\pgfusepath{stroke,fill}%
}%
\begin{pgfscope}%
\pgfsys@transformshift{2.831121in}{1.202999in}%
\pgfsys@useobject{currentmarker}{}%
\end{pgfscope}%
\end{pgfscope}%
\begin{pgfscope}%
\definecolor{textcolor}{rgb}{0.000000,0.000000,0.000000}%
\pgfsetstrokecolor{textcolor}%
\pgfsetfillcolor{textcolor}%
\pgftext[x=2.831121in,y=1.105777in,,top]{\color{textcolor}\rmfamily\fontsize{11.000000}{13.200000}\selectfont \(\displaystyle {40}\)}%
\end{pgfscope}%
\begin{pgfscope}%
\pgfsetbuttcap%
\pgfsetroundjoin%
\definecolor{currentfill}{rgb}{0.000000,0.000000,0.000000}%
\pgfsetfillcolor{currentfill}%
\pgfsetlinewidth{0.803000pt}%
\definecolor{currentstroke}{rgb}{0.000000,0.000000,0.000000}%
\pgfsetstrokecolor{currentstroke}%
\pgfsetdash{}{0pt}%
\pgfsys@defobject{currentmarker}{\pgfqpoint{0.000000in}{-0.048611in}}{\pgfqpoint{0.000000in}{0.000000in}}{%
\pgfpathmoveto{\pgfqpoint{0.000000in}{0.000000in}}%
\pgfpathlineto{\pgfqpoint{0.000000in}{-0.048611in}}%
\pgfusepath{stroke,fill}%
}%
\begin{pgfscope}%
\pgfsys@transformshift{3.452779in}{1.202999in}%
\pgfsys@useobject{currentmarker}{}%
\end{pgfscope}%
\end{pgfscope}%
\begin{pgfscope}%
\definecolor{textcolor}{rgb}{0.000000,0.000000,0.000000}%
\pgfsetstrokecolor{textcolor}%
\pgfsetfillcolor{textcolor}%
\pgftext[x=3.452779in,y=1.105777in,,top]{\color{textcolor}\rmfamily\fontsize{11.000000}{13.200000}\selectfont \(\displaystyle {45}\)}%
\end{pgfscope}%
\begin{pgfscope}%
\pgfsetbuttcap%
\pgfsetroundjoin%
\definecolor{currentfill}{rgb}{0.000000,0.000000,0.000000}%
\pgfsetfillcolor{currentfill}%
\pgfsetlinewidth{0.803000pt}%
\definecolor{currentstroke}{rgb}{0.000000,0.000000,0.000000}%
\pgfsetstrokecolor{currentstroke}%
\pgfsetdash{}{0pt}%
\pgfsys@defobject{currentmarker}{\pgfqpoint{0.000000in}{-0.048611in}}{\pgfqpoint{0.000000in}{0.000000in}}{%
\pgfpathmoveto{\pgfqpoint{0.000000in}{0.000000in}}%
\pgfpathlineto{\pgfqpoint{0.000000in}{-0.048611in}}%
\pgfusepath{stroke,fill}%
}%
\begin{pgfscope}%
\pgfsys@transformshift{4.074437in}{1.202999in}%
\pgfsys@useobject{currentmarker}{}%
\end{pgfscope}%
\end{pgfscope}%
\begin{pgfscope}%
\definecolor{textcolor}{rgb}{0.000000,0.000000,0.000000}%
\pgfsetstrokecolor{textcolor}%
\pgfsetfillcolor{textcolor}%
\pgftext[x=4.074437in,y=1.105777in,,top]{\color{textcolor}\rmfamily\fontsize{11.000000}{13.200000}\selectfont \(\displaystyle {50}\)}%
\end{pgfscope}%
\begin{pgfscope}%
\pgfsetbuttcap%
\pgfsetroundjoin%
\definecolor{currentfill}{rgb}{0.000000,0.000000,0.000000}%
\pgfsetfillcolor{currentfill}%
\pgfsetlinewidth{0.803000pt}%
\definecolor{currentstroke}{rgb}{0.000000,0.000000,0.000000}%
\pgfsetstrokecolor{currentstroke}%
\pgfsetdash{}{0pt}%
\pgfsys@defobject{currentmarker}{\pgfqpoint{0.000000in}{-0.048611in}}{\pgfqpoint{0.000000in}{0.000000in}}{%
\pgfpathmoveto{\pgfqpoint{0.000000in}{0.000000in}}%
\pgfpathlineto{\pgfqpoint{0.000000in}{-0.048611in}}%
\pgfusepath{stroke,fill}%
}%
\begin{pgfscope}%
\pgfsys@transformshift{4.696095in}{1.202999in}%
\pgfsys@useobject{currentmarker}{}%
\end{pgfscope}%
\end{pgfscope}%
\begin{pgfscope}%
\definecolor{textcolor}{rgb}{0.000000,0.000000,0.000000}%
\pgfsetstrokecolor{textcolor}%
\pgfsetfillcolor{textcolor}%
\pgftext[x=4.696095in,y=1.105777in,,top]{\color{textcolor}\rmfamily\fontsize{11.000000}{13.200000}\selectfont \(\displaystyle {55}\)}%
\end{pgfscope}%
\begin{pgfscope}%
\pgfsetbuttcap%
\pgfsetroundjoin%
\definecolor{currentfill}{rgb}{0.000000,0.000000,0.000000}%
\pgfsetfillcolor{currentfill}%
\pgfsetlinewidth{0.803000pt}%
\definecolor{currentstroke}{rgb}{0.000000,0.000000,0.000000}%
\pgfsetstrokecolor{currentstroke}%
\pgfsetdash{}{0pt}%
\pgfsys@defobject{currentmarker}{\pgfqpoint{0.000000in}{-0.048611in}}{\pgfqpoint{0.000000in}{0.000000in}}{%
\pgfpathmoveto{\pgfqpoint{0.000000in}{0.000000in}}%
\pgfpathlineto{\pgfqpoint{0.000000in}{-0.048611in}}%
\pgfusepath{stroke,fill}%
}%
\begin{pgfscope}%
\pgfsys@transformshift{5.317752in}{1.202999in}%
\pgfsys@useobject{currentmarker}{}%
\end{pgfscope}%
\end{pgfscope}%
\begin{pgfscope}%
\definecolor{textcolor}{rgb}{0.000000,0.000000,0.000000}%
\pgfsetstrokecolor{textcolor}%
\pgfsetfillcolor{textcolor}%
\pgftext[x=5.317752in,y=1.105777in,,top]{\color{textcolor}\rmfamily\fontsize{11.000000}{13.200000}\selectfont \(\displaystyle {60}\)}%
\end{pgfscope}%
\begin{pgfscope}%
\definecolor{textcolor}{rgb}{0.000000,0.000000,0.000000}%
\pgfsetstrokecolor{textcolor}%
\pgfsetfillcolor{textcolor}%
\pgftext[x=3.079784in,y=0.915037in,,top]{\color{textcolor}\rmfamily\fontsize{16.000000}{19.200000}\selectfont Age}%
\end{pgfscope}%
\begin{pgfscope}%
\pgfsetbuttcap%
\pgfsetroundjoin%
\definecolor{currentfill}{rgb}{0.000000,0.000000,0.000000}%
\pgfsetfillcolor{currentfill}%
\pgfsetlinewidth{0.803000pt}%
\definecolor{currentstroke}{rgb}{0.000000,0.000000,0.000000}%
\pgfsetstrokecolor{currentstroke}%
\pgfsetdash{}{0pt}%
\pgfsys@defobject{currentmarker}{\pgfqpoint{-0.048611in}{0.000000in}}{\pgfqpoint{0.000000in}{0.000000in}}{%
\pgfpathmoveto{\pgfqpoint{0.000000in}{0.000000in}}%
\pgfpathlineto{\pgfqpoint{-0.048611in}{0.000000in}}%
\pgfusepath{stroke,fill}%
}%
\begin{pgfscope}%
\pgfsys@transformshift{0.754784in}{1.336494in}%
\pgfsys@useobject{currentmarker}{}%
\end{pgfscope}%
\end{pgfscope}%
\begin{pgfscope}%
\definecolor{textcolor}{rgb}{0.000000,0.000000,0.000000}%
\pgfsetstrokecolor{textcolor}%
\pgfsetfillcolor{textcolor}%
\pgftext[x=0.268904in, y=1.283687in, left, base]{\color{textcolor}\rmfamily\fontsize{11.000000}{13.200000}\selectfont \(\displaystyle {-1.75}\)}%
\end{pgfscope}%
\begin{pgfscope}%
\pgfsetbuttcap%
\pgfsetroundjoin%
\definecolor{currentfill}{rgb}{0.000000,0.000000,0.000000}%
\pgfsetfillcolor{currentfill}%
\pgfsetlinewidth{0.803000pt}%
\definecolor{currentstroke}{rgb}{0.000000,0.000000,0.000000}%
\pgfsetstrokecolor{currentstroke}%
\pgfsetdash{}{0pt}%
\pgfsys@defobject{currentmarker}{\pgfqpoint{-0.048611in}{0.000000in}}{\pgfqpoint{0.000000in}{0.000000in}}{%
\pgfpathmoveto{\pgfqpoint{0.000000in}{0.000000in}}%
\pgfpathlineto{\pgfqpoint{-0.048611in}{0.000000in}}%
\pgfusepath{stroke,fill}%
}%
\begin{pgfscope}%
\pgfsys@transformshift{0.754784in}{1.678455in}%
\pgfsys@useobject{currentmarker}{}%
\end{pgfscope}%
\end{pgfscope}%
\begin{pgfscope}%
\definecolor{textcolor}{rgb}{0.000000,0.000000,0.000000}%
\pgfsetstrokecolor{textcolor}%
\pgfsetfillcolor{textcolor}%
\pgftext[x=0.268904in, y=1.625649in, left, base]{\color{textcolor}\rmfamily\fontsize{11.000000}{13.200000}\selectfont \(\displaystyle {-1.50}\)}%
\end{pgfscope}%
\begin{pgfscope}%
\pgfsetbuttcap%
\pgfsetroundjoin%
\definecolor{currentfill}{rgb}{0.000000,0.000000,0.000000}%
\pgfsetfillcolor{currentfill}%
\pgfsetlinewidth{0.803000pt}%
\definecolor{currentstroke}{rgb}{0.000000,0.000000,0.000000}%
\pgfsetstrokecolor{currentstroke}%
\pgfsetdash{}{0pt}%
\pgfsys@defobject{currentmarker}{\pgfqpoint{-0.048611in}{0.000000in}}{\pgfqpoint{0.000000in}{0.000000in}}{%
\pgfpathmoveto{\pgfqpoint{0.000000in}{0.000000in}}%
\pgfpathlineto{\pgfqpoint{-0.048611in}{0.000000in}}%
\pgfusepath{stroke,fill}%
}%
\begin{pgfscope}%
\pgfsys@transformshift{0.754784in}{2.020417in}%
\pgfsys@useobject{currentmarker}{}%
\end{pgfscope}%
\end{pgfscope}%
\begin{pgfscope}%
\definecolor{textcolor}{rgb}{0.000000,0.000000,0.000000}%
\pgfsetstrokecolor{textcolor}%
\pgfsetfillcolor{textcolor}%
\pgftext[x=0.268904in, y=1.967610in, left, base]{\color{textcolor}\rmfamily\fontsize{11.000000}{13.200000}\selectfont \(\displaystyle {-1.25}\)}%
\end{pgfscope}%
\begin{pgfscope}%
\pgfsetbuttcap%
\pgfsetroundjoin%
\definecolor{currentfill}{rgb}{0.000000,0.000000,0.000000}%
\pgfsetfillcolor{currentfill}%
\pgfsetlinewidth{0.803000pt}%
\definecolor{currentstroke}{rgb}{0.000000,0.000000,0.000000}%
\pgfsetstrokecolor{currentstroke}%
\pgfsetdash{}{0pt}%
\pgfsys@defobject{currentmarker}{\pgfqpoint{-0.048611in}{0.000000in}}{\pgfqpoint{0.000000in}{0.000000in}}{%
\pgfpathmoveto{\pgfqpoint{0.000000in}{0.000000in}}%
\pgfpathlineto{\pgfqpoint{-0.048611in}{0.000000in}}%
\pgfusepath{stroke,fill}%
}%
\begin{pgfscope}%
\pgfsys@transformshift{0.754784in}{2.362378in}%
\pgfsys@useobject{currentmarker}{}%
\end{pgfscope}%
\end{pgfscope}%
\begin{pgfscope}%
\definecolor{textcolor}{rgb}{0.000000,0.000000,0.000000}%
\pgfsetstrokecolor{textcolor}%
\pgfsetfillcolor{textcolor}%
\pgftext[x=0.268904in, y=2.309572in, left, base]{\color{textcolor}\rmfamily\fontsize{11.000000}{13.200000}\selectfont \(\displaystyle {-1.00}\)}%
\end{pgfscope}%
\begin{pgfscope}%
\pgfsetbuttcap%
\pgfsetroundjoin%
\definecolor{currentfill}{rgb}{0.000000,0.000000,0.000000}%
\pgfsetfillcolor{currentfill}%
\pgfsetlinewidth{0.803000pt}%
\definecolor{currentstroke}{rgb}{0.000000,0.000000,0.000000}%
\pgfsetstrokecolor{currentstroke}%
\pgfsetdash{}{0pt}%
\pgfsys@defobject{currentmarker}{\pgfqpoint{-0.048611in}{0.000000in}}{\pgfqpoint{0.000000in}{0.000000in}}{%
\pgfpathmoveto{\pgfqpoint{0.000000in}{0.000000in}}%
\pgfpathlineto{\pgfqpoint{-0.048611in}{0.000000in}}%
\pgfusepath{stroke,fill}%
}%
\begin{pgfscope}%
\pgfsys@transformshift{0.754784in}{2.704340in}%
\pgfsys@useobject{currentmarker}{}%
\end{pgfscope}%
\end{pgfscope}%
\begin{pgfscope}%
\definecolor{textcolor}{rgb}{0.000000,0.000000,0.000000}%
\pgfsetstrokecolor{textcolor}%
\pgfsetfillcolor{textcolor}%
\pgftext[x=0.268904in, y=2.651533in, left, base]{\color{textcolor}\rmfamily\fontsize{11.000000}{13.200000}\selectfont \(\displaystyle {-0.75}\)}%
\end{pgfscope}%
\begin{pgfscope}%
\pgfsetbuttcap%
\pgfsetroundjoin%
\definecolor{currentfill}{rgb}{0.000000,0.000000,0.000000}%
\pgfsetfillcolor{currentfill}%
\pgfsetlinewidth{0.803000pt}%
\definecolor{currentstroke}{rgb}{0.000000,0.000000,0.000000}%
\pgfsetstrokecolor{currentstroke}%
\pgfsetdash{}{0pt}%
\pgfsys@defobject{currentmarker}{\pgfqpoint{-0.048611in}{0.000000in}}{\pgfqpoint{0.000000in}{0.000000in}}{%
\pgfpathmoveto{\pgfqpoint{0.000000in}{0.000000in}}%
\pgfpathlineto{\pgfqpoint{-0.048611in}{0.000000in}}%
\pgfusepath{stroke,fill}%
}%
\begin{pgfscope}%
\pgfsys@transformshift{0.754784in}{3.046301in}%
\pgfsys@useobject{currentmarker}{}%
\end{pgfscope}%
\end{pgfscope}%
\begin{pgfscope}%
\definecolor{textcolor}{rgb}{0.000000,0.000000,0.000000}%
\pgfsetstrokecolor{textcolor}%
\pgfsetfillcolor{textcolor}%
\pgftext[x=0.268904in, y=2.993495in, left, base]{\color{textcolor}\rmfamily\fontsize{11.000000}{13.200000}\selectfont \(\displaystyle {-0.50}\)}%
\end{pgfscope}%
\begin{pgfscope}%
\pgfsetbuttcap%
\pgfsetroundjoin%
\definecolor{currentfill}{rgb}{0.000000,0.000000,0.000000}%
\pgfsetfillcolor{currentfill}%
\pgfsetlinewidth{0.803000pt}%
\definecolor{currentstroke}{rgb}{0.000000,0.000000,0.000000}%
\pgfsetstrokecolor{currentstroke}%
\pgfsetdash{}{0pt}%
\pgfsys@defobject{currentmarker}{\pgfqpoint{-0.048611in}{0.000000in}}{\pgfqpoint{0.000000in}{0.000000in}}{%
\pgfpathmoveto{\pgfqpoint{0.000000in}{0.000000in}}%
\pgfpathlineto{\pgfqpoint{-0.048611in}{0.000000in}}%
\pgfusepath{stroke,fill}%
}%
\begin{pgfscope}%
\pgfsys@transformshift{0.754784in}{3.388263in}%
\pgfsys@useobject{currentmarker}{}%
\end{pgfscope}%
\end{pgfscope}%
\begin{pgfscope}%
\definecolor{textcolor}{rgb}{0.000000,0.000000,0.000000}%
\pgfsetstrokecolor{textcolor}%
\pgfsetfillcolor{textcolor}%
\pgftext[x=0.268904in, y=3.335456in, left, base]{\color{textcolor}\rmfamily\fontsize{11.000000}{13.200000}\selectfont \(\displaystyle {-0.25}\)}%
\end{pgfscope}%
\begin{pgfscope}%
\pgfsetbuttcap%
\pgfsetroundjoin%
\definecolor{currentfill}{rgb}{0.000000,0.000000,0.000000}%
\pgfsetfillcolor{currentfill}%
\pgfsetlinewidth{0.803000pt}%
\definecolor{currentstroke}{rgb}{0.000000,0.000000,0.000000}%
\pgfsetstrokecolor{currentstroke}%
\pgfsetdash{}{0pt}%
\pgfsys@defobject{currentmarker}{\pgfqpoint{-0.048611in}{0.000000in}}{\pgfqpoint{0.000000in}{0.000000in}}{%
\pgfpathmoveto{\pgfqpoint{0.000000in}{0.000000in}}%
\pgfpathlineto{\pgfqpoint{-0.048611in}{0.000000in}}%
\pgfusepath{stroke,fill}%
}%
\begin{pgfscope}%
\pgfsys@transformshift{0.754784in}{3.730224in}%
\pgfsys@useobject{currentmarker}{}%
\end{pgfscope}%
\end{pgfscope}%
\begin{pgfscope}%
\definecolor{textcolor}{rgb}{0.000000,0.000000,0.000000}%
\pgfsetstrokecolor{textcolor}%
\pgfsetfillcolor{textcolor}%
\pgftext[x=0.387192in, y=3.677417in, left, base]{\color{textcolor}\rmfamily\fontsize{11.000000}{13.200000}\selectfont \(\displaystyle {0.00}\)}%
\end{pgfscope}%
\begin{pgfscope}%
\pgfsetbuttcap%
\pgfsetroundjoin%
\definecolor{currentfill}{rgb}{0.000000,0.000000,0.000000}%
\pgfsetfillcolor{currentfill}%
\pgfsetlinewidth{0.803000pt}%
\definecolor{currentstroke}{rgb}{0.000000,0.000000,0.000000}%
\pgfsetstrokecolor{currentstroke}%
\pgfsetdash{}{0pt}%
\pgfsys@defobject{currentmarker}{\pgfqpoint{-0.048611in}{0.000000in}}{\pgfqpoint{0.000000in}{0.000000in}}{%
\pgfpathmoveto{\pgfqpoint{0.000000in}{0.000000in}}%
\pgfpathlineto{\pgfqpoint{-0.048611in}{0.000000in}}%
\pgfusepath{stroke,fill}%
}%
\begin{pgfscope}%
\pgfsys@transformshift{0.754784in}{4.072186in}%
\pgfsys@useobject{currentmarker}{}%
\end{pgfscope}%
\end{pgfscope}%
\begin{pgfscope}%
\definecolor{textcolor}{rgb}{0.000000,0.000000,0.000000}%
\pgfsetstrokecolor{textcolor}%
\pgfsetfillcolor{textcolor}%
\pgftext[x=0.387192in, y=4.019379in, left, base]{\color{textcolor}\rmfamily\fontsize{11.000000}{13.200000}\selectfont \(\displaystyle {0.25}\)}%
\end{pgfscope}%
\begin{pgfscope}%
\definecolor{textcolor}{rgb}{0.000000,0.000000,0.000000}%
\pgfsetstrokecolor{textcolor}%
\pgfsetfillcolor{textcolor}%
\pgftext[x=0.213349in,y=2.712999in,,bottom,rotate=90.000000]{\color{textcolor}\rmfamily\fontsize{16.000000}{19.200000}\selectfont Potential log wages---mean}%
\end{pgfscope}%
\begin{pgfscope}%
\pgfpathrectangle{\pgfqpoint{0.754784in}{1.202999in}}{\pgfqpoint{4.650000in}{3.020000in}}%
\pgfusepath{clip}%
\pgfsetrectcap%
\pgfsetroundjoin%
\pgfsetlinewidth{1.505625pt}%
\definecolor{currentstroke}{rgb}{0.000000,0.000000,1.000000}%
\pgfsetstrokecolor{currentstroke}%
\pgfsetdash{}{0pt}%
\pgfpathmoveto{\pgfqpoint{0.966148in}{2.935102in}}%
\pgfpathlineto{\pgfqpoint{1.090480in}{2.955085in}}%
\pgfpathlineto{\pgfqpoint{1.214811in}{2.970472in}}%
\pgfpathlineto{\pgfqpoint{1.339143in}{2.967288in}}%
\pgfpathlineto{\pgfqpoint{1.463474in}{2.972229in}}%
\pgfpathlineto{\pgfqpoint{1.587806in}{2.971782in}}%
\pgfpathlineto{\pgfqpoint{1.712137in}{2.966277in}}%
\pgfpathlineto{\pgfqpoint{1.836469in}{2.938453in}}%
\pgfpathlineto{\pgfqpoint{1.960800in}{2.916173in}}%
\pgfpathlineto{\pgfqpoint{2.085132in}{2.885758in}}%
\pgfpathlineto{\pgfqpoint{2.209464in}{2.848205in}}%
\pgfpathlineto{\pgfqpoint{2.333795in}{2.806040in}}%
\pgfpathlineto{\pgfqpoint{2.458127in}{2.763913in}}%
\pgfpathlineto{\pgfqpoint{2.582458in}{2.707813in}}%
\pgfpathlineto{\pgfqpoint{2.706790in}{2.658960in}}%
\pgfpathlineto{\pgfqpoint{2.831121in}{2.610493in}}%
\pgfpathlineto{\pgfqpoint{2.955453in}{2.544162in}}%
\pgfpathlineto{\pgfqpoint{3.079784in}{2.478857in}}%
\pgfpathlineto{\pgfqpoint{3.204116in}{2.409612in}}%
\pgfpathlineto{\pgfqpoint{3.328448in}{2.344180in}}%
\pgfpathlineto{\pgfqpoint{3.452779in}{2.283532in}}%
\pgfpathlineto{\pgfqpoint{3.577111in}{2.208334in}}%
\pgfpathlineto{\pgfqpoint{3.701442in}{2.124526in}}%
\pgfpathlineto{\pgfqpoint{3.825774in}{2.056525in}}%
\pgfpathlineto{\pgfqpoint{3.950105in}{1.979578in}}%
\pgfpathlineto{\pgfqpoint{4.074437in}{1.912666in}}%
\pgfpathlineto{\pgfqpoint{4.198768in}{1.844021in}}%
\pgfpathlineto{\pgfqpoint{4.323100in}{1.791314in}}%
\pgfpathlineto{\pgfqpoint{4.447432in}{1.727813in}}%
\pgfpathlineto{\pgfqpoint{4.571763in}{1.693648in}}%
\pgfpathlineto{\pgfqpoint{4.696095in}{1.643360in}}%
\pgfpathlineto{\pgfqpoint{4.820426in}{1.573930in}}%
\pgfpathlineto{\pgfqpoint{4.944758in}{1.488589in}}%
\pgfpathlineto{\pgfqpoint{5.069089in}{1.415776in}}%
\pgfpathlineto{\pgfqpoint{5.193421in}{1.340272in}}%
\pgfusepath{stroke}%
\end{pgfscope}%
\begin{pgfscope}%
\pgfpathrectangle{\pgfqpoint{0.754784in}{1.202999in}}{\pgfqpoint{4.650000in}{3.020000in}}%
\pgfusepath{clip}%
\pgfsetrectcap%
\pgfsetroundjoin%
\pgfsetlinewidth{1.505625pt}%
\definecolor{currentstroke}{rgb}{0.000000,0.000000,0.000000}%
\pgfsetstrokecolor{currentstroke}%
\pgfsetdash{}{0pt}%
\pgfpathmoveto{\pgfqpoint{0.966148in}{3.542546in}}%
\pgfpathlineto{\pgfqpoint{1.090480in}{3.588391in}}%
\pgfpathlineto{\pgfqpoint{1.214811in}{3.642330in}}%
\pgfpathlineto{\pgfqpoint{1.339143in}{3.686619in}}%
\pgfpathlineto{\pgfqpoint{1.463474in}{3.736091in}}%
\pgfpathlineto{\pgfqpoint{1.587806in}{3.773423in}}%
\pgfpathlineto{\pgfqpoint{1.712137in}{3.800837in}}%
\pgfpathlineto{\pgfqpoint{1.836469in}{3.840940in}}%
\pgfpathlineto{\pgfqpoint{1.960800in}{3.871508in}}%
\pgfpathlineto{\pgfqpoint{2.085132in}{3.895612in}}%
\pgfpathlineto{\pgfqpoint{2.209464in}{3.921664in}}%
\pgfpathlineto{\pgfqpoint{2.333795in}{3.944346in}}%
\pgfpathlineto{\pgfqpoint{2.458127in}{3.961839in}}%
\pgfpathlineto{\pgfqpoint{2.582458in}{3.979407in}}%
\pgfpathlineto{\pgfqpoint{2.706790in}{3.993100in}}%
\pgfpathlineto{\pgfqpoint{2.831121in}{4.009167in}}%
\pgfpathlineto{\pgfqpoint{2.955453in}{4.018942in}}%
\pgfpathlineto{\pgfqpoint{3.079784in}{4.024913in}}%
\pgfpathlineto{\pgfqpoint{3.204116in}{4.036460in}}%
\pgfpathlineto{\pgfqpoint{3.328448in}{4.038017in}}%
\pgfpathlineto{\pgfqpoint{3.452779in}{4.038149in}}%
\pgfpathlineto{\pgfqpoint{3.577111in}{4.045219in}}%
\pgfpathlineto{\pgfqpoint{3.701442in}{4.041565in}}%
\pgfpathlineto{\pgfqpoint{3.825774in}{4.039192in}}%
\pgfpathlineto{\pgfqpoint{3.950105in}{4.035273in}}%
\pgfpathlineto{\pgfqpoint{4.074437in}{4.031916in}}%
\pgfpathlineto{\pgfqpoint{4.198768in}{4.029351in}}%
\pgfpathlineto{\pgfqpoint{4.323100in}{4.017533in}}%
\pgfpathlineto{\pgfqpoint{4.447432in}{4.006962in}}%
\pgfpathlineto{\pgfqpoint{4.571763in}{3.992221in}}%
\pgfpathlineto{\pgfqpoint{4.696095in}{3.965572in}}%
\pgfpathlineto{\pgfqpoint{4.820426in}{3.939759in}}%
\pgfpathlineto{\pgfqpoint{4.944758in}{3.910239in}}%
\pgfpathlineto{\pgfqpoint{5.069089in}{3.862589in}}%
\pgfpathlineto{\pgfqpoint{5.193421in}{3.822166in}}%
\pgfusepath{stroke}%
\end{pgfscope}%
\begin{pgfscope}%
\pgfpathrectangle{\pgfqpoint{0.754784in}{1.202999in}}{\pgfqpoint{4.650000in}{3.020000in}}%
\pgfusepath{clip}%
\pgfsetbuttcap%
\pgfsetroundjoin%
\pgfsetlinewidth{1.505625pt}%
\definecolor{currentstroke}{rgb}{1.000000,0.000000,0.000000}%
\pgfsetstrokecolor{currentstroke}%
\pgfsetdash{{5.550000pt}{2.400000pt}}{0.000000pt}%
\pgfpathmoveto{\pgfqpoint{0.966148in}{2.947045in}}%
\pgfpathlineto{\pgfqpoint{1.090480in}{2.966340in}}%
\pgfpathlineto{\pgfqpoint{1.214811in}{2.990708in}}%
\pgfpathlineto{\pgfqpoint{1.339143in}{2.992366in}}%
\pgfpathlineto{\pgfqpoint{1.463474in}{2.989657in}}%
\pgfpathlineto{\pgfqpoint{1.587806in}{2.989951in}}%
\pgfpathlineto{\pgfqpoint{1.712137in}{2.978388in}}%
\pgfpathlineto{\pgfqpoint{1.836469in}{2.968857in}}%
\pgfpathlineto{\pgfqpoint{1.960800in}{2.944751in}}%
\pgfpathlineto{\pgfqpoint{2.085132in}{2.901319in}}%
\pgfpathlineto{\pgfqpoint{2.209464in}{2.871242in}}%
\pgfpathlineto{\pgfqpoint{2.333795in}{2.832915in}}%
\pgfpathlineto{\pgfqpoint{2.458127in}{2.807950in}}%
\pgfpathlineto{\pgfqpoint{2.582458in}{2.751554in}}%
\pgfpathlineto{\pgfqpoint{2.706790in}{2.695337in}}%
\pgfpathlineto{\pgfqpoint{2.831121in}{2.637322in}}%
\pgfpathlineto{\pgfqpoint{2.955453in}{2.580491in}}%
\pgfpathlineto{\pgfqpoint{3.079784in}{2.530666in}}%
\pgfpathlineto{\pgfqpoint{3.204116in}{2.466747in}}%
\pgfpathlineto{\pgfqpoint{3.328448in}{2.397023in}}%
\pgfpathlineto{\pgfqpoint{3.452779in}{2.339627in}}%
\pgfpathlineto{\pgfqpoint{3.577111in}{2.272973in}}%
\pgfpathlineto{\pgfqpoint{3.701442in}{2.200258in}}%
\pgfpathlineto{\pgfqpoint{3.825774in}{2.129232in}}%
\pgfpathlineto{\pgfqpoint{3.950105in}{2.062917in}}%
\pgfpathlineto{\pgfqpoint{4.074437in}{1.997610in}}%
\pgfpathlineto{\pgfqpoint{4.198768in}{1.929894in}}%
\pgfpathlineto{\pgfqpoint{4.323100in}{1.881962in}}%
\pgfpathlineto{\pgfqpoint{4.447432in}{1.822657in}}%
\pgfpathlineto{\pgfqpoint{4.571763in}{1.785696in}}%
\pgfpathlineto{\pgfqpoint{4.696095in}{1.766405in}}%
\pgfpathlineto{\pgfqpoint{4.820426in}{1.739852in}}%
\pgfpathlineto{\pgfqpoint{4.944758in}{1.700106in}}%
\pgfpathlineto{\pgfqpoint{5.069089in}{1.658954in}}%
\pgfpathlineto{\pgfqpoint{5.193421in}{1.618117in}}%
\pgfusepath{stroke}%
\end{pgfscope}%
\begin{pgfscope}%
\pgfpathrectangle{\pgfqpoint{0.754784in}{1.202999in}}{\pgfqpoint{4.650000in}{3.020000in}}%
\pgfusepath{clip}%
\pgfsetbuttcap%
\pgfsetroundjoin%
\pgfsetlinewidth{1.505625pt}%
\definecolor{currentstroke}{rgb}{0.750000,0.000000,0.750000}%
\pgfsetstrokecolor{currentstroke}%
\pgfsetdash{{5.550000pt}{2.400000pt}}{0.000000pt}%
\pgfpathmoveto{\pgfqpoint{0.966148in}{3.494160in}}%
\pgfpathlineto{\pgfqpoint{1.090480in}{3.552511in}}%
\pgfpathlineto{\pgfqpoint{1.214811in}{3.606503in}}%
\pgfpathlineto{\pgfqpoint{1.339143in}{3.656012in}}%
\pgfpathlineto{\pgfqpoint{1.463474in}{3.712679in}}%
\pgfpathlineto{\pgfqpoint{1.587806in}{3.751770in}}%
\pgfpathlineto{\pgfqpoint{1.712137in}{3.788235in}}%
\pgfpathlineto{\pgfqpoint{1.836469in}{3.828394in}}%
\pgfpathlineto{\pgfqpoint{1.960800in}{3.860694in}}%
\pgfpathlineto{\pgfqpoint{2.085132in}{3.895935in}}%
\pgfpathlineto{\pgfqpoint{2.209464in}{3.928480in}}%
\pgfpathlineto{\pgfqpoint{2.333795in}{3.958300in}}%
\pgfpathlineto{\pgfqpoint{2.458127in}{3.986876in}}%
\pgfpathlineto{\pgfqpoint{2.582458in}{4.009405in}}%
\pgfpathlineto{\pgfqpoint{2.706790in}{4.029721in}}%
\pgfpathlineto{\pgfqpoint{2.831121in}{4.050441in}}%
\pgfpathlineto{\pgfqpoint{2.955453in}{4.060507in}}%
\pgfpathlineto{\pgfqpoint{3.079784in}{4.063902in}}%
\pgfpathlineto{\pgfqpoint{3.204116in}{4.074466in}}%
\pgfpathlineto{\pgfqpoint{3.328448in}{4.079706in}}%
\pgfpathlineto{\pgfqpoint{3.452779in}{4.085727in}}%
\pgfpathlineto{\pgfqpoint{3.577111in}{4.084338in}}%
\pgfpathlineto{\pgfqpoint{3.701442in}{4.083470in}}%
\pgfpathlineto{\pgfqpoint{3.825774in}{4.085170in}}%
\pgfpathlineto{\pgfqpoint{3.950105in}{4.085316in}}%
\pgfpathlineto{\pgfqpoint{4.074437in}{4.082188in}}%
\pgfpathlineto{\pgfqpoint{4.198768in}{4.078335in}}%
\pgfpathlineto{\pgfqpoint{4.323100in}{4.065809in}}%
\pgfpathlineto{\pgfqpoint{4.447432in}{4.056196in}}%
\pgfpathlineto{\pgfqpoint{4.571763in}{4.037069in}}%
\pgfpathlineto{\pgfqpoint{4.696095in}{4.017052in}}%
\pgfpathlineto{\pgfqpoint{4.820426in}{4.003772in}}%
\pgfpathlineto{\pgfqpoint{4.944758in}{3.973222in}}%
\pgfpathlineto{\pgfqpoint{5.069089in}{3.942404in}}%
\pgfpathlineto{\pgfqpoint{5.193421in}{3.904419in}}%
\pgfusepath{stroke}%
\end{pgfscope}%
\begin{pgfscope}%
\pgfsetrectcap%
\pgfsetmiterjoin%
\pgfsetlinewidth{0.803000pt}%
\definecolor{currentstroke}{rgb}{0.000000,0.000000,0.000000}%
\pgfsetstrokecolor{currentstroke}%
\pgfsetdash{}{0pt}%
\pgfpathmoveto{\pgfqpoint{0.754784in}{1.202999in}}%
\pgfpathlineto{\pgfqpoint{0.754784in}{4.222999in}}%
\pgfusepath{stroke}%
\end{pgfscope}%
\begin{pgfscope}%
\pgfsetrectcap%
\pgfsetmiterjoin%
\pgfsetlinewidth{0.803000pt}%
\definecolor{currentstroke}{rgb}{0.000000,0.000000,0.000000}%
\pgfsetstrokecolor{currentstroke}%
\pgfsetdash{}{0pt}%
\pgfpathmoveto{\pgfqpoint{5.404784in}{1.202999in}}%
\pgfpathlineto{\pgfqpoint{5.404784in}{4.222999in}}%
\pgfusepath{stroke}%
\end{pgfscope}%
\begin{pgfscope}%
\pgfsetrectcap%
\pgfsetmiterjoin%
\pgfsetlinewidth{0.803000pt}%
\definecolor{currentstroke}{rgb}{0.000000,0.000000,0.000000}%
\pgfsetstrokecolor{currentstroke}%
\pgfsetdash{}{0pt}%
\pgfpathmoveto{\pgfqpoint{0.754784in}{1.202999in}}%
\pgfpathlineto{\pgfqpoint{5.404784in}{1.202999in}}%
\pgfusepath{stroke}%
\end{pgfscope}%
\begin{pgfscope}%
\pgfsetrectcap%
\pgfsetmiterjoin%
\pgfsetlinewidth{0.803000pt}%
\definecolor{currentstroke}{rgb}{0.000000,0.000000,0.000000}%
\pgfsetstrokecolor{currentstroke}%
\pgfsetdash{}{0pt}%
\pgfpathmoveto{\pgfqpoint{0.754784in}{4.222999in}}%
\pgfpathlineto{\pgfqpoint{5.404784in}{4.222999in}}%
\pgfusepath{stroke}%
\end{pgfscope}%
\begin{pgfscope}%
\pgfsetbuttcap%
\pgfsetmiterjoin%
\definecolor{currentfill}{rgb}{0.300000,0.300000,0.300000}%
\pgfsetfillcolor{currentfill}%
\pgfsetfillopacity{0.500000}%
\pgfsetlinewidth{1.003750pt}%
\definecolor{currentstroke}{rgb}{0.300000,0.300000,0.300000}%
\pgfsetstrokecolor{currentstroke}%
\pgfsetstrokeopacity{0.500000}%
\pgfsetdash{}{0pt}%
\pgfpathmoveto{\pgfqpoint{0.468373in}{-0.027778in}}%
\pgfpathlineto{\pgfqpoint{5.746752in}{-0.027778in}}%
\pgfpathquadraticcurveto{\pgfqpoint{5.785641in}{-0.027778in}}{\pgfqpoint{5.785641in}{0.011111in}}%
\pgfpathlineto{\pgfqpoint{5.785641in}{0.586111in}}%
\pgfpathquadraticcurveto{\pgfqpoint{5.785641in}{0.624999in}}{\pgfqpoint{5.746752in}{0.624999in}}%
\pgfpathlineto{\pgfqpoint{0.468373in}{0.624999in}}%
\pgfpathquadraticcurveto{\pgfqpoint{0.429484in}{0.624999in}}{\pgfqpoint{0.429484in}{0.586111in}}%
\pgfpathlineto{\pgfqpoint{0.429484in}{0.011111in}}%
\pgfpathquadraticcurveto{\pgfqpoint{0.429484in}{-0.027778in}}{\pgfqpoint{0.468373in}{-0.027778in}}%
\pgfpathclose%
\pgfusepath{stroke,fill}%
\end{pgfscope}%
\begin{pgfscope}%
\pgfsetbuttcap%
\pgfsetmiterjoin%
\definecolor{currentfill}{rgb}{1.000000,1.000000,1.000000}%
\pgfsetfillcolor{currentfill}%
\pgfsetlinewidth{1.003750pt}%
\definecolor{currentstroke}{rgb}{0.800000,0.800000,0.800000}%
\pgfsetstrokecolor{currentstroke}%
\pgfsetdash{}{0pt}%
\pgfpathmoveto{\pgfqpoint{0.440595in}{0.000000in}}%
\pgfpathlineto{\pgfqpoint{5.718974in}{0.000000in}}%
\pgfpathquadraticcurveto{\pgfqpoint{5.757863in}{0.000000in}}{\pgfqpoint{5.757863in}{0.038889in}}%
\pgfpathlineto{\pgfqpoint{5.757863in}{0.613888in}}%
\pgfpathquadraticcurveto{\pgfqpoint{5.757863in}{0.652777in}}{\pgfqpoint{5.718974in}{0.652777in}}%
\pgfpathlineto{\pgfqpoint{0.440595in}{0.652777in}}%
\pgfpathquadraticcurveto{\pgfqpoint{0.401706in}{0.652777in}}{\pgfqpoint{0.401706in}{0.613888in}}%
\pgfpathlineto{\pgfqpoint{0.401706in}{0.038889in}}%
\pgfpathquadraticcurveto{\pgfqpoint{0.401706in}{0.000000in}}{\pgfqpoint{0.440595in}{0.000000in}}%
\pgfpathclose%
\pgfusepath{stroke,fill}%
\end{pgfscope}%
\begin{pgfscope}%
\pgfsetrectcap%
\pgfsetroundjoin%
\pgfsetlinewidth{1.505625pt}%
\definecolor{currentstroke}{rgb}{0.000000,0.000000,1.000000}%
\pgfsetstrokecolor{currentstroke}%
\pgfsetdash{}{0pt}%
\pgfpathmoveto{\pgfqpoint{0.479484in}{0.493055in}}%
\pgfpathlineto{\pgfqpoint{0.868373in}{0.493055in}}%
\pgfusepath{stroke}%
\end{pgfscope}%
\begin{pgfscope}%
\definecolor{textcolor}{rgb}{0.000000,0.000000,0.000000}%
\pgfsetstrokecolor{textcolor}%
\pgfsetfillcolor{textcolor}%
\pgftext[x=1.023928in,y=0.425000in,left,base]{\color{textcolor}\rmfamily\fontsize{14.000000}{16.800000}\selectfont Women (main person)}%
\end{pgfscope}%
\begin{pgfscope}%
\pgfsetrectcap%
\pgfsetroundjoin%
\pgfsetlinewidth{1.505625pt}%
\definecolor{currentstroke}{rgb}{0.000000,0.000000,0.000000}%
\pgfsetstrokecolor{currentstroke}%
\pgfsetdash{}{0pt}%
\pgfpathmoveto{\pgfqpoint{0.479484in}{0.195833in}}%
\pgfpathlineto{\pgfqpoint{0.868373in}{0.195833in}}%
\pgfusepath{stroke}%
\end{pgfscope}%
\begin{pgfscope}%
\definecolor{textcolor}{rgb}{0.000000,0.000000,0.000000}%
\pgfsetstrokecolor{textcolor}%
\pgfsetfillcolor{textcolor}%
\pgftext[x=1.023928in,y=0.127778in,left,base]{\color{textcolor}\rmfamily\fontsize{14.000000}{16.800000}\selectfont Men (main person)}%
\end{pgfscope}%
\begin{pgfscope}%
\pgfsetbuttcap%
\pgfsetroundjoin%
\pgfsetlinewidth{1.505625pt}%
\definecolor{currentstroke}{rgb}{1.000000,0.000000,0.000000}%
\pgfsetstrokecolor{currentstroke}%
\pgfsetdash{{5.550000pt}{2.400000pt}}{0.000000pt}%
\pgfpathmoveto{\pgfqpoint{3.323187in}{0.493055in}}%
\pgfpathlineto{\pgfqpoint{3.712076in}{0.493055in}}%
\pgfusepath{stroke}%
\end{pgfscope}%
\begin{pgfscope}%
\definecolor{textcolor}{rgb}{0.000000,0.000000,0.000000}%
\pgfsetstrokecolor{textcolor}%
\pgfsetfillcolor{textcolor}%
\pgftext[x=3.867631in,y=0.425000in,left,base]{\color{textcolor}\rmfamily\fontsize{14.000000}{16.800000}\selectfont Women (met person)}%
\end{pgfscope}%
\begin{pgfscope}%
\pgfsetbuttcap%
\pgfsetroundjoin%
\pgfsetlinewidth{1.505625pt}%
\definecolor{currentstroke}{rgb}{0.750000,0.000000,0.750000}%
\pgfsetstrokecolor{currentstroke}%
\pgfsetdash{{5.550000pt}{2.400000pt}}{0.000000pt}%
\pgfpathmoveto{\pgfqpoint{3.323187in}{0.195833in}}%
\pgfpathlineto{\pgfqpoint{3.712076in}{0.195833in}}%
\pgfusepath{stroke}%
\end{pgfscope}%
\begin{pgfscope}%
\definecolor{textcolor}{rgb}{0.000000,0.000000,0.000000}%
\pgfsetstrokecolor{textcolor}%
\pgfsetfillcolor{textcolor}%
\pgftext[x=3.867631in,y=0.127778in,left,base]{\color{textcolor}\rmfamily\fontsize{14.000000}{16.800000}\selectfont Men (met person)}%
\end{pgfscope}%
\end{pgfpicture}%
\makeatother%
\endgroup%
 } 
\end{subfigure}
\begin{subfigure}{.49\textwidth}
\centering
% include second image
\caption{Variance productivity}
\label{fig:sub-second1}
\scalebox{0.5}{%% Creator: Matplotlib, PGF backend
%%
%% To include the figure in your LaTeX document, write
%%   \input{<filename>.pgf}
%%
%% Make sure the required packages are loaded in your preamble
%%   \usepackage{pgf}
%%
%% Figures using additional raster images can only be included by \input if
%% they are in the same directory as the main LaTeX file. For loading figures
%% from other directories you can use the `import` package
%%   \usepackage{import}
%% and then include the figures with
%%   \import{<path to file>}{<filename>.pgf}
%%
%% Matplotlib used the following preamble
%%
\begingroup%
\makeatletter%
\begin{pgfpicture}%
\pgfpathrectangle{\pgfpointorigin}{\pgfqpoint{5.639576in}{4.222999in}}%
\pgfusepath{use as bounding box, clip}%
\begin{pgfscope}%
\pgfsetbuttcap%
\pgfsetmiterjoin%
\definecolor{currentfill}{rgb}{1.000000,1.000000,1.000000}%
\pgfsetfillcolor{currentfill}%
\pgfsetlinewidth{0.000000pt}%
\definecolor{currentstroke}{rgb}{1.000000,1.000000,1.000000}%
\pgfsetstrokecolor{currentstroke}%
\pgfsetdash{}{0pt}%
\pgfpathmoveto{\pgfqpoint{0.000000in}{0.000000in}}%
\pgfpathlineto{\pgfqpoint{5.639576in}{0.000000in}}%
\pgfpathlineto{\pgfqpoint{5.639576in}{4.222999in}}%
\pgfpathlineto{\pgfqpoint{0.000000in}{4.222999in}}%
\pgfpathclose%
\pgfusepath{fill}%
\end{pgfscope}%
\begin{pgfscope}%
\pgfsetbuttcap%
\pgfsetmiterjoin%
\definecolor{currentfill}{rgb}{1.000000,1.000000,1.000000}%
\pgfsetfillcolor{currentfill}%
\pgfsetlinewidth{0.000000pt}%
\definecolor{currentstroke}{rgb}{0.000000,0.000000,0.000000}%
\pgfsetstrokecolor{currentstroke}%
\pgfsetstrokeopacity{0.000000}%
\pgfsetdash{}{0pt}%
\pgfpathmoveto{\pgfqpoint{0.636497in}{1.202999in}}%
\pgfpathlineto{\pgfqpoint{5.286497in}{1.202999in}}%
\pgfpathlineto{\pgfqpoint{5.286497in}{4.222999in}}%
\pgfpathlineto{\pgfqpoint{0.636497in}{4.222999in}}%
\pgfpathclose%
\pgfusepath{fill}%
\end{pgfscope}%
\begin{pgfscope}%
\pgfsetbuttcap%
\pgfsetroundjoin%
\definecolor{currentfill}{rgb}{0.000000,0.000000,0.000000}%
\pgfsetfillcolor{currentfill}%
\pgfsetlinewidth{0.803000pt}%
\definecolor{currentstroke}{rgb}{0.000000,0.000000,0.000000}%
\pgfsetstrokecolor{currentstroke}%
\pgfsetdash{}{0pt}%
\pgfsys@defobject{currentmarker}{\pgfqpoint{0.000000in}{-0.048611in}}{\pgfqpoint{0.000000in}{0.000000in}}{%
\pgfpathmoveto{\pgfqpoint{0.000000in}{0.000000in}}%
\pgfpathlineto{\pgfqpoint{0.000000in}{-0.048611in}}%
\pgfusepath{stroke,fill}%
}%
\begin{pgfscope}%
\pgfsys@transformshift{0.847861in}{1.202999in}%
\pgfsys@useobject{currentmarker}{}%
\end{pgfscope}%
\end{pgfscope}%
\begin{pgfscope}%
\definecolor{textcolor}{rgb}{0.000000,0.000000,0.000000}%
\pgfsetstrokecolor{textcolor}%
\pgfsetfillcolor{textcolor}%
\pgftext[x=0.847861in,y=1.105777in,,top]{\color{textcolor}\rmfamily\fontsize{11.000000}{13.200000}\selectfont \(\displaystyle 25\)}%
\end{pgfscope}%
\begin{pgfscope}%
\pgfsetbuttcap%
\pgfsetroundjoin%
\definecolor{currentfill}{rgb}{0.000000,0.000000,0.000000}%
\pgfsetfillcolor{currentfill}%
\pgfsetlinewidth{0.803000pt}%
\definecolor{currentstroke}{rgb}{0.000000,0.000000,0.000000}%
\pgfsetstrokecolor{currentstroke}%
\pgfsetdash{}{0pt}%
\pgfsys@defobject{currentmarker}{\pgfqpoint{0.000000in}{-0.048611in}}{\pgfqpoint{0.000000in}{0.000000in}}{%
\pgfpathmoveto{\pgfqpoint{0.000000in}{0.000000in}}%
\pgfpathlineto{\pgfqpoint{0.000000in}{-0.048611in}}%
\pgfusepath{stroke,fill}%
}%
\begin{pgfscope}%
\pgfsys@transformshift{1.469518in}{1.202999in}%
\pgfsys@useobject{currentmarker}{}%
\end{pgfscope}%
\end{pgfscope}%
\begin{pgfscope}%
\definecolor{textcolor}{rgb}{0.000000,0.000000,0.000000}%
\pgfsetstrokecolor{textcolor}%
\pgfsetfillcolor{textcolor}%
\pgftext[x=1.469518in,y=1.105777in,,top]{\color{textcolor}\rmfamily\fontsize{11.000000}{13.200000}\selectfont \(\displaystyle 30\)}%
\end{pgfscope}%
\begin{pgfscope}%
\pgfsetbuttcap%
\pgfsetroundjoin%
\definecolor{currentfill}{rgb}{0.000000,0.000000,0.000000}%
\pgfsetfillcolor{currentfill}%
\pgfsetlinewidth{0.803000pt}%
\definecolor{currentstroke}{rgb}{0.000000,0.000000,0.000000}%
\pgfsetstrokecolor{currentstroke}%
\pgfsetdash{}{0pt}%
\pgfsys@defobject{currentmarker}{\pgfqpoint{0.000000in}{-0.048611in}}{\pgfqpoint{0.000000in}{0.000000in}}{%
\pgfpathmoveto{\pgfqpoint{0.000000in}{0.000000in}}%
\pgfpathlineto{\pgfqpoint{0.000000in}{-0.048611in}}%
\pgfusepath{stroke,fill}%
}%
\begin{pgfscope}%
\pgfsys@transformshift{2.091176in}{1.202999in}%
\pgfsys@useobject{currentmarker}{}%
\end{pgfscope}%
\end{pgfscope}%
\begin{pgfscope}%
\definecolor{textcolor}{rgb}{0.000000,0.000000,0.000000}%
\pgfsetstrokecolor{textcolor}%
\pgfsetfillcolor{textcolor}%
\pgftext[x=2.091176in,y=1.105777in,,top]{\color{textcolor}\rmfamily\fontsize{11.000000}{13.200000}\selectfont \(\displaystyle 35\)}%
\end{pgfscope}%
\begin{pgfscope}%
\pgfsetbuttcap%
\pgfsetroundjoin%
\definecolor{currentfill}{rgb}{0.000000,0.000000,0.000000}%
\pgfsetfillcolor{currentfill}%
\pgfsetlinewidth{0.803000pt}%
\definecolor{currentstroke}{rgb}{0.000000,0.000000,0.000000}%
\pgfsetstrokecolor{currentstroke}%
\pgfsetdash{}{0pt}%
\pgfsys@defobject{currentmarker}{\pgfqpoint{0.000000in}{-0.048611in}}{\pgfqpoint{0.000000in}{0.000000in}}{%
\pgfpathmoveto{\pgfqpoint{0.000000in}{0.000000in}}%
\pgfpathlineto{\pgfqpoint{0.000000in}{-0.048611in}}%
\pgfusepath{stroke,fill}%
}%
\begin{pgfscope}%
\pgfsys@transformshift{2.712834in}{1.202999in}%
\pgfsys@useobject{currentmarker}{}%
\end{pgfscope}%
\end{pgfscope}%
\begin{pgfscope}%
\definecolor{textcolor}{rgb}{0.000000,0.000000,0.000000}%
\pgfsetstrokecolor{textcolor}%
\pgfsetfillcolor{textcolor}%
\pgftext[x=2.712834in,y=1.105777in,,top]{\color{textcolor}\rmfamily\fontsize{11.000000}{13.200000}\selectfont \(\displaystyle 40\)}%
\end{pgfscope}%
\begin{pgfscope}%
\pgfsetbuttcap%
\pgfsetroundjoin%
\definecolor{currentfill}{rgb}{0.000000,0.000000,0.000000}%
\pgfsetfillcolor{currentfill}%
\pgfsetlinewidth{0.803000pt}%
\definecolor{currentstroke}{rgb}{0.000000,0.000000,0.000000}%
\pgfsetstrokecolor{currentstroke}%
\pgfsetdash{}{0pt}%
\pgfsys@defobject{currentmarker}{\pgfqpoint{0.000000in}{-0.048611in}}{\pgfqpoint{0.000000in}{0.000000in}}{%
\pgfpathmoveto{\pgfqpoint{0.000000in}{0.000000in}}%
\pgfpathlineto{\pgfqpoint{0.000000in}{-0.048611in}}%
\pgfusepath{stroke,fill}%
}%
\begin{pgfscope}%
\pgfsys@transformshift{3.334492in}{1.202999in}%
\pgfsys@useobject{currentmarker}{}%
\end{pgfscope}%
\end{pgfscope}%
\begin{pgfscope}%
\definecolor{textcolor}{rgb}{0.000000,0.000000,0.000000}%
\pgfsetstrokecolor{textcolor}%
\pgfsetfillcolor{textcolor}%
\pgftext[x=3.334492in,y=1.105777in,,top]{\color{textcolor}\rmfamily\fontsize{11.000000}{13.200000}\selectfont \(\displaystyle 45\)}%
\end{pgfscope}%
\begin{pgfscope}%
\pgfsetbuttcap%
\pgfsetroundjoin%
\definecolor{currentfill}{rgb}{0.000000,0.000000,0.000000}%
\pgfsetfillcolor{currentfill}%
\pgfsetlinewidth{0.803000pt}%
\definecolor{currentstroke}{rgb}{0.000000,0.000000,0.000000}%
\pgfsetstrokecolor{currentstroke}%
\pgfsetdash{}{0pt}%
\pgfsys@defobject{currentmarker}{\pgfqpoint{0.000000in}{-0.048611in}}{\pgfqpoint{0.000000in}{0.000000in}}{%
\pgfpathmoveto{\pgfqpoint{0.000000in}{0.000000in}}%
\pgfpathlineto{\pgfqpoint{0.000000in}{-0.048611in}}%
\pgfusepath{stroke,fill}%
}%
\begin{pgfscope}%
\pgfsys@transformshift{3.956149in}{1.202999in}%
\pgfsys@useobject{currentmarker}{}%
\end{pgfscope}%
\end{pgfscope}%
\begin{pgfscope}%
\definecolor{textcolor}{rgb}{0.000000,0.000000,0.000000}%
\pgfsetstrokecolor{textcolor}%
\pgfsetfillcolor{textcolor}%
\pgftext[x=3.956149in,y=1.105777in,,top]{\color{textcolor}\rmfamily\fontsize{11.000000}{13.200000}\selectfont \(\displaystyle 50\)}%
\end{pgfscope}%
\begin{pgfscope}%
\pgfsetbuttcap%
\pgfsetroundjoin%
\definecolor{currentfill}{rgb}{0.000000,0.000000,0.000000}%
\pgfsetfillcolor{currentfill}%
\pgfsetlinewidth{0.803000pt}%
\definecolor{currentstroke}{rgb}{0.000000,0.000000,0.000000}%
\pgfsetstrokecolor{currentstroke}%
\pgfsetdash{}{0pt}%
\pgfsys@defobject{currentmarker}{\pgfqpoint{0.000000in}{-0.048611in}}{\pgfqpoint{0.000000in}{0.000000in}}{%
\pgfpathmoveto{\pgfqpoint{0.000000in}{0.000000in}}%
\pgfpathlineto{\pgfqpoint{0.000000in}{-0.048611in}}%
\pgfusepath{stroke,fill}%
}%
\begin{pgfscope}%
\pgfsys@transformshift{4.577807in}{1.202999in}%
\pgfsys@useobject{currentmarker}{}%
\end{pgfscope}%
\end{pgfscope}%
\begin{pgfscope}%
\definecolor{textcolor}{rgb}{0.000000,0.000000,0.000000}%
\pgfsetstrokecolor{textcolor}%
\pgfsetfillcolor{textcolor}%
\pgftext[x=4.577807in,y=1.105777in,,top]{\color{textcolor}\rmfamily\fontsize{11.000000}{13.200000}\selectfont \(\displaystyle 55\)}%
\end{pgfscope}%
\begin{pgfscope}%
\pgfsetbuttcap%
\pgfsetroundjoin%
\definecolor{currentfill}{rgb}{0.000000,0.000000,0.000000}%
\pgfsetfillcolor{currentfill}%
\pgfsetlinewidth{0.803000pt}%
\definecolor{currentstroke}{rgb}{0.000000,0.000000,0.000000}%
\pgfsetstrokecolor{currentstroke}%
\pgfsetdash{}{0pt}%
\pgfsys@defobject{currentmarker}{\pgfqpoint{0.000000in}{-0.048611in}}{\pgfqpoint{0.000000in}{0.000000in}}{%
\pgfpathmoveto{\pgfqpoint{0.000000in}{0.000000in}}%
\pgfpathlineto{\pgfqpoint{0.000000in}{-0.048611in}}%
\pgfusepath{stroke,fill}%
}%
\begin{pgfscope}%
\pgfsys@transformshift{5.199465in}{1.202999in}%
\pgfsys@useobject{currentmarker}{}%
\end{pgfscope}%
\end{pgfscope}%
\begin{pgfscope}%
\definecolor{textcolor}{rgb}{0.000000,0.000000,0.000000}%
\pgfsetstrokecolor{textcolor}%
\pgfsetfillcolor{textcolor}%
\pgftext[x=5.199465in,y=1.105777in,,top]{\color{textcolor}\rmfamily\fontsize{11.000000}{13.200000}\selectfont \(\displaystyle 60\)}%
\end{pgfscope}%
\begin{pgfscope}%
\definecolor{textcolor}{rgb}{0.000000,0.000000,0.000000}%
\pgfsetstrokecolor{textcolor}%
\pgfsetfillcolor{textcolor}%
\pgftext[x=2.961497in,y=0.915037in,,top]{\color{textcolor}\rmfamily\fontsize{16.000000}{19.200000}\selectfont Age}%
\end{pgfscope}%
\begin{pgfscope}%
\pgfsetbuttcap%
\pgfsetroundjoin%
\definecolor{currentfill}{rgb}{0.000000,0.000000,0.000000}%
\pgfsetfillcolor{currentfill}%
\pgfsetlinewidth{0.803000pt}%
\definecolor{currentstroke}{rgb}{0.000000,0.000000,0.000000}%
\pgfsetstrokecolor{currentstroke}%
\pgfsetdash{}{0pt}%
\pgfsys@defobject{currentmarker}{\pgfqpoint{-0.048611in}{0.000000in}}{\pgfqpoint{0.000000in}{0.000000in}}{%
\pgfpathmoveto{\pgfqpoint{0.000000in}{0.000000in}}%
\pgfpathlineto{\pgfqpoint{-0.048611in}{0.000000in}}%
\pgfusepath{stroke,fill}%
}%
\begin{pgfscope}%
\pgfsys@transformshift{0.636497in}{1.257415in}%
\pgfsys@useobject{currentmarker}{}%
\end{pgfscope}%
\end{pgfscope}%
\begin{pgfscope}%
\definecolor{textcolor}{rgb}{0.000000,0.000000,0.000000}%
\pgfsetstrokecolor{textcolor}%
\pgfsetfillcolor{textcolor}%
\pgftext[x=0.268904in,y=1.204609in,left,base]{\color{textcolor}\rmfamily\fontsize{11.000000}{13.200000}\selectfont \(\displaystyle 0.25\)}%
\end{pgfscope}%
\begin{pgfscope}%
\pgfsetbuttcap%
\pgfsetroundjoin%
\definecolor{currentfill}{rgb}{0.000000,0.000000,0.000000}%
\pgfsetfillcolor{currentfill}%
\pgfsetlinewidth{0.803000pt}%
\definecolor{currentstroke}{rgb}{0.000000,0.000000,0.000000}%
\pgfsetstrokecolor{currentstroke}%
\pgfsetdash{}{0pt}%
\pgfsys@defobject{currentmarker}{\pgfqpoint{-0.048611in}{0.000000in}}{\pgfqpoint{0.000000in}{0.000000in}}{%
\pgfpathmoveto{\pgfqpoint{0.000000in}{0.000000in}}%
\pgfpathlineto{\pgfqpoint{-0.048611in}{0.000000in}}%
\pgfusepath{stroke,fill}%
}%
\begin{pgfscope}%
\pgfsys@transformshift{0.636497in}{1.637459in}%
\pgfsys@useobject{currentmarker}{}%
\end{pgfscope}%
\end{pgfscope}%
\begin{pgfscope}%
\definecolor{textcolor}{rgb}{0.000000,0.000000,0.000000}%
\pgfsetstrokecolor{textcolor}%
\pgfsetfillcolor{textcolor}%
\pgftext[x=0.268904in,y=1.584652in,left,base]{\color{textcolor}\rmfamily\fontsize{11.000000}{13.200000}\selectfont \(\displaystyle 0.50\)}%
\end{pgfscope}%
\begin{pgfscope}%
\pgfsetbuttcap%
\pgfsetroundjoin%
\definecolor{currentfill}{rgb}{0.000000,0.000000,0.000000}%
\pgfsetfillcolor{currentfill}%
\pgfsetlinewidth{0.803000pt}%
\definecolor{currentstroke}{rgb}{0.000000,0.000000,0.000000}%
\pgfsetstrokecolor{currentstroke}%
\pgfsetdash{}{0pt}%
\pgfsys@defobject{currentmarker}{\pgfqpoint{-0.048611in}{0.000000in}}{\pgfqpoint{0.000000in}{0.000000in}}{%
\pgfpathmoveto{\pgfqpoint{0.000000in}{0.000000in}}%
\pgfpathlineto{\pgfqpoint{-0.048611in}{0.000000in}}%
\pgfusepath{stroke,fill}%
}%
\begin{pgfscope}%
\pgfsys@transformshift{0.636497in}{2.017503in}%
\pgfsys@useobject{currentmarker}{}%
\end{pgfscope}%
\end{pgfscope}%
\begin{pgfscope}%
\definecolor{textcolor}{rgb}{0.000000,0.000000,0.000000}%
\pgfsetstrokecolor{textcolor}%
\pgfsetfillcolor{textcolor}%
\pgftext[x=0.268904in,y=1.964696in,left,base]{\color{textcolor}\rmfamily\fontsize{11.000000}{13.200000}\selectfont \(\displaystyle 0.75\)}%
\end{pgfscope}%
\begin{pgfscope}%
\pgfsetbuttcap%
\pgfsetroundjoin%
\definecolor{currentfill}{rgb}{0.000000,0.000000,0.000000}%
\pgfsetfillcolor{currentfill}%
\pgfsetlinewidth{0.803000pt}%
\definecolor{currentstroke}{rgb}{0.000000,0.000000,0.000000}%
\pgfsetstrokecolor{currentstroke}%
\pgfsetdash{}{0pt}%
\pgfsys@defobject{currentmarker}{\pgfqpoint{-0.048611in}{0.000000in}}{\pgfqpoint{0.000000in}{0.000000in}}{%
\pgfpathmoveto{\pgfqpoint{0.000000in}{0.000000in}}%
\pgfpathlineto{\pgfqpoint{-0.048611in}{0.000000in}}%
\pgfusepath{stroke,fill}%
}%
\begin{pgfscope}%
\pgfsys@transformshift{0.636497in}{2.397546in}%
\pgfsys@useobject{currentmarker}{}%
\end{pgfscope}%
\end{pgfscope}%
\begin{pgfscope}%
\definecolor{textcolor}{rgb}{0.000000,0.000000,0.000000}%
\pgfsetstrokecolor{textcolor}%
\pgfsetfillcolor{textcolor}%
\pgftext[x=0.268904in,y=2.344740in,left,base]{\color{textcolor}\rmfamily\fontsize{11.000000}{13.200000}\selectfont \(\displaystyle 1.00\)}%
\end{pgfscope}%
\begin{pgfscope}%
\pgfsetbuttcap%
\pgfsetroundjoin%
\definecolor{currentfill}{rgb}{0.000000,0.000000,0.000000}%
\pgfsetfillcolor{currentfill}%
\pgfsetlinewidth{0.803000pt}%
\definecolor{currentstroke}{rgb}{0.000000,0.000000,0.000000}%
\pgfsetstrokecolor{currentstroke}%
\pgfsetdash{}{0pt}%
\pgfsys@defobject{currentmarker}{\pgfqpoint{-0.048611in}{0.000000in}}{\pgfqpoint{0.000000in}{0.000000in}}{%
\pgfpathmoveto{\pgfqpoint{0.000000in}{0.000000in}}%
\pgfpathlineto{\pgfqpoint{-0.048611in}{0.000000in}}%
\pgfusepath{stroke,fill}%
}%
\begin{pgfscope}%
\pgfsys@transformshift{0.636497in}{2.777590in}%
\pgfsys@useobject{currentmarker}{}%
\end{pgfscope}%
\end{pgfscope}%
\begin{pgfscope}%
\definecolor{textcolor}{rgb}{0.000000,0.000000,0.000000}%
\pgfsetstrokecolor{textcolor}%
\pgfsetfillcolor{textcolor}%
\pgftext[x=0.268904in,y=2.724783in,left,base]{\color{textcolor}\rmfamily\fontsize{11.000000}{13.200000}\selectfont \(\displaystyle 1.25\)}%
\end{pgfscope}%
\begin{pgfscope}%
\pgfsetbuttcap%
\pgfsetroundjoin%
\definecolor{currentfill}{rgb}{0.000000,0.000000,0.000000}%
\pgfsetfillcolor{currentfill}%
\pgfsetlinewidth{0.803000pt}%
\definecolor{currentstroke}{rgb}{0.000000,0.000000,0.000000}%
\pgfsetstrokecolor{currentstroke}%
\pgfsetdash{}{0pt}%
\pgfsys@defobject{currentmarker}{\pgfqpoint{-0.048611in}{0.000000in}}{\pgfqpoint{0.000000in}{0.000000in}}{%
\pgfpathmoveto{\pgfqpoint{0.000000in}{0.000000in}}%
\pgfpathlineto{\pgfqpoint{-0.048611in}{0.000000in}}%
\pgfusepath{stroke,fill}%
}%
\begin{pgfscope}%
\pgfsys@transformshift{0.636497in}{3.157633in}%
\pgfsys@useobject{currentmarker}{}%
\end{pgfscope}%
\end{pgfscope}%
\begin{pgfscope}%
\definecolor{textcolor}{rgb}{0.000000,0.000000,0.000000}%
\pgfsetstrokecolor{textcolor}%
\pgfsetfillcolor{textcolor}%
\pgftext[x=0.268904in,y=3.104827in,left,base]{\color{textcolor}\rmfamily\fontsize{11.000000}{13.200000}\selectfont \(\displaystyle 1.50\)}%
\end{pgfscope}%
\begin{pgfscope}%
\pgfsetbuttcap%
\pgfsetroundjoin%
\definecolor{currentfill}{rgb}{0.000000,0.000000,0.000000}%
\pgfsetfillcolor{currentfill}%
\pgfsetlinewidth{0.803000pt}%
\definecolor{currentstroke}{rgb}{0.000000,0.000000,0.000000}%
\pgfsetstrokecolor{currentstroke}%
\pgfsetdash{}{0pt}%
\pgfsys@defobject{currentmarker}{\pgfqpoint{-0.048611in}{0.000000in}}{\pgfqpoint{0.000000in}{0.000000in}}{%
\pgfpathmoveto{\pgfqpoint{0.000000in}{0.000000in}}%
\pgfpathlineto{\pgfqpoint{-0.048611in}{0.000000in}}%
\pgfusepath{stroke,fill}%
}%
\begin{pgfscope}%
\pgfsys@transformshift{0.636497in}{3.537677in}%
\pgfsys@useobject{currentmarker}{}%
\end{pgfscope}%
\end{pgfscope}%
\begin{pgfscope}%
\definecolor{textcolor}{rgb}{0.000000,0.000000,0.000000}%
\pgfsetstrokecolor{textcolor}%
\pgfsetfillcolor{textcolor}%
\pgftext[x=0.268904in,y=3.484870in,left,base]{\color{textcolor}\rmfamily\fontsize{11.000000}{13.200000}\selectfont \(\displaystyle 1.75\)}%
\end{pgfscope}%
\begin{pgfscope}%
\pgfsetbuttcap%
\pgfsetroundjoin%
\definecolor{currentfill}{rgb}{0.000000,0.000000,0.000000}%
\pgfsetfillcolor{currentfill}%
\pgfsetlinewidth{0.803000pt}%
\definecolor{currentstroke}{rgb}{0.000000,0.000000,0.000000}%
\pgfsetstrokecolor{currentstroke}%
\pgfsetdash{}{0pt}%
\pgfsys@defobject{currentmarker}{\pgfqpoint{-0.048611in}{0.000000in}}{\pgfqpoint{0.000000in}{0.000000in}}{%
\pgfpathmoveto{\pgfqpoint{0.000000in}{0.000000in}}%
\pgfpathlineto{\pgfqpoint{-0.048611in}{0.000000in}}%
\pgfusepath{stroke,fill}%
}%
\begin{pgfscope}%
\pgfsys@transformshift{0.636497in}{3.917720in}%
\pgfsys@useobject{currentmarker}{}%
\end{pgfscope}%
\end{pgfscope}%
\begin{pgfscope}%
\definecolor{textcolor}{rgb}{0.000000,0.000000,0.000000}%
\pgfsetstrokecolor{textcolor}%
\pgfsetfillcolor{textcolor}%
\pgftext[x=0.268904in,y=3.864914in,left,base]{\color{textcolor}\rmfamily\fontsize{11.000000}{13.200000}\selectfont \(\displaystyle 2.00\)}%
\end{pgfscope}%
\begin{pgfscope}%
\definecolor{textcolor}{rgb}{0.000000,0.000000,0.000000}%
\pgfsetstrokecolor{textcolor}%
\pgfsetfillcolor{textcolor}%
\pgftext[x=0.213349in,y=2.712999in,,bottom,rotate=90.000000]{\color{textcolor}\rmfamily\fontsize{16.000000}{19.200000}\selectfont Potential log wage---variance}%
\end{pgfscope}%
\begin{pgfscope}%
\pgfpathrectangle{\pgfqpoint{0.636497in}{1.202999in}}{\pgfqpoint{4.650000in}{3.020000in}}%
\pgfusepath{clip}%
\pgfsetrectcap%
\pgfsetroundjoin%
\pgfsetlinewidth{1.505625pt}%
\definecolor{currentstroke}{rgb}{0.000000,0.000000,1.000000}%
\pgfsetstrokecolor{currentstroke}%
\pgfsetdash{}{0pt}%
\pgfpathmoveto{\pgfqpoint{0.847861in}{1.622126in}}%
\pgfpathlineto{\pgfqpoint{0.972192in}{1.663646in}}%
\pgfpathlineto{\pgfqpoint{1.096524in}{1.713432in}}%
\pgfpathlineto{\pgfqpoint{1.220855in}{1.764642in}}%
\pgfpathlineto{\pgfqpoint{1.345187in}{1.814994in}}%
\pgfpathlineto{\pgfqpoint{1.469518in}{1.875680in}}%
\pgfpathlineto{\pgfqpoint{1.593850in}{1.921995in}}%
\pgfpathlineto{\pgfqpoint{1.718182in}{1.984609in}}%
\pgfpathlineto{\pgfqpoint{1.842513in}{2.046072in}}%
\pgfpathlineto{\pgfqpoint{1.966845in}{2.104420in}}%
\pgfpathlineto{\pgfqpoint{2.091176in}{2.173383in}}%
\pgfpathlineto{\pgfqpoint{2.215508in}{2.242092in}}%
\pgfpathlineto{\pgfqpoint{2.339839in}{2.312164in}}%
\pgfpathlineto{\pgfqpoint{2.464171in}{2.383095in}}%
\pgfpathlineto{\pgfqpoint{2.588502in}{2.448584in}}%
\pgfpathlineto{\pgfqpoint{2.712834in}{2.529633in}}%
\pgfpathlineto{\pgfqpoint{2.837166in}{2.598319in}}%
\pgfpathlineto{\pgfqpoint{2.961497in}{2.673118in}}%
\pgfpathlineto{\pgfqpoint{3.085829in}{2.749368in}}%
\pgfpathlineto{\pgfqpoint{3.210160in}{2.834423in}}%
\pgfpathlineto{\pgfqpoint{3.334492in}{2.910126in}}%
\pgfpathlineto{\pgfqpoint{3.458823in}{2.983326in}}%
\pgfpathlineto{\pgfqpoint{3.583155in}{3.049823in}}%
\pgfpathlineto{\pgfqpoint{3.707486in}{3.137135in}}%
\pgfpathlineto{\pgfqpoint{3.831818in}{3.218365in}}%
\pgfpathlineto{\pgfqpoint{3.956149in}{3.304746in}}%
\pgfpathlineto{\pgfqpoint{4.080481in}{3.387500in}}%
\pgfpathlineto{\pgfqpoint{4.204813in}{3.481523in}}%
\pgfpathlineto{\pgfqpoint{4.329144in}{3.562591in}}%
\pgfpathlineto{\pgfqpoint{4.453476in}{3.663969in}}%
\pgfpathlineto{\pgfqpoint{4.577807in}{3.754085in}}%
\pgfpathlineto{\pgfqpoint{4.702139in}{3.841365in}}%
\pgfpathlineto{\pgfqpoint{4.826470in}{3.919332in}}%
\pgfpathlineto{\pgfqpoint{4.950802in}{4.003935in}}%
\pgfpathlineto{\pgfqpoint{5.075133in}{4.085727in}}%
\pgfusepath{stroke}%
\end{pgfscope}%
\begin{pgfscope}%
\pgfpathrectangle{\pgfqpoint{0.636497in}{1.202999in}}{\pgfqpoint{4.650000in}{3.020000in}}%
\pgfusepath{clip}%
\pgfsetrectcap%
\pgfsetroundjoin%
\pgfsetlinewidth{1.505625pt}%
\definecolor{currentstroke}{rgb}{0.000000,0.000000,0.000000}%
\pgfsetstrokecolor{currentstroke}%
\pgfsetdash{}{0pt}%
\pgfpathmoveto{\pgfqpoint{0.847861in}{1.340272in}}%
\pgfpathlineto{\pgfqpoint{0.972192in}{1.389868in}}%
\pgfpathlineto{\pgfqpoint{1.096524in}{1.440075in}}%
\pgfpathlineto{\pgfqpoint{1.220855in}{1.489111in}}%
\pgfpathlineto{\pgfqpoint{1.345187in}{1.535335in}}%
\pgfpathlineto{\pgfqpoint{1.469518in}{1.582407in}}%
\pgfpathlineto{\pgfqpoint{1.593850in}{1.627986in}}%
\pgfpathlineto{\pgfqpoint{1.718182in}{1.671470in}}%
\pgfpathlineto{\pgfqpoint{1.842513in}{1.709195in}}%
\pgfpathlineto{\pgfqpoint{1.966845in}{1.749604in}}%
\pgfpathlineto{\pgfqpoint{2.091176in}{1.789032in}}%
\pgfpathlineto{\pgfqpoint{2.215508in}{1.827104in}}%
\pgfpathlineto{\pgfqpoint{2.339839in}{1.865442in}}%
\pgfpathlineto{\pgfqpoint{2.464171in}{1.900654in}}%
\pgfpathlineto{\pgfqpoint{2.588502in}{1.939304in}}%
\pgfpathlineto{\pgfqpoint{2.712834in}{1.973665in}}%
\pgfpathlineto{\pgfqpoint{2.837166in}{2.007513in}}%
\pgfpathlineto{\pgfqpoint{2.961497in}{2.040006in}}%
\pgfpathlineto{\pgfqpoint{3.085829in}{2.071616in}}%
\pgfpathlineto{\pgfqpoint{3.210160in}{2.102908in}}%
\pgfpathlineto{\pgfqpoint{3.334492in}{2.136797in}}%
\pgfpathlineto{\pgfqpoint{3.458823in}{2.163675in}}%
\pgfpathlineto{\pgfqpoint{3.583155in}{2.191294in}}%
\pgfpathlineto{\pgfqpoint{3.707486in}{2.226596in}}%
\pgfpathlineto{\pgfqpoint{3.831818in}{2.268360in}}%
\pgfpathlineto{\pgfqpoint{3.956149in}{2.302103in}}%
\pgfpathlineto{\pgfqpoint{4.080481in}{2.333472in}}%
\pgfpathlineto{\pgfqpoint{4.204813in}{2.367777in}}%
\pgfpathlineto{\pgfqpoint{4.329144in}{2.404631in}}%
\pgfpathlineto{\pgfqpoint{4.453476in}{2.434858in}}%
\pgfpathlineto{\pgfqpoint{4.577807in}{2.461919in}}%
\pgfpathlineto{\pgfqpoint{4.702139in}{2.496474in}}%
\pgfpathlineto{\pgfqpoint{4.826470in}{2.529382in}}%
\pgfpathlineto{\pgfqpoint{4.950802in}{2.568373in}}%
\pgfpathlineto{\pgfqpoint{5.075133in}{2.621738in}}%
\pgfusepath{stroke}%
\end{pgfscope}%
\begin{pgfscope}%
\pgfpathrectangle{\pgfqpoint{0.636497in}{1.202999in}}{\pgfqpoint{4.650000in}{3.020000in}}%
\pgfusepath{clip}%
\pgfsetbuttcap%
\pgfsetroundjoin%
\pgfsetlinewidth{1.505625pt}%
\definecolor{currentstroke}{rgb}{1.000000,0.000000,0.000000}%
\pgfsetstrokecolor{currentstroke}%
\pgfsetdash{{5.550000pt}{2.400000pt}}{0.000000pt}%
\pgfpathmoveto{\pgfqpoint{0.847861in}{1.468854in}}%
\pgfpathlineto{\pgfqpoint{0.972192in}{1.518821in}}%
\pgfpathlineto{\pgfqpoint{1.096524in}{1.569359in}}%
\pgfpathlineto{\pgfqpoint{1.220855in}{1.620046in}}%
\pgfpathlineto{\pgfqpoint{1.345187in}{1.672214in}}%
\pgfpathlineto{\pgfqpoint{1.469518in}{1.730725in}}%
\pgfpathlineto{\pgfqpoint{1.593850in}{1.784632in}}%
\pgfpathlineto{\pgfqpoint{1.718182in}{1.843292in}}%
\pgfpathlineto{\pgfqpoint{1.842513in}{1.902329in}}%
\pgfpathlineto{\pgfqpoint{1.966845in}{1.962744in}}%
\pgfpathlineto{\pgfqpoint{2.091176in}{2.027467in}}%
\pgfpathlineto{\pgfqpoint{2.215508in}{2.093895in}}%
\pgfpathlineto{\pgfqpoint{2.339839in}{2.156038in}}%
\pgfpathlineto{\pgfqpoint{2.464171in}{2.222537in}}%
\pgfpathlineto{\pgfqpoint{2.588502in}{2.287652in}}%
\pgfpathlineto{\pgfqpoint{2.712834in}{2.365019in}}%
\pgfpathlineto{\pgfqpoint{2.837166in}{2.429309in}}%
\pgfpathlineto{\pgfqpoint{2.961497in}{2.501583in}}%
\pgfpathlineto{\pgfqpoint{3.085829in}{2.574829in}}%
\pgfpathlineto{\pgfqpoint{3.210160in}{2.652454in}}%
\pgfpathlineto{\pgfqpoint{3.334492in}{2.727632in}}%
\pgfpathlineto{\pgfqpoint{3.458823in}{2.800983in}}%
\pgfpathlineto{\pgfqpoint{3.583155in}{2.863779in}}%
\pgfpathlineto{\pgfqpoint{3.707486in}{2.943127in}}%
\pgfpathlineto{\pgfqpoint{3.831818in}{3.021901in}}%
\pgfpathlineto{\pgfqpoint{3.956149in}{3.096894in}}%
\pgfpathlineto{\pgfqpoint{4.080481in}{3.186294in}}%
\pgfpathlineto{\pgfqpoint{4.204813in}{3.283287in}}%
\pgfpathlineto{\pgfqpoint{4.329144in}{3.383920in}}%
\pgfpathlineto{\pgfqpoint{4.453476in}{3.466504in}}%
\pgfpathlineto{\pgfqpoint{4.577807in}{3.539684in}}%
\pgfpathlineto{\pgfqpoint{4.702139in}{3.606125in}}%
\pgfpathlineto{\pgfqpoint{4.826470in}{3.694966in}}%
\pgfpathlineto{\pgfqpoint{4.950802in}{3.767867in}}%
\pgfpathlineto{\pgfqpoint{5.075133in}{3.833149in}}%
\pgfusepath{stroke}%
\end{pgfscope}%
\begin{pgfscope}%
\pgfpathrectangle{\pgfqpoint{0.636497in}{1.202999in}}{\pgfqpoint{4.650000in}{3.020000in}}%
\pgfusepath{clip}%
\pgfsetbuttcap%
\pgfsetroundjoin%
\pgfsetlinewidth{1.505625pt}%
\definecolor{currentstroke}{rgb}{0.750000,0.000000,0.750000}%
\pgfsetstrokecolor{currentstroke}%
\pgfsetdash{{5.550000pt}{2.400000pt}}{0.000000pt}%
\pgfpathmoveto{\pgfqpoint{0.847861in}{1.354180in}}%
\pgfpathlineto{\pgfqpoint{0.972192in}{1.390538in}}%
\pgfpathlineto{\pgfqpoint{1.096524in}{1.429430in}}%
\pgfpathlineto{\pgfqpoint{1.220855in}{1.465702in}}%
\pgfpathlineto{\pgfqpoint{1.345187in}{1.507135in}}%
\pgfpathlineto{\pgfqpoint{1.469518in}{1.545963in}}%
\pgfpathlineto{\pgfqpoint{1.593850in}{1.582660in}}%
\pgfpathlineto{\pgfqpoint{1.718182in}{1.627846in}}%
\pgfpathlineto{\pgfqpoint{1.842513in}{1.671495in}}%
\pgfpathlineto{\pgfqpoint{1.966845in}{1.712377in}}%
\pgfpathlineto{\pgfqpoint{2.091176in}{1.757161in}}%
\pgfpathlineto{\pgfqpoint{2.215508in}{1.797404in}}%
\pgfpathlineto{\pgfqpoint{2.339839in}{1.838316in}}%
\pgfpathlineto{\pgfqpoint{2.464171in}{1.874807in}}%
\pgfpathlineto{\pgfqpoint{2.588502in}{1.907304in}}%
\pgfpathlineto{\pgfqpoint{2.712834in}{1.946109in}}%
\pgfpathlineto{\pgfqpoint{2.837166in}{1.981133in}}%
\pgfpathlineto{\pgfqpoint{2.961497in}{2.014569in}}%
\pgfpathlineto{\pgfqpoint{3.085829in}{2.051072in}}%
\pgfpathlineto{\pgfqpoint{3.210160in}{2.089451in}}%
\pgfpathlineto{\pgfqpoint{3.334492in}{2.125105in}}%
\pgfpathlineto{\pgfqpoint{3.458823in}{2.152999in}}%
\pgfpathlineto{\pgfqpoint{3.583155in}{2.188155in}}%
\pgfpathlineto{\pgfqpoint{3.707486in}{2.221285in}}%
\pgfpathlineto{\pgfqpoint{3.831818in}{2.258061in}}%
\pgfpathlineto{\pgfqpoint{3.956149in}{2.293206in}}%
\pgfpathlineto{\pgfqpoint{4.080481in}{2.324209in}}%
\pgfpathlineto{\pgfqpoint{4.204813in}{2.360003in}}%
\pgfpathlineto{\pgfqpoint{4.329144in}{2.395209in}}%
\pgfpathlineto{\pgfqpoint{4.453476in}{2.430140in}}%
\pgfpathlineto{\pgfqpoint{4.577807in}{2.464732in}}%
\pgfpathlineto{\pgfqpoint{4.702139in}{2.504772in}}%
\pgfpathlineto{\pgfqpoint{4.826470in}{2.523546in}}%
\pgfpathlineto{\pgfqpoint{4.950802in}{2.551064in}}%
\pgfpathlineto{\pgfqpoint{5.075133in}{2.595009in}}%
\pgfusepath{stroke}%
\end{pgfscope}%
\begin{pgfscope}%
\pgfsetrectcap%
\pgfsetmiterjoin%
\pgfsetlinewidth{0.803000pt}%
\definecolor{currentstroke}{rgb}{0.000000,0.000000,0.000000}%
\pgfsetstrokecolor{currentstroke}%
\pgfsetdash{}{0pt}%
\pgfpathmoveto{\pgfqpoint{0.636497in}{1.202999in}}%
\pgfpathlineto{\pgfqpoint{0.636497in}{4.222999in}}%
\pgfusepath{stroke}%
\end{pgfscope}%
\begin{pgfscope}%
\pgfsetrectcap%
\pgfsetmiterjoin%
\pgfsetlinewidth{0.803000pt}%
\definecolor{currentstroke}{rgb}{0.000000,0.000000,0.000000}%
\pgfsetstrokecolor{currentstroke}%
\pgfsetdash{}{0pt}%
\pgfpathmoveto{\pgfqpoint{5.286497in}{1.202999in}}%
\pgfpathlineto{\pgfqpoint{5.286497in}{4.222999in}}%
\pgfusepath{stroke}%
\end{pgfscope}%
\begin{pgfscope}%
\pgfsetrectcap%
\pgfsetmiterjoin%
\pgfsetlinewidth{0.803000pt}%
\definecolor{currentstroke}{rgb}{0.000000,0.000000,0.000000}%
\pgfsetstrokecolor{currentstroke}%
\pgfsetdash{}{0pt}%
\pgfpathmoveto{\pgfqpoint{0.636497in}{1.202999in}}%
\pgfpathlineto{\pgfqpoint{5.286497in}{1.202999in}}%
\pgfusepath{stroke}%
\end{pgfscope}%
\begin{pgfscope}%
\pgfsetrectcap%
\pgfsetmiterjoin%
\pgfsetlinewidth{0.803000pt}%
\definecolor{currentstroke}{rgb}{0.000000,0.000000,0.000000}%
\pgfsetstrokecolor{currentstroke}%
\pgfsetdash{}{0pt}%
\pgfpathmoveto{\pgfqpoint{0.636497in}{4.222999in}}%
\pgfpathlineto{\pgfqpoint{5.286497in}{4.222999in}}%
\pgfusepath{stroke}%
\end{pgfscope}%
\begin{pgfscope}%
\pgfsetbuttcap%
\pgfsetmiterjoin%
\definecolor{currentfill}{rgb}{0.300000,0.300000,0.300000}%
\pgfsetfillcolor{currentfill}%
\pgfsetfillopacity{0.500000}%
\pgfsetlinewidth{1.003750pt}%
\definecolor{currentstroke}{rgb}{0.300000,0.300000,0.300000}%
\pgfsetstrokecolor{currentstroke}%
\pgfsetstrokeopacity{0.500000}%
\pgfsetdash{}{0pt}%
\pgfpathmoveto{\pgfqpoint{0.350085in}{-0.027778in}}%
\pgfpathlineto{\pgfqpoint{5.628465in}{-0.027778in}}%
\pgfpathquadraticcurveto{\pgfqpoint{5.667353in}{-0.027778in}}{\pgfqpoint{5.667353in}{0.011111in}}%
\pgfpathlineto{\pgfqpoint{5.667353in}{0.586111in}}%
\pgfpathquadraticcurveto{\pgfqpoint{5.667353in}{0.624999in}}{\pgfqpoint{5.628465in}{0.624999in}}%
\pgfpathlineto{\pgfqpoint{0.350085in}{0.624999in}}%
\pgfpathquadraticcurveto{\pgfqpoint{0.311196in}{0.624999in}}{\pgfqpoint{0.311196in}{0.586111in}}%
\pgfpathlineto{\pgfqpoint{0.311196in}{0.011111in}}%
\pgfpathquadraticcurveto{\pgfqpoint{0.311196in}{-0.027778in}}{\pgfqpoint{0.350085in}{-0.027778in}}%
\pgfpathclose%
\pgfusepath{stroke,fill}%
\end{pgfscope}%
\begin{pgfscope}%
\pgfsetbuttcap%
\pgfsetmiterjoin%
\definecolor{currentfill}{rgb}{1.000000,1.000000,1.000000}%
\pgfsetfillcolor{currentfill}%
\pgfsetlinewidth{1.003750pt}%
\definecolor{currentstroke}{rgb}{0.800000,0.800000,0.800000}%
\pgfsetstrokecolor{currentstroke}%
\pgfsetdash{}{0pt}%
\pgfpathmoveto{\pgfqpoint{0.322307in}{0.000000in}}%
\pgfpathlineto{\pgfqpoint{5.600687in}{0.000000in}}%
\pgfpathquadraticcurveto{\pgfqpoint{5.639576in}{0.000000in}}{\pgfqpoint{5.639576in}{0.038889in}}%
\pgfpathlineto{\pgfqpoint{5.639576in}{0.613888in}}%
\pgfpathquadraticcurveto{\pgfqpoint{5.639576in}{0.652777in}}{\pgfqpoint{5.600687in}{0.652777in}}%
\pgfpathlineto{\pgfqpoint{0.322307in}{0.652777in}}%
\pgfpathquadraticcurveto{\pgfqpoint{0.283418in}{0.652777in}}{\pgfqpoint{0.283418in}{0.613888in}}%
\pgfpathlineto{\pgfqpoint{0.283418in}{0.038889in}}%
\pgfpathquadraticcurveto{\pgfqpoint{0.283418in}{0.000000in}}{\pgfqpoint{0.322307in}{0.000000in}}%
\pgfpathclose%
\pgfusepath{stroke,fill}%
\end{pgfscope}%
\begin{pgfscope}%
\pgfsetrectcap%
\pgfsetroundjoin%
\pgfsetlinewidth{1.505625pt}%
\definecolor{currentstroke}{rgb}{0.000000,0.000000,1.000000}%
\pgfsetstrokecolor{currentstroke}%
\pgfsetdash{}{0pt}%
\pgfpathmoveto{\pgfqpoint{0.361196in}{0.493055in}}%
\pgfpathlineto{\pgfqpoint{0.750085in}{0.493055in}}%
\pgfusepath{stroke}%
\end{pgfscope}%
\begin{pgfscope}%
\definecolor{textcolor}{rgb}{0.000000,0.000000,0.000000}%
\pgfsetstrokecolor{textcolor}%
\pgfsetfillcolor{textcolor}%
\pgftext[x=0.905641in,y=0.425000in,left,base]{\color{textcolor}\rmfamily\fontsize{14.000000}{16.800000}\selectfont Women (main person)}%
\end{pgfscope}%
\begin{pgfscope}%
\pgfsetrectcap%
\pgfsetroundjoin%
\pgfsetlinewidth{1.505625pt}%
\definecolor{currentstroke}{rgb}{0.000000,0.000000,0.000000}%
\pgfsetstrokecolor{currentstroke}%
\pgfsetdash{}{0pt}%
\pgfpathmoveto{\pgfqpoint{0.361196in}{0.195833in}}%
\pgfpathlineto{\pgfqpoint{0.750085in}{0.195833in}}%
\pgfusepath{stroke}%
\end{pgfscope}%
\begin{pgfscope}%
\definecolor{textcolor}{rgb}{0.000000,0.000000,0.000000}%
\pgfsetstrokecolor{textcolor}%
\pgfsetfillcolor{textcolor}%
\pgftext[x=0.905641in,y=0.127778in,left,base]{\color{textcolor}\rmfamily\fontsize{14.000000}{16.800000}\selectfont Men (main person)}%
\end{pgfscope}%
\begin{pgfscope}%
\pgfsetbuttcap%
\pgfsetroundjoin%
\pgfsetlinewidth{1.505625pt}%
\definecolor{currentstroke}{rgb}{1.000000,0.000000,0.000000}%
\pgfsetstrokecolor{currentstroke}%
\pgfsetdash{{5.550000pt}{2.400000pt}}{0.000000pt}%
\pgfpathmoveto{\pgfqpoint{3.204899in}{0.493055in}}%
\pgfpathlineto{\pgfqpoint{3.593788in}{0.493055in}}%
\pgfusepath{stroke}%
\end{pgfscope}%
\begin{pgfscope}%
\definecolor{textcolor}{rgb}{0.000000,0.000000,0.000000}%
\pgfsetstrokecolor{textcolor}%
\pgfsetfillcolor{textcolor}%
\pgftext[x=3.749344in,y=0.425000in,left,base]{\color{textcolor}\rmfamily\fontsize{14.000000}{16.800000}\selectfont Women (met person)}%
\end{pgfscope}%
\begin{pgfscope}%
\pgfsetbuttcap%
\pgfsetroundjoin%
\pgfsetlinewidth{1.505625pt}%
\definecolor{currentstroke}{rgb}{0.750000,0.000000,0.750000}%
\pgfsetstrokecolor{currentstroke}%
\pgfsetdash{{5.550000pt}{2.400000pt}}{0.000000pt}%
\pgfpathmoveto{\pgfqpoint{3.204899in}{0.195833in}}%
\pgfpathlineto{\pgfqpoint{3.593788in}{0.195833in}}%
\pgfusepath{stroke}%
\end{pgfscope}%
\begin{pgfscope}%
\definecolor{textcolor}{rgb}{0.000000,0.000000,0.000000}%
\pgfsetstrokecolor{textcolor}%
\pgfsetfillcolor{textcolor}%
\pgftext[x=3.749344in,y=0.127778in,left,base]{\color{textcolor}\rmfamily\fontsize{14.000000}{16.800000}\selectfont Men (met person)}%
\end{pgfscope}%
\end{pgfpicture}%
\makeatother%
\endgroup%
 } 
\end{subfigure}
\end{center}

\hspace{20em}

\begin{center}
\begin{subfigure}{.49\textwidth}
\centering
% include second image
\caption{Average assets}
\label{fig:sub-third12}
\scalebox{0.5}{%% Creator: Matplotlib, PGF backend
%%
%% To include the figure in your LaTeX document, write
%%   \input{<filename>.pgf}
%%
%% Make sure the required packages are loaded in your preamble
%%   \usepackage{pgf}
%%
%% Figures using additional raster images can only be included by \input if
%% they are in the same directory as the main LaTeX file. For loading figures
%% from other directories you can use the `import` package
%%   \usepackage{import}
%% and then include the figures with
%%   \import{<path to file>}{<filename>.pgf}
%%
%% Matplotlib used the following preamble
%%
\begingroup%
\makeatletter%
\begin{pgfpicture}%
\pgfpathrectangle{\pgfpointorigin}{\pgfqpoint{5.445247in}{4.222999in}}%
\pgfusepath{use as bounding box, clip}%
\begin{pgfscope}%
\pgfsetbuttcap%
\pgfsetmiterjoin%
\definecolor{currentfill}{rgb}{1.000000,1.000000,1.000000}%
\pgfsetfillcolor{currentfill}%
\pgfsetlinewidth{0.000000pt}%
\definecolor{currentstroke}{rgb}{1.000000,1.000000,1.000000}%
\pgfsetstrokecolor{currentstroke}%
\pgfsetdash{}{0pt}%
\pgfpathmoveto{\pgfqpoint{0.000000in}{0.000000in}}%
\pgfpathlineto{\pgfqpoint{5.445247in}{0.000000in}}%
\pgfpathlineto{\pgfqpoint{5.445247in}{4.222999in}}%
\pgfpathlineto{\pgfqpoint{0.000000in}{4.222999in}}%
\pgfpathclose%
\pgfusepath{fill}%
\end{pgfscope}%
\begin{pgfscope}%
\pgfsetbuttcap%
\pgfsetmiterjoin%
\definecolor{currentfill}{rgb}{1.000000,1.000000,1.000000}%
\pgfsetfillcolor{currentfill}%
\pgfsetlinewidth{0.000000pt}%
\definecolor{currentstroke}{rgb}{0.000000,0.000000,0.000000}%
\pgfsetstrokecolor{currentstroke}%
\pgfsetstrokeopacity{0.000000}%
\pgfsetdash{}{0pt}%
\pgfpathmoveto{\pgfqpoint{0.442168in}{1.202999in}}%
\pgfpathlineto{\pgfqpoint{5.092168in}{1.202999in}}%
\pgfpathlineto{\pgfqpoint{5.092168in}{4.222999in}}%
\pgfpathlineto{\pgfqpoint{0.442168in}{4.222999in}}%
\pgfpathclose%
\pgfusepath{fill}%
\end{pgfscope}%
\begin{pgfscope}%
\pgfsetbuttcap%
\pgfsetroundjoin%
\definecolor{currentfill}{rgb}{0.000000,0.000000,0.000000}%
\pgfsetfillcolor{currentfill}%
\pgfsetlinewidth{0.803000pt}%
\definecolor{currentstroke}{rgb}{0.000000,0.000000,0.000000}%
\pgfsetstrokecolor{currentstroke}%
\pgfsetdash{}{0pt}%
\pgfsys@defobject{currentmarker}{\pgfqpoint{0.000000in}{-0.048611in}}{\pgfqpoint{0.000000in}{0.000000in}}{%
\pgfpathmoveto{\pgfqpoint{0.000000in}{0.000000in}}%
\pgfpathlineto{\pgfqpoint{0.000000in}{-0.048611in}}%
\pgfusepath{stroke,fill}%
}%
\begin{pgfscope}%
\pgfsys@transformshift{0.653532in}{1.202999in}%
\pgfsys@useobject{currentmarker}{}%
\end{pgfscope}%
\end{pgfscope}%
\begin{pgfscope}%
\definecolor{textcolor}{rgb}{0.000000,0.000000,0.000000}%
\pgfsetstrokecolor{textcolor}%
\pgfsetfillcolor{textcolor}%
\pgftext[x=0.653532in,y=1.105777in,,top]{\color{textcolor}\rmfamily\fontsize{11.000000}{13.200000}\selectfont \(\displaystyle 25\)}%
\end{pgfscope}%
\begin{pgfscope}%
\pgfsetbuttcap%
\pgfsetroundjoin%
\definecolor{currentfill}{rgb}{0.000000,0.000000,0.000000}%
\pgfsetfillcolor{currentfill}%
\pgfsetlinewidth{0.803000pt}%
\definecolor{currentstroke}{rgb}{0.000000,0.000000,0.000000}%
\pgfsetstrokecolor{currentstroke}%
\pgfsetdash{}{0pt}%
\pgfsys@defobject{currentmarker}{\pgfqpoint{0.000000in}{-0.048611in}}{\pgfqpoint{0.000000in}{0.000000in}}{%
\pgfpathmoveto{\pgfqpoint{0.000000in}{0.000000in}}%
\pgfpathlineto{\pgfqpoint{0.000000in}{-0.048611in}}%
\pgfusepath{stroke,fill}%
}%
\begin{pgfscope}%
\pgfsys@transformshift{1.275189in}{1.202999in}%
\pgfsys@useobject{currentmarker}{}%
\end{pgfscope}%
\end{pgfscope}%
\begin{pgfscope}%
\definecolor{textcolor}{rgb}{0.000000,0.000000,0.000000}%
\pgfsetstrokecolor{textcolor}%
\pgfsetfillcolor{textcolor}%
\pgftext[x=1.275189in,y=1.105777in,,top]{\color{textcolor}\rmfamily\fontsize{11.000000}{13.200000}\selectfont \(\displaystyle 30\)}%
\end{pgfscope}%
\begin{pgfscope}%
\pgfsetbuttcap%
\pgfsetroundjoin%
\definecolor{currentfill}{rgb}{0.000000,0.000000,0.000000}%
\pgfsetfillcolor{currentfill}%
\pgfsetlinewidth{0.803000pt}%
\definecolor{currentstroke}{rgb}{0.000000,0.000000,0.000000}%
\pgfsetstrokecolor{currentstroke}%
\pgfsetdash{}{0pt}%
\pgfsys@defobject{currentmarker}{\pgfqpoint{0.000000in}{-0.048611in}}{\pgfqpoint{0.000000in}{0.000000in}}{%
\pgfpathmoveto{\pgfqpoint{0.000000in}{0.000000in}}%
\pgfpathlineto{\pgfqpoint{0.000000in}{-0.048611in}}%
\pgfusepath{stroke,fill}%
}%
\begin{pgfscope}%
\pgfsys@transformshift{1.896847in}{1.202999in}%
\pgfsys@useobject{currentmarker}{}%
\end{pgfscope}%
\end{pgfscope}%
\begin{pgfscope}%
\definecolor{textcolor}{rgb}{0.000000,0.000000,0.000000}%
\pgfsetstrokecolor{textcolor}%
\pgfsetfillcolor{textcolor}%
\pgftext[x=1.896847in,y=1.105777in,,top]{\color{textcolor}\rmfamily\fontsize{11.000000}{13.200000}\selectfont \(\displaystyle 35\)}%
\end{pgfscope}%
\begin{pgfscope}%
\pgfsetbuttcap%
\pgfsetroundjoin%
\definecolor{currentfill}{rgb}{0.000000,0.000000,0.000000}%
\pgfsetfillcolor{currentfill}%
\pgfsetlinewidth{0.803000pt}%
\definecolor{currentstroke}{rgb}{0.000000,0.000000,0.000000}%
\pgfsetstrokecolor{currentstroke}%
\pgfsetdash{}{0pt}%
\pgfsys@defobject{currentmarker}{\pgfqpoint{0.000000in}{-0.048611in}}{\pgfqpoint{0.000000in}{0.000000in}}{%
\pgfpathmoveto{\pgfqpoint{0.000000in}{0.000000in}}%
\pgfpathlineto{\pgfqpoint{0.000000in}{-0.048611in}}%
\pgfusepath{stroke,fill}%
}%
\begin{pgfscope}%
\pgfsys@transformshift{2.518505in}{1.202999in}%
\pgfsys@useobject{currentmarker}{}%
\end{pgfscope}%
\end{pgfscope}%
\begin{pgfscope}%
\definecolor{textcolor}{rgb}{0.000000,0.000000,0.000000}%
\pgfsetstrokecolor{textcolor}%
\pgfsetfillcolor{textcolor}%
\pgftext[x=2.518505in,y=1.105777in,,top]{\color{textcolor}\rmfamily\fontsize{11.000000}{13.200000}\selectfont \(\displaystyle 40\)}%
\end{pgfscope}%
\begin{pgfscope}%
\pgfsetbuttcap%
\pgfsetroundjoin%
\definecolor{currentfill}{rgb}{0.000000,0.000000,0.000000}%
\pgfsetfillcolor{currentfill}%
\pgfsetlinewidth{0.803000pt}%
\definecolor{currentstroke}{rgb}{0.000000,0.000000,0.000000}%
\pgfsetstrokecolor{currentstroke}%
\pgfsetdash{}{0pt}%
\pgfsys@defobject{currentmarker}{\pgfqpoint{0.000000in}{-0.048611in}}{\pgfqpoint{0.000000in}{0.000000in}}{%
\pgfpathmoveto{\pgfqpoint{0.000000in}{0.000000in}}%
\pgfpathlineto{\pgfqpoint{0.000000in}{-0.048611in}}%
\pgfusepath{stroke,fill}%
}%
\begin{pgfscope}%
\pgfsys@transformshift{3.140163in}{1.202999in}%
\pgfsys@useobject{currentmarker}{}%
\end{pgfscope}%
\end{pgfscope}%
\begin{pgfscope}%
\definecolor{textcolor}{rgb}{0.000000,0.000000,0.000000}%
\pgfsetstrokecolor{textcolor}%
\pgfsetfillcolor{textcolor}%
\pgftext[x=3.140163in,y=1.105777in,,top]{\color{textcolor}\rmfamily\fontsize{11.000000}{13.200000}\selectfont \(\displaystyle 45\)}%
\end{pgfscope}%
\begin{pgfscope}%
\pgfsetbuttcap%
\pgfsetroundjoin%
\definecolor{currentfill}{rgb}{0.000000,0.000000,0.000000}%
\pgfsetfillcolor{currentfill}%
\pgfsetlinewidth{0.803000pt}%
\definecolor{currentstroke}{rgb}{0.000000,0.000000,0.000000}%
\pgfsetstrokecolor{currentstroke}%
\pgfsetdash{}{0pt}%
\pgfsys@defobject{currentmarker}{\pgfqpoint{0.000000in}{-0.048611in}}{\pgfqpoint{0.000000in}{0.000000in}}{%
\pgfpathmoveto{\pgfqpoint{0.000000in}{0.000000in}}%
\pgfpathlineto{\pgfqpoint{0.000000in}{-0.048611in}}%
\pgfusepath{stroke,fill}%
}%
\begin{pgfscope}%
\pgfsys@transformshift{3.761820in}{1.202999in}%
\pgfsys@useobject{currentmarker}{}%
\end{pgfscope}%
\end{pgfscope}%
\begin{pgfscope}%
\definecolor{textcolor}{rgb}{0.000000,0.000000,0.000000}%
\pgfsetstrokecolor{textcolor}%
\pgfsetfillcolor{textcolor}%
\pgftext[x=3.761820in,y=1.105777in,,top]{\color{textcolor}\rmfamily\fontsize{11.000000}{13.200000}\selectfont \(\displaystyle 50\)}%
\end{pgfscope}%
\begin{pgfscope}%
\pgfsetbuttcap%
\pgfsetroundjoin%
\definecolor{currentfill}{rgb}{0.000000,0.000000,0.000000}%
\pgfsetfillcolor{currentfill}%
\pgfsetlinewidth{0.803000pt}%
\definecolor{currentstroke}{rgb}{0.000000,0.000000,0.000000}%
\pgfsetstrokecolor{currentstroke}%
\pgfsetdash{}{0pt}%
\pgfsys@defobject{currentmarker}{\pgfqpoint{0.000000in}{-0.048611in}}{\pgfqpoint{0.000000in}{0.000000in}}{%
\pgfpathmoveto{\pgfqpoint{0.000000in}{0.000000in}}%
\pgfpathlineto{\pgfqpoint{0.000000in}{-0.048611in}}%
\pgfusepath{stroke,fill}%
}%
\begin{pgfscope}%
\pgfsys@transformshift{4.383478in}{1.202999in}%
\pgfsys@useobject{currentmarker}{}%
\end{pgfscope}%
\end{pgfscope}%
\begin{pgfscope}%
\definecolor{textcolor}{rgb}{0.000000,0.000000,0.000000}%
\pgfsetstrokecolor{textcolor}%
\pgfsetfillcolor{textcolor}%
\pgftext[x=4.383478in,y=1.105777in,,top]{\color{textcolor}\rmfamily\fontsize{11.000000}{13.200000}\selectfont \(\displaystyle 55\)}%
\end{pgfscope}%
\begin{pgfscope}%
\pgfsetbuttcap%
\pgfsetroundjoin%
\definecolor{currentfill}{rgb}{0.000000,0.000000,0.000000}%
\pgfsetfillcolor{currentfill}%
\pgfsetlinewidth{0.803000pt}%
\definecolor{currentstroke}{rgb}{0.000000,0.000000,0.000000}%
\pgfsetstrokecolor{currentstroke}%
\pgfsetdash{}{0pt}%
\pgfsys@defobject{currentmarker}{\pgfqpoint{0.000000in}{-0.048611in}}{\pgfqpoint{0.000000in}{0.000000in}}{%
\pgfpathmoveto{\pgfqpoint{0.000000in}{0.000000in}}%
\pgfpathlineto{\pgfqpoint{0.000000in}{-0.048611in}}%
\pgfusepath{stroke,fill}%
}%
\begin{pgfscope}%
\pgfsys@transformshift{5.005136in}{1.202999in}%
\pgfsys@useobject{currentmarker}{}%
\end{pgfscope}%
\end{pgfscope}%
\begin{pgfscope}%
\definecolor{textcolor}{rgb}{0.000000,0.000000,0.000000}%
\pgfsetstrokecolor{textcolor}%
\pgfsetfillcolor{textcolor}%
\pgftext[x=5.005136in,y=1.105777in,,top]{\color{textcolor}\rmfamily\fontsize{11.000000}{13.200000}\selectfont \(\displaystyle 60\)}%
\end{pgfscope}%
\begin{pgfscope}%
\definecolor{textcolor}{rgb}{0.000000,0.000000,0.000000}%
\pgfsetstrokecolor{textcolor}%
\pgfsetfillcolor{textcolor}%
\pgftext[x=2.767168in,y=0.915037in,,top]{\color{textcolor}\rmfamily\fontsize{16.000000}{19.200000}\selectfont Age}%
\end{pgfscope}%
\begin{pgfscope}%
\pgfsetbuttcap%
\pgfsetroundjoin%
\definecolor{currentfill}{rgb}{0.000000,0.000000,0.000000}%
\pgfsetfillcolor{currentfill}%
\pgfsetlinewidth{0.803000pt}%
\definecolor{currentstroke}{rgb}{0.000000,0.000000,0.000000}%
\pgfsetstrokecolor{currentstroke}%
\pgfsetdash{}{0pt}%
\pgfsys@defobject{currentmarker}{\pgfqpoint{-0.048611in}{0.000000in}}{\pgfqpoint{0.000000in}{0.000000in}}{%
\pgfpathmoveto{\pgfqpoint{0.000000in}{0.000000in}}%
\pgfpathlineto{\pgfqpoint{-0.048611in}{0.000000in}}%
\pgfusepath{stroke,fill}%
}%
\begin{pgfscope}%
\pgfsys@transformshift{0.442168in}{1.339949in}%
\pgfsys@useobject{currentmarker}{}%
\end{pgfscope}%
\end{pgfscope}%
\begin{pgfscope}%
\definecolor{textcolor}{rgb}{0.000000,0.000000,0.000000}%
\pgfsetstrokecolor{textcolor}%
\pgfsetfillcolor{textcolor}%
\pgftext[x=0.268904in,y=1.287143in,left,base]{\color{textcolor}\rmfamily\fontsize{11.000000}{13.200000}\selectfont \(\displaystyle 0\)}%
\end{pgfscope}%
\begin{pgfscope}%
\pgfsetbuttcap%
\pgfsetroundjoin%
\definecolor{currentfill}{rgb}{0.000000,0.000000,0.000000}%
\pgfsetfillcolor{currentfill}%
\pgfsetlinewidth{0.803000pt}%
\definecolor{currentstroke}{rgb}{0.000000,0.000000,0.000000}%
\pgfsetstrokecolor{currentstroke}%
\pgfsetdash{}{0pt}%
\pgfsys@defobject{currentmarker}{\pgfqpoint{-0.048611in}{0.000000in}}{\pgfqpoint{0.000000in}{0.000000in}}{%
\pgfpathmoveto{\pgfqpoint{0.000000in}{0.000000in}}%
\pgfpathlineto{\pgfqpoint{-0.048611in}{0.000000in}}%
\pgfusepath{stroke,fill}%
}%
\begin{pgfscope}%
\pgfsys@transformshift{0.442168in}{1.989571in}%
\pgfsys@useobject{currentmarker}{}%
\end{pgfscope}%
\end{pgfscope}%
\begin{pgfscope}%
\definecolor{textcolor}{rgb}{0.000000,0.000000,0.000000}%
\pgfsetstrokecolor{textcolor}%
\pgfsetfillcolor{textcolor}%
\pgftext[x=0.268904in,y=1.936764in,left,base]{\color{textcolor}\rmfamily\fontsize{11.000000}{13.200000}\selectfont \(\displaystyle 2\)}%
\end{pgfscope}%
\begin{pgfscope}%
\pgfsetbuttcap%
\pgfsetroundjoin%
\definecolor{currentfill}{rgb}{0.000000,0.000000,0.000000}%
\pgfsetfillcolor{currentfill}%
\pgfsetlinewidth{0.803000pt}%
\definecolor{currentstroke}{rgb}{0.000000,0.000000,0.000000}%
\pgfsetstrokecolor{currentstroke}%
\pgfsetdash{}{0pt}%
\pgfsys@defobject{currentmarker}{\pgfqpoint{-0.048611in}{0.000000in}}{\pgfqpoint{0.000000in}{0.000000in}}{%
\pgfpathmoveto{\pgfqpoint{0.000000in}{0.000000in}}%
\pgfpathlineto{\pgfqpoint{-0.048611in}{0.000000in}}%
\pgfusepath{stroke,fill}%
}%
\begin{pgfscope}%
\pgfsys@transformshift{0.442168in}{2.639193in}%
\pgfsys@useobject{currentmarker}{}%
\end{pgfscope}%
\end{pgfscope}%
\begin{pgfscope}%
\definecolor{textcolor}{rgb}{0.000000,0.000000,0.000000}%
\pgfsetstrokecolor{textcolor}%
\pgfsetfillcolor{textcolor}%
\pgftext[x=0.268904in,y=2.586386in,left,base]{\color{textcolor}\rmfamily\fontsize{11.000000}{13.200000}\selectfont \(\displaystyle 4\)}%
\end{pgfscope}%
\begin{pgfscope}%
\pgfsetbuttcap%
\pgfsetroundjoin%
\definecolor{currentfill}{rgb}{0.000000,0.000000,0.000000}%
\pgfsetfillcolor{currentfill}%
\pgfsetlinewidth{0.803000pt}%
\definecolor{currentstroke}{rgb}{0.000000,0.000000,0.000000}%
\pgfsetstrokecolor{currentstroke}%
\pgfsetdash{}{0pt}%
\pgfsys@defobject{currentmarker}{\pgfqpoint{-0.048611in}{0.000000in}}{\pgfqpoint{0.000000in}{0.000000in}}{%
\pgfpathmoveto{\pgfqpoint{0.000000in}{0.000000in}}%
\pgfpathlineto{\pgfqpoint{-0.048611in}{0.000000in}}%
\pgfusepath{stroke,fill}%
}%
\begin{pgfscope}%
\pgfsys@transformshift{0.442168in}{3.288814in}%
\pgfsys@useobject{currentmarker}{}%
\end{pgfscope}%
\end{pgfscope}%
\begin{pgfscope}%
\definecolor{textcolor}{rgb}{0.000000,0.000000,0.000000}%
\pgfsetstrokecolor{textcolor}%
\pgfsetfillcolor{textcolor}%
\pgftext[x=0.268904in,y=3.236008in,left,base]{\color{textcolor}\rmfamily\fontsize{11.000000}{13.200000}\selectfont \(\displaystyle 6\)}%
\end{pgfscope}%
\begin{pgfscope}%
\pgfsetbuttcap%
\pgfsetroundjoin%
\definecolor{currentfill}{rgb}{0.000000,0.000000,0.000000}%
\pgfsetfillcolor{currentfill}%
\pgfsetlinewidth{0.803000pt}%
\definecolor{currentstroke}{rgb}{0.000000,0.000000,0.000000}%
\pgfsetstrokecolor{currentstroke}%
\pgfsetdash{}{0pt}%
\pgfsys@defobject{currentmarker}{\pgfqpoint{-0.048611in}{0.000000in}}{\pgfqpoint{0.000000in}{0.000000in}}{%
\pgfpathmoveto{\pgfqpoint{0.000000in}{0.000000in}}%
\pgfpathlineto{\pgfqpoint{-0.048611in}{0.000000in}}%
\pgfusepath{stroke,fill}%
}%
\begin{pgfscope}%
\pgfsys@transformshift{0.442168in}{3.938436in}%
\pgfsys@useobject{currentmarker}{}%
\end{pgfscope}%
\end{pgfscope}%
\begin{pgfscope}%
\definecolor{textcolor}{rgb}{0.000000,0.000000,0.000000}%
\pgfsetstrokecolor{textcolor}%
\pgfsetfillcolor{textcolor}%
\pgftext[x=0.268904in,y=3.885629in,left,base]{\color{textcolor}\rmfamily\fontsize{11.000000}{13.200000}\selectfont \(\displaystyle 8\)}%
\end{pgfscope}%
\begin{pgfscope}%
\definecolor{textcolor}{rgb}{0.000000,0.000000,0.000000}%
\pgfsetstrokecolor{textcolor}%
\pgfsetfillcolor{textcolor}%
\pgftext[x=0.213349in,y=2.712999in,,bottom,rotate=90.000000]{\color{textcolor}\rmfamily\fontsize{16.000000}{19.200000}\selectfont Assets at meeting---mean}%
\end{pgfscope}%
\begin{pgfscope}%
\pgfpathrectangle{\pgfqpoint{0.442168in}{1.202999in}}{\pgfqpoint{4.650000in}{3.020000in}}%
\pgfusepath{clip}%
\pgfsetrectcap%
\pgfsetroundjoin%
\pgfsetlinewidth{1.505625pt}%
\definecolor{currentstroke}{rgb}{0.000000,0.000000,1.000000}%
\pgfsetstrokecolor{currentstroke}%
\pgfsetdash{}{0pt}%
\pgfpathmoveto{\pgfqpoint{0.653532in}{1.340821in}}%
\pgfpathlineto{\pgfqpoint{0.777863in}{1.340346in}}%
\pgfpathlineto{\pgfqpoint{0.902195in}{1.340272in}}%
\pgfpathlineto{\pgfqpoint{1.026526in}{1.349425in}}%
\pgfpathlineto{\pgfqpoint{1.150858in}{1.352561in}}%
\pgfpathlineto{\pgfqpoint{1.275189in}{1.363317in}}%
\pgfpathlineto{\pgfqpoint{1.399521in}{1.360829in}}%
\pgfpathlineto{\pgfqpoint{1.523853in}{1.392601in}}%
\pgfpathlineto{\pgfqpoint{1.648184in}{1.428244in}}%
\pgfpathlineto{\pgfqpoint{1.772516in}{1.436894in}}%
\pgfpathlineto{\pgfqpoint{1.896847in}{1.447661in}}%
\pgfpathlineto{\pgfqpoint{2.021179in}{1.464352in}}%
\pgfpathlineto{\pgfqpoint{2.145510in}{1.501415in}}%
\pgfpathlineto{\pgfqpoint{2.269842in}{1.531928in}}%
\pgfpathlineto{\pgfqpoint{2.394173in}{1.552445in}}%
\pgfpathlineto{\pgfqpoint{2.518505in}{1.581647in}}%
\pgfpathlineto{\pgfqpoint{2.642837in}{1.545783in}}%
\pgfpathlineto{\pgfqpoint{2.767168in}{1.553402in}}%
\pgfpathlineto{\pgfqpoint{2.891500in}{1.629974in}}%
\pgfpathlineto{\pgfqpoint{3.015831in}{1.560781in}}%
\pgfpathlineto{\pgfqpoint{3.140163in}{1.567947in}}%
\pgfpathlineto{\pgfqpoint{3.264494in}{1.589035in}}%
\pgfpathlineto{\pgfqpoint{3.388826in}{1.602866in}}%
\pgfpathlineto{\pgfqpoint{3.513157in}{1.635781in}}%
\pgfpathlineto{\pgfqpoint{3.637489in}{1.567635in}}%
\pgfpathlineto{\pgfqpoint{3.761820in}{1.591787in}}%
\pgfpathlineto{\pgfqpoint{3.886152in}{1.606634in}}%
\pgfpathlineto{\pgfqpoint{4.010484in}{1.671008in}}%
\pgfpathlineto{\pgfqpoint{4.134815in}{1.658347in}}%
\pgfpathlineto{\pgfqpoint{4.259147in}{1.677135in}}%
\pgfpathlineto{\pgfqpoint{4.383478in}{1.812442in}}%
\pgfpathlineto{\pgfqpoint{4.507810in}{1.800682in}}%
\pgfpathlineto{\pgfqpoint{4.632141in}{1.734283in}}%
\pgfpathlineto{\pgfqpoint{4.756473in}{1.759016in}}%
\pgfpathlineto{\pgfqpoint{4.880804in}{1.800473in}}%
\pgfusepath{stroke}%
\end{pgfscope}%
\begin{pgfscope}%
\pgfpathrectangle{\pgfqpoint{0.442168in}{1.202999in}}{\pgfqpoint{4.650000in}{3.020000in}}%
\pgfusepath{clip}%
\pgfsetrectcap%
\pgfsetroundjoin%
\pgfsetlinewidth{1.505625pt}%
\definecolor{currentstroke}{rgb}{0.000000,0.000000,0.000000}%
\pgfsetstrokecolor{currentstroke}%
\pgfsetdash{}{0pt}%
\pgfpathmoveto{\pgfqpoint{0.653532in}{1.421957in}}%
\pgfpathlineto{\pgfqpoint{0.777863in}{1.453738in}}%
\pgfpathlineto{\pgfqpoint{0.902195in}{1.468975in}}%
\pgfpathlineto{\pgfqpoint{1.026526in}{1.492342in}}%
\pgfpathlineto{\pgfqpoint{1.150858in}{1.520396in}}%
\pgfpathlineto{\pgfqpoint{1.275189in}{1.560186in}}%
\pgfpathlineto{\pgfqpoint{1.399521in}{1.601368in}}%
\pgfpathlineto{\pgfqpoint{1.523853in}{1.634259in}}%
\pgfpathlineto{\pgfqpoint{1.648184in}{1.667023in}}%
\pgfpathlineto{\pgfqpoint{1.772516in}{1.654373in}}%
\pgfpathlineto{\pgfqpoint{1.896847in}{1.664114in}}%
\pgfpathlineto{\pgfqpoint{2.021179in}{1.797674in}}%
\pgfpathlineto{\pgfqpoint{2.145510in}{1.889213in}}%
\pgfpathlineto{\pgfqpoint{2.269842in}{1.841471in}}%
\pgfpathlineto{\pgfqpoint{2.394173in}{1.879810in}}%
\pgfpathlineto{\pgfqpoint{2.518505in}{1.951579in}}%
\pgfpathlineto{\pgfqpoint{2.642837in}{2.177363in}}%
\pgfpathlineto{\pgfqpoint{2.767168in}{2.120683in}}%
\pgfpathlineto{\pgfqpoint{2.891500in}{2.189434in}}%
\pgfpathlineto{\pgfqpoint{3.015831in}{2.250801in}}%
\pgfpathlineto{\pgfqpoint{3.140163in}{2.378866in}}%
\pgfpathlineto{\pgfqpoint{3.264494in}{2.367633in}}%
\pgfpathlineto{\pgfqpoint{3.388826in}{2.383824in}}%
\pgfpathlineto{\pgfqpoint{3.513157in}{2.547394in}}%
\pgfpathlineto{\pgfqpoint{3.637489in}{2.540139in}}%
\pgfpathlineto{\pgfqpoint{3.761820in}{2.547717in}}%
\pgfpathlineto{\pgfqpoint{3.886152in}{2.657490in}}%
\pgfpathlineto{\pgfqpoint{4.010484in}{2.974140in}}%
\pgfpathlineto{\pgfqpoint{4.134815in}{3.532123in}}%
\pgfpathlineto{\pgfqpoint{4.259147in}{4.085727in}}%
\pgfpathlineto{\pgfqpoint{4.383478in}{3.741289in}}%
\pgfpathlineto{\pgfqpoint{4.507810in}{3.765286in}}%
\pgfpathlineto{\pgfqpoint{4.632141in}{3.452131in}}%
\pgfpathlineto{\pgfqpoint{4.756473in}{3.133871in}}%
\pgfpathlineto{\pgfqpoint{4.880804in}{3.168481in}}%
\pgfusepath{stroke}%
\end{pgfscope}%
\begin{pgfscope}%
\pgfpathrectangle{\pgfqpoint{0.442168in}{1.202999in}}{\pgfqpoint{4.650000in}{3.020000in}}%
\pgfusepath{clip}%
\pgfsetbuttcap%
\pgfsetroundjoin%
\pgfsetlinewidth{1.505625pt}%
\definecolor{currentstroke}{rgb}{1.000000,0.000000,0.000000}%
\pgfsetstrokecolor{currentstroke}%
\pgfsetdash{{5.550000pt}{2.400000pt}}{0.000000pt}%
\pgfpathmoveto{\pgfqpoint{0.653532in}{1.421756in}}%
\pgfpathlineto{\pgfqpoint{0.777863in}{1.452988in}}%
\pgfpathlineto{\pgfqpoint{0.902195in}{1.468381in}}%
\pgfpathlineto{\pgfqpoint{1.026526in}{1.492106in}}%
\pgfpathlineto{\pgfqpoint{1.150858in}{1.517866in}}%
\pgfpathlineto{\pgfqpoint{1.275189in}{1.557178in}}%
\pgfpathlineto{\pgfqpoint{1.399521in}{1.595297in}}%
\pgfpathlineto{\pgfqpoint{1.523853in}{1.630435in}}%
\pgfpathlineto{\pgfqpoint{1.648184in}{1.659978in}}%
\pgfpathlineto{\pgfqpoint{1.772516in}{1.648188in}}%
\pgfpathlineto{\pgfqpoint{1.896847in}{1.654023in}}%
\pgfpathlineto{\pgfqpoint{2.021179in}{1.779976in}}%
\pgfpathlineto{\pgfqpoint{2.145510in}{1.867886in}}%
\pgfpathlineto{\pgfqpoint{2.269842in}{1.821670in}}%
\pgfpathlineto{\pgfqpoint{2.394173in}{1.864241in}}%
\pgfpathlineto{\pgfqpoint{2.518505in}{1.928462in}}%
\pgfpathlineto{\pgfqpoint{2.642837in}{2.149070in}}%
\pgfpathlineto{\pgfqpoint{2.767168in}{2.093303in}}%
\pgfpathlineto{\pgfqpoint{2.891500in}{2.168497in}}%
\pgfpathlineto{\pgfqpoint{3.015831in}{2.232765in}}%
\pgfpathlineto{\pgfqpoint{3.140163in}{2.346053in}}%
\pgfpathlineto{\pgfqpoint{3.264494in}{2.335910in}}%
\pgfpathlineto{\pgfqpoint{3.388826in}{2.346877in}}%
\pgfpathlineto{\pgfqpoint{3.513157in}{2.496360in}}%
\pgfpathlineto{\pgfqpoint{3.637489in}{2.482243in}}%
\pgfpathlineto{\pgfqpoint{3.761820in}{2.513801in}}%
\pgfpathlineto{\pgfqpoint{3.886152in}{2.626516in}}%
\pgfpathlineto{\pgfqpoint{4.010484in}{2.946422in}}%
\pgfpathlineto{\pgfqpoint{4.134815in}{3.477676in}}%
\pgfpathlineto{\pgfqpoint{4.259147in}{3.996419in}}%
\pgfpathlineto{\pgfqpoint{4.383478in}{3.641222in}}%
\pgfpathlineto{\pgfqpoint{4.507810in}{3.668185in}}%
\pgfpathlineto{\pgfqpoint{4.632141in}{3.377139in}}%
\pgfpathlineto{\pgfqpoint{4.756473in}{3.069142in}}%
\pgfpathlineto{\pgfqpoint{4.880804in}{3.111157in}}%
\pgfusepath{stroke}%
\end{pgfscope}%
\begin{pgfscope}%
\pgfpathrectangle{\pgfqpoint{0.442168in}{1.202999in}}{\pgfqpoint{4.650000in}{3.020000in}}%
\pgfusepath{clip}%
\pgfsetbuttcap%
\pgfsetroundjoin%
\pgfsetlinewidth{1.505625pt}%
\definecolor{currentstroke}{rgb}{0.750000,0.000000,0.750000}%
\pgfsetstrokecolor{currentstroke}%
\pgfsetdash{{5.550000pt}{2.400000pt}}{0.000000pt}%
\pgfpathmoveto{\pgfqpoint{0.653532in}{1.340821in}}%
\pgfpathlineto{\pgfqpoint{0.777863in}{1.340346in}}%
\pgfpathlineto{\pgfqpoint{0.902195in}{1.340272in}}%
\pgfpathlineto{\pgfqpoint{1.026526in}{1.349513in}}%
\pgfpathlineto{\pgfqpoint{1.150858in}{1.352658in}}%
\pgfpathlineto{\pgfqpoint{1.275189in}{1.363821in}}%
\pgfpathlineto{\pgfqpoint{1.399521in}{1.361209in}}%
\pgfpathlineto{\pgfqpoint{1.523853in}{1.393162in}}%
\pgfpathlineto{\pgfqpoint{1.648184in}{1.429282in}}%
\pgfpathlineto{\pgfqpoint{1.772516in}{1.437956in}}%
\pgfpathlineto{\pgfqpoint{1.896847in}{1.449858in}}%
\pgfpathlineto{\pgfqpoint{2.021179in}{1.467940in}}%
\pgfpathlineto{\pgfqpoint{2.145510in}{1.505095in}}%
\pgfpathlineto{\pgfqpoint{2.269842in}{1.536648in}}%
\pgfpathlineto{\pgfqpoint{2.394173in}{1.558779in}}%
\pgfpathlineto{\pgfqpoint{2.518505in}{1.594353in}}%
\pgfpathlineto{\pgfqpoint{2.642837in}{1.556595in}}%
\pgfpathlineto{\pgfqpoint{2.767168in}{1.560352in}}%
\pgfpathlineto{\pgfqpoint{2.891500in}{1.639536in}}%
\pgfpathlineto{\pgfqpoint{3.015831in}{1.568392in}}%
\pgfpathlineto{\pgfqpoint{3.140163in}{1.581942in}}%
\pgfpathlineto{\pgfqpoint{3.264494in}{1.601112in}}%
\pgfpathlineto{\pgfqpoint{3.388826in}{1.614449in}}%
\pgfpathlineto{\pgfqpoint{3.513157in}{1.649331in}}%
\pgfpathlineto{\pgfqpoint{3.637489in}{1.575630in}}%
\pgfpathlineto{\pgfqpoint{3.761820in}{1.599458in}}%
\pgfpathlineto{\pgfqpoint{3.886152in}{1.618315in}}%
\pgfpathlineto{\pgfqpoint{4.010484in}{1.684388in}}%
\pgfpathlineto{\pgfqpoint{4.134815in}{1.667734in}}%
\pgfpathlineto{\pgfqpoint{4.259147in}{1.697353in}}%
\pgfpathlineto{\pgfqpoint{4.383478in}{1.847338in}}%
\pgfpathlineto{\pgfqpoint{4.507810in}{1.822451in}}%
\pgfpathlineto{\pgfqpoint{4.632141in}{1.749796in}}%
\pgfpathlineto{\pgfqpoint{4.756473in}{1.780171in}}%
\pgfpathlineto{\pgfqpoint{4.880804in}{1.824602in}}%
\pgfusepath{stroke}%
\end{pgfscope}%
\begin{pgfscope}%
\pgfsetrectcap%
\pgfsetmiterjoin%
\pgfsetlinewidth{0.803000pt}%
\definecolor{currentstroke}{rgb}{0.000000,0.000000,0.000000}%
\pgfsetstrokecolor{currentstroke}%
\pgfsetdash{}{0pt}%
\pgfpathmoveto{\pgfqpoint{0.442168in}{1.202999in}}%
\pgfpathlineto{\pgfqpoint{0.442168in}{4.222999in}}%
\pgfusepath{stroke}%
\end{pgfscope}%
\begin{pgfscope}%
\pgfsetrectcap%
\pgfsetmiterjoin%
\pgfsetlinewidth{0.803000pt}%
\definecolor{currentstroke}{rgb}{0.000000,0.000000,0.000000}%
\pgfsetstrokecolor{currentstroke}%
\pgfsetdash{}{0pt}%
\pgfpathmoveto{\pgfqpoint{5.092168in}{1.202999in}}%
\pgfpathlineto{\pgfqpoint{5.092168in}{4.222999in}}%
\pgfusepath{stroke}%
\end{pgfscope}%
\begin{pgfscope}%
\pgfsetrectcap%
\pgfsetmiterjoin%
\pgfsetlinewidth{0.803000pt}%
\definecolor{currentstroke}{rgb}{0.000000,0.000000,0.000000}%
\pgfsetstrokecolor{currentstroke}%
\pgfsetdash{}{0pt}%
\pgfpathmoveto{\pgfqpoint{0.442168in}{1.202999in}}%
\pgfpathlineto{\pgfqpoint{5.092168in}{1.202999in}}%
\pgfusepath{stroke}%
\end{pgfscope}%
\begin{pgfscope}%
\pgfsetrectcap%
\pgfsetmiterjoin%
\pgfsetlinewidth{0.803000pt}%
\definecolor{currentstroke}{rgb}{0.000000,0.000000,0.000000}%
\pgfsetstrokecolor{currentstroke}%
\pgfsetdash{}{0pt}%
\pgfpathmoveto{\pgfqpoint{0.442168in}{4.222999in}}%
\pgfpathlineto{\pgfqpoint{5.092168in}{4.222999in}}%
\pgfusepath{stroke}%
\end{pgfscope}%
\begin{pgfscope}%
\pgfsetbuttcap%
\pgfsetmiterjoin%
\definecolor{currentfill}{rgb}{0.300000,0.300000,0.300000}%
\pgfsetfillcolor{currentfill}%
\pgfsetfillopacity{0.500000}%
\pgfsetlinewidth{1.003750pt}%
\definecolor{currentstroke}{rgb}{0.300000,0.300000,0.300000}%
\pgfsetstrokecolor{currentstroke}%
\pgfsetstrokeopacity{0.500000}%
\pgfsetdash{}{0pt}%
\pgfpathmoveto{\pgfqpoint{0.155756in}{-0.027778in}}%
\pgfpathlineto{\pgfqpoint{5.434136in}{-0.027778in}}%
\pgfpathquadraticcurveto{\pgfqpoint{5.473024in}{-0.027778in}}{\pgfqpoint{5.473024in}{0.011111in}}%
\pgfpathlineto{\pgfqpoint{5.473024in}{0.586111in}}%
\pgfpathquadraticcurveto{\pgfqpoint{5.473024in}{0.624999in}}{\pgfqpoint{5.434136in}{0.624999in}}%
\pgfpathlineto{\pgfqpoint{0.155756in}{0.624999in}}%
\pgfpathquadraticcurveto{\pgfqpoint{0.116867in}{0.624999in}}{\pgfqpoint{0.116867in}{0.586111in}}%
\pgfpathlineto{\pgfqpoint{0.116867in}{0.011111in}}%
\pgfpathquadraticcurveto{\pgfqpoint{0.116867in}{-0.027778in}}{\pgfqpoint{0.155756in}{-0.027778in}}%
\pgfpathclose%
\pgfusepath{stroke,fill}%
\end{pgfscope}%
\begin{pgfscope}%
\pgfsetbuttcap%
\pgfsetmiterjoin%
\definecolor{currentfill}{rgb}{1.000000,1.000000,1.000000}%
\pgfsetfillcolor{currentfill}%
\pgfsetlinewidth{1.003750pt}%
\definecolor{currentstroke}{rgb}{0.800000,0.800000,0.800000}%
\pgfsetstrokecolor{currentstroke}%
\pgfsetdash{}{0pt}%
\pgfpathmoveto{\pgfqpoint{0.127978in}{0.000000in}}%
\pgfpathlineto{\pgfqpoint{5.406358in}{0.000000in}}%
\pgfpathquadraticcurveto{\pgfqpoint{5.445247in}{0.000000in}}{\pgfqpoint{5.445247in}{0.038889in}}%
\pgfpathlineto{\pgfqpoint{5.445247in}{0.613888in}}%
\pgfpathquadraticcurveto{\pgfqpoint{5.445247in}{0.652777in}}{\pgfqpoint{5.406358in}{0.652777in}}%
\pgfpathlineto{\pgfqpoint{0.127978in}{0.652777in}}%
\pgfpathquadraticcurveto{\pgfqpoint{0.089089in}{0.652777in}}{\pgfqpoint{0.089089in}{0.613888in}}%
\pgfpathlineto{\pgfqpoint{0.089089in}{0.038889in}}%
\pgfpathquadraticcurveto{\pgfqpoint{0.089089in}{0.000000in}}{\pgfqpoint{0.127978in}{0.000000in}}%
\pgfpathclose%
\pgfusepath{stroke,fill}%
\end{pgfscope}%
\begin{pgfscope}%
\pgfsetrectcap%
\pgfsetroundjoin%
\pgfsetlinewidth{1.505625pt}%
\definecolor{currentstroke}{rgb}{0.000000,0.000000,1.000000}%
\pgfsetstrokecolor{currentstroke}%
\pgfsetdash{}{0pt}%
\pgfpathmoveto{\pgfqpoint{0.166867in}{0.493055in}}%
\pgfpathlineto{\pgfqpoint{0.555756in}{0.493055in}}%
\pgfusepath{stroke}%
\end{pgfscope}%
\begin{pgfscope}%
\definecolor{textcolor}{rgb}{0.000000,0.000000,0.000000}%
\pgfsetstrokecolor{textcolor}%
\pgfsetfillcolor{textcolor}%
\pgftext[x=0.711312in,y=0.425000in,left,base]{\color{textcolor}\rmfamily\fontsize{14.000000}{16.800000}\selectfont Women (main person)}%
\end{pgfscope}%
\begin{pgfscope}%
\pgfsetrectcap%
\pgfsetroundjoin%
\pgfsetlinewidth{1.505625pt}%
\definecolor{currentstroke}{rgb}{0.000000,0.000000,0.000000}%
\pgfsetstrokecolor{currentstroke}%
\pgfsetdash{}{0pt}%
\pgfpathmoveto{\pgfqpoint{0.166867in}{0.195833in}}%
\pgfpathlineto{\pgfqpoint{0.555756in}{0.195833in}}%
\pgfusepath{stroke}%
\end{pgfscope}%
\begin{pgfscope}%
\definecolor{textcolor}{rgb}{0.000000,0.000000,0.000000}%
\pgfsetstrokecolor{textcolor}%
\pgfsetfillcolor{textcolor}%
\pgftext[x=0.711312in,y=0.127778in,left,base]{\color{textcolor}\rmfamily\fontsize{14.000000}{16.800000}\selectfont Men (main person)}%
\end{pgfscope}%
\begin{pgfscope}%
\pgfsetbuttcap%
\pgfsetroundjoin%
\pgfsetlinewidth{1.505625pt}%
\definecolor{currentstroke}{rgb}{1.000000,0.000000,0.000000}%
\pgfsetstrokecolor{currentstroke}%
\pgfsetdash{{5.550000pt}{2.400000pt}}{0.000000pt}%
\pgfpathmoveto{\pgfqpoint{3.010570in}{0.493055in}}%
\pgfpathlineto{\pgfqpoint{3.399459in}{0.493055in}}%
\pgfusepath{stroke}%
\end{pgfscope}%
\begin{pgfscope}%
\definecolor{textcolor}{rgb}{0.000000,0.000000,0.000000}%
\pgfsetstrokecolor{textcolor}%
\pgfsetfillcolor{textcolor}%
\pgftext[x=3.555015in,y=0.425000in,left,base]{\color{textcolor}\rmfamily\fontsize{14.000000}{16.800000}\selectfont Women (met person)}%
\end{pgfscope}%
\begin{pgfscope}%
\pgfsetbuttcap%
\pgfsetroundjoin%
\pgfsetlinewidth{1.505625pt}%
\definecolor{currentstroke}{rgb}{0.750000,0.000000,0.750000}%
\pgfsetstrokecolor{currentstroke}%
\pgfsetdash{{5.550000pt}{2.400000pt}}{0.000000pt}%
\pgfpathmoveto{\pgfqpoint{3.010570in}{0.195833in}}%
\pgfpathlineto{\pgfqpoint{3.399459in}{0.195833in}}%
\pgfusepath{stroke}%
\end{pgfscope}%
\begin{pgfscope}%
\definecolor{textcolor}{rgb}{0.000000,0.000000,0.000000}%
\pgfsetstrokecolor{textcolor}%
\pgfsetfillcolor{textcolor}%
\pgftext[x=3.555015in,y=0.127778in,left,base]{\color{textcolor}\rmfamily\fontsize{14.000000}{16.800000}\selectfont Men (met person)}%
\end{pgfscope}%
\end{pgfpicture}%
\makeatother%
\endgroup%
 } 
\end{subfigure}
\begin{subfigure}{.49\textwidth}
\centering
% include second image
\caption{Variance assets}
\label{fig:sub-third13}
\scalebox{0.5}{%% Creator: Matplotlib, PGF backend
%%
%% To include the figure in your LaTeX document, write
%%   \input{<filename>.pgf}
%%
%% Make sure the required packages are loaded in your preamble
%%   \usepackage{pgf}
%%
%% and, on pdftex
%%   \usepackage[utf8]{inputenc}\DeclareUnicodeCharacter{2212}{-}
%%
%% or, on luatex and xetex
%%   \usepackage{unicode-math}
%%
%% Figures using additional raster images can only be included by \input if
%% they are in the same directory as the main LaTeX file. For loading figures
%% from other directories you can use the `import` package
%%   \usepackage{import}
%%
%% and then include the figures with
%%   \import{<path to file>}{<filename>.pgf}
%%
%% Matplotlib used the following preamble
%%
\begingroup%
\makeatletter%
\begin{pgfpicture}%
\pgfpathrectangle{\pgfpointorigin}{\pgfqpoint{5.443125in}{4.222999in}}%
\pgfusepath{use as bounding box, clip}%
\begin{pgfscope}%
\pgfsetbuttcap%
\pgfsetmiterjoin%
\definecolor{currentfill}{rgb}{1.000000,1.000000,1.000000}%
\pgfsetfillcolor{currentfill}%
\pgfsetlinewidth{0.000000pt}%
\definecolor{currentstroke}{rgb}{1.000000,1.000000,1.000000}%
\pgfsetstrokecolor{currentstroke}%
\pgfsetdash{}{0pt}%
\pgfpathmoveto{\pgfqpoint{0.000000in}{0.000000in}}%
\pgfpathlineto{\pgfqpoint{5.443125in}{0.000000in}}%
\pgfpathlineto{\pgfqpoint{5.443125in}{4.222999in}}%
\pgfpathlineto{\pgfqpoint{0.000000in}{4.222999in}}%
\pgfpathclose%
\pgfusepath{fill}%
\end{pgfscope}%
\begin{pgfscope}%
\pgfsetbuttcap%
\pgfsetmiterjoin%
\definecolor{currentfill}{rgb}{1.000000,1.000000,1.000000}%
\pgfsetfillcolor{currentfill}%
\pgfsetlinewidth{0.000000pt}%
\definecolor{currentstroke}{rgb}{0.000000,0.000000,0.000000}%
\pgfsetstrokecolor{currentstroke}%
\pgfsetstrokeopacity{0.000000}%
\pgfsetdash{}{0pt}%
\pgfpathmoveto{\pgfqpoint{0.440046in}{1.202999in}}%
\pgfpathlineto{\pgfqpoint{5.090046in}{1.202999in}}%
\pgfpathlineto{\pgfqpoint{5.090046in}{4.222999in}}%
\pgfpathlineto{\pgfqpoint{0.440046in}{4.222999in}}%
\pgfpathclose%
\pgfusepath{fill}%
\end{pgfscope}%
\begin{pgfscope}%
\pgfsetbuttcap%
\pgfsetroundjoin%
\definecolor{currentfill}{rgb}{0.000000,0.000000,0.000000}%
\pgfsetfillcolor{currentfill}%
\pgfsetlinewidth{0.803000pt}%
\definecolor{currentstroke}{rgb}{0.000000,0.000000,0.000000}%
\pgfsetstrokecolor{currentstroke}%
\pgfsetdash{}{0pt}%
\pgfsys@defobject{currentmarker}{\pgfqpoint{0.000000in}{-0.048611in}}{\pgfqpoint{0.000000in}{0.000000in}}{%
\pgfpathmoveto{\pgfqpoint{0.000000in}{0.000000in}}%
\pgfpathlineto{\pgfqpoint{0.000000in}{-0.048611in}}%
\pgfusepath{stroke,fill}%
}%
\begin{pgfscope}%
\pgfsys@transformshift{0.651410in}{1.202999in}%
\pgfsys@useobject{currentmarker}{}%
\end{pgfscope}%
\end{pgfscope}%
\begin{pgfscope}%
\definecolor{textcolor}{rgb}{0.000000,0.000000,0.000000}%
\pgfsetstrokecolor{textcolor}%
\pgfsetfillcolor{textcolor}%
\pgftext[x=0.651410in,y=1.105777in,,top]{\color{textcolor}\rmfamily\fontsize{11.000000}{13.200000}\selectfont \(\displaystyle {25}\)}%
\end{pgfscope}%
\begin{pgfscope}%
\pgfsetbuttcap%
\pgfsetroundjoin%
\definecolor{currentfill}{rgb}{0.000000,0.000000,0.000000}%
\pgfsetfillcolor{currentfill}%
\pgfsetlinewidth{0.803000pt}%
\definecolor{currentstroke}{rgb}{0.000000,0.000000,0.000000}%
\pgfsetstrokecolor{currentstroke}%
\pgfsetdash{}{0pt}%
\pgfsys@defobject{currentmarker}{\pgfqpoint{0.000000in}{-0.048611in}}{\pgfqpoint{0.000000in}{0.000000in}}{%
\pgfpathmoveto{\pgfqpoint{0.000000in}{0.000000in}}%
\pgfpathlineto{\pgfqpoint{0.000000in}{-0.048611in}}%
\pgfusepath{stroke,fill}%
}%
\begin{pgfscope}%
\pgfsys@transformshift{1.273068in}{1.202999in}%
\pgfsys@useobject{currentmarker}{}%
\end{pgfscope}%
\end{pgfscope}%
\begin{pgfscope}%
\definecolor{textcolor}{rgb}{0.000000,0.000000,0.000000}%
\pgfsetstrokecolor{textcolor}%
\pgfsetfillcolor{textcolor}%
\pgftext[x=1.273068in,y=1.105777in,,top]{\color{textcolor}\rmfamily\fontsize{11.000000}{13.200000}\selectfont \(\displaystyle {30}\)}%
\end{pgfscope}%
\begin{pgfscope}%
\pgfsetbuttcap%
\pgfsetroundjoin%
\definecolor{currentfill}{rgb}{0.000000,0.000000,0.000000}%
\pgfsetfillcolor{currentfill}%
\pgfsetlinewidth{0.803000pt}%
\definecolor{currentstroke}{rgb}{0.000000,0.000000,0.000000}%
\pgfsetstrokecolor{currentstroke}%
\pgfsetdash{}{0pt}%
\pgfsys@defobject{currentmarker}{\pgfqpoint{0.000000in}{-0.048611in}}{\pgfqpoint{0.000000in}{0.000000in}}{%
\pgfpathmoveto{\pgfqpoint{0.000000in}{0.000000in}}%
\pgfpathlineto{\pgfqpoint{0.000000in}{-0.048611in}}%
\pgfusepath{stroke,fill}%
}%
\begin{pgfscope}%
\pgfsys@transformshift{1.894726in}{1.202999in}%
\pgfsys@useobject{currentmarker}{}%
\end{pgfscope}%
\end{pgfscope}%
\begin{pgfscope}%
\definecolor{textcolor}{rgb}{0.000000,0.000000,0.000000}%
\pgfsetstrokecolor{textcolor}%
\pgfsetfillcolor{textcolor}%
\pgftext[x=1.894726in,y=1.105777in,,top]{\color{textcolor}\rmfamily\fontsize{11.000000}{13.200000}\selectfont \(\displaystyle {35}\)}%
\end{pgfscope}%
\begin{pgfscope}%
\pgfsetbuttcap%
\pgfsetroundjoin%
\definecolor{currentfill}{rgb}{0.000000,0.000000,0.000000}%
\pgfsetfillcolor{currentfill}%
\pgfsetlinewidth{0.803000pt}%
\definecolor{currentstroke}{rgb}{0.000000,0.000000,0.000000}%
\pgfsetstrokecolor{currentstroke}%
\pgfsetdash{}{0pt}%
\pgfsys@defobject{currentmarker}{\pgfqpoint{0.000000in}{-0.048611in}}{\pgfqpoint{0.000000in}{0.000000in}}{%
\pgfpathmoveto{\pgfqpoint{0.000000in}{0.000000in}}%
\pgfpathlineto{\pgfqpoint{0.000000in}{-0.048611in}}%
\pgfusepath{stroke,fill}%
}%
\begin{pgfscope}%
\pgfsys@transformshift{2.516383in}{1.202999in}%
\pgfsys@useobject{currentmarker}{}%
\end{pgfscope}%
\end{pgfscope}%
\begin{pgfscope}%
\definecolor{textcolor}{rgb}{0.000000,0.000000,0.000000}%
\pgfsetstrokecolor{textcolor}%
\pgfsetfillcolor{textcolor}%
\pgftext[x=2.516383in,y=1.105777in,,top]{\color{textcolor}\rmfamily\fontsize{11.000000}{13.200000}\selectfont \(\displaystyle {40}\)}%
\end{pgfscope}%
\begin{pgfscope}%
\pgfsetbuttcap%
\pgfsetroundjoin%
\definecolor{currentfill}{rgb}{0.000000,0.000000,0.000000}%
\pgfsetfillcolor{currentfill}%
\pgfsetlinewidth{0.803000pt}%
\definecolor{currentstroke}{rgb}{0.000000,0.000000,0.000000}%
\pgfsetstrokecolor{currentstroke}%
\pgfsetdash{}{0pt}%
\pgfsys@defobject{currentmarker}{\pgfqpoint{0.000000in}{-0.048611in}}{\pgfqpoint{0.000000in}{0.000000in}}{%
\pgfpathmoveto{\pgfqpoint{0.000000in}{0.000000in}}%
\pgfpathlineto{\pgfqpoint{0.000000in}{-0.048611in}}%
\pgfusepath{stroke,fill}%
}%
\begin{pgfscope}%
\pgfsys@transformshift{3.138041in}{1.202999in}%
\pgfsys@useobject{currentmarker}{}%
\end{pgfscope}%
\end{pgfscope}%
\begin{pgfscope}%
\definecolor{textcolor}{rgb}{0.000000,0.000000,0.000000}%
\pgfsetstrokecolor{textcolor}%
\pgfsetfillcolor{textcolor}%
\pgftext[x=3.138041in,y=1.105777in,,top]{\color{textcolor}\rmfamily\fontsize{11.000000}{13.200000}\selectfont \(\displaystyle {45}\)}%
\end{pgfscope}%
\begin{pgfscope}%
\pgfsetbuttcap%
\pgfsetroundjoin%
\definecolor{currentfill}{rgb}{0.000000,0.000000,0.000000}%
\pgfsetfillcolor{currentfill}%
\pgfsetlinewidth{0.803000pt}%
\definecolor{currentstroke}{rgb}{0.000000,0.000000,0.000000}%
\pgfsetstrokecolor{currentstroke}%
\pgfsetdash{}{0pt}%
\pgfsys@defobject{currentmarker}{\pgfqpoint{0.000000in}{-0.048611in}}{\pgfqpoint{0.000000in}{0.000000in}}{%
\pgfpathmoveto{\pgfqpoint{0.000000in}{0.000000in}}%
\pgfpathlineto{\pgfqpoint{0.000000in}{-0.048611in}}%
\pgfusepath{stroke,fill}%
}%
\begin{pgfscope}%
\pgfsys@transformshift{3.759699in}{1.202999in}%
\pgfsys@useobject{currentmarker}{}%
\end{pgfscope}%
\end{pgfscope}%
\begin{pgfscope}%
\definecolor{textcolor}{rgb}{0.000000,0.000000,0.000000}%
\pgfsetstrokecolor{textcolor}%
\pgfsetfillcolor{textcolor}%
\pgftext[x=3.759699in,y=1.105777in,,top]{\color{textcolor}\rmfamily\fontsize{11.000000}{13.200000}\selectfont \(\displaystyle {50}\)}%
\end{pgfscope}%
\begin{pgfscope}%
\pgfsetbuttcap%
\pgfsetroundjoin%
\definecolor{currentfill}{rgb}{0.000000,0.000000,0.000000}%
\pgfsetfillcolor{currentfill}%
\pgfsetlinewidth{0.803000pt}%
\definecolor{currentstroke}{rgb}{0.000000,0.000000,0.000000}%
\pgfsetstrokecolor{currentstroke}%
\pgfsetdash{}{0pt}%
\pgfsys@defobject{currentmarker}{\pgfqpoint{0.000000in}{-0.048611in}}{\pgfqpoint{0.000000in}{0.000000in}}{%
\pgfpathmoveto{\pgfqpoint{0.000000in}{0.000000in}}%
\pgfpathlineto{\pgfqpoint{0.000000in}{-0.048611in}}%
\pgfusepath{stroke,fill}%
}%
\begin{pgfscope}%
\pgfsys@transformshift{4.381357in}{1.202999in}%
\pgfsys@useobject{currentmarker}{}%
\end{pgfscope}%
\end{pgfscope}%
\begin{pgfscope}%
\definecolor{textcolor}{rgb}{0.000000,0.000000,0.000000}%
\pgfsetstrokecolor{textcolor}%
\pgfsetfillcolor{textcolor}%
\pgftext[x=4.381357in,y=1.105777in,,top]{\color{textcolor}\rmfamily\fontsize{11.000000}{13.200000}\selectfont \(\displaystyle {55}\)}%
\end{pgfscope}%
\begin{pgfscope}%
\pgfsetbuttcap%
\pgfsetroundjoin%
\definecolor{currentfill}{rgb}{0.000000,0.000000,0.000000}%
\pgfsetfillcolor{currentfill}%
\pgfsetlinewidth{0.803000pt}%
\definecolor{currentstroke}{rgb}{0.000000,0.000000,0.000000}%
\pgfsetstrokecolor{currentstroke}%
\pgfsetdash{}{0pt}%
\pgfsys@defobject{currentmarker}{\pgfqpoint{0.000000in}{-0.048611in}}{\pgfqpoint{0.000000in}{0.000000in}}{%
\pgfpathmoveto{\pgfqpoint{0.000000in}{0.000000in}}%
\pgfpathlineto{\pgfqpoint{0.000000in}{-0.048611in}}%
\pgfusepath{stroke,fill}%
}%
\begin{pgfscope}%
\pgfsys@transformshift{5.003014in}{1.202999in}%
\pgfsys@useobject{currentmarker}{}%
\end{pgfscope}%
\end{pgfscope}%
\begin{pgfscope}%
\definecolor{textcolor}{rgb}{0.000000,0.000000,0.000000}%
\pgfsetstrokecolor{textcolor}%
\pgfsetfillcolor{textcolor}%
\pgftext[x=5.003014in,y=1.105777in,,top]{\color{textcolor}\rmfamily\fontsize{11.000000}{13.200000}\selectfont \(\displaystyle {60}\)}%
\end{pgfscope}%
\begin{pgfscope}%
\definecolor{textcolor}{rgb}{0.000000,0.000000,0.000000}%
\pgfsetstrokecolor{textcolor}%
\pgfsetfillcolor{textcolor}%
\pgftext[x=2.765046in,y=0.915037in,,top]{\color{textcolor}\rmfamily\fontsize{11.000000}{13.200000}\selectfont Age}%
\end{pgfscope}%
\begin{pgfscope}%
\pgfsetbuttcap%
\pgfsetroundjoin%
\definecolor{currentfill}{rgb}{0.000000,0.000000,0.000000}%
\pgfsetfillcolor{currentfill}%
\pgfsetlinewidth{0.803000pt}%
\definecolor{currentstroke}{rgb}{0.000000,0.000000,0.000000}%
\pgfsetstrokecolor{currentstroke}%
\pgfsetdash{}{0pt}%
\pgfsys@defobject{currentmarker}{\pgfqpoint{-0.048611in}{0.000000in}}{\pgfqpoint{0.000000in}{0.000000in}}{%
\pgfpathmoveto{\pgfqpoint{0.000000in}{0.000000in}}%
\pgfpathlineto{\pgfqpoint{-0.048611in}{0.000000in}}%
\pgfusepath{stroke,fill}%
}%
\begin{pgfscope}%
\pgfsys@transformshift{0.440046in}{1.339644in}%
\pgfsys@useobject{currentmarker}{}%
\end{pgfscope}%
\end{pgfscope}%
\begin{pgfscope}%
\definecolor{textcolor}{rgb}{0.000000,0.000000,0.000000}%
\pgfsetstrokecolor{textcolor}%
\pgfsetfillcolor{textcolor}%
\pgftext[x=0.266782in, y=1.286837in, left, base]{\color{textcolor}\rmfamily\fontsize{11.000000}{13.200000}\selectfont \(\displaystyle {0}\)}%
\end{pgfscope}%
\begin{pgfscope}%
\pgfsetbuttcap%
\pgfsetroundjoin%
\definecolor{currentfill}{rgb}{0.000000,0.000000,0.000000}%
\pgfsetfillcolor{currentfill}%
\pgfsetlinewidth{0.803000pt}%
\definecolor{currentstroke}{rgb}{0.000000,0.000000,0.000000}%
\pgfsetstrokecolor{currentstroke}%
\pgfsetdash{}{0pt}%
\pgfsys@defobject{currentmarker}{\pgfqpoint{-0.048611in}{0.000000in}}{\pgfqpoint{0.000000in}{0.000000in}}{%
\pgfpathmoveto{\pgfqpoint{0.000000in}{0.000000in}}%
\pgfpathlineto{\pgfqpoint{-0.048611in}{0.000000in}}%
\pgfusepath{stroke,fill}%
}%
\begin{pgfscope}%
\pgfsys@transformshift{0.440046in}{1.955477in}%
\pgfsys@useobject{currentmarker}{}%
\end{pgfscope}%
\end{pgfscope}%
\begin{pgfscope}%
\definecolor{textcolor}{rgb}{0.000000,0.000000,0.000000}%
\pgfsetstrokecolor{textcolor}%
\pgfsetfillcolor{textcolor}%
\pgftext[x=0.190741in, y=1.902670in, left, base]{\color{textcolor}\rmfamily\fontsize{11.000000}{13.200000}\selectfont \(\displaystyle {10}\)}%
\end{pgfscope}%
\begin{pgfscope}%
\pgfsetbuttcap%
\pgfsetroundjoin%
\definecolor{currentfill}{rgb}{0.000000,0.000000,0.000000}%
\pgfsetfillcolor{currentfill}%
\pgfsetlinewidth{0.803000pt}%
\definecolor{currentstroke}{rgb}{0.000000,0.000000,0.000000}%
\pgfsetstrokecolor{currentstroke}%
\pgfsetdash{}{0pt}%
\pgfsys@defobject{currentmarker}{\pgfqpoint{-0.048611in}{0.000000in}}{\pgfqpoint{0.000000in}{0.000000in}}{%
\pgfpathmoveto{\pgfqpoint{0.000000in}{0.000000in}}%
\pgfpathlineto{\pgfqpoint{-0.048611in}{0.000000in}}%
\pgfusepath{stroke,fill}%
}%
\begin{pgfscope}%
\pgfsys@transformshift{0.440046in}{2.571309in}%
\pgfsys@useobject{currentmarker}{}%
\end{pgfscope}%
\end{pgfscope}%
\begin{pgfscope}%
\definecolor{textcolor}{rgb}{0.000000,0.000000,0.000000}%
\pgfsetstrokecolor{textcolor}%
\pgfsetfillcolor{textcolor}%
\pgftext[x=0.190741in, y=2.518502in, left, base]{\color{textcolor}\rmfamily\fontsize{11.000000}{13.200000}\selectfont \(\displaystyle {20}\)}%
\end{pgfscope}%
\begin{pgfscope}%
\pgfsetbuttcap%
\pgfsetroundjoin%
\definecolor{currentfill}{rgb}{0.000000,0.000000,0.000000}%
\pgfsetfillcolor{currentfill}%
\pgfsetlinewidth{0.803000pt}%
\definecolor{currentstroke}{rgb}{0.000000,0.000000,0.000000}%
\pgfsetstrokecolor{currentstroke}%
\pgfsetdash{}{0pt}%
\pgfsys@defobject{currentmarker}{\pgfqpoint{-0.048611in}{0.000000in}}{\pgfqpoint{0.000000in}{0.000000in}}{%
\pgfpathmoveto{\pgfqpoint{0.000000in}{0.000000in}}%
\pgfpathlineto{\pgfqpoint{-0.048611in}{0.000000in}}%
\pgfusepath{stroke,fill}%
}%
\begin{pgfscope}%
\pgfsys@transformshift{0.440046in}{3.187141in}%
\pgfsys@useobject{currentmarker}{}%
\end{pgfscope}%
\end{pgfscope}%
\begin{pgfscope}%
\definecolor{textcolor}{rgb}{0.000000,0.000000,0.000000}%
\pgfsetstrokecolor{textcolor}%
\pgfsetfillcolor{textcolor}%
\pgftext[x=0.190741in, y=3.134335in, left, base]{\color{textcolor}\rmfamily\fontsize{11.000000}{13.200000}\selectfont \(\displaystyle {30}\)}%
\end{pgfscope}%
\begin{pgfscope}%
\pgfsetbuttcap%
\pgfsetroundjoin%
\definecolor{currentfill}{rgb}{0.000000,0.000000,0.000000}%
\pgfsetfillcolor{currentfill}%
\pgfsetlinewidth{0.803000pt}%
\definecolor{currentstroke}{rgb}{0.000000,0.000000,0.000000}%
\pgfsetstrokecolor{currentstroke}%
\pgfsetdash{}{0pt}%
\pgfsys@defobject{currentmarker}{\pgfqpoint{-0.048611in}{0.000000in}}{\pgfqpoint{0.000000in}{0.000000in}}{%
\pgfpathmoveto{\pgfqpoint{0.000000in}{0.000000in}}%
\pgfpathlineto{\pgfqpoint{-0.048611in}{0.000000in}}%
\pgfusepath{stroke,fill}%
}%
\begin{pgfscope}%
\pgfsys@transformshift{0.440046in}{3.802974in}%
\pgfsys@useobject{currentmarker}{}%
\end{pgfscope}%
\end{pgfscope}%
\begin{pgfscope}%
\definecolor{textcolor}{rgb}{0.000000,0.000000,0.000000}%
\pgfsetstrokecolor{textcolor}%
\pgfsetfillcolor{textcolor}%
\pgftext[x=0.190741in, y=3.750167in, left, base]{\color{textcolor}\rmfamily\fontsize{11.000000}{13.200000}\selectfont \(\displaystyle {40}\)}%
\end{pgfscope}%
\begin{pgfscope}%
\definecolor{textcolor}{rgb}{0.000000,0.000000,0.000000}%
\pgfsetstrokecolor{textcolor}%
\pgfsetfillcolor{textcolor}%
\pgftext[x=0.135185in,y=2.712999in,,bottom,rotate=90.000000]{\color{textcolor}\rmfamily\fontsize{11.000000}{13.200000}\selectfont Asset at meeting---variance}%
\end{pgfscope}%
\begin{pgfscope}%
\pgfpathrectangle{\pgfqpoint{0.440046in}{1.202999in}}{\pgfqpoint{4.650000in}{3.020000in}}%
\pgfusepath{clip}%
\pgfsetrectcap%
\pgfsetroundjoin%
\pgfsetlinewidth{1.505625pt}%
\definecolor{currentstroke}{rgb}{0.000000,0.000000,1.000000}%
\pgfsetstrokecolor{currentstroke}%
\pgfsetdash{}{0pt}%
\pgfpathmoveto{\pgfqpoint{0.651410in}{1.341032in}}%
\pgfpathlineto{\pgfqpoint{0.775742in}{1.342127in}}%
\pgfpathlineto{\pgfqpoint{0.900073in}{1.344071in}}%
\pgfpathlineto{\pgfqpoint{1.024405in}{1.345837in}}%
\pgfpathlineto{\pgfqpoint{1.148736in}{1.353829in}}%
\pgfpathlineto{\pgfqpoint{1.273068in}{1.359209in}}%
\pgfpathlineto{\pgfqpoint{1.397399in}{1.381395in}}%
\pgfpathlineto{\pgfqpoint{1.521731in}{1.395925in}}%
\pgfpathlineto{\pgfqpoint{1.646062in}{1.402653in}}%
\pgfpathlineto{\pgfqpoint{1.770394in}{1.424128in}}%
\pgfpathlineto{\pgfqpoint{1.894726in}{1.426973in}}%
\pgfpathlineto{\pgfqpoint{2.019057in}{1.436339in}}%
\pgfpathlineto{\pgfqpoint{2.143389in}{1.458502in}}%
\pgfpathlineto{\pgfqpoint{2.267720in}{1.459567in}}%
\pgfpathlineto{\pgfqpoint{2.392052in}{1.524490in}}%
\pgfpathlineto{\pgfqpoint{2.516383in}{1.592710in}}%
\pgfpathlineto{\pgfqpoint{2.640715in}{1.579515in}}%
\pgfpathlineto{\pgfqpoint{2.765046in}{1.532248in}}%
\pgfpathlineto{\pgfqpoint{2.889378in}{1.666254in}}%
\pgfpathlineto{\pgfqpoint{3.013709in}{1.727999in}}%
\pgfpathlineto{\pgfqpoint{3.138041in}{1.776499in}}%
\pgfpathlineto{\pgfqpoint{3.262373in}{1.988954in}}%
\pgfpathlineto{\pgfqpoint{3.386704in}{1.755751in}}%
\pgfpathlineto{\pgfqpoint{3.511036in}{1.712739in}}%
\pgfpathlineto{\pgfqpoint{3.635367in}{2.057121in}}%
\pgfpathlineto{\pgfqpoint{3.759699in}{2.119776in}}%
\pgfpathlineto{\pgfqpoint{3.884030in}{2.697469in}}%
\pgfpathlineto{\pgfqpoint{4.008362in}{2.530509in}}%
\pgfpathlineto{\pgfqpoint{4.132693in}{2.793749in}}%
\pgfpathlineto{\pgfqpoint{4.257025in}{3.229301in}}%
\pgfpathlineto{\pgfqpoint{4.381357in}{2.789279in}}%
\pgfpathlineto{\pgfqpoint{4.505688in}{3.261814in}}%
\pgfpathlineto{\pgfqpoint{4.630020in}{2.865831in}}%
\pgfpathlineto{\pgfqpoint{4.754351in}{2.283072in}}%
\pgfpathlineto{\pgfqpoint{4.878683in}{2.540564in}}%
\pgfusepath{stroke}%
\end{pgfscope}%
\begin{pgfscope}%
\pgfpathrectangle{\pgfqpoint{0.440046in}{1.202999in}}{\pgfqpoint{4.650000in}{3.020000in}}%
\pgfusepath{clip}%
\pgfsetrectcap%
\pgfsetroundjoin%
\pgfsetlinewidth{1.505625pt}%
\definecolor{currentstroke}{rgb}{0.000000,0.000000,0.000000}%
\pgfsetstrokecolor{currentstroke}%
\pgfsetdash{}{0pt}%
\pgfpathmoveto{\pgfqpoint{0.651410in}{1.340272in}}%
\pgfpathlineto{\pgfqpoint{0.775742in}{1.341937in}}%
\pgfpathlineto{\pgfqpoint{0.900073in}{1.343373in}}%
\pgfpathlineto{\pgfqpoint{1.024405in}{1.344631in}}%
\pgfpathlineto{\pgfqpoint{1.148736in}{1.345729in}}%
\pgfpathlineto{\pgfqpoint{1.273068in}{1.348408in}}%
\pgfpathlineto{\pgfqpoint{1.397399in}{1.355760in}}%
\pgfpathlineto{\pgfqpoint{1.521731in}{1.368974in}}%
\pgfpathlineto{\pgfqpoint{1.646062in}{1.362408in}}%
\pgfpathlineto{\pgfqpoint{1.770394in}{1.390032in}}%
\pgfpathlineto{\pgfqpoint{1.894726in}{1.411644in}}%
\pgfpathlineto{\pgfqpoint{2.019057in}{1.426018in}}%
\pgfpathlineto{\pgfqpoint{2.143389in}{1.461102in}}%
\pgfpathlineto{\pgfqpoint{2.267720in}{1.515915in}}%
\pgfpathlineto{\pgfqpoint{2.392052in}{1.546965in}}%
\pgfpathlineto{\pgfqpoint{2.516383in}{1.615505in}}%
\pgfpathlineto{\pgfqpoint{2.640715in}{1.540632in}}%
\pgfpathlineto{\pgfqpoint{2.765046in}{1.866031in}}%
\pgfpathlineto{\pgfqpoint{2.889378in}{1.607648in}}%
\pgfpathlineto{\pgfqpoint{3.013709in}{1.652633in}}%
\pgfpathlineto{\pgfqpoint{3.138041in}{1.703614in}}%
\pgfpathlineto{\pgfqpoint{3.262373in}{1.647559in}}%
\pgfpathlineto{\pgfqpoint{3.386704in}{1.863549in}}%
\pgfpathlineto{\pgfqpoint{3.511036in}{2.036724in}}%
\pgfpathlineto{\pgfqpoint{3.635367in}{2.204630in}}%
\pgfpathlineto{\pgfqpoint{3.759699in}{2.349288in}}%
\pgfpathlineto{\pgfqpoint{3.884030in}{2.312427in}}%
\pgfpathlineto{\pgfqpoint{4.008362in}{2.554365in}}%
\pgfpathlineto{\pgfqpoint{4.132693in}{2.617438in}}%
\pgfpathlineto{\pgfqpoint{4.257025in}{2.707990in}}%
\pgfpathlineto{\pgfqpoint{4.381357in}{2.450662in}}%
\pgfpathlineto{\pgfqpoint{4.505688in}{2.240314in}}%
\pgfpathlineto{\pgfqpoint{4.630020in}{2.398225in}}%
\pgfpathlineto{\pgfqpoint{4.754351in}{2.642256in}}%
\pgfpathlineto{\pgfqpoint{4.878683in}{2.661323in}}%
\pgfusepath{stroke}%
\end{pgfscope}%
\begin{pgfscope}%
\pgfpathrectangle{\pgfqpoint{0.440046in}{1.202999in}}{\pgfqpoint{4.650000in}{3.020000in}}%
\pgfusepath{clip}%
\pgfsetbuttcap%
\pgfsetroundjoin%
\pgfsetlinewidth{1.505625pt}%
\definecolor{currentstroke}{rgb}{1.000000,0.000000,0.000000}%
\pgfsetstrokecolor{currentstroke}%
\pgfsetdash{{5.550000pt}{2.400000pt}}{0.000000pt}%
\pgfpathmoveto{\pgfqpoint{0.651410in}{1.340323in}}%
\pgfpathlineto{\pgfqpoint{0.775742in}{1.342488in}}%
\pgfpathlineto{\pgfqpoint{0.900073in}{1.343839in}}%
\pgfpathlineto{\pgfqpoint{1.024405in}{1.344394in}}%
\pgfpathlineto{\pgfqpoint{1.148736in}{1.346245in}}%
\pgfpathlineto{\pgfqpoint{1.273068in}{1.349279in}}%
\pgfpathlineto{\pgfqpoint{1.397399in}{1.350626in}}%
\pgfpathlineto{\pgfqpoint{1.521731in}{1.360892in}}%
\pgfpathlineto{\pgfqpoint{1.646062in}{1.353185in}}%
\pgfpathlineto{\pgfqpoint{1.770394in}{1.370253in}}%
\pgfpathlineto{\pgfqpoint{1.894726in}{1.388552in}}%
\pgfpathlineto{\pgfqpoint{2.019057in}{1.394667in}}%
\pgfpathlineto{\pgfqpoint{2.143389in}{1.419652in}}%
\pgfpathlineto{\pgfqpoint{2.267720in}{1.460117in}}%
\pgfpathlineto{\pgfqpoint{2.392052in}{1.473531in}}%
\pgfpathlineto{\pgfqpoint{2.516383in}{1.511141in}}%
\pgfpathlineto{\pgfqpoint{2.640715in}{1.474155in}}%
\pgfpathlineto{\pgfqpoint{2.765046in}{1.687299in}}%
\pgfpathlineto{\pgfqpoint{2.889378in}{1.521131in}}%
\pgfpathlineto{\pgfqpoint{3.013709in}{1.556526in}}%
\pgfpathlineto{\pgfqpoint{3.138041in}{1.579254in}}%
\pgfpathlineto{\pgfqpoint{3.262373in}{1.519623in}}%
\pgfpathlineto{\pgfqpoint{3.386704in}{1.714276in}}%
\pgfpathlineto{\pgfqpoint{3.511036in}{1.809685in}}%
\pgfpathlineto{\pgfqpoint{3.635367in}{1.934282in}}%
\pgfpathlineto{\pgfqpoint{3.759699in}{2.043704in}}%
\pgfpathlineto{\pgfqpoint{3.884030in}{2.035923in}}%
\pgfpathlineto{\pgfqpoint{4.008362in}{2.250702in}}%
\pgfpathlineto{\pgfqpoint{4.132693in}{2.242833in}}%
\pgfpathlineto{\pgfqpoint{4.257025in}{2.236095in}}%
\pgfpathlineto{\pgfqpoint{4.381357in}{2.130133in}}%
\pgfpathlineto{\pgfqpoint{4.505688in}{1.984085in}}%
\pgfpathlineto{\pgfqpoint{4.630020in}{2.089931in}}%
\pgfpathlineto{\pgfqpoint{4.754351in}{2.217925in}}%
\pgfpathlineto{\pgfqpoint{4.878683in}{2.263070in}}%
\pgfusepath{stroke}%
\end{pgfscope}%
\begin{pgfscope}%
\pgfpathrectangle{\pgfqpoint{0.440046in}{1.202999in}}{\pgfqpoint{4.650000in}{3.020000in}}%
\pgfusepath{clip}%
\pgfsetbuttcap%
\pgfsetroundjoin%
\pgfsetlinewidth{1.505625pt}%
\definecolor{currentstroke}{rgb}{0.750000,0.000000,0.750000}%
\pgfsetstrokecolor{currentstroke}%
\pgfsetdash{{5.550000pt}{2.400000pt}}{0.000000pt}%
\pgfpathmoveto{\pgfqpoint{0.651410in}{1.349866in}}%
\pgfpathlineto{\pgfqpoint{0.775742in}{1.352062in}}%
\pgfpathlineto{\pgfqpoint{0.900073in}{1.355555in}}%
\pgfpathlineto{\pgfqpoint{1.024405in}{1.357839in}}%
\pgfpathlineto{\pgfqpoint{1.148736in}{1.371659in}}%
\pgfpathlineto{\pgfqpoint{1.273068in}{1.379268in}}%
\pgfpathlineto{\pgfqpoint{1.397399in}{1.410583in}}%
\pgfpathlineto{\pgfqpoint{1.521731in}{1.429254in}}%
\pgfpathlineto{\pgfqpoint{1.646062in}{1.443630in}}%
\pgfpathlineto{\pgfqpoint{1.770394in}{1.471097in}}%
\pgfpathlineto{\pgfqpoint{1.894726in}{1.472618in}}%
\pgfpathlineto{\pgfqpoint{2.019057in}{1.482145in}}%
\pgfpathlineto{\pgfqpoint{2.143389in}{1.509419in}}%
\pgfpathlineto{\pgfqpoint{2.267720in}{1.519669in}}%
\pgfpathlineto{\pgfqpoint{2.392052in}{1.614785in}}%
\pgfpathlineto{\pgfqpoint{2.516383in}{1.693586in}}%
\pgfpathlineto{\pgfqpoint{2.640715in}{1.671660in}}%
\pgfpathlineto{\pgfqpoint{2.765046in}{1.612636in}}%
\pgfpathlineto{\pgfqpoint{2.889378in}{1.802632in}}%
\pgfpathlineto{\pgfqpoint{3.013709in}{1.912707in}}%
\pgfpathlineto{\pgfqpoint{3.138041in}{2.029877in}}%
\pgfpathlineto{\pgfqpoint{3.262373in}{2.255411in}}%
\pgfpathlineto{\pgfqpoint{3.386704in}{1.964083in}}%
\pgfpathlineto{\pgfqpoint{3.511036in}{1.889404in}}%
\pgfpathlineto{\pgfqpoint{3.635367in}{2.444422in}}%
\pgfpathlineto{\pgfqpoint{3.759699in}{2.444905in}}%
\pgfpathlineto{\pgfqpoint{3.884030in}{3.267686in}}%
\pgfpathlineto{\pgfqpoint{4.008362in}{2.998592in}}%
\pgfpathlineto{\pgfqpoint{4.132693in}{3.419305in}}%
\pgfpathlineto{\pgfqpoint{4.257025in}{4.037315in}}%
\pgfpathlineto{\pgfqpoint{4.381357in}{3.388968in}}%
\pgfpathlineto{\pgfqpoint{4.505688in}{4.085727in}}%
\pgfpathlineto{\pgfqpoint{4.630020in}{3.523403in}}%
\pgfpathlineto{\pgfqpoint{4.754351in}{2.619405in}}%
\pgfpathlineto{\pgfqpoint{4.878683in}{2.992489in}}%
\pgfusepath{stroke}%
\end{pgfscope}%
\begin{pgfscope}%
\pgfsetrectcap%
\pgfsetmiterjoin%
\pgfsetlinewidth{0.803000pt}%
\definecolor{currentstroke}{rgb}{0.000000,0.000000,0.000000}%
\pgfsetstrokecolor{currentstroke}%
\pgfsetdash{}{0pt}%
\pgfpathmoveto{\pgfqpoint{0.440046in}{1.202999in}}%
\pgfpathlineto{\pgfqpoint{0.440046in}{4.222999in}}%
\pgfusepath{stroke}%
\end{pgfscope}%
\begin{pgfscope}%
\pgfsetrectcap%
\pgfsetmiterjoin%
\pgfsetlinewidth{0.803000pt}%
\definecolor{currentstroke}{rgb}{0.000000,0.000000,0.000000}%
\pgfsetstrokecolor{currentstroke}%
\pgfsetdash{}{0pt}%
\pgfpathmoveto{\pgfqpoint{5.090046in}{1.202999in}}%
\pgfpathlineto{\pgfqpoint{5.090046in}{4.222999in}}%
\pgfusepath{stroke}%
\end{pgfscope}%
\begin{pgfscope}%
\pgfsetrectcap%
\pgfsetmiterjoin%
\pgfsetlinewidth{0.803000pt}%
\definecolor{currentstroke}{rgb}{0.000000,0.000000,0.000000}%
\pgfsetstrokecolor{currentstroke}%
\pgfsetdash{}{0pt}%
\pgfpathmoveto{\pgfqpoint{0.440046in}{1.202999in}}%
\pgfpathlineto{\pgfqpoint{5.090046in}{1.202999in}}%
\pgfusepath{stroke}%
\end{pgfscope}%
\begin{pgfscope}%
\pgfsetrectcap%
\pgfsetmiterjoin%
\pgfsetlinewidth{0.803000pt}%
\definecolor{currentstroke}{rgb}{0.000000,0.000000,0.000000}%
\pgfsetstrokecolor{currentstroke}%
\pgfsetdash{}{0pt}%
\pgfpathmoveto{\pgfqpoint{0.440046in}{4.222999in}}%
\pgfpathlineto{\pgfqpoint{5.090046in}{4.222999in}}%
\pgfusepath{stroke}%
\end{pgfscope}%
\begin{pgfscope}%
\pgfsetbuttcap%
\pgfsetmiterjoin%
\definecolor{currentfill}{rgb}{0.300000,0.300000,0.300000}%
\pgfsetfillcolor{currentfill}%
\pgfsetfillopacity{0.500000}%
\pgfsetlinewidth{1.003750pt}%
\definecolor{currentstroke}{rgb}{0.300000,0.300000,0.300000}%
\pgfsetstrokecolor{currentstroke}%
\pgfsetstrokeopacity{0.500000}%
\pgfsetdash{}{0pt}%
\pgfpathmoveto{\pgfqpoint{0.153634in}{-0.027778in}}%
\pgfpathlineto{\pgfqpoint{5.432014in}{-0.027778in}}%
\pgfpathquadraticcurveto{\pgfqpoint{5.470903in}{-0.027778in}}{\pgfqpoint{5.470903in}{0.011111in}}%
\pgfpathlineto{\pgfqpoint{5.470903in}{0.586111in}}%
\pgfpathquadraticcurveto{\pgfqpoint{5.470903in}{0.624999in}}{\pgfqpoint{5.432014in}{0.624999in}}%
\pgfpathlineto{\pgfqpoint{0.153634in}{0.624999in}}%
\pgfpathquadraticcurveto{\pgfqpoint{0.114746in}{0.624999in}}{\pgfqpoint{0.114746in}{0.586111in}}%
\pgfpathlineto{\pgfqpoint{0.114746in}{0.011111in}}%
\pgfpathquadraticcurveto{\pgfqpoint{0.114746in}{-0.027778in}}{\pgfqpoint{0.153634in}{-0.027778in}}%
\pgfpathclose%
\pgfusepath{stroke,fill}%
\end{pgfscope}%
\begin{pgfscope}%
\pgfsetbuttcap%
\pgfsetmiterjoin%
\definecolor{currentfill}{rgb}{1.000000,1.000000,1.000000}%
\pgfsetfillcolor{currentfill}%
\pgfsetlinewidth{1.003750pt}%
\definecolor{currentstroke}{rgb}{0.800000,0.800000,0.800000}%
\pgfsetstrokecolor{currentstroke}%
\pgfsetdash{}{0pt}%
\pgfpathmoveto{\pgfqpoint{0.125857in}{0.000000in}}%
\pgfpathlineto{\pgfqpoint{5.404236in}{0.000000in}}%
\pgfpathquadraticcurveto{\pgfqpoint{5.443125in}{0.000000in}}{\pgfqpoint{5.443125in}{0.038889in}}%
\pgfpathlineto{\pgfqpoint{5.443125in}{0.613888in}}%
\pgfpathquadraticcurveto{\pgfqpoint{5.443125in}{0.652777in}}{\pgfqpoint{5.404236in}{0.652777in}}%
\pgfpathlineto{\pgfqpoint{0.125857in}{0.652777in}}%
\pgfpathquadraticcurveto{\pgfqpoint{0.086968in}{0.652777in}}{\pgfqpoint{0.086968in}{0.613888in}}%
\pgfpathlineto{\pgfqpoint{0.086968in}{0.038889in}}%
\pgfpathquadraticcurveto{\pgfqpoint{0.086968in}{0.000000in}}{\pgfqpoint{0.125857in}{0.000000in}}%
\pgfpathclose%
\pgfusepath{stroke,fill}%
\end{pgfscope}%
\begin{pgfscope}%
\pgfsetrectcap%
\pgfsetroundjoin%
\pgfsetlinewidth{1.505625pt}%
\definecolor{currentstroke}{rgb}{0.000000,0.000000,1.000000}%
\pgfsetstrokecolor{currentstroke}%
\pgfsetdash{}{0pt}%
\pgfpathmoveto{\pgfqpoint{0.164746in}{0.493055in}}%
\pgfpathlineto{\pgfqpoint{0.553634in}{0.493055in}}%
\pgfusepath{stroke}%
\end{pgfscope}%
\begin{pgfscope}%
\definecolor{textcolor}{rgb}{0.000000,0.000000,0.000000}%
\pgfsetstrokecolor{textcolor}%
\pgfsetfillcolor{textcolor}%
\pgftext[x=0.709190in,y=0.425000in,left,base]{\color{textcolor}\rmfamily\fontsize{14.000000}{16.800000}\selectfont Women (main person)}%
\end{pgfscope}%
\begin{pgfscope}%
\pgfsetrectcap%
\pgfsetroundjoin%
\pgfsetlinewidth{1.505625pt}%
\definecolor{currentstroke}{rgb}{0.000000,0.000000,0.000000}%
\pgfsetstrokecolor{currentstroke}%
\pgfsetdash{}{0pt}%
\pgfpathmoveto{\pgfqpoint{0.164746in}{0.195833in}}%
\pgfpathlineto{\pgfqpoint{0.553634in}{0.195833in}}%
\pgfusepath{stroke}%
\end{pgfscope}%
\begin{pgfscope}%
\definecolor{textcolor}{rgb}{0.000000,0.000000,0.000000}%
\pgfsetstrokecolor{textcolor}%
\pgfsetfillcolor{textcolor}%
\pgftext[x=0.709190in,y=0.127778in,left,base]{\color{textcolor}\rmfamily\fontsize{14.000000}{16.800000}\selectfont Men (main person)}%
\end{pgfscope}%
\begin{pgfscope}%
\pgfsetbuttcap%
\pgfsetroundjoin%
\pgfsetlinewidth{1.505625pt}%
\definecolor{currentstroke}{rgb}{1.000000,0.000000,0.000000}%
\pgfsetstrokecolor{currentstroke}%
\pgfsetdash{{5.550000pt}{2.400000pt}}{0.000000pt}%
\pgfpathmoveto{\pgfqpoint{3.008449in}{0.493055in}}%
\pgfpathlineto{\pgfqpoint{3.397337in}{0.493055in}}%
\pgfusepath{stroke}%
\end{pgfscope}%
\begin{pgfscope}%
\definecolor{textcolor}{rgb}{0.000000,0.000000,0.000000}%
\pgfsetstrokecolor{textcolor}%
\pgfsetfillcolor{textcolor}%
\pgftext[x=3.552893in,y=0.425000in,left,base]{\color{textcolor}\rmfamily\fontsize{14.000000}{16.800000}\selectfont Women (met person)}%
\end{pgfscope}%
\begin{pgfscope}%
\pgfsetbuttcap%
\pgfsetroundjoin%
\pgfsetlinewidth{1.505625pt}%
\definecolor{currentstroke}{rgb}{0.750000,0.000000,0.750000}%
\pgfsetstrokecolor{currentstroke}%
\pgfsetdash{{5.550000pt}{2.400000pt}}{0.000000pt}%
\pgfpathmoveto{\pgfqpoint{3.008449in}{0.195833in}}%
\pgfpathlineto{\pgfqpoint{3.397337in}{0.195833in}}%
\pgfusepath{stroke}%
\end{pgfscope}%
\begin{pgfscope}%
\definecolor{textcolor}{rgb}{0.000000,0.000000,0.000000}%
\pgfsetstrokecolor{textcolor}%
\pgfsetfillcolor{textcolor}%
\pgftext[x=3.552893in,y=0.127778in,left,base]{\color{textcolor}\rmfamily\fontsize{14.000000}{16.800000}\selectfont Men (met person)}%
\end{pgfscope}%
\end{pgfpicture}%
\makeatother%
\endgroup%
 } 
\end{subfigure}
\end{center}

\begin{minipage}{0.99\textwidth} % choose width suitably
{\footnotesize \textsc{Notes.} The figures display means and variances of simulated log wages and assets of men and women in a couple over their age. We label as ``main person" the variables that are computed from agents that are simulated and followed through their whole life-cycle, while we label as ``met person" the variables constructed using the partners met by the people whose behavior is simulated for their whole life-cycle. Wage variables are constructed using couples at any point of their relationship, while for assets we use only the period the couple met, where we can still distinguish the title of ownership of assets. \par}
\end{minipage}
\end{figure}
\FloatBarrier



\end{document}